\title{Talnafræði}
\subtitle{Veldishafning}
\author{Bergur Snorrason}
\date{\today}

\begin{document}

\frame{\titlepage}

\env{frame}
{
	\env{itemize}
	{
		\item<1-> Í talningarfræðidæmum er algengt að skila eigi leif svars.
		\item<2-> Þetta er gert því leif er takmörkuð, en í talningarfræði er algengt að fá mjög stórar tölur.
		\item<3-> Tökum dæmi.
	}
}

\env{frame}
{
	\frametitle{Veldishafning}
	\env{itemize}
	{
		\item<1-> Gefnar eru þrjár jákvæðar heiltölur $x$, $n$ og $m$.
		\item<2-> Finnið $x^n \mod m$.
	}
}

\env{frame}
{
	\env{itemize}
	{
		\item<1-> Þetta er lítið mál að gera í $\mathcal{O}(n)$ tíma.
		\item<2->[] \selectcode{code/exp-linear.c}{4}{11}
		\item<3-> Við getum þó leyst þetta hraðar.
		\item<4-> Sú lausn byggir á að deila og drottna.
		\item<5-> Takið eftir að $x^{2n} = x^n \cdot x^n$ og $x^{2n + 1} = x^n \cdot x^n \cdot x$.
		\item<6-> Því getum við í hverju skrefi helmingað veldisvísinn.
	}
}

\env{frame}
{
	\env{itemize}
	{
		\item<1-> Við getum útfært þetta endurkvæmt.
		\item<2->[] \selectcode{code/exp-log-rec.c}{4}{10}
		\item<3-> Við getum líka gert þetta með einfaldri \texttt{for}-lykkju.
		\item<4->[] \selectcode{code/exp-log-iter.c}{4}{14}
	}
}

\env{frame}
{
	\env{itemize}
	{
		\item<1-> Eins og sagt var áðan þá helmingast veldisvísirinn í hverju skrefi.
		\item<2-> Svo tímaflækjan er $\mathcal{O}($\onslide<3->{$\log n$}$)$.
	}
}

\env{frame}
{
}

\end{document}
