\title{Miðmisseriskönnun}
\author{Bergur Snorrason}
\date{\today}

\begin{document}

\frame{\titlepage}

\env{frame}
{
	\env{itemize}
	{
		\item<1-> Miðmisseriskönnun var að ljúka.
		\item<2-> Það er ein athugasemd sem ég vil ræða.
	}
}

\env{frame}
{
	Mætti kenna aðeins meira markvisst, taka fyrir alla algórithmana sem maður þarf að kunna fyrir næstu vikudæmi.
	Oft eru einhverjar pælingar sem maður hefur aldrei séð áður,
	eins og svokallaða ,,Two egg'' vandamálið sem var í raun endurskrifað í dæmið ,,Exploding Batteries'' á Kattis.
	Þá þurfti maður að eyða miklum tíma í að googla og finna svipað vandamál (þannig rakst ég á ,,Two egg problem'').
}

\env{frame}
{
	\code{code/batteries.c}
}

\env{frame}
{
	\env{itemize}
	{
		\item<1-> Ég mun fara að nota styttingunar \texttt{rep(...)} restina af námskeiðinu.
		\item<2-> Hún er notuð til að spara skriftir í \texttt{for}-lykkjum.
		\item<3-> Hún er skilgreind með \texttt{\#define rep(E, F) for (E = 0; E < (F); E++)}.
		\item<4-> Þetta þýðir að \texttt{rep(i, n)} er jafngilt því að skrifa \texttt{for (i = 0; i < n; i++)}.
		\item<5-> Sem dæmi um hagnýtingu þessara styttingu má sjá útfærsluna mína á bakstrengsfylkjasmið (e. suffix array constructor).
	}
}

\env{frame}
{
	\code{code/suffix-array.c}
}

\env{frame}
{
}

\end{document}
