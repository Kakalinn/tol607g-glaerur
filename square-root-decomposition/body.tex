\title{Sniðmát}
\author{Bergur Snorrason}
\date{\today}

\begin{document}

\frame{\titlepage}

\env{frame}
{
	\frametitle{Rótarþáttun}
	\env{itemize}
	{
		\item<1-> Skoðum aftur dæmið sem við skoðuðum í biltrjáa kaflanum.
		\item<2-> Hvað ef við skiptum listanum okkar upp í nokkurn hólf og geymum summu hvers hólfs fyrir sig.
		\item<3-> Segjum að við höfum $k$ hólf, og $n$ tölur.
		\item<4-> Ef við viljum finna summu yfir hlutbil nægir okkur leggja saman þau gildi frá endapunktum bilsins upp að næstu hólfamörkum,
			svo leggjum við saman hólfin á milli.
		\item<5-> Þessa aðgerð er því $\mathcal{O}(k + n/k)$, og ef við veljum $k = \sqrt{n}$ fæst að hún er $\mathcal{O}(\sqrt{n})$.
		\item<6-> Til að uppfæra gildi í listanum leggjum við saman öll stökin í hólfinu, sem tekur $\mathcal{O}(\sqrt{n})$.
	}
}

\env{frame}
{
	\frametitle{Rótarþáttun}
	\env{itemize}
	{
		\item<1-> Þessa almennu aðferð má nota í flestum dæmum sem eru leysanleg með biltrjám og kallast hún \emph{rótarþáttun} (e. squareroot decomposition).
		\item<2-> Þetta er þó hægara en biltréin (virkar t.d. ekki fyrir $n = 10^6$.
		\item<3-> Kosturinn við þessa aðferð er að hún er létt í útfærlsu eftir smá æfingu og er almennari en biltréin.
	}
}

\env{frame}
{
	\frametitle{Rótarþáttun}
	\code{code/sq.c}
}

\end{document}
