\title{Hrúgur}
\author{Bergur Snorrason}
\date{\today}

\begin{document}

\frame{\titlepage}

\env{frame}
{
	\frametitle{Hrúgur}
	\env{itemize}
	{
		\item<1-> Rótartvíundatré sem uppfyllir að sérhver nóða er stærri en börnin sín er sagt uppfylla \emph{hrúguskilyrðið}.
		\item<2-> Við köllum slík tré \emph{hrúgur} (e. heap).
		\item<3-> Hrúgur eru heppilega auðveldar í útfærslu.
		\item<4-> Við geymum tréð sem fylki og eina erfiðið er að viðhalda hrúguskilyrðinu.
	}
}

\env{frame}
{
	\frametitle{Fylki sem tré}
	\env{itemize}
	{
		\item<1-> Þegar við geymum tréð sem fylki notum við eina af tveimur aðferðum.
		\item<2-> Sú fyrri:
		\env{itemize}
		{
			\item<3-> Rótin er í staki $1$ í fylkinu.
			\item<4-> Vinstra barn staksins $i$ er stak $2\times i$.
			\item<5-> Hægra barn staksins $i$ er stak $2\times i + 1$.
			\item<6-> Foreldri staks $i$ er stakið $\left \lfloor \dfrac{i}{2} \right \rfloor$.
		}
		\item<7-> Sú seinni:
		\env{itemize}
		{
			\item<8-> Rótin er í staki $0$ í fylkinu.
			\item<9-> Vinstra barn staksins $i$ er stak $2\times i + 1$.
			\item<10-> Hægra barn staksins $i$ er stak $2\times i + 2$.
			\item<11-> Foreldri staks $i$ er stakið $\left \lfloor \dfrac{i - 1}{2} \right \rfloor$.
		}
	}
}

\env{frame}
{
	\frametitle{Hrúga í \texttt{C}}
	\code{code/heap-skel.c}
}

\env{frame}
{
	\frametitle{Hrúga í \texttt{C}}
	\code{code/pop-peek-push.c}
}

\env{frame}
{
	\frametitle{Hrúga í \texttt{C}}
	\code{code/fix-down.c}
}

\env{frame}
{
	\frametitle{Hrúga í \texttt{C}}
	\code{code/fix-up.c}
}

\end{document}
