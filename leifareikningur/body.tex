\title{Talnafræði}
\subtitle{Leifareikningur}
\author{Bergur Snorrason}
\date{\today}

\begin{document}

\frame{\titlepage}

\env{frame}
{
	\frametitle{Leifareikningur}
	\env{itemize}
	{
		\item<1-> Látum $a$ og $m$ vera jákvæðar heiltölur.
		\item<2-> Alltaf eru til jákvæðar heiltölur $r < m$ og $q$ þannig $a = q \cdot m + r$.
		\item<3-> Við segjum þá að $r$ sé \emph{leif $a$ með tilliti til $m$}.
		\item<4-> Við skrifum svo $b = a \mod m$ ef $a$ og $b$ hafa sömu leif með tilliti til $m$.
		\item<5-> Flest forritunarmál reikna þessa leif með \texttt{a\%m}.
		\item<6-> Gerum nú ráð fyrir að við séum með jákvæðar heiltölur $a_1$, $a_2$, $m$, og $r_1 = a_1 \mod m$ og $r_2 = a_2 \mod m$.
		\item<7-> Þá gildir að
		\[
			r_1 + r_2 = a_1 + a_2 \mod m
		\]
		 og
		\[
			r_1 \cdot r_2 = a_1 \cdot a_2 \mod m.
		\]
	}
}

\env{frame}
{
	\env{itemize}
	{
		\item<1-> Þið þurfið að passa ykkur ef þið eruð með neikvæðar tölur.
		\item<2-> Til dæmis er ekki skilgreint hverju \texttt{(-10)\%3} skilar í \texttt{C}.
		\item<3-> Við vitum ekki hvort það skili \texttt{-1} eða \texttt{2}.
		\item<4-> Til að komast í kringum þessa óvissu notum við frekar \texttt{(a\%m + m)\%m} ef \texttt{a} getur verið neikvæð.
		\item<5-> Þetta virkar því \texttt{a\%m + m} verður alltaf jákvæð.
	}
}

\env{frame}
{
	\env{itemize}
	{
		\item<1-> Einnig þarf að passa sig að tölurnar verði ekki of stórar.
		\item<2-> Til dæmis er algengt í keppnisforritun að reikna leif með tilliti til $m = 10^9 + 7$.
		\item<3-> Takið eftir að $m$ er ekki of stór fyrir \texttt{int}.
		\item<4-> Ef við erum með tvær \texttt{int} tölur, \texttt{a} og \texttt{b}, og viljum reikna \texttt{(a*b)\%m}
					þá gæti \texttt{a*b} orðið of stór fyrir \texttt{int}.
		\item<5-> Til að komast hjá þessu þurfum við að nota \texttt{long long}.
		\item<6-> Ef tölurnar eru \texttt{long long} í stað \texttt{int} þurfum við að nota \texttt{\_\_int128}.
		\item<7->[] \selectcode{code/bigprod.c}{2}{8}
	}
}

\env{frame}
{
	\env{itemize}
	{
		\item<1-> Stundum þurfum við að geta deilt í leifareikningi.
		\item<2-> Þetta er ekki hægt að gera með hefðbundinni deilingu.
		\item<3-> Við látum $b^{-1}$ tákna þá tölu sem uppfyllir að $1 = b \cdot b^{-1} \mod m$.
		\item<4-> Þessi tala er ekki alltaf til.
		\item<5-> Hún er þó alltaf til ef $m$ er frumtala.
		\item<6-> Við köllum $b^{-1}$ \emph{margföldunar andhverfu $b$ með tilliti til $m$}.
		\item<7-> Við skrifum svo stundum $a/b$ í stað $ab^{-1}$.
		\item<8-> En hvernig finnum við þessa tölu?
	}
}

\env{frame}
{
	\env{itemize}
	{
		\item<1-> Látum $p$ vera frumtölu.
		\item<2-> Litla setning Fermats segir okkur að $a^p = a \mod p$.
		\item<3-> Ef við margföldum báðum megin með $a^{-2}$ fæst að $a^{p - 2} = a^{-1} \mod p$.
		\item<4-> Svo eina sem við þurfum að gera er að reikna $a^{p - 2} \mod p$.
		\item<5-> Gerum ráð fyrir að við séum með fall \texttt{modpow(x, n, m)} sem reiknar $x^n \mod m$ (við útfærum það á eftir).
		\item<6->[] \selectcode{code/mulinv.c}{16}{19}
		\item<7-> Tímaflækjan á þessari aðferð verður síðan sú sama og tímaflækjan á \texttt{modpow(...)}.
	}
}

\env{frame}
{
	\env{itemize}
	{
		\item<1-> Til að finna $a^{-1} \mod m$ ef $m$ er ekki frumtala er ögn flóknara.
		\item<2-> Við skoðum það á eftir þegar við skoðum reiknirit Evklíðs.
	}
}

\env{frame}
{
}

\end{document}
