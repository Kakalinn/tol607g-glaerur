\title{Reiknirit Ahos og Corasicks ($1975$)}
\author{Bergur Snorrason}
\date{\today}

\begin{document}

\frame{\titlepage}

\env{frame}
{
	\env{itemize}
	{
		\item<1-> Gerum ráð fyrir að við séum með stafróð $\Sigma$, streng $s$ og list af $n$ strengjum $p$, þar sem $j$-ti strengurinn kallast $p_j$.
		\item<2-> Látum $|s|$ tákna lengd strengsins $s$ og $|p| = |p_1| + \dots + |p_n|$.
		\item<3-> Við viljum finna alla hlutstrengi $s$ sem eru í listanum $p$.
		\item<4-> Við getum notað reiknirit Knuths, Morrisar og Pratts $n$ sinnum.
		\item<5-> Þessi aðferð hefur tímaflækjuna $\mathcal{O}(n \cdot |s| + |p|)$.
		\item<6-> Reiknirit Ahos og Corasicks bætir þetta.
	}
}

\env{frame}
{
	\env{itemize}
	{
		\item<1-> Byrjum á að setja all strengina í $p$ inn í forgangstré $T$.
		\item<2-> Við viljum síðan búa til stöðuvél úr $T$.
		\item<3-> Hnútarnir í trénu eru stöðurnar, en okkur vantar að finna færslur fyrir hverja stöðu og bókstaf í $\Sigma$.
		\item<4-> Gerum ráð fyrir að við séum í hnút $v$ í $T$ og viljum með fyrir staf $c$ í $\Sigma$.
		\item<5-> Ef það er leggur úr $v$ merktur með $c$ þá ferðumst við eftir honum.
		\item<6-> Annars þurfum við að fara aftur í hnút $w$ þannig að strengurinn sem svarar til $w$ er bakstrengur strengsins sem svarar til $v$.
		\item<7-> Við viljum hafa strenginn $w$ sem lengstan (með öðrum orðum viljum við fara sem styst aftur).
		\item<8-> Við köllum þessa leggi \emph{basktrengsleggi} (e. \emph{suffix links}).
		\item<9-> Takið eftir að þeir eru í raun óháðir bókstafnum $c$.
		\item<10-> Við látum bakstrengslegg rótarinn benda á sjálfa sig.
	}
}

\env{frame}
{
	\env{itemize}
	{
		\item<1-> Hvernig finnum við alla bakstrengsleggina?
		\item<2-> Við notum kvika bestun.
		\item<3-> Látum $f(w, c)$ tákna færslu úr stöðu $w$ með staf $c$ og $g(w)$ tákna bakstrengslegg $w$.
		\item<4-> Gerum einnig ráð fyrir að foreldri $v$ sé $p$ og $f(p, a) = v$.
		\item<5-> Við sjáum þá að $g(v) = f(g(p), a)$.
		\item<6-> Með öðrum orðum förum við upp í foreldrið, ferðumst eftir bakstrenglegg þaðan og færum okkur í stöðuvélinni eftir $a$.
		\item<7-> Við höfum þá rakningarformúlu sem við getum notað til að reikna bakstrengshlekkina.
	}
}

\env{frame}
{
	\env{tikzpicture}
	{
		\node[draw, circle, thick, white] (0) at (-1, 0) {};

		\only<all:2, 3> { \node[draw, circle, fill, blue, thick, inner sep = 1.0pt] at (2, -1) {}; }
		\only<all:4, 5> { \node[draw, circle, fill, blue, thick, inner sep = 1.0pt] at (3, -1) {}; }
		\only<all:6, 7> { \node[draw, circle, fill, blue, thick, inner sep = 1.0pt] at (4, -1) {}; }
		\only<all:8, 9> { \node[draw, circle, fill, blue, thick, inner sep = 1.0pt] at (2, 1) {}; }
		\only<all:10, 11> { \node[draw, circle, fill, blue, thick, inner sep = 1.0pt] at (3, 1) {}; }
		\only<all:12, 13> { \node[draw, circle, fill, blue, thick, inner sep = 1.0pt] at (4, 1) {}; }

		\node[draw, circle, thick, inner sep = 1.0pt] (1) at (0, 0) {};
		\node[draw, circle, thick, inner sep = 1.0pt] (2) at (1, 1) {};
		\node[draw, circle, thick, inner sep = 1.0pt] (3) at (2, 1) {};
		\node[draw, circle, thick, inner sep = 1.0pt] (4) at (3, 1) {};
		\node[draw, circle, thick, inner sep = 1.0pt] (5) at (4, 1) {};
		\node[draw, circle, thick, inner sep = 1.0pt] (6) at (1, -1) {};
		\node[draw, circle, thick, inner sep = 1.0pt] (7) at (2, -1) {};
		\node[draw, circle, thick, inner sep = 1.0pt] (8) at (3, -1) {};
		\node[draw, circle, thick, inner sep = 1.0pt] (9) at (4, -1) {};

		\path[draw, thick, ->] (0) -- (1);
		\path[draw, thick] (1) -- (2);
		\path[draw, thick] (2) -- (3);
		\path[draw, thick] (3) -- (4);
		\path[draw, thick] (4) -- (5);
		\path[draw, thick] (1) -- (6);
		\path[draw, thick] (6) -- (7);
		\path[draw, thick] (7) -- (8);
		\path[draw, thick] (8) -- (9);
		\only<all:1-> { \path[draw, thick, white, dotted] (9) edge[bend left = 90] node {} (1); } %alignment
		\only<all:3-> { \path[draw, thick, green, dotted] (7) edge[bend left = 90] node {} (1); }
		\only<all:5-> { \path[draw, thick, green, dotted] (8) edge[bend left = 90] node {} (1); }
		\only<all:7-> { \path[draw, thick, green, dotted] (9) edge[bend left = 90] node {} (1); }
		\only<all:9-> { \path[draw, thick, green, dotted] (3) -- (6); }
		\only<all:11-> { \path[draw, thick, green, dotted] (4) -- (7); }
		\only<all:13-> { \path[draw, thick, green, dotted] (5) -- (8); }

		\node at (0.5, 0.8) {a};
		\node at (1.5, 1.2) {b};
		\node at (2.5, 1.2) {c};
		\node at (3.5, 1.2) {d};
		\node at (0.6, -0.4) {b};
		\node at (1.5, -0.8) {c};
		\node at (2.5, -0.8) {d};
		\node at (3.5, -0.8) {e};
	}
}

\env{frame}
{
	\env{itemize}
	{
		\item<1-> Í $T$ merkjum við lokastöður þar sem strengir enda.
		\item<2-> Við ferðumst svo í gegnum strenginn $s$ og hliðrum stöðvélinni miðað við stafina í $s$.
	}
}

\env{frame}
{
	\env{tikzpicture}
	{
		\node[draw, circle, thick, white] (0) at (-1, 0) {};

		\only<all:1-2> { \node at (3, 2) {,,abcdcdeaaabcdeabcxab''}; }
		\only<all:3> { \node at (3, 2)    {,,bcdcdeaaabcdeabcxab''}; }
		\only<all:4> { \node at (3, 2)     {,,cdcdeaaabcdeabcxab''}; }
		\only<all:5> { \node at (3, 2)      {,,dcdeaaabcdeabcxab''}; }
		\only<all:6-8> { \node at (3, 2)     {,,cdeaaabcdeabcxab''}; }
		\only<all:9> { \node at (3, 2)        {,,deaaabcdeabcxab''}; }
		\only<all:10> { \node at (3, 2)        {,,eaaabcdeabcxab''}; }
		\only<all:11> { \node at (3, 2)         {,,aaabcdeabcxab''}; }
		\only<all:12-13> { \node at (3, 2)       {,,aabcdeabcxab''}; }
		\only<all:14-15> { \node at (3, 2)        {,,abcdeabcxab''}; }
		\only<all:16> { \node at (3, 2)            {,,bcdeabcxab''}; }
		\only<all:17> { \node at (3, 2)             {,,cdeabcxab''}; }
		\only<all:18> { \node at (3, 2)              {,,deabcxab''}; }
		\only<all:19-20> { \node at (3, 2)            {,,eabcxab''}; }
		\only<all:21-22> { \node at (3, 2)             {,,abcxab''}; }
		\only<all:23> { \node at (3, 2)                 {,,bcxab''}; }
		\only<all:24> { \node at (3, 2)                  {,,cxab''}; }
		\only<all:25-26> { \node at (3, 2)                {,,xab''}; }
		\only<all:27> { \node at (3, 2)                    {,,ab''}; }
		\only<all:28> { \node at (3, 2)                     {,,b''}; }
		\only<all:29-30> { \node at (3, 2)                   {,,''}; }

		\only<all:2> { \node[draw, circle, fill, blue, thick, inner sep = 1.0pt] at (0, 0) {}; }
		\only<all:3> { \node[draw, circle, fill, blue, thick, inner sep = 1.0pt] at (1, 1) {}; }
		\only<all:4> { \node[draw, circle, fill, blue, thick, inner sep = 1.0pt] at (2, 1) {}; }
		\only<all:5> { \node[draw, circle, fill, blue, thick, inner sep = 1.0pt] at (3, 1) {}; }
		\only<all:6> { \node[draw, circle, fill, blue, thick, inner sep = 1.0pt] at (4, 1) {}; }
		\only<all:7> { \node[draw, circle, fill, blue, thick, inner sep = 1.0pt] at (3, -1) {}; }
		\only<all:8-11, 13, 15> { \node[draw, circle, fill, blue, thick, inner sep = 1.0pt] at (0, 0) {}; }
		\only<all:12, 14, 16> { \node[draw, circle, fill, blue, thick, inner sep = 1.0pt] at (1, 1) {}; }
		\only<all:17> { \node[draw, circle, fill, blue, thick, inner sep = 1.0pt] at (2, 1) {}; }
		\only<all:18> { \node[draw, circle, fill, blue, thick, inner sep = 1.0pt] at (3, 1) {}; }
		\only<all:19> { \node[draw, circle, fill, blue, thick, inner sep = 1.0pt] at (4, 1) {}; }
		\only<all:20> { \node[draw, circle, fill, blue, thick, inner sep = 1.0pt] at (3, -1) {}; }
		\only<all:21> { \node[draw, circle, fill, blue, thick, inner sep = 1.0pt] at (4, -1) {}; }
		\only<all:22> { \node[draw, circle, fill, blue, thick, inner sep = 1.0pt] at (0, 0) {}; }
		\only<all:23> { \node[draw, circle, fill, blue, thick, inner sep = 1.0pt] at (1, 1) {}; }
		\only<all:24> { \node[draw, circle, fill, blue, thick, inner sep = 1.0pt] at (2, 1) {}; }
		\only<all:25> { \node[draw, circle, fill, blue, thick, inner sep = 1.0pt] at (3, 1) {}; }
		\only<all:26-27> { \node[draw, circle, fill, blue, thick, inner sep = 1.0pt] at (0, 0) {}; }
		\only<all:28> { \node[draw, circle, fill, blue, thick, inner sep = 1.0pt] at (1, 1) {}; }
		\only<all:29> { \node[draw, circle, fill, blue, thick, inner sep = 1.0pt] at (2, 1) {}; }

		\node[draw, circle, thick, inner sep = 1.0pt] (1) at (0, 0) {};
		\node[draw, circle, thick, inner sep = 1.0pt] (2) at (1, 1) {};
		\node[draw, circle, thick, inner sep = 1.0pt] (3) at (2, 1) {};
		\node[draw, circle, thick, inner sep = 1.0pt] (4) at (3, 1) {};
		\node[draw, circle, thick, inner sep = 1.0pt] (5) at (4, 1) {};
		\node[draw, circle, thick, inner sep = 1.0pt] (6) at (1, -1) {};
		\node[draw, circle, thick, inner sep = 1.0pt] (7) at (2, -1) {};
		\node[draw, circle, thick, inner sep = 1.0pt] (8) at (3, -1) {};
		\node[draw, circle, thick, inner sep = 1.0pt] (9) at (4, -1) {};

		\path[draw, thick, ->] (0) -- (1);
		\path[draw, thick] (1) -- (2);
		\path[draw, thick] (2) -- (3);
		\path[draw, thick] (3) -- (4);
		\path[draw, thick] (4) -- (5);
		\path[draw, thick] (1) -- (6);
		\path[draw, thick] (6) -- (7);
		\path[draw, thick] (7) -- (8);
		\path[draw, thick] (8) -- (9);
		\path[draw, thick, green, dotted] (2) edge[bend left = 40] node {} (1);
		\path[draw, thick, green, dotted] (3) -- (6);
		\path[draw, thick, green, dotted] (4) -- (7);
		\path[draw, thick, green, dotted] (5) -- (8);
		\path[draw, thick, green, dotted] (6) edge[bend left = 40] node {} (1);
		\path[draw, thick, green, dotted] (7) edge[bend left = 90] node {} (1);
		\path[draw, thick, green, dotted] (8) edge[bend left = 90] node {} (1);
		\path[draw, thick, green, dotted] (9) edge[bend left = 90] node {} (1);

		\node at (0.5, 0.8) {a};
		\node at (1.5, 1.2) {b};
		\node at (2.5, 1.2) {c};
		\node at (3.5, 1.2) {d};
		\node at (0.6, -0.4) {b};
		\node at (1.5, -0.8) {c};
		\node at (2.5, -0.8) {d};
		\node at (3.5, -0.8) {e};
	}
}

\env{frame}
{
	\env{itemize}
	{
		\item<1-> Ljóst er að alltaf þegar við erum í lokstöðu þá erum við með hlutstreng í $s$ sem er í $p$.
		\item<2-> En eru þetta einu slíku hlutstrengirnir?
		\item<3-> Nei, við þurfum líka, í hverju skrefi, að athuga hvort við getum komist í lokastöðu ef við ferðumst eftir bakstrengsleggjum.
		\item<4->[]
		\env{tikzpicture}
		{
			\node[draw, circle, thick, white] (0) at (-1, 0) {};
			\onslide<all:6, 7> { \node at (3, 2) {,,bcb''}; }
			\onslide<all:8> { \node at (3, 2)     {,,cb''}; }
			\onslide<all:9> { \node at (3, 2)      {,,b''}; }
			\onslide<all:10-11> { \node at (3, 2)   {,,''}; }

			\onslide<all:7> { \node[draw, circle, fill, blue, thick, inner sep = 1.0pt] at (0, 0) {}; }
			\onslide<all:8> { \node[draw, circle, fill, blue, thick, inner sep = 1.0pt] at (1, 1) {}; }
			\onslide<all:9> { \node[draw, circle, fill, blue, thick, inner sep = 1.0pt] at (2, 1) {}; }
			\onslide<all:10> { \node[draw, circle, fill, blue, thick, inner sep = 1.0pt] at (3, 1) {}; }

			\node[draw, circle, thick, inner sep = 1.0pt] (1) at (0, 0) {};
			\node[draw, circle, thick, inner sep = 1.0pt] (2) at (1, 1) {};
			\node[draw, circle, thick, inner sep = 1.0pt] (3) at (2, 1) {};
			\node[draw, circle, thick, inner sep = 1.0pt] (4) at (3, 1) {};
			\node[draw, circle, thick, inner sep = 1.0pt] (5) at (1, -1) {};

			\path[draw, thick, ->] (0) -- (1);
			\path[draw, thick] (1) -- (2);
			\path[draw, thick] (2) -- (3);
			\path[draw, thick] (3) -- (4);
			\path[draw, thick] (1) -- (5);
			\path[draw, thick, green, dotted] (2) edge[bend left = 40] node {} (1);
			\path[draw, thick, green, dotted] (3) -- (5);
			%\path[draw, thick, green, dotted] (4) -- (6);
			\path[draw, thick, green, dotted] (4) edge[bend right = 90] node {} (1);
			\path[draw, thick, green, dotted] (5) edge[bend left = 40] node {} (1);

			\node at (0.5, 0.8) {b};
			\node at (1.5, 1.2) {c};
			\node at (2.5, 1.2) {b};
			\node at (0.6, -0.4) {c};
		}
		\item<12-> Til að koma í vega fyrir að tímaflækjan verði of slæm þá notum við aftur kvika bestun.
		\item<13-> Við bætum í raun leggjum inn í tréð, sem við köllum \emph{lokaleggi} (e. \emph{exit link}).
	}
}

\env{frame}
{
	\env{itemize}
	{
		\item<1-> Í útfærslunni munum við notast við þrjú hjálparföll.
		\item<2-> Fyrsta heitir \texttt{trie\_step(...)} og er það notað til að ferðast um stöðuvélina.
		\item<3-> Annað heitir \texttt{trie\_step(...)} og er það notað til að finna bakstrengsleggi.
		\item<4-> Þriðja heitir \texttt{trie\_exit(...)} og er það notað til að finna lokaleggina.
		\item<5-> Öll þessi föll eru endurkvæm og notast við minnun.
	}
}

\env{frame}
{
	\selectcode{code/aho-corasick.c}{24}{52}
}

\env{frame}
{
	\selectcode{code/aho-corasick.c}{54}{74}
}

\env{frame}
{
	\selectcode{code/aho-corasick.c}{76}{102}
}

\env{frame}
{
	\env{itemize}
	{
		\item<1-> Gerum ráð fyrir að strengirnar í $p$ komi fyrir $k$ sinnum í $s$.
		\item<2-> Þá er tímaflækjan $\mathcal{O}($\onslide<3->{$|s| + |\Sigma| \cdot |p| + k$}$)$.
		\item<4-> Ef við höfum bara áhuga á að finna töluna $k$ getum við breytt \texttt{trie\_exit(...)}
					þannig að það reikni fjölda lokastaða á leiðinni að rót eftir bakstrengsleggjum.
		\item<5-> Þá verður tímaflækjan $\mathcal{O}($\onslide<6->{$|s| + |\Sigma| \cdot |p|$}$)$.
		\item<7-> Takið eftir að, ef stafrófið er takmarkað, þá er seinni tímaflækjan línuleg.
	}
}

\env{frame}
{
}

\end{document}
