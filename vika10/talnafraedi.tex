\documentclass{beamer}
\usefonttheme[onlymath]{serif}
\usepackage[T1]{fontenc}
\usepackage[utf8]{inputenc}
\usepackage[english, icelandic]{babel}
\usepackage{amsmath}
\usepackage{amssymb}
\usepackage{amsthm}
\usepackage{gensymb}
\usepackage{parskip}
\usepackage{mathtools}
\usepackage{xfrac}
\usepackage{graphicx}
\usepackage{xcolor}
\usepackage{tikz}
\usetikzlibrary{calc}
\usetikzlibrary{positioning}
\usepackage{verbatim}
\usepackage{multicol}
\usepackage{minted}
\parskip 0pt

\DeclareMathOperator{\lcm}{lcm}
\DeclareMathOperator{\diam}{diam}
\DeclareMathOperator{\dist}{dist}
\DeclareMathOperator{\ord}{ord}
\DeclareMathOperator{\Aut}{Aut}
\DeclareMathOperator{\Inn}{Inn}
\DeclareMathOperator{\Ker}{Ker}
\DeclareMathOperator{\trace}{trace}
\DeclareMathOperator{\fix}{fix}
\DeclareMathOperator{\Log}{Log}
\newcommand\floor[1]{\left\lfloor#1\right\rfloor}
\newcommand\ceil[1]{\left\lceil#1\right\rceil}
\newcommand\abs[1]{\left|#1\right|}
\newcommand\p[1]{\left(#1\right)}
\newcommand\sqp[1]{\left[#1\right]}
\newcommand\cp[1]{\left\{#1\right\}}
\newcommand\norm[1]{\left\lVert#1\right\rVert}
\renewcommand\qedsymbol{$\blacksquare$}
\renewcommand\Im{\operatorname{Im}}
\renewcommand\Re{\operatorname{Re}}

\usetheme{Madrid}
\definecolor{dark yellow}{rgb} {0.6,0.6,0.0}
\definecolor{dark green}{rgb} {0.0,0.6,0.0}

\title{Talnafræði}
\subtitle{Leifareikningur, frumtölur o.fl.}
\author{Atli Fannar Franklín}
\date{\today}

\graphicspath{{myndir/}}

\AtBeginSection[] {
  \begin{frame}
    \frametitle{Efnisyfirlit}
    \tableofcontents[currentsection]
  \end{frame}
}


\begin{document}

\frame{\titlepage}

\section[Leifareikningur]{Leifareikningur}

\begin{frame}
\frametitle{Hvað er leifareikningur?}

\begin{itemize}

\item<1-> Þegar við tölum um leifareikning erum við að tala um útreikninga sem eru allir gerðir á leifum einhverrar frumtölu $p$. Þá geymum við ávallt bara afgang tölunnar okkur þegar henni er deilt með $p$ í stað þess að geyma töluna alla.

\item<2-> Samlagning virkar með hvaða tölu $n$ sem er en til þess að margföldun virki eins og við viljum þurfum við að nota frumtölu $p$. Þetta er vegna þess að ef við vinnum með frumtölu þýðir $ab = 0$ að $a = 0$ eða $b = 0$. Ef við værum að reikna með t.d. 15 væri $5 \cdot 3 = 0$ hins vegar, sem við viljum ekki.

\item<3-> Það að geyma 'bara' leifina getur samt sagt okkur heilmikið um niðurstöðuna. Einnig kemur oft fyrir að svar í dæmi sé svo stórt að það sé bara beðið um að skila leifinni.
\end{itemize}

\end{frame}

\begin{frame}
\frametitle{Fleiri leifaaðgerðir}

\begin{itemize}

\item<1-> Þegar við hefjum í veldi í leifareikningi hækkar talan ekki endilega, svo við erum oft beðin um að hefja tölu í afar stór veldi þegar við erum að framkvæma leifareikning. Hvernig má gera slíkt hratt? Það er augljóst hvernig má hefja í $n$-ta veldi í $\mathcal{O}(n)$ tíma. Hins vegar má gera það í $\mathcal{O}(\log(n))$ tíma!

\item<2-> Við hefjum bara í annað veldi aftur og aftur til að komast nær veldinu í $\log$ tíma, með smá auka til þess að við endum örugglega í nákvæmlega rétta veldinu!

\end{itemize}

\end{frame}

\begin{frame}[fragile]
\frametitle{Fast mod pow}

\begin{minted}{cpp}
#include<bits/stdc++.h>
using namespace std;
typedef long long ll;

ll modpow(ll b, ll e, ll m) {
    ll res = 1;
    while(e) {
        if(e & 1) res = ((res * b) % m + m) % m;
        b = ((b * b) % m + m) % m;
        e >>= 1;
    }
    return res;
}
\end{minted}

\end{frame}

\begin{frame}
\frametitle{En deiling?}

\begin{itemize}

\item<1-> Hvernig deilum við þegar við erum að vinna með leif?

\item<2-> Það er svolítið snúnara og við þurfum að skoða nokkra aðra hluti fyrst til að geta svarað þessu. Fyrst skoðum við stærsta samdeili. Vitið þið hvernig má finna stærsta samdeili í tölvu?

\item<3-> Við notum reiknirit Evklíðs bara eins og það leggur sig! 

\item<4-> Æfing: Sýna að þetta sé $\mathcal{O}(\log(n))$ (hint: sýna að $a + b$ minnki um fjórðung eftir tvær ítranir sama hvað)

\end{itemize}

\end{frame}

\begin{frame}[fragile]
\frametitle{Reiknirit Evklíðs}

\begin{minted}{cpp}
#include<bits/stdc++.h>
using namespace std;
typedef long long ll;

ll gcd(ll a, ll b) { 
    return b == 0 ? a : gcd(b, a % b); 
}
\end{minted}

\end{frame}

\begin{frame}
\frametitle{Reiknirit Evklíðs}

\begin{itemize}

\item<1-> Skoðum þetta reiknirit samt aðeins nánar. Ef $a, b \in \mathbb{Z}$ og $d = gcd(a, b)$ kallast jafnan $ax + by = d$ Bézout-jafnan. Ef við greinum nákvæmlega hvað reiknirit Evklíðs gerir þá getum við séð að við getum lesið lausn á Bézout-jöfnunni út úr því sem reiknirit Evklíðs gerir.

\item<2-> Því ef við breitum forritinu að ofan bara aðeins, svo það haldi aðeins meira bókhald, þá getum við leyst þessa jöfnu frítt með.

\end{itemize}

\end{frame}

\begin{frame}[fragile]
\frametitle{Útvíkkaða reiknirit Evklíðs}

\begin{minted}{cpp}
#include<bits/stdc++.h>
using namespace std;
typedef long long ll;

ll egcd(ll a, ll b, ll& x, ll& y) {
    if (b == 0) { 
        x = 1; y = 0; 
        return a; 
    }
    ll d = egcd(b, a % b, x, y);
    x -= a / b * y; 
    swap(x, y); 
    return d; 
}
\end{minted}

\end{frame}

\begin{frame}
\frametitle{Deiling núna?}

\begin{itemize}

\item<1-> Hvernig deilum við þá? Það er til ein auðveld leið sem notar litlu setningu Fermats. Hún segir að fyrir frumtölu $p$ og $a < p$ þá sé $a^{p - 1} = 1 \ (mod \ p)$. En þá sést að $a^{p - 2}$ sé margföldunarandhverfa. En þetta er yfirleitt ekki leiðin sem notuð er. Við notum reiknirit Evklíðs.

\item<2-> Viljum finna margföldunarandhverfu $a$ þegar við erum að vinna modulo $p$. Þá er stærsti samdeilir $a$ og $p$ jafn einum, svo útvíkkaða reiknirit Evklíðs gefur okkur lausn á $ax + py = 1$. En þessi jafna skoðuð modulo $p$ er bara $ax = 1$, svo $x$ er margföldunarandhverfa $a$ modulo $p$.

\item<3-> Því fáum við mjög stutt og þægilegt forrit til að gera þetta með því að nota \texttt{egcd} fallið okkar.

\end{itemize}

\end{frame}

\begin{frame}[fragile]
\frametitle{Leifamargföldunarandhverfa}

\begin{minted}{cpp}
#include<bits/stdc++.h>
using namespace std;
typedef long long ll;

ll mod_inv(ll a, ll m) {
    ll x, y;
    ll d = egcd(a, m, x, y);
    return (x % m + m) % m; 
}
\end{minted}

\end{frame}

\section[Frumtölur]{Frumtölur}

\begin{frame}
\frametitle{Frumtölutékk}

\begin{itemize}

\item<1-> Hvernig gáum við hvort tala sé frumtala?

\item<2-> Við skulum skoða þrjár leiðir til þess, byrjum á þeirri 'augljósu'.

\item<3-> Við deilum bara upp í töluna með smærri tölum og sjáum hvort við fáum alltaf einhvern afgang.

\item<4-> Það er samt margt sem má gera til að spara. T.d. þarf bara að skoða deila upp í kvaðratrótina af tölunni því ef tala er ekki frumtala er einhver þáttur minna en eða jafn kvaðratrót hennar.

\item<5-> Við þurfum líka (fyrir utan $2$ og $3$) bara að skoða deila á forminu $6k + 1$ og $6k + 5$ því $2$ deilir $6k$, $3$ deilir $6k + 3$ og $2$ deilir $6k + 4$.

\item<6-> Saman gefur þetta okkur reiknirit sem keyrir í $\mathcal{O}(\sqrt{n})$ tíma.

\end{itemize}

\end{frame}

\begin{frame}[fragile]
\frametitle{Leifamargföldunarandhverfa}

\begin{minted}{cpp}
#include<bits/stdc++.h>
using namespace std;
typedef long long ll;

bool isprime(ll n) {
    if(n < 2) return false;
    if(n == 2 || n == 3) return true;
    if(n % 2 == 0 || n % 3 == 0) return false;
    for(ll i = 5; i * i <= n; i += 6) {
        if(n % i == 0) return false;
        if(n % (i + 2) == 0) return false;
    }
    return true;
}
\end{minted}

\end{frame}

\begin{frame}
\frametitle{Hraðar?}

\begin{itemize}

\item<1-> Getum við gert þetta hraðar?

\item<2-> Jáá... samt ekki.

\item<3-> Við getum notað probabilistic prime testing, s.s. fall sem segir að inntakið sé \textit{líklega} frumtala.

\item<4-> Það hljómar kannski grunsamlega en maður getur keyrt þetta forrit nokkra tugi skipta og þá eru líkurnar á því að forritið skili rangri niðurstöðu öll skiptin svo stjarnfræðilega litlar að maður myndi nýta tíma sinn betur við að hafa áhyggjur af tölvubilana af völdum geimgeisla.

\end{itemize}

\end{frame}

\begin{frame}
\frametitle{Miller-Rabin pælingar}

\begin{itemize}

\item<1-> Bendum fyrst á að ef $x^2 = 1 \ (mod \ p)$ þá má þátta þetta sem $(x-1)(x+1) = 0 \ (mod \ p)$ og þar sem $p$ er frumtala þýðir það að $x = \pm 1 \ (mod \ p)$.

\item<2-> Tökum nú eitthvað $p > 2$ og $a < p$. Ritum $p - 1 = 2^sd$ þ.a. $d$ sé odda. Þá með því að taka ferningsrót af jöfnunni $a^{p - 1} = 1 \ (mod \ p)$ (sem er bara litla setning Fermats) þá annað hvort verður hægri hliðin einhvern tímann $-1$ og þá verðum við að stoppa, eða á endanum deilum við út öllum tvistum í veldi $a$. Því er annað hvort $a^d = 1 \ (mod \ p)$ eða $a^{2^r d} = -1 \ (mod \ p)$ fyrir eitthvað $0 \leq r \leq s - 1$.

\item<3-> Því til að afsanna að $n$ sé frumtala reynum við að finna $a < n$ þ.a. $a^d \neq 1 \ (mod \ n)$ og $a^{2^rd} \neq -1 \ (mod \ n)$ fyrir öll $0 \leq r \leq s - 1$.

\end{itemize}

\end{frame}

\begin{frame}
\frametitle{Miller-Rabin pælingar frh.}

\begin{itemize}

\item<1-> Að finna svona $a$ hljómar langsótt, en í ljós kemur að hátt hlutfall talna virkar sem slíkt $a$ ef $n$ er ekki frumtala (að sýna þetta er aðeins of flókið til að fara í það hér).

\item<2-> Því virkar reiknirit Miller-Rabin með því að velja handahófskennt $a$ og sjá hvort það útiloki það að $n$ sé frumtala.

\item<3-> Því er reikniritið þannig að ef það segir $p$ er ekki frumtala, þá er það nauðsynlega satt. En ef reikniritið segir $p$ er frumtala þýðir það 'ég gat ekki útilokað það að $p$ sé frumtala, það gæti samt verið ekki frumtala'.

\item<4-> En ef við prófum mörg $a$ þá eru líkurnar mjög svo okkur í hag. Láta má forritið taka inn breytu \texttt{it} sem segir hversu oft hún á að reyna og er þá hægt að stilla það eftir því hversu paranoid maður er. Þetta keyrir þá í $\mathcal{O}(it\log(n)^3)$ tíma fyrir stórar tölur $n$ (s.s. þegar við erum farin að nota bigInt).

\end{itemize}

\end{frame}

\begin{frame}[fragile]
\frametitle{Miller-Rabin útfærsla}

\begin{small}
\begin{minted}{cpp}
bool isprobablyprime(ll n, int it) {
    if(n % 2 == 0) return n == 2;
    if(n <= 3) return n == 3;
    ll d = n - 1, r = 0;
    while(d % 2 == 0) d >>= 1, r++;
    for(int i = 0; i < it; ++i) {
        ll a = (n - 3) * rand() / RAND_MAX + 2;
        ll x = modpow(a, d, n);
        if(x == 1 || x == n - 1) continue;
        for(ll j = 0; j < r - 1; ++j) {
            x = (x * x % n + n) % n;
            if(x == n - 1) continue;
        }
        return false;
    }
    return true;
}
\end{minted}
\end{small}

\end{frame}

\begin{frame}
\frametitle{Magnafsláttur}

\begin{itemize}

\item<1-> En ef við viljum finna allar frumtölur undir $n$? Getum við fengið einhverskonar magnafslátt þá?

\item<2-> Já! Við notum sáldur Eratosthenes!

\item<3-> Það er mjög einfalt reiknirit. Við byrjum í 2 og krossum út allar sléttar tölur upp að $n$. Svo förum við áfram að næstu tölu sem er ekki búið að krossa út og krossum út öll margfeldi af henni, rinse and repeat. Þegar við erum komin upp að $n$ er búið að krossa út allt nema frumtölurnar.

\item<4-> Svo gerum við þetta aðeins hraðara með því að fara bara upp í rótina af $n$, krossa bara út tölur þar sem frumþátturinn okkar getur verið minnsti frumþátturinn o.s.frv. Þegar við förum að hellast yfir í bigInt verður tímaflækjan á þessu $\mathcal{O}(n\log(n)\log(\log(n)))$.

\end{itemize}

\end{frame}

\begin{frame}[fragile]
\frametitle{Eratosthenes útfærsla}

\begin{small}
\begin{minted}{cpp}
#include<bits/stdc++.h>
using namespace std;
typedef long long ll;

vector<bool> eratosthenes(ll n) {
    vector<bool> res(n + 1, true);
    res[0] = res[1] = false;
    for(ll i = 2; i * i <= n; ++i) {
        if(res[i]) {
            for(ll j = i * i; j <= n; j += i) {
                res[j] = false;
            }
        }
    }
    return res;
}
\end{minted}
\end{small}

\end{frame}

\begin{frame}
\frametitle{Frumþáttun}

\begin{itemize}

\item<1-> Sáldur Eratosthenes er mjög gott til að frumþátta tölur því þá geymir maður bara frumþátt tölunnar í stað bool og er þá tala frumtala ef hún er sjálf geymd í sætinu sínu. Þá má frumþátta með því að deila út tölunni í sæti þess í sáldrinu aftur og aftur þar til maður fær út ás.

\item<2-> En er til einhver hraðari leið til að frumþátta $n$?

\item<3-> Aftur er svarið jááaaa, samt ekki.

\item<4-> Við notum afmælisdagaþversögnina! Ef við höfum $n$ möguleg gildi er viðbúið að við þurfum að taka tilviljanakennt gildi $\mathcal{O}(\sqrt{n})$ sinnum til að fá endurtekningu. Við munum nýta þetta til að finna frumþátt í $n$. En fyrst smá hliðarspor.

\end{itemize}

\end{frame}

\begin{frame}
\frametitle{Floyd rásarleit}

\begin{itemize}

\item<1-> Ef við höfum fall $f$ sem tekur inn heiltölur, hvernig finnum við hvort $f(x + n) = f(x)$ fyrir eitthvað $n$ (fyrir öll $x$ stærri en eitthvað $k$)?

\item<2-> Þetta er oft nytsamlegt að geta gert hratt og til þess notum við rásarleitarreiknirit Floyds, einnig þekkt sem skjaldböku- og hérareikniritið.

\item<3-> Trikkið er að $i$ er margfeldi af rásarlengd $f$ þ.þ.a.a. $f(i) = f(2i)$. Því dugar að skoða þá jöfnu til að finna rásarlengd $f$. Þegar það er búið getum við farið til baka til að hvar rásin byrjar og útfrá því er auðvelt að finna raunverulega rásarlengd.

\end{itemize}

\end{frame}

\begin{frame}[fragile]
\frametitle{Floyd útfærsla}

\begin{tiny}
\begin{minted}{cpp}
#include<bits/stdc++.h>
using namespace std;
typedef pair<int,int> ii;

ii floyd(int (*f)(int), int x0) {
    int t = (*f)(x0), h = (*f)((*f)(x0));
    while(t != h) {
        t = (*f)(t); 
        h = (*f)((*f)(h));
    }
    int mu = 0; t = x0;
    while(t != h) {
        t = (*f)(t);
        h = (*f)(h);
        mu++;
    }
    int lam = 1; h = (*f)(t);
    while(t != h) {
        h = (*f)(h);
        lam++;
    }
    // lengd rásar, upphafspunktur rásarhegðunar
    return ii(lam, mu);
}

\end{minted}
\end{tiny}

\end{frame}

\begin{frame}
\frametitle{Pollard rho þáttun}

\begin{itemize}

\item<1-> Hugmyndin er nú sú að við látum $g(x) = x^2 + 1 \ (mod \ n)$ og búum til runu $x, g(x), g(g(x)), \dots$ þar sem $x$ er valið bara einhvern veginn. Köllum þessar tölur $x_1,x_2,\dots$. Við reiknum svona gildi og tékkum hvort við fáum einhvern tímann $\operatorname{gcd}(x_i - x_j, n) > 1$. Þá er runan farin að endurtaka sig ekki bara modulo $n$ heldur modulo $d$ þar sem $d$ deilir $n$. Þá gefur þessi stærsti samdeilir okkur tölu sem deilir $n$.

\item<2-> Ef $\operatorname{gcd}(x_i - x_j, n) = 1$ fyrir öll gildin sem við skoðum er annað hvort $n$ frumtala eða okkur tókst bara ekki að finna frumþátt í $n$. Því prófum við nokkur upphafsgildi fyrir $x$ yfirleitt og gefumst upp ef ekkert þeirra virkar. Þetta reiknirit er almennt bara notað á bigInt því það borgar sig ekki fyrir en þá, en við sýnum þetta hér fyrir long long bara til að sýna einhverja útfærslu.

\end{itemize}

\end{frame}

\begin{frame}[fragile]
\frametitle{Pollard rho útfærsla}

\begin{minted}{cpp}
ll rho(ll n) {
    vector<ll> seed = {2, 3, 4, 5, 7, 11, 13, 1031};
    for(ll s : seed) {
        ll x = s, y = x, d = 1;
        while(d == 1) {
            x = ((x * x + 1) % n + n) % n;
            y = ((y * y + 1) % n + n) % n;
            y = ((y * y + 1) % n + n) % n;
            d = gcd(abs(x - y), n);
        }
        if(d == n) continue;
        return d;
    }
    return -1;
}
\end{minted}

\end{frame}

\section[Ýmist fleira]{Ýmist fleira}

\begin{frame}
\frametitle{Talnafræðiföll}

\begin{itemize}

\item<1-> Nú með þessum aðferður hér á undan getum við frumþáttað tölu $n = p_1^{e_1}\dots p_r^{e_r}$. En hvað má nýta þetta í?

\item<2-> Þetta má nýta í föll eins og:

\end{itemize}

\only<3-> {\[\textrm{Fjöldi deila } n: d(n) = \prod_{i = 1}^r (e_i + 1)\]}

\only<4-> {\[\textrm{Summa deila } n: \sigma(n) = \prod_{i = 1}^r \frac{p_i^{e_i + 1} - 1}{p_i - 1}\]}

\only<4-> {\[\textrm{Fjöldi talna minni en } n \textrm{ ósamþátta } n: \varphi(n) = n\prod_{i = 1}^r \p{1 - \frac{1}{p_i}}\]}

\end{frame}

\begin{frame}
\frametitle{Kínverska leifasetningin}

\begin{itemize}

\item<1-> Hvað ef við höfum jöfnuhneppi á forminu $a_ix = b_i \ (mod \ m_i)$ þar sem $i = 1,\dots,n$? Hvernig leysum við slíkt?

\item<2-> Í slíkt notum við kínversku leifasetninguna!

\item<3-> Það að reikna lausnina er samt rosalega ljótt og leiðinlegt. Við látum því duga að segja að ef $m_i$-in eru ósamþátta tvö og tvö, þá \textit{er til} ótvírætt ákvörðuð lausn, og það dugar okkur yfirleitt. Mjög sjaldan þarf maður að reikna út raunverulegu lausnina.

\end{itemize}

\end{frame}

\begin{frame}
\frametitle{Fast Fourier Transform}

\begin{itemize}

\item<1-> Þetta, eins og flæðireikniritin förum við ekki í.

\item<2-> Hins vegar, eins og flæðireikniritin, eru mörg þyngri keppnisforritunardæmi sem byggja á þessu.

\item<3-> Því mælum við eindregið með að áhugasamir kynni sér þetta. Þetta er svolítið þungt efni, en þið getið rætt það við okkur ef þið viljið aðstoð við að átta ykkur á því.

\end{itemize}

\end{frame}

\end{document}