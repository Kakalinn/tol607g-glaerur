\title{Lausnir á lokakeppnisdæmum}
\author{Bergur Snorrason, Atli FF}
\date{\today}

\begin{document}

\frame{\titlepage}

\env{frame}
{
	\frametitle{Skálagerð}
	\env{itemize}
	{
		\item<1-> Þér er gefinn hringur með geisla $r$ og þú átt að dreifa fjórum punktum jafnt á hringinn.
		\item<2-> Hver verður fjarlægðin milli aðliggjandi punkta?
	}
}

\env{frame}
{
	\frametitle{Skálagerð}
	\env{center}
	{
		\env{tikzpicture}
		{

			\node at (0.8, 0) {$x$};
			\node at (0, -0.8) {$x$};
			\node at (-0.4, 0) {$2r$};
			\draw[dotted] (1, 1) -- (-1, -1);
			\draw[dotted] (1, -1) -- (-1, -1);
			\draw[dotted] (1, 1) -- (1, -1);
			\draw (0, 0) circle (1.4142135623730951);
			\node[draw, fill, circle, inner sep = 0.5pt] at (0, 0) {};
			\node[draw, fill, circle, inner sep = 0.5pt] at (1, 1) {};
			\node[draw, fill, circle, inner sep = 0.5pt] at (1, -1) {};
			\node[draw, fill, circle, inner sep = 0.5pt] at (-1, -1) {};
			\node[draw, fill, circle, inner sep = 0.5pt] at (-1, 1) {};
		}
	}
	\env{itemize}
	{
		\item<2-> Við vitum að þríhyrningurinn merktur með punktalínum er rétthyrndur (setning Þalesar).
		\item<3-> Svo $2x^2 = 4r^2$ (setning Pýþagorasar).
		\item<4-> Svarið er því $x = r\sqrt{2}$.
	}
}

\env{frame}
{
    \frametitle{DCC líkur}
    \env{itemize}
    {
        \item<1-> Gefna teninga í DCC kerfinu og líkur $p$, hvað þarf að hækka annan teninginn mikið svo hann hafi $p\%$ vinningslíkur?
    }
}

\env{frame}
{
	\frametitle{DCC líkur}
	\env{itemize}
	{
		\item<1-> Líkurnar á að $n$ hliða teningur sigri $m$ hliða tening má hér einfaldlega reikna með tvöfaldri for-lykkju því tölurnar eru svo smáar.
        \item<2-> Ytri lykkjan fer frá $1$ til $n$, hin frá $1$ til $m$ og þegar ytri breytan er stærri (strangt!) hækkum við teljara um $1$. Deilum með $nm$ í lokin og fáum líkurnar. 
        \item<3-> Þá er bara að gera þetta aftur og aftur þar til líkurnar eru stærri en eða jöfn (ekki strangt!) $p\%$, hækka tening um einn í keðjunni í einu.
        \item<4-> Passa nákvæmni, betra jafnvel að nota almenn brot heldur en fleytitölur. Passa að ekki sé hægt að fara uppfyrir $30$ hliðar.
	}
}

\env{frame}
{
	\frametitle{Vatnskubbur}
	\env{itemize}
	{
		\item<1-> Gefin heiltala $n \leq 10^{18}$, eru til heiltölur $a, b > 1$ þannig að $n = ab^2$?
	}
}

\env{frame}
{
	\frametitle{Vatnskubbur}
	\env{itemize}
	{
		\item<1-> Frumþáttum þannig að $n = p_1^{e_1} \dots p_m^{e_m}$.
		\item<2-> Tökum eftir að ef $e_1 = \dots = e_m = 1$ þá er þetta ekki hægt.
		\item<3-> Ef $m = 1$ og $e_1 = 2$ þá er þetta heldur ekki hægt.
		\item<4-> Annars er þetta hægt.
		\item<5-> Þá er til $j$ þannig að $e_j \geq 2$ svo við getum látið $b = p_j$.
		\item<6-> Við þurfum að passa að $n$ er stór, svo við þurfum reiknirit Pollards til að lausnin verði nógu hröð.
        \item<7-> Reiknirit Pollards er of hægt fyrir stóra frumtölu þar að auki, svo byrja þarf á að nota reiknirit Miller-Rabin.
	}
}

\env{frame}
{
    \frametitle{Önnur lausn}
    \env{itemize}
    {
        \item<1-> Einnig má vera aðeins sniðugur og sleppa öllu flottu reikniritunum.
        \item<2-> Það þarf aðeins að fjarlægja þættina úr $n$ sem eru $\leq \sqrt[3]{n}$.
        \item<3-> Eftirlátum restina af þessarri lausn sem æfingu fyrir lesanda.
    }
}

\env{frame}
{
	\frametitle{Bíórugl}
	\env{itemize}
	{
		\item<1-> Það eru $n \leq 10^{18}$ einstaklingar í bíó og þeir sitja allar í sömu röð og fylla akkúrat röðina.
		\item<2-> Í hlé fara allir á klóið og vilja svo sæti sem er í mesta lagi tveimur sætum frá upprunalega sætinu sínu.
		\item<3-> Á hversu marga vegu geta þeir sest aftur?
	}
}

\env{frame}
{
	\frametitle{Bíórugl}
	\env{itemize}
	{
		\item<1-> Við leysum þetta með því að finna rakningarvensl sem lýsa dæminu.
		\item<2-> Með því að skoða hvernig dæmið skiptist í smærri tilfelli (og handreikna grunntilfellin) fæst að
		\[
			c_n = 14c_{n - 1} + 2c_{n - 3} - c_{n - 5},
		\]
		ef $n > 4$ og $c_0 = 1, c_1 = 1, c_2 = 2, c_3 = 6$ og $c_4 = 14$.
		\item<3-> Við getum síðan notað fylkjamargföldun til að reikna $c_n$ í logratíma.
		\item<4-> Ef við viljum ekki reikna grunntilfellin í höndunum getum við notað tæmandi leit til þessa að finna þau.
		\item<5-> Við getum líka fundið stuðlana með Gauss-Jordan eyðingu.
	}
}

\env{frame}
{
	\frametitle{Réttur krappi er rangur}
	\env{itemize}
	{
		\item<1-> Gefnir eru $n \leq 3\, 000$ punktar í plani.
		\item<2-> Hversu margar þrenndir í punkta safninu mynda rétthyrndan þríhyrning?
	}
}

\env{frame}
{
	\frametitle{Réttur krappi er rangur}
	\env{itemize}
	{
		\item<1-> Það er lítið mál að skoða allar þrenndir punkta, en sú lausn er $\mathcal{O}(n^3)$ sem er of hægt.
		\item<2-> Veljum einhver punkt sem vendipunkt og skoðum allar línur sem liggja gegnum vendi punktinn og einhvern annan punkt í safninu.
		\item<3-> Ef tvær línur skerast í réttu horni þá svara þær til þrenndar í punktasafninu sem myndar rétthyrning.
		\item<4-> Við getum fundið, fyrir tiltekna línu, hversu margar línur hún sker undir réttu horni með helmingunarleit (tveimur leitum reyndar)
					eða gagngrindum á borða við leitartré (\texttt{set<...>}) eða hakkatöflu (\texttt{unordered\_map<...>}).
		\item<5-> Endurtökum svo þannig að allir punktar verði vendipunktar og styttum svo út endurtekningar.
		\item<6-> Þessi lausn er $\mathcal{O}(n^2 \log n)$.
	}
}

\env{frame}
{
    \frametitle{Önnur lausn}
    \env{itemize}
    {
        \item<1-> Einnig má nýta sér að þetta séu allt heiltölur. Ef við erum með línu gegnum $(0, 0)$ og tvo punkta verða hnit annars punktsins
            að vera margfeldi af hnitum hins. 
        \item<2-> Veljum þá einn vendipunkt í einu og styttum út stærsta samdeili hnita allra punkta til að fá punktasafn, höldum utan um hvað
            það eru mörg af hverjum punkti því við fáum mögulega endurtekningar.
        \item<3-> Svo fyrir hvern punkt skoðum við bara hvað það eru margir af honum og af honum snúið um $\pi/2$, leggjum það við niðurstöðu.
        \item<4-> Þessi lausn er $\mathcal{O}(n^2 \log(w))$ þar sem $w$ er stærsta leyfilega hnit talnanna, sem gengur einnig.
    }
}

\env{frame}
{
	\frametitle{Leiðinda rigning}
	\env{itemize}
	{
		\item<1-> Finna á leiðina heim fyrir Atla sem bleytir hann sem minnst. Höfum net með $n \leq 5 \cdot 10^4$ hnúta, $m \leq 10^5$ leggi
            og svo $q \leq 5 \cdot 10^4$ fyrirspurnar. Þær biðja annað hvort um að breyta hvort hnútur sé strætóstöð eða að finna hvaða stöð
            er næst gefnum hnút.
	}
}

\env{frame}
{
    \frametitle{Drög að lausn}
    \env{itemize}
    {
        \item<1-> Fjarlægðin er ekki summa vigtanna, heldur bara hæsta vigtin sem kemur fyrir á leiðinni. Við getum því hent öllum leggjum
            sem eru ekki í minnsta spannandi tré netsins. 
        \item<2-> Þá erum við með tré. Ímyndum okkur að við viljum reikna fjarlægð í næstu stöð fyrir alla hnúta í byrjun. Getum gert þetta
            með reikniriti Dijkstra, setjum allar strætóstöðvar sem fjarlægð $0$ í byrjun.
        \item<3-> En við getum ekki uppfært þetta nógu hratt. Hvað ef við viljum skoða fjarlægð í eina tiltekna stöð hratt?
        \item<4-> Setjum upp LCA töflu fyrir tréð! Þá getum við fundið hámarksvigtina milli upphafspunkts og að strætóstöð í logratíma.
        \item<5-> Hvernig má nú sameina þetta tvennt til að fá skikkanlega tímaflkju?
    }
}

\env{frame}
{
    \frametitle{Rótarþáttun}
    \env{itemize}
    {
        \item<1-> Við skiptum fyrirspurnunum í $\sqrt{q}$ fötur. Í byrjuninni á hverri fötu reiknum við allar fjarlægðir í stöðvar sem munu
            ekki breytast í þessarri fötu með því að nota Dijkstra.
        \item<2-> Löbbum svo í gegnum fötuna. Flettum upp gildi í Dijkstra niðurstöðum fyrir hverja fyrirspurn, en höldum einnig utan um
            lista fyrir allar breyttar stöðvar. Við reiknum fjarlægðirnar í þær allar til viðbótar með LCA og tökum besta gildið. Þessi
            listi verður aldrei lengri en $\sqrt{q}$ því hann getur aðeins breyst um eitt stak í hverri fyrirspurn.
        \item<3-> Reiknum Dijkstra $\sqrt{q}$ sinnum, það tekur samtals $\mathcal{O}(\sqrt{q}n\log(n))$ tíma. Reiknum LCA töflu í byrjun
            í $\mathcal{O}(n\log(n))$ tíma. Flettum upp í henni fyrir hvert stak listans og í hverri fyrirspurn, það tekur 
            $\mathcal{O}(q\sqrt{q}\log(n))$. Ef við reiknum upp úr þessu sést að þetta allt saman er undir tímamörkum.
    }
}

\end{document}
