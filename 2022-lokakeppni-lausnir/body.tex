\title{Lausnir á rúmfræðidæmi}
\author{Bergur Snorrason}
\date{\today}

\begin{document}

\frame{\titlepage}

\env{frame}
{
	\frametitle{Skálagerð}
	\env{itemize}
	{
		\item<1-> Þér er gefinn hringur með geisla $r$ og þú átt að dreifa fjórum punktum jafnt á hringinn.
		\item<2-> Hver verður fjarlægðin milli aðliggjandi punkta?
	}
}

\env{frame}
{
	\frametitle{Skálagerð}
	\env{center}
	{
		\env{tikzpicture}
		{

			\node at (0.8, 0) {$x$};
			\node at (0, -0.8) {$x$};
			\node at (-0.4, 0) {$2r$};
			\draw[dotted] (1, 1) -- (-1, -1);
			\draw[dotted] (1, -1) -- (-1, -1);
			\draw[dotted] (1, 1) -- (1, -1);
			\draw (0, 0) circle (1.4142135623730951);
			\node[draw, fill, circle, inner sep = 0.5pt] at (0, 0) {};
			\node[draw, fill, circle, inner sep = 0.5pt] at (1, 1) {};
			\node[draw, fill, circle, inner sep = 0.5pt] at (1, -1) {};
			\node[draw, fill, circle, inner sep = 0.5pt] at (-1, -1) {};
			\node[draw, fill, circle, inner sep = 0.5pt] at (-1, 1) {};
		}
	}
	\env{itemize}
	{
		\item<2-> Við vitum að þríhyrningurinn merktur með punktalínum er rétthyrndur (setning Þalesar).
		\item<3-> Svo $2x^2 = 4r^2$ (setning Pýþagorasar).
		\item<4-> Svarið er því $x = r\sqrt{2}$.
	}
}

\env{frame}
{
	\frametitle{DCC líkur}
	\env{itemize}
	{
		\item<1-> 
	}
}

\env{frame}
{
	\frametitle{Vatnskubbur}
	\env{itemize}
	{
		\item<1-> Gefin heiltala $n \leq 10^{18}$, eru til heiltölur $a, b > 1$ þannig að $n = ab^2$?
	}
}

\env{frame}
{
	\frametitle{Vatnskubbur}
	\env{itemize}
	{
		\item<1-> Frumþáttum þannig að $n = p_1^{e_1} \dots p_m^{e_m}$.
		\item<2-> Tökum eftir að ef $e_1 = \dots = e_m = 1$ þá er þetta ekki hægt.
		\item<3-> Ef $m = 1$ og $e_1 = 2$ þá er þetta heldur ekki hægt.
		\item<4-> Annar er þetta hægt.
		\item<5-> Þá er til $j$ þannig að $e_j \geq 2$ svo við getum látið $b = p_j$.
		\item<6-> Við þurfum að passa að $n$ er stór, svo við þurfum reiknirit Pollards til að lausnin verði nógu hröð.
	}
}

\env{frame}
{
	\frametitle{Bíórugl}
	\env{itemize}
	{
		\item<1-> Það eru $n \leq 10^{18}$ einstaklingar í bíó og þeir sitja allar í sömu röð og fylla akkúrat röðina.
		\item<2-> Í hlé fara allir á klóið og vilja svo sæti sem er í mesta lagi tveimur sætum frá upprunalega sætinu sínu.
		\item<3-> Á hversu marga vegu geta þeir sest aftur?
	}
}

\env{frame}
{
	\frametitle{Bíórugl}
	\env{itemize}
	{
		\item<1-> Við leysum þetta með því að finna rakningarvensl sem lýsa dæminu.
		\item<2-> Með því að skoða hvernig dæmið skiptist í smærri tilfelli (og handreikna grunntilfellin) fæst að
		\[
			c_n = 14c_{n - 1} + 2c_{n - 3} - c_{n - 5},
		\]
		ef $n > 4$ og $c_0 = 1, c_1 = 1, c_2 = 2, c_3 = 6$ og $c_4 = 14$.
		\item<3-> Við getum síðan notað fylkjamargföldun til að reikna $c_n$ í logratíma.
		\item<4-> Ef við viljum ekki reikna grunntilfellin í höndunum getum við notað tæmandi leit til þessa að finna þau.
		\item<5-> Við getum líka fundið stuðlana með Gauss-Jordan eyðingu.
	}
}

\env{frame}
{
	\frametitle{Réttur krappi er rangur}
	\env{itemize}
	{
		\item<1-> Gefnir eru $n \leq 3\, 000$ punktar í plani.
		\item<2-> Hversu margar þrenndir í punkta safninu mynda rétthyrndan þríhyrning?
	}
}

\env{frame}
{
	\frametitle{Réttur krappi er rangur}
	\env{itemize}
	{
		\item<1-> Það er lítið mál að skoða allar þrenndir punkta, en sú lausn er $\mathcal{O}(n^3)$ sem er of hægt.
		\item<2-> 

	}
}

\env{frame}
{
	\frametitle{Leiðinda rigning}
	\env{itemize}
	{
		\item<1-> 
	}
}

\env{frame}
{
}

\end{document}
