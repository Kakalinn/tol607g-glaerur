\title{Inngangur að netafræði}
\author{Bergur Snorrason}
\date{\today}

\begin{document}

\frame{\titlepage}

\env{frame}
{
	\frametitle{Net}
	\env{itemize}
	{
		\item<1-> Tvennd $(V, E)$, þar sem $V$ er endanlegt mengi og $E \subset V \times V$, kallast \emph{net} (e. \emph{graph}).
		\item<2-> Stökin í $V$ köllum við \emph{hnúta} (e. \emph{node}) og stökin í $E$ köllum við \emph{leggi} (e. \emph{edge}).
		\item<3-> Ef venslin $E$ eru samhverf, það er að segja ef 
					\[
						(u, v) \in E \Rightarrow (v, u) \in E,
					\]
					þá segjum við að netið sé \emph{óstefnt} (e. \emph{undirected}).
		\item<4-> Net sem er ekki óstefnt kallast \emph{stefnt} (e. \emph{directed}).
		\item<5-> Við segjum að hnúturinn $v$ sé \emph{nágranni} (e. \emph{neighbour}) hnútsins $u$ ef $(u, v)$ er í $E$.
		\item<6-> Við segjum að hnútar $u$ og $v$ í óstefndu neti séu \emph{nágrannar} ef $(u, v)$ er í $E$.
		\item<7-> Við segjum einnig að það liggi leggur á milli $u$ og $v$.
	}
}

\env{frame}
{
	\env{itemize}
	{
		\item<1-> Takið eftir að netið $G = (V, E)$ hefur $|V|$ fjölda hnúta og $|E|$ fjölda leggja (ögn flóknara ef netið er óstefnt).
		\item<2-> Tímaflækjur reinkirita í netafræði eru því iðulega gefnar sem föll af $|V|$ og $|E|$ (hingað til höfum við aðalega notað $n$).
		\item<3-> Þegar ég forrita netafræði dæmi læt ég yfirleitt $n$ tákna fjölda hnúta og $m$ tákna fjölda leggja.
		\item<4-> Því kemur fyrir að ég lýsa tímaflækjum reikniritana með þessum breytum.
		\item<5-> Ég mun einnig leyfa mér að nota $V$ í stað $|V|$ og $E$ í stað $|E|$, því þetta ætti aldrei valda ruglningi.
		\item<6-> Sem dæmi skrifa ég $\mathcal{O}(E + V)$ í stað $\mathcal{O}(|E| + |V|)$.
	}
}

\env{frame}
{
	\env{itemize}
	{
		\item<1-> Oft hjálpar að skilja tiltekið net með því að teikna það.
		\item<2-> Við byrjum á að teikna punkta fyrir hnútana.
		\item<3-> Ef netið er óstefnt teiknum við svo línu á milli nágranna (svo hver lína svarar til leggs).
		\item<4-> Ef netið er stefnt þá teiknum við ör í stað línu.
		\item<5-> Leggur $(u, v)$ er þá táknaður með ör frá hnút $u$ til hnúts $v$.
	}
}

\env{frame}
{
	\env{center}
	{
		\env{tikzpicture}
		{
			\node[draw, circle, thick] (1) at (0, 0) {$1$};
			\node[draw, circle, thick] (2) at (2, 0) {$2$};
			\node[draw, circle, thick] (3) at (2, 2) {$3$};
			\node[draw, circle, thick] (4) at (4, 0) {$4$};

			\path[draw] (1) -- (2) -- (3);
			\path[draw] (1) edge[in=45, out=135, looseness=10] (1);
			%\path[draw] (2) edge[bend left=30] (6);
			%\path[draw] (2) edge[bend right=30] (6);
			%\path[draw] (7) -- (8);
		}
		\env{itemize}
		{
			\item<1-> Hér má sjá teikningu sem svarar til $V = \{1, 2, 3, 4\}$ og $E = \{(1, 1), (1, 2), (2, 1), (2, 3), (3, 2)\}$.
		}
	}
}

\env{frame}
{
	\env{center}
	{
		\env{tikzpicture}
		{
			\node[draw, circle, thick] (1) at (0, 0) {$1$};
			\node[draw, circle, thick] (2) at (2, 0) {$2$};
			\node[draw, circle, thick] (3) at (2, 2) {$3$};
			\node[draw, circle, thick] (4) at (4, 0) {$4$};

			\path[draw, ->] (1) -- (2);
			\path[draw, ->] (2) -- (1);
			\path[draw, ->] (3) -- (2);
		}
		\env{itemize}
		{
			\item<1-> Hér má sjá teikningu sem svarar til $V = \{1, 2, 3, 4\}$ og $E = \{(1, 2), (2, 1), (3, 2)\}$.
		}
	}
}

\env{frame}
{
	\env{itemize}
	{
		\item<1-> Leggir af gerðinni $(u, u)$ kallast \emph{lykkjur} (e. \emph{loop}) (ástæða nafngiftarinnar sést af fyrri myndinni).
		\item<2-> Net án leggja kallast \emph{einfalt} (e. \emph{simple}).
		\item<3-> Í umfjöllun okkar gerum við ráð fyrir að öll net séu einföld nema annað sé tekið fram.
		\item<4-> Yfirleitt eru net skilgreind á veg sem leyfa fleiri en einn legg milli hnúta.
		\item<5-> Einföld net þurfa þá líka að hafa í mesta lagi einn legg milli hnúta.
	}
}

\env{frame}
{
	\env{itemize}
	{
		\item<1-> Runa hnúta $v_1, v_2, ..., v_n$ kallast \emph{vegur} (e. \emph{path}) ef $(v_j, v_{j + 1}) \in E$, fyrir $j = 1, 2, ..., n - 1$.
		\item<2-> Vegur kallast \emph{rás} (e. \emph{cycle}) ef $v_1 = v_n$.
		\item<3-> Vegur kallast \emph{einfaldur} (e. \emph{simple}) ef engir tveir hnútar í $v_1, v_2, ..., v_n$ eru eins.
		\item<4-> Rás kallast \emph{einföld} (e. \emph{simple}) ef engir tveir hnútar í $v_1, v_2, ..., v_{n - 1}$ eru eins.
		\item<5-> Við segjum að vegurinn $v_1, v_2, ..., v_n$ \emph{liggi á milli hnútanna $v_1$ og $v_n$}.
		\item<6-> Óstefnt net er sagt vera \emph{samanhangandi} (e. \emph{connected}) ef til er vegur milli sérhverja tveggja hnúta.
		\item<7-> Óstefnt net er sagt vera \emph{tré} (e. \emph{tree}) ef það er samanhangandi og inniheldur enga rás.
	}
}
\env{frame}
{
	\frametitle{Framsetning neta í tölvum}
	\env{itemize}
	{
		\item<1-> Það er engin stöðluð leið til að geyma net í minni.
		\item<2-> Yfirleitt er notað eina af þremur gagnagrindum:
		\env{itemize}
		{
			\item<3-> Leggjalista.
			\item<4-> Nágrannafylki.
			\item<5-> Nágrannalista (algengust).
		}
	}
}

\env{frame}
{
	\frametitle{Leggjalisti}
	\env{itemize}
	{
		\item<1-> Látum $G = (V, E)$ tákna netið okkar.
		\item<2-> Þar sem $V$ er endanlegt megum við gera ráð fyrir að $V = \{1, 2, ..., n\}$, þar sem $n$ er fjöldi hnúta í $G$.
		\item<3-> Látum $m$ vera fjölda leggja í $G$.
		\item<4-> Listi af tvenndum sem inniheldur nákvæmlega sömu stök og $E$ kallast \emph{leggjalisti} netsins $G$.
		\item<5-> Við notum leggjalist sjaldan, en það kemur fyrir (til dæmis í reikniriti Kruskals).
		\item<6-> Net í dæmum í keppnisforritun eru þó oftast gefin með leggja lista.
		\item<7-> Í óstefndum netum er hver leggur tvítekinn í $E$ og við leyfum okkur að sleppa annari endurtekningunni í listanum.
	}
}

\env{frame}
{
	\env{center}
	{
		\env{tikzpicture}
		{
			[scale=0.75]
			\node[draw, circle, thick] (1) at (0,0) {1};
			\node[draw, circle, thick] (2) at (2,0) {2};
			\node[draw, circle, thick] (3) at (2,-2) {3};
			\node[draw, circle, thick] (4) at (4,-1) {4};

			\path[draw] (1) -- (2) -- (4) -- (3);
			\path[draw] (2) -- (3);

		}
	}
	\env{align*}
	{
	L = [&\\
			&(1, 2),\\
			&(2, 3),\\
			&(2, 4),\\
			&(3, 4)\\
		]&
	}
}

\env{frame}
{
	\env{itemize}
	{
		\item<1-> Helsti galli leggjalistans er að það tekur $\mathcal{O}($\onslide<2->{$\,m\,$}$)$ tíma að ákvarða hvort hnútar séu nágrannar
					eða finna nágranna tiltekins hnúts.
	}
}

\env{frame}
{
	\frametitle{Nágrannafylki}
	\env{itemize}
	{
		\item<1-> Látum $A$ vera $n \times n$ fylki þannig að $A_{uv} = 1$ ef $(u, v)$ er í $E$, en $A_{uv} = 0$ annars.
		\item<2-> Við köllum $A$ \emph{nágrannafylki} netsins $G$.
		\item<3-> Takið eftir að það tekur $\mathcal{O}($\onslide<4->{$n^2$}$)$ tíma að upphafsstilla $A$.
		\item<5-> Svo þessi aðferð er alferið of hæg ef, til dæmis, $n = 10^5$ (sem er oft raunin).
		\item<6-> Þegar $n$ er nógu lítið eru nágrannafylki nytsamleg því við getum ákvarðað hvort tveir hnútar séu nágrannar í
					$\mathcal{O}($\onslide<7->{$\,1\,$}$)$ tíma.
		\item<8-> Einnig hefur $A^p$ (fylkjamargföldun) hefur einnig áhugaverða talningarfræðilega merkingu sem við skoðum síðar.
	}
}

\env{frame}
{
	\env{center}
	{
		\env{tikzpicture}
		{
			[scale=0.75]
			\node[draw, circle, thick] (1) at (0,0) {1};
			\node[draw, circle, thick] (2) at (2,0) {2};
			\node[draw, circle, thick] (3) at (2,-2) {3};
			\node[draw, circle, thick] (4) at (4,-1) {4};

			\path[draw] (1) -- (2) -- (4) -- (3);
			\path[draw] (2) -- (3);

		}
	}
	\[
		A = \left (
		\env{array}
		{
			{l l l l}
			0 & 1 & 0 & 0\\
			1 & 0 & 1 & 1\\
			0 & 1 & 0 & 1\\
			0 & 1 & 1 & 0
		}
		\right )
	\]
}

\env{frame}
{
	\frametitle{Nágrannalistar}
	\env{itemize}
	{
		\item<1-> Látum nú $V$ tákna lista af $n$ listum.
		\item<2-> Táknum $j$-ta lista $V$ með $V_j$.
		\item<3-> Látum nú $V_u$ innihalda alla nágranna hnútsins $u$ í netinu $G$, án endurtekningar.
		\item<4-> Við köllum $V$ \emph{nágrannalista} (fleirtölu) netsins $G$ og $V_u$ \emph{nágrannalista} (eintölu) hnútsins $u$ í netinu $G$.
		\item<5-> Helsti kostur nágrannalistanna er að við getum skoðað alla nágranna tiltekins hnúts án þess að skoða neitt annað.
		\item<6-> Við getum því ítrað í gegnum alla nágranna allra hnúta í $\mathcal{O}($\onslide<7->{$E + V$}$)$ tíma,
					óháð röð hnútanna.
		\item<8-> Þetta kemur að góðum notum þegar við erum að ferðast í gegnum netið.
	}
}

\env{frame}
{
	\env{center}
	{
		\env{tikzpicture}
		{
			[scale=0.75]
			\node[draw, circle, thick] (1) at (0,0) {1};
			\node[draw, circle, thick] (2) at (2,0) {2};
			\node[draw, circle, thick] (3) at (2,-2) {3};
			\node[draw, circle, thick] (4) at (4,-1) {4};

			\path[draw] (1) -- (2) -- (4) -- (3);
			\path[draw] (2) -- (3);

		}
	}
	\env{align*}
	{
	L = [&\\
			&[2]\\
			&[1, 3, 4]\\
			&[2, 4]\\
			&[2, 3]\\
		]&
	}
}

\env{frame}
{
	\env{itemize}
	{
		\item<1-> Eins og minnst var á áðan eru net oftast gefin með leggjalista, en við vinnum yfirleitt með nágrannalista.
		\item<2-> Til að breyta á milli látum nágrannalistann okkar vera af tagi \texttt{vector<vector<int>\!\!\!\! >}.
		\item<3-> Við upphafsstillum hann með $n$ tómum listum.
		\item<4-> Við lesum svo í gegnum alla leggina og bætum viðeigandi hnútum í tilheyrandi nágrannalista.
		\item<5-> Ef leggur $(u, v)$ er í leggjalista stefnds nets þá bætum við $v$ við $V_u$.
		\item<6-> Ef leggur $(u, v)$ er í leggjalista óstefnds nets þá bætum við $v$ við $V_u$ og $u$ við $V_v$.
	}
}

\env{frame}
{
	\code{code/ll2nl.cpp}
}

\env{frame}
{
	\frametitle{Vegin net}
	\env{itemize}
	{
		\item<1-> Við segjum að þrenndin $(V, E, w)$, þar sem $(V, E)$ er net og $w \colon E \rightarrow \mathbb{R} \cup \{-\infty, \infty\}$, sé
					\emph{vegið net} (e. \emph{weighted graph}).
		\item<2-> Það sem við erum í rauninni að gera er að gefa hverjum legg í netinu vigt.
		\item<3-> Ef við erum, til dæmis, með net sem lýsir gönguleiðum í Mosfellsbæ þá gætu vigtirnar verið lengd hvers leggs gönguleiðanna
					eða tíminn sem það tekur að labba legginn.
		\item<4-> Þegar við teiknum vegin net þá teiknum við netið líkt og áður og bætum vigtunum við hliðina á tilheyrandi leggjum.
	}
}

\env{frame}
{
	\frametitle{Vegin net}
	\env{itemize}
	{
		\item<1-> Þegar við geymum vegin net í einhverri gagnagrind þá bætum við yfirleitt bara vigtinni við.
		\item<2-> Leggjalisti vegins nets er því listi af þrenndum.
		\item<3-> Þrennd $(u, v, w)$ þýðir þá að það liggi leggur frá hnút $u$ til hnúts $v$ með vigt $w$.
		\item<4-> Nágrannafylki vegins nets er gefið með $A_{uv} = \infty$ ef enginn leggur liggur á frá hnút $u$ til hnúts $v$ og
					$A_{uv} = w(e)$ er leggurinn $e$ liggur frá hnút $u$ til hnúts $v$.
		\item<5-> Nágrannalistar vegins nets eru listar af tvenndum, fyrra stak tvenndarinnar er nágranninn og seinna stakið er vigtin á leggnum
					til nágrannans.
		\item<6-> Nágrannalistarnir eru því geymdir með \texttt{vector<vector<pair<int, int>\!\!\!\! >\!\!\!\! >}.
		\item<7-> Við notum \texttt{typedef} til að stytta þetta niður í \texttt{vvii}.
	}
}

\env{frame}
{
	\code{code/ll2nlw.cpp}
}

\env{frame}
{
}

\end{document}
