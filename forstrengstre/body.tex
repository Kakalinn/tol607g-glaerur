\title{Forstrengstré}
\author{Bergur Snorrason}
\date{\today}

\begin{document}

\frame{\titlepage}

\env{frame}
{
	\env{itemize}
	{
		\item<1-> Við segjum að ekki tómt rótartré $T = (V, E)$ ásamt vörpun $\tau \colon E \rightarrow \Sigma$,
					þar sem $\Sigma$ er eitthvað stafróf, sé \emph{forstrengstré} (e. \emph{prefix tree}).
		\item<2-> Forstrengstré eru oft kölluð \emph{Trie}.
		\item<3-> Við segjum að strengur $s$ sé í trénu ef það er til hnútur $v$ í trénu þannig að
						\[
							s = \tau(e_1) \dots \tau(e_k)
						\]
					þar sem $e_1, \dots e_k$ eru leggirnir á veginu frá rótar til $v$, í réttri röð.
		\item<4-> Það er mjög algengt að geyma aukagögn í hnútunum í trénu.
		\item<5-> Skoðum dæmi um forstrengstré sem hefur engin aukagögn í hnútunum.
	}
}

\env{frame}
{
	\env{tikzpicture}
	{
		\node[white] at (9,-5) {.};
		\node[white] at (9,5) {.};
		\node[white] at (-2,-5) {.};
		\node[white] at (-2,5) {.};

		\node[draw, circle, thick, white] (0) at (-1, 0) {};
		\node[draw, circle, thick, inner sep = 1.0pt] (1) at (0, 0) {};
		\node[draw, circle, thick, inner sep = 1.0pt] (2) at (1, 1) {};
		\node[draw, circle, thick, inner sep = 1.0pt] (3) at (1, 0) {};
		\node[draw, circle, thick, inner sep = 1.0pt] (4) at (1, -1) {};
		\node[draw, circle, thick, inner sep = 1.0pt] (5) at (2, 0) {};
		\node[draw, circle, thick, inner sep = 1.0pt] (6) at (3, 0) {};
		\node[draw, circle, thick, inner sep = 1.0pt] (7) at (4, 0) {};
		\node[draw, circle, thick, inner sep = 1.0pt] (8) at (5, 0) {};
		\node[draw, circle, thick, inner sep = 1.0pt] (9) at (6, 0) {};
		\node[draw, circle, thick, inner sep = 1.0pt] (10) at (2, -1) {};
		\node[draw, circle, thick, inner sep = 1.0pt] (11) at (3, -1) {};
		\node[draw, circle, thick, inner sep = 1.0pt] (12) at (4, -1) {};
		\node[draw, circle, thick, inner sep = 1.0pt] (13) at (2, 1) {};
		\node[draw, circle, thick, inner sep = 1.0pt] (14) at (3, 1) {};
		\node[draw, circle, thick, inner sep = 1.0pt] (15) at (4, 1) {};
		\node[draw, circle, thick, inner sep = 1.0pt] (16) at (5, 1) {};
		\node[draw, circle, thick, inner sep = 1.0pt] (17) at (6, 1) {};
		\node[draw, circle, thick, inner sep = 1.0pt] (18) at (5, -1) {};
		\node[draw, circle, thick, inner sep = 1.0pt] (19) at (6, -1) {};
		\node[draw, circle, thick, inner sep = 1.0pt] (20) at (7, -1) {};

		\path[draw, thick] (0) -- (1);
		\path[draw, thick] (1) -- (2);
		\path[draw, thick] (1) -- (3);
		\path[draw, thick] (3) -- (5);
		\path[draw, thick] (5) -- (6);
		\path[draw, thick] (6) -- (7);
		\path[draw, thick] (7) -- (8);
		\path[draw, thick] (8) -- (9);
		\path[draw, thick] (1) -- (4);
		\path[draw, thick] (4) -- (10);
		\path[draw, thick] (10) -- (11);
		\path[draw, thick] (11) -- (12);
		\path[draw, thick] (2) -- (13);
		\path[draw, thick] (13) -- (14);
		\path[draw, thick] (14) -- (15);
		\path[draw, thick] (15) -- (16);
		\path[draw, thick] (16) -- (17);
		\path[draw, thick] (7) -- (18);
		\path[draw, thick] (18) -- (19);
		\path[draw, thick] (19) -- (20);

		\node at (0.5, 0.2) {b};
		\node at (1.5, 0.2) {e};
		\node at (2.5, 0.2) {r};
		\node at (3.5, 0.2) {g};
		\node at (4.5, 0.2) {u};
		\node at (5.5, 0.2) {r};
		\node at (0.6, -0.4) {n};
		\node at (1.5, -0.8) {a};
		\node at (2.5, -0.8) {l};
		\node at (3.5, -0.8) {a};
		\node at (0.5, 0.8) {s};
		\node at (1.5, 1.2) {a};
		\node at (2.5, 1.2) {n};
		\node at (3.5, 1.2) {d};
		\node at (4.5, 1.2) {r};
		\node at (5.5, 1.2) {a};
		\node at (4.7, -0.4) {þ};
		\node at (5.5, -0.8) {ó};
		\node at (6.5, -0.8) {r};
		%\only<all:2-3> { \node at (6,3) {$v$}; }
		%\only<all:3> { \node at (5,1) {$x$}; }
	}
}

\env{frame}
{
}

\end{document}
