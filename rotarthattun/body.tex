\title{Rótarþáttun}
\author{Bergur Snorrason}
\date{\today}

\begin{document}

\frame{\titlepage}

\env{frame}
{
	\frametitle{Dæmi}
	\env{itemize}
	{
		\item<1-> Gefinn er listi með $n$ tölum.
		\item<2-> Næst koma $q$ fyrirspurnir, þar sem hver er af einni af tveimur gerðum:
		\env{itemize}
		{
			\item<3-> Bættu $k$ við $i$-tu töluna.
			\item<4-> Reiknaðu summu allra talna á bilinu $[i, j]$.
		}
	}
}

\env{frame}
{
	\frametitle{Almenn $k$-þáttun}
	\env{itemize}
	{
		\item<1-> Hvað ef við skiptum fylkinu upp í $k$ (næstum) jafnstór hólf.
		\item<2-> Við getum þá haldið utan um, og uppfært, summu hvers hólfs auðveldlega.
		\item<3-> Til að finna summu á einhverju bili í fylkinu nægir að reikna summu hólfana á milli
			endapunktana og leggja svo afganginn við (afgangurinn er í mesta lagi lengd tveggja hólfa).
		\item<4-> Tökum eftirfarandi sýnidæmi sem skipt hefur verið í þrjú hólf,
			\[
				p = [0\ 1\ 4\ |\ 3\ 4\ 5\ |\ 0\ 1\ 8\ 9].
			\]
		\item<5-> Köllum fylkið sem geymir summu hvers hólfs \texttt{s}, sem verður þá
			\[
				s = [5\ 12\ 18].
			\]
	}
}

\env{frame}
{
	\env{itemize}
	{
		\item<1-> Ef við viljum uppfæra, til dæmis bæta $5$ við stak $2$, þá þurfum við að sjálfsögðu að uppfæra
			\texttt{p}, en það þarf líka breyta \texttt{s}.
		\item<2-> Til að breyta \texttt{p} gerum við einfaldlega \texttt{p[2] += 5}.
		\item<3-> Til að uppfæra $s$ þurfum við að finna hólfið sem stak \texttt{2} tilheyrir. Þar sem það er í hólfi \texttt{0}
			notum við \texttt{s[0] += 5}.
		\item<4-> Svona líta svo fylkin út, fyrir og eftir uppfærslu.
			\[
			\begin {array}{c c}
			\text{Fyrir breytingu} & \text{Eftir breytingu}\\
				p = [0\ 1\ 4\ |\ 3\ 4\ 5\ |\ 0\ 1\ 8\ 9] & p = [0\ 1\ 9\ |\ 3\ 4\ 5\ |\ 0\ 1\ 8\ 9]\\
				s = [5\ 12\ 18] & s = [10\ 12\ 18]
			\end {array}.
			\]
	}
}

\env{frame}
{
	\env{itemize}
	{
		\item<1-> Ég fór mjög losaralega í hvernig ætti að framkvæma seinni aðgerðina.
		\item<2-> Skoðum, sem dæmi, hverju eigi að skila fyrir \texttt{2 1 8}.
		\item<3-> Það er aðeins eitt hólf á milli staks \texttt{1} og staks \texttt{8}, hólf \texttt{1}.
		\item<4-> ,,Afgangurinn'', eins og ég kallaði hann áðan, eru þau stök sem ekki eru í hólfi \texttt{1}
			en eru þó á bilinu frá \texttt{1} til \texttt{8}.
		\item<5-> Þetta eru stök \texttt{1}, \texttt{2}, \texttt{6}, \texttt{7} og \texttt{8} (samtals summan er þá $31$).
		\item<6-> Við erum því að leggja saman rauðu stökin á myndinni fyrir neðan,
			\[
			\begin {array}{c}
				p = [0\ {\color{red} 1 \ 9\ }|\ {\color{blue}3\ 4\ 5\ } |\ {\color{red} 0\ 1\ 8\ } 9]\\ 
				s = [10\ \alert{12}\ 18]
		\end {array}.
			\]
	}
}

\env{frame}
{
	\env{itemize}
	{
		\item<1-> En er þetta hraðar en frumstæða aðferðin sem við skoðuðum í upphafi?
		\item<2-> Það fer að sjálfsögðu allt eftir því hversu stór hólf við veljum.
		\item<3-> Ef fylkinu er skipt upp í $n$ hólf er nokkuð ljóst að þessi aðferð er jafngild frumstæðu aðferðinni.
		\item<4-> Ef fylkinu er skipt upp í $1$ hólf gildir það sama.
		\item<5-> Munið að við létum $k$ tákna fjölda hólfa.
		\item<6-> Fyrri aðgerðin er ennþá $\mathcal{O}($\onslide<7->{$\,1\,$}$)$,
			en seinni aðgerðin verður $\mathcal{O}($\onslide<8->{$n/k + k$}$)$,
			svo tímaflækjan er $\mathcal{O}($\onslide<9->{$qn/k + qk$}$)$.
	}
}

\env{frame}
{
	\frametitle{Skynsamlegt val á $k$}
	\env{itemize}
	{
		\item Þar sem að fyrri aðgerðin er ekki háð skiptingunni þá nægir að lágmarka $\frac{n}{k} + k$.
		\pause\item Látum $f(k) = \frac{n}{k} + k$.
		\pause\item Við höfum $f'(k) = -\frac{n}{k^2} + 1$.
		\pause\item Útgildispunktar fást í
			\[
				\begin{array}{l l}
				\pause& f'(k) = 0\\
				\pause\Rightarrow & 1 - \frac{n}{k^2} = 0\\
				\pause\Rightarrow & 1 = \frac{n}{k^2}\\
				\pause\Rightarrow & k^2 = n\\
				\pause\Rightarrow & k = \sqrt{n}.
			\end{array}
			\]
		\pause\item Nú þarf bara að ganga úr skugga um að þessi skipting sé betri en línuleg.
}
}

\env{frame}
{
	\env{itemize}
	{
		\item<1-> Ef við veljum $k = \sqrt{n}$ þá er tímaflækja seinni aðgerðarinnar
			\[
				\mathcal{O} \left (\frac{n}{\sqrt{n}} + \sqrt{n}\right ) = \mathcal{O} (\sqrt{n} + \sqrt{n}) = \mathcal{O} (\sqrt{n}).
			\]
		\item<2-> Því er tímaflækjan á lausninni $\mathcal{O}($\onslide<3->{$q\sqrt{n}$}$)$.
		\item<4-> Svo þessi aðferð er betri en sú frumstæða, ef við skiptum í $\sqrt{n}$ hólf.
		\item<5-> Við köllum það \emph{rótarþáttun} (e. \emph{squareroot decomposition}) þegar við skiptum upp í $\sqrt{n}$ hólf.
	}
}

\env{frame}
{
	\selectcode{code/sq.c}{6}{23}
}

\env{frame}
{
	\frametitle{Lygn uppfærlsa}
	\env{itemize}
	{
		\item<1-> Við getum einnig framkvæmt lygnar uppfærlsur þegar við notum rótarþáttun (líkt og með biltré).
		\item<2-> Við uppfærum þá beint þau gildi sem eru í sömu hólfum og endarpunktar bilsins sem við uppfærum yfir.
		\item<3-> Við framkvæmum svo lygna uppfærslu á þau hólf sem liggja þar á milli.
	}
}

\env{frame}
{
	\selectcode{code/sq-le.c}{6}{35}
}

\env{frame}
{
}

\end{document}
