\documentclass{beamer}
\usefonttheme[onlymath]{serif}
\usepackage[T1]{fontenc}
\usepackage[utf8]{inputenc}
\usepackage[english, icelandic]{babel}
\usepackage{amsmath}
\usepackage{amssymb}
\usepackage{amsthm}
\usepackage{gensymb}
\usepackage{parskip}
\usepackage{mathtools}
\usepackage{listings}
\usepackage{xfrac}
\usepackage{graphicx}
\usepackage{xcolor}
\usepackage{tikz}
\usetikzlibrary{calc}
\usepackage{multicol}

\DeclareMathOperator{\lcm}{lcm}
\DeclareMathOperator{\diam}{diam}
\DeclareMathOperator{\dist}{dist}
\DeclareMathOperator{\ord}{ord}
\DeclareMathOperator{\Aut}{Aut}
\DeclareMathOperator{\Inn}{Inn}
\DeclareMathOperator{\Ker}{Ker}
\DeclareMathOperator{\trace}{trace}
\DeclareMathOperator{\fix}{fix}
\DeclareMathOperator{\Log}{Log}
\renewcommand\O{\mathcal{O}}
\newcommand\floor[1]{\left\lfloor#1\right\rfloor}
\newcommand\ceil[1]{\left\lceil#1\right\rceil}
\newcommand\abs[1]{\left|#1\right|}
\newcommand\p[1]{\left(#1\right)}
\newcommand\sqp[1]{\left[#1\right]}
\newcommand\cp[1]{\left\{#1\right\}}
\newcommand\norm[1]{\left\lVert#1\right\rVert}
\renewcommand\qedsymbol{$\blacksquare$}
\renewcommand\Im{\operatorname{Im}}
\renewcommand\Re{\operatorname{Re}}
\usepackage{color}

\definecolor{mygray}{rgb}{0.4,0.4,0.4}
\definecolor{mygreen}{rgb}{0, 0, 1}
\definecolor{myorange}{rgb}{1.0,0.4,0}

\lstset{
	commentstyle=\color{mygray},
	numbersep=5pt,
	numberstyle=\tiny\color{mygray},
	keywordstyle=\color{mygreen},
	showspaces=false,
	showstringspaces=false,
	stringstyle=\color{myorange},
	tabsize=4
}
\lstset{literate=
{æ}{{\ae}}1
{í}{{\'{i}}}1
{ó}{{\'{o}}}1
{á}{{\'{a}}}1
{é}{{\'{e}}}1
{ú}{{\'{u}}}1
{ý}{{\'{y}}}1
{ð}{{\dh}}1
{þ}{{\th}}1
{ö}{{\"o}}1
{Á}{{\'{A}}}1
{Í}{{\'{I}}}1
{Ó}{{\'{O}}}1
{Ú}{{\'{U}}}1
{Æ}{{\AE}}1
{Ö}{{\"O}}1
{Ø}{{\O}}1
{Þ}{{\TH}}1
}

\usetheme{Madrid}

\title{Strengir et cetera}
\author{Bergur Snorrason}
\date{\today}

\AtBeginSubsection[] {
	\begin{frame}
		\frametitle{Efnisyfirlit}
		\tableofcontents[currentsubsection]
	\end{frame}
}

\begin{document}

\frame{\titlepage}

\begin{frame}
	\frametitle{Google Code Jam}
	\begin{itemize}
		\item<1-> Aðfaranótt 6. apríl hefst fyrsta umferð í Google Code Jam.
		\item<2-> Hún stendur yfir í 24 tíma.
		\item<3-> Ég mæli með.
	\end{itemize}
\end{frame}

\section[Strengir]{Strengir}
\subsection{Ad hoc}

\begin{frame}
	\frametitle{Ad hoc strengjadæmi}
	\begin{itemize}
		\item<1-> Í keppnum koma stundum létt strengjadæmi. 
		\item<2-> Til dæmis gæti þurft að
			\begin{itemize}
				\item<3-> "Dulkóðun".
				\item<4-> Finna tíðna (stafa eða orða).
				\item<5-> Lesa inn leiðinlegt inntak.
				\item<6-> Skila leiðinlegu úttaki.
			\end{itemize}
		\item<7-> Við höfum nú þegar látið ykkur gera flest þetta áður.
	\end{itemize}
\end{frame}

\begin{frame}
	\frametitle{Brögð}
	\begin{itemize}
		\item<1-> Ef ég gef ykkur streng og bið ykkur um að finna lengsta hlutbil í strengum sem hefur alla stafi eins,
			hvernig mynduð þið leysa það?
		\item<2-> Hvað ef strengurinn er skrifaður á hring, þ.e.a.s. hlutbil getur farið út fyrir end strengsins og heldur
			þá áfram í byrjun hans (,,urBe'' er þá hlutbil í ,,Bergur'')?
		\item<3-> Sígild leið til að leysa þessa gerð dæmi er að skeyta strengum við sjálfan sig.
		\item<4-> Svo, t.d., ,,Bergur'' verður ,,BergurBergur'' og ,,urBe'' er bersýnilega hlutbil í seinni strengnum.
		\item<5-> Þessi aðferð virkar einnig fyrir aðra hluti en strengjaleit.
	\end{itemize}
\end{frame}

\subsection{Strengjaleit}

\begin{frame}
	\frametitle{Strengjaleit}
	\begin{itemize}
		\item<1-> Gefum okkur langan streng $s$ og styttri streng $p$.
		\item<2-> Hvernig getum við fundið alla hlutstrengi $s$ sem eru jafnir $p$.
		\item<3-> Fyrsta sem manni dettur í hug er að bera $p$ saman við alla hlutstrengi
			$s$ af sömu lengd og $p$.
	\end{itemize}
\end{frame}

\begin{frame}[fragile]
	\frametitle{Strengjaleit}
	\tiny
	\begin{lstlisting}[language=C]
void naive(char* s, int n, char* p, int m)
{
	int i, j;
	for (i = 0; i < n - m + 1; i++)
	{
		for (j = 0; j < m; j++)
		{
			if (s[i + j] != p[j])
			{
				break;
			}
		}
		if (j >= m)
		{
			printf("%d ", i - j);
		}
	}
	printf("\n");
}
	\end{lstlisting}
\end{frame}

\begin{frame}
	\frametitle{Strengjaleit}
	\begin{itemize}
		\item<1-> Gerum ráð fyrir að $s$ sé af lengd $n$ og $p$ sé af lengd $m$.
		\item<2-> Þá hefur $s$ $n - m + 1$ hlutstrengi af lengd $m$.
		\item<3-> Strengja samanburðurinn tekur línulegan tíma.
		\item<4-> Svo tímaflækja leitarinnar er $\mathcal{O}(nm - m^2)$.
		\item<5-> Ef $m = \dfrac{n}{2}$ þá er $nm - m^2 = \dfrac{n^2}{2} - \dfrac{n^2}{4} = \dfrac{n^2}{4}$
			svo $\mathcal{O}(nm - m^2) = \mathcal{O}(n^2)$.
		\item<6-> Dæmi um leiðinlega strengi væri $s = "aaaaaaaaaaaaaaaa"$ og $p = "aaaaaaab"$.
	\end{itemize}
\end{frame}

\begin{frame}
	\frametitle{Strengjaleit}
	\begin{itemize}
		\item<1-> Þó þessi aðferð sé ekki góð þá er hún stundum nógu góð.
		\item<2-> Í þeim tilfellum er samt ekki mælt með því að útfæra hana.
		\item<3-> Það er ekki mælt með því því hún er útfærð í mörgum málum, t.d.:
			\begin{itemize}
				\item<4-> Í \text{string.h} í \texttt{C} er \text{strstr}.
				\item<5-> Í \text{string} í \texttt{C++} er \text{find}.
				\item<6-> Í \text{String} í \texttt{Java} er \text{indexOf}.
			\end{itemize}
		\item<7-> Munið bara að ef $n > 10\ 000$ er þetta yfirleitt of hægt.
	\end{itemize}
\end{frame}

\subsection{Knuth-Morris-Pratt-Matiyasevich}

\begin{frame}
	\frametitle{\texttt{KMP}}
	\begin{itemize}
		\item<1-> Er einhver leið til að bæta strengjaleitina úr fyrri glærum?
		\item<2-> Skoðum betur sértilfellið $p = "aaaabbbb"$.
		\item<3-> Ef strengja samanburðurinn misheppnast í $p[3]$ þá myndi einfalda strengjaleitin
			okkar hliðra $p$ um einn og reyna aftur.
		\item<4-> En við vitum að $(i + 3)$-ji stafur $s$ er ekki $"a"$ (því strengja samanburðurinn
			misheppnaðist þar) svo við getum í raun hliðrað $p$ um $3$.
		\item<5-> Reiknirit Knuth-Morris-Pratt (\texttt{KMP}) notar sér þessa hugmynd til að framkvæma strengjaleit
			í línulegum tíma.
		\item<6-> Í \texttt{KMP} er forreiknað (í línulegum tíma) hversu mikið maður getur hliðrað 
			þegar samanburðurinn misheppnast.
	\end{itemize}
\end{frame}

\begin{frame}[fragile]
	\frametitle{\texttt{KMP} forreikningur}
	\tiny
	\begin{lstlisting}[language=C]
void _kmp(char* p, int* b, int m)
{
	int i = 0, j = -1; b[0] = -1;
	while (i < m)
	{
		while (j >= 0 && p[i] != p[j])
		{
			j = b[j];
		}
		i++; j++; b[i] = j;
	}
}
	\end{lstlisting}
\end{frame}

\begin{frame}
	\frametitle{\texttt{KMP}}
	\begin{itemize}
		\item<1-> Svo þurfum við einfaldlega að labba í gegnum $s$ og hliðra þegar á við.
	\end{itemize}
\end{frame}

\begin{frame}[fragile]
	\frametitle{\texttt{KMP} leit}
	\tiny
	\begin{lstlisting}[language=C]
void kmp(char* s, int n, char* p, int m, int* b)
{
	int i = 0, j = 0;
	while (i < n)
	{
		while (j >= 0 && s[i] != p[j])
		{
			j = b[j];
		}
		i++; j++;
		if (j == m)
		{
			printf("%d\n", i - j); j = b[j];
		}
	}
}
	\end{lstlisting}
\end{frame}

\begin{frame}[fragile]
	\frametitle{\texttt{KMP} sýniforrit}
	\tiny
	\begin{lstlisting}[language=C]
#include <stdio.h>
int get_string(char* b, char t)
{
	int i = 0, c = getchar();
	while (c != t) { b[i++] = c; c = getchar(); }
	b[i] = '\0';
	return i;
}
void _kmp(char* p, int* b, int m)
{
	...
}
void kmp(char* s, int n, char* p, int m, int* b)
{
	...
}
int main()
{
	char s[1000001], p[1001];
	int i;
	printf("Langi strengur: "); fflush(stdout);
	int n = get_string(s, 10);
	printf("Stutti strengur: "); fflush(stdout);
	int m = get_string(p, 10);
	int b[m];
	_kmp(p, b, m);
	for (i = 0; i < m; i++) printf("%4d ", b[i]); printf("\n");
	for (i = 0; i < m; i++) printf("   %c ", p[i]); printf("\n");
	kmp(s, n, p, m, b);
}
	\end{lstlisting}
\end{frame}

\begin{frame}
	\frametitle{Aho-Corasick}
	\begin{itemize}
		\item<1-> Til er önnur aðferð, svipuð og \texttt{KMP}, sem finnur staðsetningar margra orða í einu í streng.
		\item<2-> Hún er kennd við Aho og Corasick.
		\item<3-> Ég fer ekki í hana hér en hún byggir á því að gera stöðuvél.
		\item<4-> \texttt{KMP} er í raun sértilfelli af Aho-Corasick sem vill svo heppilega til að það sé þægilegt
			að útfæra.
	\end{itemize}
\end{frame}

\section[Et cetera]{Et cetera}
\subsection{Hlaupabil}

\begin{frame}
	\frametitle{Hlaupabil}
	\begin{itemize}
			\item<1-> Aðferð hlaupabila (e. sliding window) er stundum hægt að nota til að taka
				dæmi sem hafa augljósa $\mathcal{O}(n^2)$ og gera þau $\mathcal{O}(n)$ eða $\mathcal{O}(n\log n)$.
	\end{itemize}
\end{frame}

\begin{frame}
	\frametitle{Hlaupabil - Dæmi}
	\begin{itemize}
			\item<1-> Skoðum dæmi:
			\item<2-> Gefið $n$, $k$ og svo $n$ tölur $a_i$, þ.a. $a_i \in \{0, 1\}$ finndu
				lengd lengsta bils í $(a_n)_n$ sem inniheldur bara $1$ ef þú mátt breyta allt að $k$ tölum.
			\item<3-> Sjáum strax að maður vill alltaf breyta $0$ í $1$ og aldrei öfugt.
			\item<4-> Sjáum því að við erum að leita að lengsta bili í $(a_n)_n$ sem hefur mesta
				$k$ stök jöfn $0$.
			\item<5-> Gefum okkur nú hlaupabil. Það byrjar tómt.
			\item<6-> Við löbbum svo í gegnum $(a_n)_n$ og lengjum bilið að aftan. Ef það eru einhvern tímann
				fleiri en $k$ stök í bilinu sem eru $0$ þá minnkum við bilið að aftan þar til svo er ekki lengur.
		\end{itemize}
\end{frame}

\begin{frame}[fragile]
	\frametitle{Hlaupabil - Dæmi}
\begin{verbatim}
        k = 2
        l = 1
        [0 1 1 0 1 0 0 0 1 1 1 1 0 0 1 1]
        | |
\end{verbatim}
\end{frame}
\addtocounter{framenumber}{-1}

\begin{frame}[fragile]
	\frametitle{Hlaupabil - Dæmi}
\begin{verbatim}
        k = 2
        l = 2
        [0 1 1 0 1 0 0 0 1 1 1 1 0 0 1 1]
        |   |
\end{verbatim}
\end{frame}
\addtocounter{framenumber}{-1}

\begin{frame}[fragile]
	\frametitle{Hlaupabil - Dæmi}
\begin{verbatim}
        k = 2
        l = 3
        [0 1 1 0 1 0 0 0 1 1 1 1 0 0 1 1]
        |     |
\end{verbatim}
\end{frame}
\addtocounter{framenumber}{-1}

\begin{frame}[fragile]
	\frametitle{Hlaupabil - Dæmi}
\begin{verbatim}
        k = 2
        l = 4
        [0 1 1 0 1 0 0 0 1 1 1 1 0 0 1 1]
        |       |
\end{verbatim}
\end{frame}
\addtocounter{framenumber}{-1}

\begin{frame}[fragile]
	\frametitle{Hlaupabil - Dæmi}
\begin{verbatim}
        k = 2
        l = 5
        [0 1 1 0 1 0 0 0 1 1 1 1 0 0 1 1]
        |         |
\end{verbatim}
\end{frame}
\addtocounter{framenumber}{-1}

\begin{frame}[fragile]
	\frametitle{Hlaupabil - Dæmi}
\begin{verbatim}
        k = 2
        l = 5
        [0 1 1 0 1 0 0 0 1 1 1 1 0 0 1 1]
          |         |
\end{verbatim}
\end{frame}
\addtocounter{framenumber}{-1}

\begin{frame}[fragile]
	\frametitle{Hlaupabil - Dæmi}
\begin{verbatim}
        k = 2
        l = 2
        [0 1 1 0 1 0 0 0 1 1 1 1 0 0 1 1]
                |     |
\end{verbatim}
\end{frame}
\addtocounter{framenumber}{-1}

\begin{frame}[fragile]
	\frametitle{Hlaupabil - Dæmi}
\begin{verbatim}
        k = 2
        l = 2
        [0 1 1 0 1 0 0 0 1 1 1 1 0 0 1 1]
                    |   |
\end{verbatim}
\end{frame}
\addtocounter{framenumber}{-1}

\begin{frame}[fragile]
	\frametitle{Hlaupabil - Dæmi}
\begin{verbatim}
        k = 2
        l = 3
        [0 1 1 0 1 0 0 0 1 1 1 1 0 0 1 1]
                    |     |
\end{verbatim}
\end{frame}
\addtocounter{framenumber}{-1}

\begin{frame}[fragile]
	\frametitle{Hlaupabil - Dæmi}
\begin{verbatim}
        k = 2
        l = 4
        [0 1 1 0 1 0 0 0 1 1 1 1 0 0 1 1]
                    |       |
\end{verbatim}
\end{frame}
\addtocounter{framenumber}{-1}

\begin{frame}[fragile]
	\frametitle{Hlaupabil - Dæmi}
\begin{verbatim}
        k = 2
        l = 5
        [0 1 1 0 1 0 0 0 1 1 1 1 0 0 1 1]
                    |         |
\end{verbatim}
\end{frame}
\addtocounter{framenumber}{-1}

\begin{frame}[fragile]
	\frametitle{Hlaupabil - Dæmi}
\begin{verbatim}
        k = 2
        l = 6
        [0 1 1 0 1 0 0 0 1 1 1 1 0 0 1 1]
                    |           |
\end{verbatim}
\end{frame}
\addtocounter{framenumber}{-1}

\begin{frame}[fragile]
	\frametitle{Hlaupabil - Dæmi}
\begin{verbatim}
        k = 2
        l = 6
        [0 1 1 0 1 0 0 0 1 1 1 1 0 0 1 1]
                      |           |
\end{verbatim}
\end{frame}
\addtocounter{framenumber}{-1}

\begin{frame}[fragile]
	\frametitle{Hlaupabil - Dæmi}
\begin{verbatim}
        k = 2
        l = 6
        [0 1 1 0 1 0 0 0 1 1 1 1 0 0 1 1]
                        |           |
\end{verbatim}
\end{frame}
\addtocounter{framenumber}{-1}

\begin{frame}[fragile]
	\frametitle{Hlaupabil - Dæmi}
\begin{verbatim}
        k = 2
        l = 7
        [0 1 1 0 1 0 0 0 1 1 1 1 0 0 1 1]
                        |             |
\end{verbatim}
\end{frame}
\addtocounter{framenumber}{-1}

\begin{frame}[fragile]
	\frametitle{Hlaupabil - Dæmi}
\begin{verbatim}
        k = 2
        l = 8
        [0 1 1 0 1 0 0 0 1 1 1 1 0 0 1 1]
                        |               |
\end{verbatim}
\end{frame}

\begin{frame}[fragile]
	\frametitle{Hlaupabil - Útfærsla á dæmi}
	\tiny
	\begin{lstlisting}[language=C]
#include <stdio.h>

int main()
{
	int n, k, i;
	scanf("%d %d", &n, &k);
	int a[n];
	for (i = 0; i < n; i++) scanf("%d", &(a[i]));

	int b = 0, z = 0, mx = 0;
	for (i = 0; i < n; i++)
	{
		if (a[i] == 0) z++;
		while (z > k)
		{
			if (a[b] == 0) z--;
			b++;
		}

		if (i - b + 1 > mx) mx = i - b + 1;
	}
	printf("%d\n", mx);
}
	\end{lstlisting}
\end{frame}

\begin{frame}
	\frametitle{Hlaupabil - Annað dæmi}
	\begin{itemize}
			\item<1-> Þetta dæmi er nú kannski í auðveldari kantinum.
			\item<2-> Skoðum annað dæmi:
			\item<3-> Byjrum á nokkrum undirstöðu atriðum.
			\item<4-> Tvö bil kallast \emph{næstum sundurlæg} ef sniðmengi þeirra er tómt eða bara einn punktur.
			\item<5-> Sammengi bila má skrifa sem sammengi næstu sundurlægra bila.
			\item<6-> \emph{Lengd bilsins} $[a, b]$ er $b - a$.
			\item<7-> Til að finna \emph{lengd sammengis bila} skrifum við sammengið sem sammengi næstum sundurlægra bila
				og tökum summu lengda þeirra.
			\item<8-> Til dæmis eru bilin $[1, 2]$ og $[2, 3]$ næstum sundurlæg (en þó ekki sundurlæg) en 
				$[1, 3]$ og $[2, 4]$ eru það ekki. Nú $[1, 3] \cup [2, 4] = [1, 4]$ svo lengd 
				$[1, 3] \cup [2, 4]$ er $3$.
		\end{itemize}
\end{frame}

\begin{frame}
	\frametitle{Hlaupabil - Annað dæmi}
	\begin{itemize}
			\item<1-> Gefið $n$ bil hver er lengd sammengis þeirra.
			\item<2-> Stillum $n$ þannig að $\mathcal{O}(n^2)$ sé of hægt.
			\item<3-> Einhverjar hugmyndir?
			\item<4-> Þetta er hægt að leysa á fleiri en eina vegu, en hér er mín lausn:
			\item<5-> Geymum í lista tvenndir þar sem fyrra stakið er endapunktur einhversbils
				og seinna stakið segir hvaða bili punkturinn tilleyrir.
			\item<6-> Röðum þessum punktum svo í vaxandi röð.
			\item<7-> Hvað gerum við næst?
		\end{itemize}
\end{frame}

\begin{frame}
	\frametitle{Hlaupabil - Annað dæmi}
	\begin{itemize}
			\item<1-> Við löbbum í gegnum þennan raðaða lista og höldum utan um hlaupabil þannig að
				við bætum við bili í hlaupabilið þegar við rekumst á vinstri endapunkt þess og fjarlægjum það 
				þegar við rekumst á hægri endapunkt þess. 
			\item<2-> Við skoðum svo sérstaklega tilfellin þegar við erum ekki með nein bil í hlaupabilinu okkur.
			\item<3-> Sammengi þeirra bila sem við höfum farið í gegnum þá síðan hlaupabilið var síðast tómt er nú
				sundurlægt öllum öðrum bilum sem okkur var gefið í byrjun.
			\item<4-> Við skilum því summu lengda þessara sammengja.
			\item<5-> Þessi lausn er $\mathcal{O}(n \log n)$ því við þurftum að raða.
		\end{itemize}
\end{frame}

\begin{frame}[fragile]
	\frametitle{Hlaupabil - Annað dæmi}
\begin{verbatim}
   |
 1:  x----------------x
 2:     x--------x
 3:  x----x
 4:              x----------x
 5:                             x------x
 6:                                                    x--x
 7:                                                x------x
 8:                               x--x
 9:                                          x------------x
10:                             x------x
   |
[]
r = 0
\end{verbatim}
\end{frame}
\addtocounter{framenumber}{-1}

\begin{frame}[fragile]
	\frametitle{Hlaupabil - Annað dæmi}
\begin{verbatim}
     |
 1:  x----------------x
 2:     x--------x
 3:  x----x
 4:              x----------x
 5:                             x------x
 6:                                                    x--x
 7:                                                x------x
 8:                               x--x
 9:                                          x------------x
10:                             x------x
     |
[1]
r = 0
\end{verbatim}
\end{frame}
\addtocounter{framenumber}{-1}

\begin{frame}[fragile]
	\frametitle{Hlaupabil - Annað dæmi}
\begin{verbatim}
     |
 1:  x----------------x
 2:     x--------x
 3:  x----x
 4:              x----------x
 5:                             x------x
 6:                                                    x--x
 7:                                                x------x
 8:                               x--x
 9:                                          x------------x
10:                             x------x
     |
[1, 3]
r = 0
\end{verbatim}
\end{frame}
\addtocounter{framenumber}{-1}

\begin{frame}[fragile]
	\frametitle{Hlaupabil - Annað dæmi}
\begin{verbatim}
        |
 1:  x----------------x
 2:     x--------x
 3:  x----x
 4:              x----------x
 5:                             x------x
 6:                                                    x--x
 7:                                                x------x
 8:                               x--x
 9:                                          x------------x
10:                             x------x
        |
[1, 2, 3]
r = 0
\end{verbatim}
\end{frame}
\addtocounter{framenumber}{-1}

\begin{frame}[fragile]
	\frametitle{Hlaupabil - Annað dæmi}
\begin{verbatim}
          |
 1:  x----------------x
 2:     x--------x
 3:  x----x
 4:              x----------x
 5:                             x------x
 6:                                                    x--x
 7:                                                x------x
 8:                               x--x
 9:                                          x------------x
10:                             x------x
          |
[1, 2]
r = 0
\end{verbatim}
\end{frame}
\addtocounter{framenumber}{-1}

\begin{frame}[fragile]
	\frametitle{Hlaupabil - Annað dæmi}
\begin{verbatim}
                 |
 1:  x----------------x
 2:     x--------x
 3:  x----x
 4:              x----------x
 5:                             x------x
 6:                                                    x--x
 7:                                                x------x
 8:                               x--x
 9:                                          x------------x
10:                             x------x
                 |
[1, 2, 4]
r = 0
\end{verbatim}
\end{frame}
\addtocounter{framenumber}{-1}

\begin{frame}[fragile]
	\frametitle{Hlaupabil - Annað dæmi}
\begin{verbatim}
                 |
 1:  x----------------x
 2:     x--------x
 3:  x----x
 4:              x----------x
 5:                             x------x
 6:                                                    x--x
 7:                                                x------x
 8:                               x--x
 9:                                          x------------x
10:                             x------x
                 |
[1, 4]
r = 0
\end{verbatim}
\end{frame}
\addtocounter{framenumber}{-1}

\begin{frame}[fragile]
	\frametitle{Hlaupabil - Annað dæmi}
\begin{verbatim}
                      |
 1:  x----------------x
 2:     x--------x
 3:  x----x
 4:              x----------x
 5:                             x------x
 6:                                                    x--x
 7:                                                x------x
 8:                               x--x
 9:                                          x------------x
10:                             x------x
                      |
[4]
r = 0
\end{verbatim}
\end{frame}
\addtocounter{framenumber}{-1}

\begin{frame}[fragile]
	\frametitle{Hlaupabil - Annað dæmi}
\begin{verbatim}
                            |
 1:  x----------------x
 2:     x--------x
 3:  x----x
 4:              x----------x
 5:                             x------x
 6:                                                    x--x
 7:                                                x------x
 8:                               x--x
 9:                                          x------------x
10:                             x------x
                            |
[]
r = 24
\end{verbatim}
\end{frame}
\addtocounter{framenumber}{-1}

\begin{frame}[fragile]
	\frametitle{Hlaupabil - Annað dæmi}
\begin{verbatim}
                                |
 1:  x----------------x
 2:     x--------x
 3:  x----x
 4:              x----------x
 5:                             x------x
 6:                                                    x--x
 7:                                                x------x
 8:                               x--x
 9:                                          x------------x
10:                             x------x
                                |
[5]
r = 24
\end{verbatim}
\end{frame}
\addtocounter{framenumber}{-1}

\begin{frame}[fragile]
	\frametitle{Hlaupabil - Annað dæmi}
\begin{verbatim}
                                |
 1:  x----------------x
 2:     x--------x
 3:  x----x
 4:              x----------x
 5:                             x------x
 6:                                                    x--x
 7:                                                x------x
 8:                               x--x
 9:                                          x------------x
10:                             x------x
                                |
[5, 10]
r = 24
\end{verbatim}
\end{frame}
\addtocounter{framenumber}{-1}

\begin{frame}[fragile]
	\frametitle{Hlaupabil - Annað dæmi}
\begin{verbatim}
                                  |
 1:  x----------------x
 2:     x--------x
 3:  x----x
 4:              x----------x
 5:                             x------x
 6:                                                    x--x
 7:                                                x------x
 8:                               x--x
 9:                                          x------------x
10:                             x------x
                                  |
[5, 8, 10]
r = 24
\end{verbatim}
\end{frame}
\addtocounter{framenumber}{-1}

\begin{frame}[fragile]
	\frametitle{Hlaupabil - Annað dæmi}
\begin{verbatim}
                                     |
 1:  x----------------x
 2:     x--------x
 3:  x----x
 4:              x----------x
 5:                             x------x
 6:                                                    x--x
 7:                                                x------x
 8:                               x--x
 9:                                          x------------x
10:                             x------x
                                     |
[5, 10]
r = 24
\end{verbatim}
\end{frame}
\addtocounter{framenumber}{-1}

\begin{frame}[fragile]
	\frametitle{Hlaupabil - Annað dæmi}
\begin{verbatim}
                                       |
 1:  x----------------x
 2:     x--------x
 3:  x----x
 4:              x----------x
 5:                             x------x
 6:                                                    x--x
 7:                                                x------x
 8:                               x--x
 9:                                          x------------x
10:                             x------x
                                       |
[5]
r = 24
\end{verbatim}
\end{frame}
\addtocounter{framenumber}{-1}

\begin{frame}[fragile]
	\frametitle{Hlaupabil - Annað dæmi}
\begin{verbatim}
                                       |
 1:  x----------------x
 2:     x--------x
 3:  x----x
 4:              x----------x
 5:                             x------x
 6:                                                    x--x
 7:                                                x------x
 8:                               x--x
 9:                                          x------------x
10:                             x------x
                                       |
[]
r = 32
\end{verbatim}
\end{frame}
\addtocounter{framenumber}{-1}

\begin{frame}[fragile]
	\frametitle{Hlaupabil - Annað dæmi}
\begin{verbatim}
                                             |
 1:  x----------------x
 2:     x--------x
 3:  x----x
 4:              x----------x
 5:                             x------x
 6:                                                    x--x
 7:                                                x------x
 8:                               x--x
 9:                                          x------------x
10:                             x------x
                                             |
[9]
r = 32
\end{verbatim}
\end{frame}
\addtocounter{framenumber}{-1}

\begin{frame}[fragile]
	\frametitle{Hlaupabil - Annað dæmi}
\begin{verbatim}
                                                   |
 1:  x----------------x
 2:     x--------x
 3:  x----x
 4:              x----------x
 5:                             x------x
 6:                                                    x--x
 7:                                                x------x
 8:                               x--x
 9:                                          x------------x
10:                             x------x
                                                   |
[7, 9]
r = 32
\end{verbatim}
\end{frame}
\addtocounter{framenumber}{-1}

\begin{frame}[fragile]
	\frametitle{Hlaupabil - Annað dæmi}
\begin{verbatim}
                                                       |
 1:  x----------------x
 2:     x--------x
 3:  x----x
 4:              x----------x
 5:                             x------x
 6:                                                    x--x
 7:                                                x------x
 8:                               x--x
 9:                                          x------------x
10:                             x------x
                                                       |
[6, 7, 9]
r = 32
\end{verbatim}
\end{frame}
\addtocounter{framenumber}{-1}

\begin{frame}[fragile]
	\frametitle{Hlaupabil - Annað dæmi}
\begin{verbatim}
                                                          |
 1:  x----------------x
 2:     x--------x
 3:  x----x
 4:              x----------x
 5:                             x------x
 6:                                                    x--x
 7:                                                x------x
 8:                               x--x
 9:                                          x------------x
10:                             x------x
                                                          |
[7, 9]
r = 32
\end{verbatim}
\end{frame}
\addtocounter{framenumber}{-1}

\begin{frame}[fragile]
	\frametitle{Hlaupabil - Annað dæmi}
\begin{verbatim}
                                                          |
 1:  x----------------x
 2:     x--------x
 3:  x----x
 4:              x----------x
 5:                             x------x
 6:                                                    x--x
 7:                                                x------x
 8:                               x--x
 9:                                          x------------x
10:                             x------x
                                                          |
[9]
r = 32
\end{verbatim}
\end{frame}
\addtocounter{framenumber}{-1}

\begin{frame}[fragile]
	\frametitle{Hlaupabil - Annað dæmi}
\begin{verbatim}
                                                          |
 1:  x----------------x
 2:     x--------x
 3:  x----x
 4:              x----------x
 5:                             x------x
 6:                                                    x--x
 7:                                                x------x
 8:                               x--x
 9:                                          x------------x
10:                             x------x
                                                          |
[]
r = 46
\end{verbatim}
\end{frame}

\begin{frame}[fragile]
	\frametitle{Hlaupabil - Útfærsla á dæmi}
	\tiny
	\begin{lstlisting}[language=C]
#include <stdlib.h>
#include <stdio.h>
typedef struct { int x, y; } par;
int cmp(const void* p1, const void* p2) { return ((par*)p1)->x - ((par*)p2)->x; }

int main()
{
	int n, r, i, j, k;
	scanf("%d", &n);
	par a[2*n]; int b[n];
	for (i = 0; i < n; i++)
	{
		scanf("%d %d", &(a[2*i].x), &(a[2*i + 1].x));
		a[2*i].y = i; a[2*i + 1].y = i; b[i] = 0;
	}
	qsort(a, 2*n, sizeof(a[0]), cmp);
	i = 0, r = 0;
	while (i < 2*n)
	{
		k = 1, j = i + 1, b[a[i].y] = 1;
		while (k > 0)
		{
			if (b[a[j].y] == 1) k--;
			else b[a[j].y] = 1, k++;
			j++;
		}
		r = r + a[j - 1].x - a[i].x; i = j;
	}
	printf("%d\n", r);
}
	\end{lstlisting}
\end{frame}

\subsection{Lengsta vaxandi hlutruna}

\begin{frame}
	\frametitle{\texttt{LIS}}
	\begin{itemize}
		\item<1-> Hlutruna í talnarunu er runa af tölum, allar úr upprunalegu rununni, sem eru í sömu röð og
			í upprunalegu rununni.
		\item<2-> Hvernig getum við fundið lengtsu vaxandi hlutrunu (e. longest increasing subsequence (\texttt{LIS}))
			í gefinni runu?
		\item<3-> Sem dæmi er $[2\ 3\ 5\ 9]$  ein ef lengstu vaxandi hlutrunum $[2\ 3\ 1\ 5\ 9\ 8\ 7]$.
	\end{itemize}
\end{frame}

\begin{frame}
	\frametitle{\texttt{LIS} $n2^n$}
	\begin{itemize}
		\item<1-> Gerum ráð fyrir að við höfum talnarunu af lengd $1 \leq n \leq 15$.
		\item<2-> Við getum þá prófað allar hlutrunur, skoðað hvort þær séu vaxandi og geymt þá lengstu.
	\end{itemize}
\end{frame}

\begin{frame}[fragile]
	\frametitle{\texttt{LIS} $n2^n$}
	\tiny
	\begin{lstlisting}[language=C]
#include <stdio.h>
int main() {
	int n, i, j; scanf("%d", &n); int a[n];
	for (i = 0; i < n; i++) {
		scanf("%d", &(a[i]));
	}
	int mx = 0, mxi;
	for (i = 1; i < (1 << n); i++) {
		int s[n], sc = 0;
		for (j = 0; j < n; j++) {
			if (((1 << j)&i) != 0) {
				s[sc++] = j;
			}
		}
		for (j = 1; j < sc; j++)
			if (a[s[j]] < a[s[j - 1]]) break;
		if (j == sc && sc > mx) {
			mx = sc;
			mxi = i;
		}
	}
	printf("%d\n", mx);
	for (i = 0; i < n; i++) {
		if (((1 << i)&mxi) != 0) {
			printf("%d ", a[i]);
		}
	}
	printf("\n");
}
	\end{lstlisting}
\end{frame}

\begin{frame}
	\frametitle{\texttt{LIS} $n^2$}
	\begin{itemize}
		\item<1-> Það er, ótrúlegt en satt, hægt að gera þetta hraðar.
		\item<2-> Við getum notað kvika bestun!
		\item<3-> Látum $a$ vera runu af $n$ tölum.
		\item<4-> Látum $f(k)$ vera lengd lengstu hlutrunu sem endar í $k$-ta staki $a$.
		\item<5-> Við getum þá reiknað gildi $f$ í línulegum tíma og gerum það fyrir sérhverja tölu $1, 2, ..., n$.
		\item<6-> {\bf Æfing:} Finna rununa.
	\end{itemize}
\end{frame}

\begin{frame}[fragile]
	\frametitle{\texttt{LIS} $n^2$}
	\tiny
	\begin{lstlisting}[language=C]
#include <stdio.h>

int main()
{
	int n, i, j;
	scanf("%d", &n);
	int a[n], c[n];
	for (i = 0; i < n; i++)
	{
		scanf("%d", &(a[i]));
	}
	for (i = 0; i < n; i++)
	{
		c[i] = 1;
		for (j = 0; j < i; j++)
		{
			if (a[i] > a[j] && c[i] < c[j] + 1)
			{
				c[i] = c[j] + 1;
			}
		}
	}
	for (i = 0; i < n; i++) printf("%4d", a[i]); printf("\n");
	for (i = 0; i < n; i++) printf("%4d", c[i]); printf("\n");
}
\end{lstlisting}
\begin{verbatim}
ARCH% ./lisn2
7 2 3 1 5 9 8 7
   2   3   1   5   9   8   7
   1   2   1   3   4   4   4
\end{verbatim}
\end{frame}

\begin{frame}
	\frametitle{\texttt{LIS} $n\log n$}
	\begin{itemize}
		\item<1-> En er hægt að gera þetta ennþá hraðar?
		\item<2-> Heldur betur!
		\item<3-> Við getum notað helmingunarleit.
		\item<4-> Skoðum first reiknirit sem er ekki hentugt að útfæra.
	\end{itemize}
\end{frame}

\begin{frame}
	\frametitle{\texttt{LIS} $n\log n$}
	\begin{itemize}
		\item<1-> Höfum lista af listum.
		\item<2-> Skilgreinum röðun þannig að listar eru bornir saman eftir síðasta staki.
		\item<3-> Listalistinn okkar byrjar tómur.
		\item<4-> Löbbum í gegnum $a$ í réttri röð og fyrir hvert stak $a[i]$ finnum við (með helmingunarleit)
			þann lista sem hefur stærsta aftasta stakið sem er minna en $a[i]$.
		\item<5-> Við afritum nú listann sem við fundum, setjum hann fyrir aftan, bætum stakinu okkar við hann
			og fjarlægjum listann fyrir aftann nýja listann (ef það er einhver).
		\item<6-> Hvert skref í þessu reiknu riti er $\mathcal{O}(\log n)$ ef við getum afritað lista
			í föstum tíma (sem er ekki beint eðlilegt).
		\item<7-> Rúllum í gegnum þetta fyrir listann $[0, 8, 4, 12, 2, 10, 6, 14, 1, 9, 5, 13, 3, 11, 7, 15]$.
	\end{itemize}
\end{frame}

\begin{frame}[fragile]
	\frametitle{\texttt{LIS} $n\log n$}
\begin{verbatim}
[0, 8, 4, 12, 2, 10, 6, 14, 1, 9, 5, 13, 3, 11, 7, 15]
 |

[0]





\end{verbatim}
\end{frame}
\addtocounter{framenumber}{-1}

\begin{frame}[fragile]
	\frametitle{\texttt{LIS} $n\log n$}
\begin{verbatim}
[0, 8, 4, 12, 2, 10, 6, 14, 1, 9, 5, 13, 3, 11, 7, 15]
    |

[0]
[0, 8]




\end{verbatim}
\end{frame}
\addtocounter{framenumber}{-1}

\begin{frame}[fragile]
	\frametitle{\texttt{LIS} $n\log n$}
\begin{verbatim}
[0, 8, 4, 12, 2, 10, 6, 14, 1, 9, 5, 13, 3, 11, 7, 15]
       |

[0]
[0, 4]




\end{verbatim}
\end{frame}
\addtocounter{framenumber}{-1}

\begin{frame}[fragile]
	\frametitle{\texttt{LIS} $n\log n$}
\begin{verbatim}
[0, 8, 4, 12, 2, 10, 6, 14, 1, 9, 5, 13, 3, 11, 7, 15]
           |

[0]
[0, 4]
[0, 4, 12]



\end{verbatim}
\end{frame}
\addtocounter{framenumber}{-1}

\begin{frame}[fragile]
	\frametitle{\texttt{LIS} $n\log n$}
\begin{verbatim}
[0, 8, 4, 12, 2, 10, 6, 14, 1, 9, 5, 13, 3, 11, 7, 15]
              |

[0]
[0, 2]
[0, 4, 12]



\end{verbatim}
\end{frame}
\addtocounter{framenumber}{-1}

\begin{frame}[fragile]
	\frametitle{\texttt{LIS} $n\log n$}
\begin{verbatim}
[0, 8, 4, 12, 2, 10, 6, 14, 1, 9, 5, 13, 3, 11, 7, 15]
                  |

[0]
[0, 2]
[0, 2, 10]



\end{verbatim}
\end{frame}
\addtocounter{framenumber}{-1}

\begin{frame}[fragile]
	\frametitle{\texttt{LIS} $n\log n$}
\begin{verbatim}
[0, 8, 4, 12, 2, 10, 6, 14, 1, 9, 5, 13, 3, 11, 7, 15]
                     |

[0]
[0, 2]
[0, 2, 6]



\end{verbatim}
\end{frame}
\addtocounter{framenumber}{-1}

\begin{frame}[fragile]
	\frametitle{\texttt{LIS} $n\log n$}
\begin{verbatim}
[0, 8, 4, 12, 2, 10, 6, 14, 1, 9, 5, 13, 3, 11, 7, 15]
                         |

[0]
[0, 2]
[0, 2, 6]
[0, 2, 6, 14]


\end{verbatim}
\end{frame}
\addtocounter{framenumber}{-1}

\begin{frame}[fragile]
	\frametitle{\texttt{LIS} $n\log n$}
\begin{verbatim}
[0, 8, 4, 12, 2, 10, 6, 14, 1, 9, 5, 13, 3, 11, 7, 15]
                            |

[0]
[0, 1]
[0, 2, 6]
[0, 2, 6, 14]


\end{verbatim}
\end{frame}
\addtocounter{framenumber}{-1}

\begin{frame}[fragile]
	\frametitle{\texttt{LIS} $n\log n$}
\begin{verbatim}
[0, 8, 4, 12, 2, 10, 6, 14, 1, 9, 5, 13, 3, 11, 7, 15]
                               |

[0]
[0, 1]
[0, 2, 6]
[0, 2, 6, 9]


\end{verbatim}
\end{frame}
\addtocounter{framenumber}{-1}

\begin{frame}[fragile]
	\frametitle{\texttt{LIS} $n\log n$}
\begin{verbatim}
[0, 8, 4, 12, 2, 10, 6, 14, 1, 9, 5, 13, 3, 11, 7, 15]
                                  |

[0]
[0, 1]
[0, 1, 5]
[0, 2, 6, 9]


\end{verbatim}
\end{frame}
\addtocounter{framenumber}{-1}

\begin{frame}[fragile]
	\frametitle{\texttt{LIS} $n\log n$}
\begin{verbatim}
[0, 8, 4, 12, 2, 10, 6, 14, 1, 9, 5, 13, 3, 11, 7, 15]
                                      |

[0]
[0, 1]
[0, 1, 5]
[0, 2, 6, 9]
[0, 2, 6, 9, 13]

\end{verbatim}
\end{frame}
\addtocounter{framenumber}{-1}

\begin{frame}[fragile]
	\frametitle{\texttt{LIS} $n\log n$}
\begin{verbatim}
[0, 8, 4, 12, 2, 10, 6, 14, 1, 9, 5, 13, 3, 11, 7, 15]
                                         |

[0]
[0, 1]
[0, 1, 3]
[0, 2, 6, 9]
[0, 2, 6, 9, 13]

\end{verbatim}
\end{frame}
\addtocounter{framenumber}{-1}

\begin{frame}[fragile]
	\frametitle{\texttt{LIS} $n\log n$}
\begin{verbatim}
[0, 8, 4, 12, 2, 10, 6, 14, 1, 9, 5, 13, 3, 11, 7, 15]
                                             |

[0]
[0, 1]
[0, 1, 3]
[0, 2, 6, 9]
[0, 2, 6, 9, 11]

\end{verbatim}
\end{frame}
\addtocounter{framenumber}{-1}

\begin{frame}[fragile]
	\frametitle{\texttt{LIS} $n\log n$}
\begin{verbatim}
[0, 8, 4, 12, 2, 10, 6, 14, 1, 9, 5, 13, 3, 11, 7, 15]
                                                |

[0]
[0, 1]
[0, 1, 3]
[0, 1, 3, 7]
[0, 2, 6, 9, 11]

\end{verbatim}
\end{frame}
\addtocounter{framenumber}{-1}

\begin{frame}[fragile]
	\frametitle{\texttt{LIS} $n\log n$}
\begin{verbatim}
[0, 8, 4, 12, 2, 10, 6, 14, 1, 9, 5, 13, 3, 11, 7, 15]
                                                    |

[0]
[0, 1]
[0, 1, 3]
[0, 1, 3, 7]
[0, 2, 6, 9, 11]
[0, 2, 6, 9, 11, 15]
\end{verbatim}
\end{frame}

\begin{frame}
	\frametitle{\texttt{LIS} $n\log n$}
	\begin{itemize}
		\item<1-> Hvernig útfærum við þetta?
		\item<2-> Við nýtum okkur það að allir listarnir sem við vorum með
			eru (að mestu) óþarfir því við þurfum bara aftasta stakið í hverjum þeirra.
		\item<3-> Við erum þá með lista af tölum, sem gerir allt mun auðveldara.
	\end{itemize}
\end{frame}

\begin{frame}[fragile]
	\frametitle{\texttt{LIS} $n\log n$}
	\tiny
	\begin{lstlisting}[language=C]
#define INF 2000000000

int bs(int* a, int n, int t)
{
	int r = 0, s = n;
	while (r < s)
		if (a[(r + s)/2] < t) r = (r + s)/2 + 1;
		else s = (r + s)/2;
	return r;
}

int lis(int* a, int n)
{
	int i, j;
	int b[n + 2];
	for (i = 0; i < n + 2; i++)
	{
		b[i] = INF;
	}
	b[0] = -INF;
	for (i = 0; i < n; i++)
	{
		j = bs(b, n + 1, a[i]);
		if (b[j - 1] < a[i] && a[i] < b[j]) b[j] = a[i];
	}
	for (i = 0; b[i] != INF; i++);
	return i - 1;
}
	\end{lstlisting}
\end{frame}

\subsection{Næsta stærra stak}

\begin{frame}
	\frametitle{Næsta stærra stak}
	\begin{itemize}
		\item<1-> Látum $a$ vera lista af $n$ tölum.
		\item<2-> Við segjum að næsta stak stærra en $a[i]$
			(next greater element (\texttt{NGE})) sé minnsta stak $a[j]$ þ.a. $j > i$.
		\item<3-> Sem dæmi er \texttt{NGE} $4$ í $[2, 3, 4, 8, 5]$ er $8$.
		\item<4-> Til þæginda segjum við að
			\texttt{NGE} $8$ í $[2, 3, 4, 8, 5]$ er $-1$.
		\item<5-> Það er auðséð að við getum reiknað \texttt{NGE} allra talnanna í
			$\mathcal{O}(n^2)$.
	\end{itemize}
\end{frame}

\begin{frame}[fragile]
	\frametitle{\texttt{NGE} $n^2$}
	\tiny
	\begin{lstlisting}[language=C]
void nge(int* a, int* b, int n)
{
	int i, j;
	for (i = 0; i < n; i++)
	{
		b[i] = -1;
		for (j = i + 1; j < n; j++)
		{
			if (a[i] < a[j])
			{
				b[i] = j;
				break;
			}
		}
	}
}
	\end{lstlisting}
\end{frame}

\begin{frame}
	\frametitle{\texttt{NGE} $n$}
	\begin{itemize}
		\item<1-> En að sjálfsögðu getum við gert þetta betur.
		\item<2-> Gefum okkur hlaða $h$. 
		\item<3-> Löbbum í gegnum $a$ í réttri röð.
		\item<4-> Tökum nú tölur úr hlaðan og setjum \texttt{NGE} þeirra talna
			sem $a[i]$ á meðan $a[i]$ er stærri en toppurinn á hlaðanum. Þegar toppurinn
			á hlaðanum er stærri en $a[i]$ þá látum við $a[i]$ á hlaðann og höldum svo áfram.
		\item<5-> Bersýnilega er hlaðinn ávallt raðaður, svo þú færð allar tölur
			sem eiga að hafa $a[i]$ sem \texttt{NGE}.
		\item<6-> Þegar búið er að fara í gegnum $a$ látum við \texttt{NGE} þeirra staka sem
			eftir eru í $h$ vera $-1$.
		\item<7-> Þar sem maður setur hverja tölu einu sinni á hlaðann og tekur hana svo
			af þá er þetta reiknirit $\mathcal{O}(n)$.
	\end{itemize}
\end{frame}

\begin{frame}[fragile]
	\frametitle{\texttt{NGE} $n$}
\begin{verbatim}
 0 1 2 3 4 5 6 7
[2 3 1 5 9 6 8 7]
|

 0 1 2 3 4 5 6 7
[x x x x x x x x]


h: []
\end{verbatim}
\end{frame}
\addtocounter{framenumber}{-1}

\begin{frame}[fragile]
	\frametitle{\texttt{NGE} $n$}
\begin{verbatim}
 0 1 2 3 4 5 6 7
[2 3 1 5 9 6 8 7]
 |

 0 1 2 3 4 5 6 7
[x x x x x x x x]


h: [2]
\end{verbatim}
\end{frame}
\addtocounter{framenumber}{-1}

\begin{frame}[fragile]
	\frametitle{\texttt{NGE} $n$}
\begin{verbatim}
 0 1 2 3 4 5 6 7
[2 3 1 5 9 6 8 7]
   |

 0 1 2 3 4 5 6 7
[1 x x x x x x x]


h: [3]
\end{verbatim}
\end{frame}
\addtocounter{framenumber}{-1}

\begin{frame}[fragile]
	\frametitle{\texttt{NGE} $n$}
\begin{verbatim}
 0 1 2 3 4 5 6 7
[2 3 1 5 9 6 8 7]
     |

 0 1 2 3 4 5 6 7
[1 x x x x x x x]


h: [3 1]
\end{verbatim}
\end{frame}
\addtocounter{framenumber}{-1}

\begin{frame}[fragile]
	\frametitle{\texttt{NGE} $n$}
\begin{verbatim}
 0 1 2 3 4 5 6 7
[2 3 1 5 9 6 8 7]
       |

 0 1 2 3 4 5 6 7
[1 3 3 x x x x x]


h: [5]
\end{verbatim}
\end{frame}
\addtocounter{framenumber}{-1}

\begin{frame}[fragile]
	\frametitle{\texttt{NGE} $n$}
\begin{verbatim}
 0 1 2 3 4 5 6 7
[2 3 1 5 9 6 8 7]
         |

 0 1 2 3 4 5 6 7
[1 3 3 4 x x x x]


h: [9]
\end{verbatim}
\end{frame}
\addtocounter{framenumber}{-1}

\begin{frame}[fragile]
	\frametitle{\texttt{NGE} $n$}
\begin{verbatim}
 0 1 2 3 4 5 6 7
[2 3 1 5 9 6 8 7]
           |

 0 1 2 3 4 5 6 7
[1 3 3 4 x x x x]


h: [9 6]
\end{verbatim}
\end{frame}
\addtocounter{framenumber}{-1}

\begin{frame}[fragile]
	\frametitle{\texttt{NGE} $n$}
\begin{verbatim}
 0 1 2 3 4 5 6 7
[2 3 1 5 9 6 8 7]
             |

 0 1 2 3 4 5 6 7
[1 3 3 4 x 6 x x]


h: [9 8]
\end{verbatim}
\end{frame}
\addtocounter{framenumber}{-1}

\begin{frame}[fragile]
	\frametitle{\texttt{NGE} $n$}
\begin{verbatim}
 0 1 2 3 4 5 6 7
[2 3 1 5 9 6 8 7]
               |

 0 1 2 3 4 5 6 7
[1 3 3 4 x 6 x x]


h: [9 8 7]
\end{verbatim}
\end{frame}

\begin{frame}[fragile]
	\frametitle{\texttt{NGE} $n$}
	\tiny
	\begin{lstlisting}[language=C]
#include <stdio.h>

void nge(int* a, int* b, int n)
{
	int s[n], c = 0, i;
	for (i = 0; i < n; i++)
	{
		while (c > 0 && a[s[c - 1]] < a[i]) b[s[--c]] = i;
		s[c++] = i;
	}
	while (c > 0) b[s[--c]] = -1;
}

int main()
{
	int i, n;
	printf("Staerd listans: "); fflush(stdout);
	scanf("%d", &n);
	int a[n], b[n];
	printf("%d tolur: ", n); fflush(stdout);
	for (i = 0; i < n; i++) scanf("%d", &(a[i]));
	nge(a, b, n);

	printf("NGE:\n");
	for (i = 0; i < n; i++) printf("%8d ", a[i]);
	printf("\n");
	for (i = 0; i < n; i++) printf("%8d ", (b[i] != -1 ? a[b[i]] : -1));
	printf("\n");
}
	\end{lstlisting}
\end{frame}

\subsection{Línuleg rakningarvensl}

\begin{frame}
	\frametitle{Fibonacci í $\log n$}
	\begin{itemize}
		\item<1-> Munið þið þegar Atli talaði um að unnt væri að reikna $n$-tu Fibonacci-töluna í $\log n$?
		\item<2-> Skoðum það aðeins nánar.
		\item<3-> Hugmyndin er vigurinn 
			\[
				\left (
				\begin{array}{c c}
					1 & 1\\
					1 & 0
				\end{array}
				\right )^n
				\left (
				\begin{array}{c}
					a\\
					b
				\end{array}
				\right )
			\]
		\item<4-> inniheldur $n$-tu og $(n + 1)$-tu Fibonacci-töluna með fræi $a$ og $b$ 
			(s.s. venjulegu Fibonacci fást með $a = b = 1$).
		\item<5-> Við getum síðan reiknað þetta í $\log n$ með aðferðum úr síðasta fyrirlestri.
		\item<6-> Nánar tiltekið reiknum við þetta $k^3 \log n$ þar sem $k$ er vídd fylkisins ($k = 2$ hér).
	\end{itemize}
\end{frame}

\begin{frame}
	\frametitle{Fibonacci í $\log n$}
	\begin{itemize}
		\item<1-> Málið er að þetta gildir almennt fyrir línuleg rakningar vensl.
		\item<2-> Látum $(a_n)_{n \geq 1}$ þ.a. $a_n = \sum_{i = 1}^{m}c_ia_{n - i}$, fyrir $n > m$
			og $a_i = k_i$ annars. Hér eru $c_i$ og $k_i$ fastar.
		\item<3-> Við segjum nú að $a_n$ ákvarðast af $m$-ta stigs línulegum rakningarvenslum með upphafsskilyrði
			$k_1, k_2, ..., k_m$ og fasta $c_1, c_2, ..., c_m$.
		\item<4-> Við getum notað jöfnuna 
			\[
				\left (
				\begin{array}{c c c c c}
					c_1 & c_2 & ... & c_{m - 1} & c_m\\
					1   & 0   & ... & 0 & 0\\
					0   & 1   & ... & 0 & 0\\
					\vdots & \vdots & \ddots & \vdots & \vdots\\
					0   & 0   & ... & 1 & 0\\
				\end{array}
				\right )^n
				\left (
				\begin{array}{c}
					k_1\\
					k_2\\
					\vdots\\
					k_m
				\end{array}
				\right )
				=
				\left (
				\begin{array}{c}
					a_{n + 1}\\
					a_{n + 2}\\
					\vdots\\
					a_{n + m + 1}
				\end{array}
				\right )
			\]
			til að reikna $a_n$ í $m^3 \log n$. Þar sem $m$ er yfirleitt lítið getum við oftast notað þetta fyrir mjög stór $n$.
	\end{itemize}
\end{frame}

\subsection{Röðun}

\begin{frame}
	\frametitle{Röðun}
	\begin{itemize}
		\item<1-> Ég reikna með að þið kunnið öll að raða tölum.
		\item<2-> Stundum þegar maður raðar tölum vill maður viðhalda einhverskonar röð.
		\item<3-> Röðunarreiknirt kallast \emph{stöðugt} ef jafn stór stök viðhalda upprunalegu röð sinni.
		\item<4-> Sem dæmi tökum við lista af tvenndum
			\[
				(2, 6)\ (1, 3)\ (2, 3)\ (1, 1)\ (2, 2)
			\]
			og röðum eftir fyrst stakinu.
		\item<5-> Ef við röðum stöðugt þá fæst
			\[
				(1, 3)\ (1, 1)\ (2, 6)\ (2, 3)\ (2, 2).
			\]
	\end{itemize}
\end{frame}

\begin{frame}
	\frametitle{Röðun}
	\begin{itemize}
		\item<1-> Í \texttt{C++} er \texttt{stable\_sort} stöðugt.
		\item<2-> Í \texttt{Python} er \texttt{sort} stöðugt.
		\item<3-> Í \texttt{C} er hægt að raða og nota stöðu staka í upprunalega listanum
			til að gera upp á milli jafn stórra talna.
	\end{itemize}
\end{frame}

\subsection{Gauss-Jordan útfærsla}

\begin{frame}
	\frametitle{Gauss-Jordan}
	\begin{itemize}
		\item<1-> Fyrir nokkrum vikum sýndum við ykkur útfærslu á Gauss-Jordan eyðingu.
		\item<2-> Fólki fannst hún nokkuð torlesin svo við skrifuðum aðra, mögulega þægilegri.
		\item<3-> Við erum búnir að prófa hana á nokkrum dæmum og Kattis hefur ekki ennþá kvartað.
	\end{itemize}
\end{frame}

\begin{frame}[fragile]
	\frametitle{Gauss-Jordan}
	\tiny
	\begin{lstlisting}[language=C]
#define EPS 1e-9
// a er n x m fylki
void gauss(double* a, int n, int m)
{
	int i, j, k, t; double p;
	for (i = 0; i < n; i++)
	{
		t = -1;
		while (++t < m && fabs(a[i*m + t]) < EPS);

		if (t == m) continue;
		p = a[i*m + t];
		for (j = t; j < m; j++)
		{
			a[i*m + j] = a[i*m + j]/p;
		}

		for (j = 0; j < n; j++)
		{
			if (i != j)
			{
				p = a[j*m + t];
				for (k = t; k < m; k++)
				{
					a[j*m + k] = a[j*m + k] - a[i*m + k]*p;
				}
			}
		}
	}
}
	\end{lstlisting}
\end{frame}

\end{document}
