\title{Ad hoc}
\author{Bergur Snorrason}
\date{\today}

\begin{document}

\frame{\titlepage}

\env{frame}
{
	\frametitle{Lausnar aðferðir}
	\env{itemize}
	{
		\item<1-> Þegar við leysum dæmi í keppnisforritun notumst við oftast við eina af eftirfarandi aðferðum:
		\env{itemize}
		{
			\item<2-> \emph{Ad hoc},
			\item<3-> \emph{Tæmandi leit} eða \emph{ofbeldis aðferðin} (e. \emph{complete search, brute force}),
			\item<4-> \emph{Gráðug reiknirit} (e. \emph{greedy algorithms}),
			\item<5-> \emph{Deila og drottna} (e. \emph{divide and conquer}),
			\item<6-> \emph{Kvik bestun} (e. \emph{dynamic programming}).
		}
		\item<7-> Þessi skipting er ekki fullkomin, en það er þó gott að hafa hana í huga.
		\item<8-> Til dæmis má færa rök fyrir því að gráðugar lausnir og \texttt{D\&C} séu sértilfelli af kvikri bestun.
		\item<9-> Við munum byrja á því að fjalla almennt um þessar aðferðir og fara svo í sértækara efni.
		\item<10-> Þá er oft gott að hafa í huga hvernig flokka megi reikniritin.
	}
}

\env{frame}
{
	\frametitle{Ad hoc}
	\env{itemize}
	{
		\item<1-> Ef lausn dæmisins byggir ekki á sérþekkingu flokkast dæmið sem \emph{Ad hoc}.
		\item<2-> Þessi dæmi eru stundum flokkuð undir ``implementation'', eða sem \emph{útfærsludæmi}.
		\item<3-> Þetta er gert því flest Ad hoc dæmi snúast um að fylgja beint leiðbeiningum.
		\item<4-> Það eru þó undantekningar.
		\item<5-> Í NCPC $2020$ var Ad hoc dæmi sem mætti ekki flokkast sem útfærsludæmi.
		\item<6-> Ad hoc dæmi flokkast oft til léttari dæma í keppnum.
		\item<7-> Áðurnefnt NCPC dæmi er þó aftur undanteking, því engin keppandi náði að leysa það dæmi.
		\item<8-> Samkvæmt skilgreiningu getum við ekki rætt Ad hoc dæmi ítarlega. Tökum því nokkur dæmi.
	}
}

\env{frame}
{
	\frametitle{Blandað brot}
	\env{itemize}
	{
		\item<1-> Þú átt að breyta almennu broti í blandað brot.
		\item<2-> Munið að almenna brotið $p/q$, og blandaða brotið $a\ b/c$ tákna sömu töluna ef $p/q = a + b/c$.
		\item<3-> Munið einnig að ef $a\ b/c$ er almennt brot þá gildir $b < c$.
		\item<4-> Blandaða brotið ykkar á að hafa sama nefnara og upprunarlega brotið.
		\item<5-> Inntakið inniheldur tvær heiltölur $1 \leq p, q \leq 10^9$.
		\item<6-> Úttakið skal innihalda blandaða brotið sem svarar til $p/q$.
		\item<7->[]
		\env{tabular}
		{
			{l | l | l}
			& Inntak & Úttak\\
			\hline
			Sýnidæmi 1 & \texttt{27 12} & \texttt{2 3 / 12}\\
			Sýnidæmi 2 & \texttt{2460000 98400} & \texttt{25 0 / 98400}\\
			Sýnidæmi 3 & \texttt{3 4000} & \texttt{0 3 / 4000}\\
		}
	}
}

\env{frame}
{
	\frametitle{Lausn á blandað brot}
	\env{itemize}
	{
		\item<1-> Hér nægir okkur að reikna.
		\item<2-> Við getum aðeins stytt okkur leið með því að nota heiltöludeilingu.
		\item<3-> Við fáum þá að $a$ er heiltalan sem fæst með deilingunni $p/q$ og $b$ er afgangurinn.
	}
}

\env{frame}
{
	\code{code/mixedfractions.c}
}

\env{frame}
{
	\frametitle{Barnahjal}
	\env{itemize}
	{
		\item<1-> Þið eruð að reyna að kenna barni að telja.
		\item<2-> Það er þó ekki alltaf hægt að heyra hvað barnið segir.
		\item<3-> Þið viljið ákvarða hvort það sem barnið er að segja gæti mögulega verið rétt.
		\item<4-> Fyrsta lína inntaksins inniheldur heiltölu $1 \leq n \leq 10^3$.
		\item<5-> Síðan fylgir ein lína með $n$ strengjum.
		\item<6-> Hver strengur er annaðhvort heiltala á bilinu $[0, 10^4]$ eða strengurinn ``mumble''.
		\item<7-> Ef það er hægt að skipta út öllum ``mumble'' fyrir tölu þannig að talningin sé rétt skal prenta ``jebb''.
		\item<8-> Annars skal prenta ``neibb''.
	}
}

\env{frame}
{
	\frametitle{Lausn á Barnahjal}
	\env{itemize}
	{
		\item<1-> Ef $i$-ti strengurinn inniheldur strenginn sem svarar til tölurnnar $i$ eða ``mumble'', fyrir öll $i$,
			þá er barnið kannski að telja rétt.
		\item<2-> Annars er barnið að telja rangt.
	}
}

\env{frame}
{
	\code{code/babybites.py}
}

\env{frame}
{
}

\end{document}
