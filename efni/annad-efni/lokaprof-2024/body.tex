\title{Lokakeppnin $2024$}
\author{Bergur Snorrason}
\date{\today}

\begin{document}

\frame{\titlepage}

\section{Lokaprófið}
\subsection{Inngangur}
\env{frame}
{
    \env{itemize}
    {
        \item<1-> Eftir viku er lokaprófið okkar.
        \item<2-> Prófið er þrjár klukkustundir.
        \item<3-> Það eru sex dæmi og þið eigið að leysa þrjú eða fleiri.
        \item<4-> Reglurnar verða strangari en þær hafa verið í vikukeppnunum.
    }
}

\subsection{Reglur}
\env{frame}
{
    \env{itemize}
    {
        \item<1-> Þið megið nota allt kennsluefni námskeiðins (allt sem er á Canvas síðunni okkar).
        \item<2-> Þið megið nota kóðan sem þið hafið skrifað í námskeiðinu.
        \item<3-> Einu samskiptin sem þið megið eiga eru við prófdómara:
        \env{itemize}
        {
            \item<4-> Þið megið ekki samskiptamiðla á netinu.
            \item<5-> Þið megið ekki nota Google.
            \item<6-> Þið megið ekki nota ykkur mállíkön.
        }
        \item<7-> Sér í lagi megið þið ekki nota neina tækni sem byggir á mállíkönum.
        \item<8-> Þið megið því ekki nota neina ritla sem nýta sér gervigreind.
        \item<9-> Til dæmis má ekki nota Visual Studio Code.
    }
}

\subsection{Ritlar}
\env{frame}
{
    \env{itemize}
    {
        \item<1-> Dæmi um ritil sem má nota er Notepad++ (Windows) og Notepadqq (Mac eða Linux).
        \item<2-> Ef þið eru ekki viss hvort það megi nota eitthva þá má það örugglega ekki.
        \item<3-> Prófið er samið með þetta í huga.
        \item<4-> Dæmin má leysa án þessa að skrifa mikinn og flókinn kóða, og byggja aðallega á að nýta þann kóða sem ég hef gefið ykkur á Canvas.
    }
}

\section{Upprifjun}
\subsection{Inngangur}
\env{frame}
{
    \env{itemize}
    {
        \item<1-> Rifjum upp námsefnið.
    }
}

\subsection{Vika 03, 04 og 05 - Almennar lausnaraðferðir}
\env{frame}
{
    \frametitle{Vika $03$, $04$ og $05$ - Almennar lausnaraðferðir}
    \env{itemize}
    {
        \item<1-> Hér ber helst að nefna kvika bestun, en einnig er gott að kunna að nota helmingunarleit til að umorða dæmi.
        \item<2-> Helstu reikniritin til að kunna eru dæmin sem flokkast til bakpokaverkefnisins:
        \env{itemize}
        {
            \item<3-> \ilcode{knapsack.c}, \ilcode{knapsack.py}, \ilcode{knapsack.java}
            \item<4-> \ilcode{subsetsum.c}
            \item<5-> \ilcode{partition.c}
        }
        \item<6-> Einnig fórum við í farandsölumannaverkefnið: \ilcode{tsp.c}.
    }
}

\subsection{Vika 06 - Gagnagrindur}
\env{frame}
{
    \frametitle{Vika $06$ - Gagnagrindur}
    \env{itemize}
    {
        \item<1-> Við fórum helst í sammengisleit og biltré.
        \item<2-> Við útfærðum fimm biltré, í vaxandi flækjuröð, helst ber að skoða fyrstu þrjú dæmin.
        \env{itemize}
        {
            \item<3-> \ilcode{biltre*.c}, \ilcode{biltre.py}, \ilcode{biltre.java}
            \item<4-> \ilcode{sammengisleit.c}
        }
    }
}

\subsection{Vika 07 - Vaxandi hlutrunur}
\env{frame}
{
    \frametitle{Vika $07$ - Vaxandi hlutrunur}
    \env{itemize}
    {
        \item<1-> Við fórum meira í gagnagrindur, en helst fjölluðum við um vaxandi hlutrunur.
        \env{itemize}
        {
            \item<2-> \ilcode{lis.c}, \ilcode{lis.py}, \ilcode{lis.java}
        }
    }
}

\subsection{Vika 08 og 09 - Netfræði}
\env{frame}
{
    \frametitle{Vika $08$ og $09$ - Netfræði}
    \env{itemize}
    {
        \item<1-> Við fórum í mörg af frægustu reikniritum í netafræði.
        \env{itemize}
        {
            \item<2-> \ilcode{bfs.cpp}, \ilcode{dfs.cpp}
            \item<3-> \ilcode{samhengisthaettir*.cpp}, \ilcode{lidhnutar-og-bryr.cpp}, \ilcode{stefndir-samhengisthaettir.cpp}
            \item<4-> \ilcode{grannrodun.cpp}
            \item<5-> \ilcode{dijkstra.cpp} \ilcode{bellman-ford*.cpp} \ilcode{floyd-warshall.cpp}
            \item<6-> \ilcode{kruskal.cpp}
        }
    }
}

\subsection{Vika 10 - Talnafræði}
\env{frame}
{
    \frametitle{Vika $10$ - Talnafræði}
    \env{itemize}
    {
        \item<1-> Við fjölluðum um mikilvægar undirstöðu niðurstöður í talnafærði.
        \item<2-> Þetta var aðallega gert sem undirbúningur fyrir reiknirit til að frumþátta tölur.
        \env{itemize}
        {
            \item<3-> \ilcode{eratosthenes*.c}
            \item<4-> \ilcode{miller-rabin.c} \ilcode{miller-rabin.py} \ilcode{miller\_rabin.java}
            \item<5-> \ilcode{pollar-rho.c} \ilcode{pollar-rho.py} \ilcode{pollar\_rho.java}
        }
    }
}

\subsection{Vika 11 - Talningarfræði}
\env{frame}
{
    \frametitle{Vika $11$ - Talningarfræði}
    \env{itemize}
    {
        \item<1-> Helst skoðuðum við hvernig reikna má upp úr línulega rakningarvenslum og hvernig telja megi umhverfingar í lista.
        \env{itemize}
        {
            \item<2-> \ilcode{fibonacci.c} \ilcode{fibonacci.py} \ilcode{fibonacci.java}
            \item<3-> \ilcode{umhverfingar.c} \ilcode{umhverfingar.py} \ilcode{umhverfingar.java}
        }
        \item<4-> Einnig er gott að kunna að nota kvika bestun til að leysa talningarfræði dæmi.
    }
}

\subsection{Vika 12 - Rúmfræði}
\env{frame}
{
    \frametitle{Vika $12$ - Rúmfræði}
    \env{itemize}
    {
        \item<1-> Ásamt því að fjalla um reiknirit fyrir marghyrninga sáum við hvernig væri hægt að finn nálægustu punkta í punktasafni.
        \env{itemize}
        {
            \item<2-> \ilcode{flatarmal-og-ummal.c}
            \item<3-> \ilcode{punktur-i-marghyrningi.c}
            \item<4-> \ilcode{kuptur-hjupur.c} \ilcode{kuptur-hjupur.cpp}
            \item<5-> \ilcode{nalaegustu-punktar.c}
        }
    }
}

\subsection{Vika 13}
\env{frame}
{
    \frametitle{Vika $13$ - Samansóp}
    \env{itemize}
    {
        \item<1-> Í lokinn tóku við saman efni sem passaði ekki inn í neina aðra viku.
        \env{itemize}
        {
            \item<2-> \ilcode{kmp.c}
            \item<3-> \ilcode{naesta-staerra-stak.c}
            \item<4-> \ilcode{naesti-sameiginlegi-forfadir*.c}
        }
    }
}

\section{Þessi glæra er viljandi auð}
\env{frame}
{
}

\end{document}
