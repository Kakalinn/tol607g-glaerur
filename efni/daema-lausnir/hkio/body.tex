\title{Lausn á \emph{HKIO}}
\author{Bergur Snorrason}
\date{\today}

\begin{document}

\frame{\titlepage}

\env{frame}
{
	\frametitle{HKIO}
	\env{itemize}
	{
		\item<1-> Gefnar eru $n$ heiltölur $a_1, \dots, a_n$.
		\item<2-> Finnið $j \leq k$ þannig að meðaltalið $\operatorname{avg}(a_j, a_{j + 1}, \dots, a_k)$ sé hámarkað.
		\item<3-> Gefið er að $1 \leq n \leq 10^5$.
	}
}

\env{frame}
{
	\frametitle{HKIO}
	\env{itemize}
	{
		\item<1-> Látum $m$ vera heiltölu þannig að $a_m = \max(a_1, \dots, a_n)$.
		\item<2-> Takið þá eftir að 
		\env{align*}
		{
				\operatorname{avg}(a_j, a_{j + 1}, \dots, a_k)
				&= \frac{a_j + a_{j + 1} + \dots + a_k}{k - j + 1}\\
				&\leq \frac{a_m + a_m + \dots + a_m}{k - j + 1}\\
				&= \frac{(k - j + 1) \cdot a_m}{k - j + 1}\\
				&= a_m.
		}
		\item<3-> Svo meðaltalið verður aldrei stærra en $a_m$.
		\item<4-> En einnig gildir að $\operatorname{avg}(a_m) = a_m$.
		\item<5-> Svo okkur nægir að finna stærstu töluna í listanum.
	}
}

\env{frame}
{
	\frametitle{HKIO}
	\code{code/hkio.c}
}

\env{frame}
{
}

\end{document}
