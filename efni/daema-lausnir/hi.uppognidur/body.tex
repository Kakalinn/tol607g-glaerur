\title{Lausn á \emph{Upp og niður}}
\author{Bergur Snorrason}
\date{\today}

\begin{document}

\frame{\titlepage}

\section{Dæmalýsing}
\env{frame}
{
	\env{itemize}
	{
		\item<1-> Gefnar eru $n \leq 10^5$ ólíkar heiltölur $a_1, \dots, a_n$, $1 \leq a_j \leq 10^9$.
		\item<2-> Finnið heiltölur $1 \leq i < j < k \leq n$ þannig að $x_i < x_k < x_j$.
	}
}

\section{Lausn}
\subsection{Smækkun talnanna}
\env{frame}
{
	\env{itemize}
	{
		\item<1-> Byrjum á að minnka $a_j$ þannig að $a_j \leq n$, án þess að breyta innbyrðis röðun talnanna.
		\item<2-> Í grófum dráttum breytum við minnstu tölunni í $0$, næst minnstu tölunni í $1$ og svo framvegis.
		\env{itemize}
		{
			\item<3-> Röðum $(a_i, i)$ eftir fyrra hnitinu.
			\item<4-> Látum nú fyrra hnit $i$-ta staksins í tvenndunum (eftir röðun) vera $i$.
			\item<5-> Röðum eftir seinna hnitinu.
			\item<6-> Látum nú $a_i$ vera fyrra hnit $i$-ta staksins af tvenndunum (eftir að raða tvisvar).
		}
		\item<7-> Með þessari aðferð getum við gert ráð fyrir að $0 \leq a_j \leq 10^5$.
		\selectcode{code/compress.py}{3}{7}
	}
}

\subsection{Lýsing}
\env{frame}
{
	\env{itemize}
	{
		\item<1-> Festum nú eitthvað $j$.
		\item<2-> Við getum gráðugt valið $i < j$ sem vísinn á minnsta stakið af fyrstu $j - 1$ tölunum.
		\item<3-> Við þurfum svo að geta athugað hvort til sé $k$ þannig að $x_i < x_k < x_j$.
		\item<4-> Við getum gert þetta með biltréi sem styður...
		\env{itemize}
		{
			\item<5-> ...punktuppfærsluna ,,bætum $k$ við $i$-ta stakið í trénu''.
			\item<6-> ...bilfyrirspurnina ,,hver er summan yfir bil $[i, j]$''.
		}
		\item<7-> Þetta biltré er eins og fyrsta dæmið í glærunum um biltré.
	}
}

\env{frame}
{
	\env{itemize}
	{
		\item<1-> Við byrjum á að láta allar tölurnar $a_i$ í biltréð (með \ilcode{update(a[i], 1)}).
		\item<2-> Við gefum okkur einnig breytu \ilcode{mn} sem er upphafstillt sem $\infty$.
		\item<3-> Við ítrum síðan í gegnum $j = 2, 3, \dots, n - 1$ og ...
		\env{itemize}
		{
			\item<4-> ...breytum \ilcode{mn} í $a_{j - 1}$ ef það er minna en \ilcode{mn}.
			\item<5-> ...fjarlægjum $a_j$ úr biltrénu (með \ilcode{update(a[j], -1)}).
			\item<6-> ...athugum hvort \ilcode{mn} er minna en $a_j$ og summan yfir bilið \ilcode{mn + 1} og $a_j - 1$ er stærra en núll.
							Ef svo er erum við komin með vísanna.
		}
		\item<7-> Við getum nú fundið $i$ og $k$ með því að leita línulega.
	}
}

\subsection{Tímaflækja}
\env{frame}
{
	\env{itemize}
	{
		\item<1-> Þegar við upphafstillum biltréð köllum við $n$ sinnum á \ilcode{update(...)}.
		\item<2-> Þetta tekur $\mathcal{O}(\onslide<3->{n \log n})$ tíma.
		\item<4-> Síðan ítrum við í gegnum öll stökin, nema tvö, og fyrir hvert stak köllum við á \ilcode{update(...)} og \ilcode{query(...)}.
		\item<5-> Þetta tekur $\mathcal{O}(\onslide<6->{n \log n})$ tíma.
		\item<7-> Svo heildar tímaflækjan er $\mathcal{O}(\onslide<8->{n \log n})$ tíma.
	}
}

\section{Þessi glæra er viljandi auð}
\env{frame}
{
}

\end{document}
