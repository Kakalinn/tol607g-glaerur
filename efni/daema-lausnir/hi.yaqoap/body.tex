\title{Lausn á \emph{Yet Another Query on Array Problem}}
\author{Bergur Snorrason}
\date{\today}

\begin{document}

\frame{\titlepage}

\section{Dæmalýsing}
\env{frame}
{
	\env{itemize}
	{
		\item<1-> Gefnar eru $n$ heiltölur $a_1, \dots, a_n$.
		\item<2-> Síðan eru gefnar eru $q$ fyrirspurnir $x_1, y_1, \dots, x_n, y_n$.
		\item<3-> Fyrirspurnin $x, y$ svarar til þess að breyta öllum tölunum í listanum sem eru jafnar $a_x$ í $a_y$.
		\item<4-> Fyrir hverja fyrirspurn skal prenta fjölda $a_x$ í listanum fyrir fyrirspurna og fjölda $a_y$ í listanum eftir fyrirspurnina.
	}
}

\section{Sýniinntak}
\env{frame}
{
	\env{itemize}
	{
		\item<1-> Sýniinntakið er
		\item<2->[] \code{code/sample.in}
		\item<3-> Listinn er þá, á hverju tímapunkti:
		\item<4->[] \ilcode{1 2 3 4 5 6}
		\item<5->[] \ilcode{1 4 3 4 5 6}
		\item<6->[] \ilcode{1 4 4 4 5 6}
		\item<7->[] \ilcode{1 4 4 4 5 5}
		\item<8->[] \ilcode{1 5 5 5 5 5}
	}
}

\section{Lausn}
\subsection{Lýsing}
\env{frame}
{
	\env{itemize}
	{
		\item<1-> Lausnin byggir á \onslide<2->{sammengisleit}.
		\item<3-> Fyrir fyrirspurnina $x, y$ prentum við \ilcode{size(x)} og \ilcode{size(x) + size(y)} ef $x$ og $y$ eru ekki í sama samhengisþætti.
		\item<4-> Annars prentum við \ilcode{size(x)} tvisvar.
		\item<5-> Síðan sameinum við $x$ og $y$ (með \ilcode{join(x, y)}).
		\item<6-> \bf{ATH:} Tölurnar í inntakinu eru ekki nauðsynlega ólíkar.
		\item<7-> Við þurfum að byrja á að sameina tölur sem eru eins.
		\item<8-> Ein leið til að gera þetta er að raða tvenndunum $(a_i, i)$ eftir fyrra hnitinu.
		\item<9-> Þá eru eins stök aðliggjandi.
		\item<10-> Síðan fæst upprunalegi listinn aftur með því að raða eftir seinna hnitinu.
		\item<11-> Einnig má nota \ilcode{map} til að geyma minnsta vísinn á tiltekna tölu, og svo alltaf sameina við þann vísi.
	}
}

\subsection{Tímaflækja}
\env{frame}
{
	\env{itemize}
	{
		\item<1-> Röðun tekur $\mathcal{O}(\onslide<2->{n \log n})$ tíma.
		\item<3-> Sammengisleitin tekur styttri tíma.
		\item<4-> Heildartímaflækjan er því $\mathcal{O}(\onslide<5->{n \log n})$ tíma.
	}
}

\section{Þessi glæra er viljandi auð}
\env{frame}
{
}

\end{document}
