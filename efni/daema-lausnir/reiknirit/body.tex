\title{Lausn á \emph{Reiknirit}}
\author{Bergur Snorrason}
\date{\today}

\begin{document}

\frame{\titlepage}

\env{frame}
{
	\env{itemize}
	{
		\item<1-> Skoðum aftur skiladæmið \emph{Reiknirit}.
		\item<2-> Fyrsta lína inntaksins inniheldur heiltölu $n$.
		\item<3-> Síðan koma $n$ heiltölur $a_1, a_2, \dots, a_n$.
		\item<4-> Gerum ráð fyrir að við séum með forrit sem gerir eftirfarandi:
		\env{itemize}
		{
			\item<5-> Prentar tölurnar.
			\item<6-> Fjarlægir öll eintök af algengustu tölunni í listan.
			\item<7-> Endurtekur skrefin að ofan þar til listinn er tómur.
		}
		\item<8-> Dæmið snýst um að finna hversu margar tölur eru prentaðar í heildina.
	}
}

\env{frame}
{
	\env{itemize}
	{
		\item<1-> Síðast leystum við þetta dæmi með því að útfæra forritið sem er lýst í dæminu.
		\item<2-> Við komumst þó að því að sú lausn var $\mathcal{O}(n^2)$ sem reyndist of hæg.
		\item<3-> Tökum eftir að við getum notað svipaða hugmynd og í hægu útfærslunni til að telja hversu oft hver tala kemur fyrir.
		\item<4-> Tökum einnig eftir að hvert eintak af algengustu tölunni er prentað einu sinni,
					hvert eintak af næst algengustu tölunni er prentað tvisvar
					og svo framvegis.
		\item<5-> Einnig skiptir talan sjálf ekki máli, heldur eingöngu hversu oft hún kemur fyrir.
	}
}

\env{frame}
{
	\env{itemize}
	{
		\item<1-> Látum því $h_1 \geq h_2 \geq \dots \geq h_k$ þannig að algengast talan kemur $h_1$ sinni fyrir,
					næst algengasta talan kemur $h_2$ sinnum fyrir
					og svo framvegis.
		\item<2-> Svarið er því
		\[
			\sum_{i = 1}^k i \cdot h_i.
		\]
		\item<3-> Við þurfum þó að passa okkur aðeins.
		\item<4-> Ef $h_1 = h_2 = \dots h_k = 1$ (þá er einnig $k = n$) fæst að svarið er
		\[
			\sum_{i = 1}^k i \cdot h_i
			=
			\sum_{i = 1}^k i
			=
			\frac{n \cdot (n + 1)}{2}.
		\]
		\item<5-> Svo, þar sem $n$ getur verið allt að $10^6$, getur svarið okkar orðið of stórt fyrir \texttt{int}.
		\item<6-> Við þurfum því að nota \texttt{long long}.
	}
}

\env{frame}
{
	\code{code/reiknirit.c}
}

\env{frame}
{
	\env{itemize}
	{
		\item<1-> Sjáum við byrjum á að raða, sem er $\mathcal{O}($\onslide<2->{$n \log n$}$)$.
		\item<3-> Við teljum síðan hvað hver tala kemur oft fyrir, sem er $\mathcal{O}($\onslide<4->{$\ n\ $}$)$.
		\item<5-> Síðan röðum við aftur.
		\item<6-> Að lokum reiknum við summuna í $\mathcal{O}($\onslide<7->{$\ n\ $}$)$.
		\item<8-> Tímaflækjan er því í heildina $\mathcal{O}($\onslide<9->{$n \log n$}$)$.
		\item<10-> Nú er $10^{-8} \cdot 10^6 \cdot \log 10^6 \sim 0,2$, svo þessi lausn er nógu hröð.
	}
}


\env{frame}
{
}

\end{document}
