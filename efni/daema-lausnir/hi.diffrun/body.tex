\title{Lausn á \emph{Diffrun}}
\author{Bergur Snorrason}
\date{\today}

\begin{document}

\frame{\titlepage}

\section{Dæmalýsing}
\env{frame}
{
	\env{itemize}
	{
		\item<1-> Gefin er margliða $p$ í breytunum $a, b, \dots, z$, ásamt tiltekinni breytu $\alpha$.
		\item<2-> Prenta á afleiðuna $\partial p/\partial \alpha$, ásamt ýmsum reglum.
		\item<3-> Helstu reglurnar eru að það á ekki að prenta út fastann við lið ef hann er $1$ eða $-1$
					og það á ekki að prenta út liðinn ef fastinn er $0$.
	}
}

\section{Lausn}
\env{frame}
{
	\env{itemize}
	{
		\item<1-> Í dæminu er lýst hvernig á að framkvæma diffrun.
		\item<2-> Helsti vandinn í dæminu er að lesa inn margliðuna og að prenta út margliðuna á réttu sniði.
		\item<3-> Dæmi um inntök sem geta fellt mann:
		\item<4->[] \code{code/sample-01.in}
		\item<5->[] \code{code/sample-02.in}
		\item<6->[] \code{code/sample-03.in}
		\item<7->[] \code{code/sample-04.in}
		\item<8-> Til að raða nægir að túlka liðinn $ca^{e_a}b^{e_b} \dots z^{e_z}$ sem $n$-dina $(e_a, e_b, \dots, e_z, c)$ og raða í orðabókaröð.
		\item<9-> Eftir röðun er líka þægilegt að taka saman þá liði sem á að taka saman.
	}
}

\section{Þessi glæra er viljandi auð}
\env{frame}
{
}

\end{document}
