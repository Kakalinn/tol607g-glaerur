\title{Lausn á \emph{Bruni íslenskra fræða}}
\author{Bergur Snorrason}
\date{\today}

\begin{document}

\frame{\titlepage}

\section{Dæmalýsing}
\env{frame}
{
	\env{itemize}
	{
		\item<1-> Þú hefur eld sem getur logað í $t$ sekúndur.
		\item<2-> Síðan hefur þú $n$ bækur, hverri lýst með tveimur heiltölum.
		\item<3-> Bók lýst með tölunum $t_i$ og $f_i$ lengir líftíma eldsins um $f_i$ sekúndur, en það tekur $t_i$ sekúndur að sækja bókina.
		\item<4-> Ef líftími eldsins er $t_0$ sekúndur getur þú notað bókina til að lengja líftímann ef $t_i \leq t_0$.
		\item<5-> Eftir að nota þá bók verður líftími eldsins $t_0 + f_i - t_i$.
	}
}

\section{Lausn}
\subsection{Gráugi helmingurinn}
\env{frame}
{
	\env{itemize}
	{
		\item<1-> Tökum eftir að það borgar sig alltaf að taka bækur þannig að $t_i \leq f_i$, ef það er nægur tími til að sækja þær.
		\item<2-> Skiptum því bókunum í tvennt.
		\item<3-> Látum $j_1 < j_2 < \dots < j_m$ þannig að $t_i \leq f_i$ þá og því aðeins að $i = j_k$ fyrir eitthvert $k$.
		\item<4-> Röðum svo tvenndunum $(t_{j_1}, f_{j_1}), \dots, (t_{j_m}, f_{j_m})$ í vaxandi röð eftir fyrra hnitinu.
		\item<5-> Þá getum við gráðugt gengið á listann og tekið þær bækur sem við höfum tíma til að sækja.
		\item<6-> Eftir þetta skoðum við bækurnar sem uppfylla að $t_i > f_i$.
	}
}

\subsection{Kvik bestunar helmingurinn}
\env{frame}
{
	\env{itemize}
	{
		\item<1-> Við getum núna ímyndað okkur að $t_i > f_i$ gildi alltaf.
		\item<2-> Röðum tvenndunum $(t_1, f_1), \dots, (t_n, f_n)$ í vaxandi röð eftir seinna hnitinu.
		\item<3-> Þá getum við gengið á tvenndirnar í öfugri röð og prófað bæði að taka tiltekna bók og ekki.
		\item<4-> Við fáum þá rakningarformúluna
\[
	f(x, y)
	=
	\left \{
	\begin{array}{c c}
		-\infty, & y < 0,\\
		0, & x = 0,\\
		f(x - 1, y), & t_x > y\\
		\max(f(x - 1, y), &\\
			f_x + f(x - 1, y + f_x - t_x)), & \text{annars,}
	\end{array}
	\right .
\]
					þar sem $f(x, y)$ segir okkur hversu mikið við getum lengt líftíma eldsins ef hann hefur líftíma $y$ og 
						við getum notað bækur $1, 2, \dots, x$ (eftir röðun).
		\item<5-> Við getum svo reiknað upp úr þessum venslum með \onslide<6->{kvikri bestun}.
		\item<7-> Svarið er þá $t + f(n - 1, t)$ (svo þarf að bæta við því sem við fengum í gráðuga hlutanum).
	}
}

\section{Þessi glæra er viljandi auð}
\env{frame}
{
}

\end{document}
