\title{Lausn á \emph{Planetaris}}
\author{Bergur Snorrason}
\date{\today}

\begin{document}

\frame{\titlepage}

\env{frame}
{
	\frametitle{Planetaris}
	\env{itemize}
	{
		\item<1-> Atli og Finnur eru að spila tölvuleik sem snýst um að fanga sólkerfi.
		\item<2-> Í leiknum eru $1 \leq n \leq 10^5$ sólkerfi.
		\item<3-> Atli og Finnur senda einhvern fjölda skipa sinna á hvert sólkerfi.
		\item<4-> Atli fangar tiltekið sólkerfi ef hann sendir strangt fleiri skip þangað.
		\item<5-> Atli hefur $a$ skip og veit að Finnur mun senda $e_i$ skip á $i$-ta skólkerfið.
		\item<6-> Hver er mesti fjöldi sólkerfa sem Atli getur fangað?
	}
}

\env{frame}
{
	\frametitle{Planetaris}
	\env{itemize}
	{
		\item<1-> Við græðum jafn mikið að fanga hvert sólkerfi, svo það er best að fanga þau sólkerfi sem Finnur sendir fá skip á.
		\item<2-> Við föngum því einfaldlega sólkerfin í röð, byrjum á því sem Finnur sendir fæst skip á,
					svo næst það sem hann sendir næst fæst skip á, og svo framvegis.
		\item<3-> Þegar við föngum $i$-ta sólkerfið verðum við að passa að senda $e_i + 1$ skip, til að það verði ekki jafntefli.
		\item<4-> Við verðum líka að passa að hætta að fanga sólkerfi þegar við höfum ekki nóg af skipum.
	}
}

\env{frame}
{
	\frametitle{Planetaris}
	\selectcode{code/planetaris.c}{10}{23}
}

\env{frame}
{
	\frametitle{Planetaris}
	\env{itemize}
	{
		\item<1-> Tímaflækjan á þessari lausn er $\mathcal{O}($\onslide<2->{$n \log n$}$)$ \onslide<2->{sökum röðunar.}
		\item<3-> Takið eftir að það er mjög auðvelt að gera litlar villur sem gera lausnin ranga.
		\item<4-> Til dæmis fær eftirfarandi lausn rétt í sýnidæmum en rangt á fyrsta huldudæminu.
	}
}

\env{frame}
{
	\frametitle{Planetaris, röng lausn}
	\selectcode{code/planetaris1.c}{10}{23}
}

\env{frame}
{
	\frametitle{Planetaris, rétt lausn aftur, til samanburðar}
	\selectcode{code/planetaris.c}{10}{23}
}

\env{frame}
{
}

\end{document}
