\title{Grannröðun}
\author{Bergur Snorrason}
\date{\today}

\begin{document}

\frame{\titlepage}

\section{Skilgreining}
\env{frame}
{
    \env{itemize}
    {
        \item<1-> Gerum ráð fyrir að við séum með stefnt net.
        \item<2-> Við viljum getað raðað hnútunum þannig að ef það er leggur frá hnútnum $u$ til hnútsins $v$ í netinu þá sé $u$ á undan $v$ í röðuninni.
        \item<3-> Þetta má gera þá og því aðeins að netið sé órásað.
        \item<4-> Við köllum slíka röðun \emph{grannröðun} (e. \emph{topological sort}).
        \item<5-> Takið þó eftir að þessi röðun þarf ekki að vera ótvírætt ákvörðuð.
    }
}

\section{Útskýring á lausn}
\env{frame}
{
    \env{itemize}
    {
        \item<1-> Fjöldi leggja afgerðinni $(u, v)$, fyrir fast $v$, kallast \emph{innstig} hnútsins $v$.
        \item<2-> Tökum eftir að stefnt órásað net hefur alltaf hnút með innstig núll.
        \item<3-> Einnig gildir að ef við fjarlægjum hnút úr stefndu órásuðu neti þá er netið ennþá órásað.
        \item<4-> Ef hnútur hefur innstig núll má hann alltaf fara fremst í grannröðunina.
        \item<5-> Við finnum því grannröðun með því að:
        \env{itemize}
        {
            \item<6-> Finna hnút með innstig núll.
            \item<7-> Setja hann aftast í grannröðunina.
            \item<8-> Fjarlægja hann úr netinu.
            \item<9-> Endurtökum þangað til enginn hnútur er eftir.
        }
    }
}

\section{Sýnidæmi}
\env{frame}
{
    \env{center}
    {
        \env{tikzpicture}
        {
            \onslide<all:1-2> { \node[draw, circle, thick] (1) at (-1, -1) {$1$}; }
            \onslide<all:1-5> { \node[draw, circle, thick] (2) at (1, 1) {$2$}; }
            \onslide<all:1-7> { \node[draw, circle, thick] (3) at (-1, 1) {$3$}; }
            \onslide<all:1-8> { \node[draw, circle, thick] (4) at (0, 1) {$4$}; }
            \onslide<all:1-4> { \node[draw, circle, thick] (5) at (0, 0) {$5$}; }
            \onslide<all:1-3> { \node[draw, circle, thick] (6) at (0, -1) {$6$}; }
            \onslide<all:1-1> { \node[draw, circle, thick] (7) at (1, -1) {$7$}; }
            \onslide<all:1-6> { \node[draw, circle, thick] (8) at (-1, 0) {$8$}; }


            \onslide<all:1-6> { \path[draw, ->] (8) -- (3); }
            \onslide<all:1-4> { \path[draw, ->] (5) -- (4); }
            \onslide<all:1-6> { \path[draw, ->] (8) -- (4); }
            \onslide<all:1-1> { \path[draw, ->] (7) -- (2); }
            \onslide<all:1-4> { \path[draw, ->] (5) -- (2); }
            \onslide<all:1-3> { \path[draw, ->] (6) -- (5); }
            \onslide<all:1-2> { \path[draw, ->] (1) -- (8); }
            \onslide<all:1-3> { \path[draw, ->] (6) -- (8); }
            \onslide<all:1-1> { \path[draw, ->] (7) -- (8); }
        }
    }
    \env{center}
    {
        \onslide<all:2-> { $7$ }
        \onslide<all:3-> { $1$ }
        \onslide<all:4-> { $6$ }
        \onslide<all:5-> { $5$ }
        \onslide<all:6-> { $2$ }
        \onslide<all:7-> { $8$ }
        \onslide<all:8-> { $3$ }
        \onslide<all:9-> { $4$ }
    }
}

\section{Lausn}
\env{frame}
{
    \env{itemize}
    {
        \item<1-> Þegar við útfærum þetta byrjum við á að setja alla hnúta með innstig núll í biðröð.
        \item<2-> Tökum svo hnút úr biðröðinni og bætum við þeim hnútum sem hafa bara legg í sig frá þeim hnút.
        \item<3-> Við fáum einnig að grannröðunin er ótvírætt ákvörðuð þá og því aðeins að biðröðin innihaldi aldrei fleiri en eitt stak.
    }
}

\section{Útfærsla}
\env{frame}
{
    \selectcode{code/grannrodun.cpp}{6}{23}
}

\section{Tímaflækja}
\env{frame}
{
    \env{itemize}
    {
        \item<1-> Við heimsækjum hvern hnút að mestu einu sinni og ferðumst eftir hverjum legg einu sinni.
        \item<2-> Svo forritið er $\mathcal{O}($\onslide<3->{$E + V$}$)$.
    }
}

\section{Dæmi um hagnýtingu}
\env{frame}
{
    \env{itemize}
    {
        \item<1-> Gerum ráð fyrir að þú sért að forrita verkefni.
        \item<2-> Verkefnið samanstendur af $n$ skrám.
        \item<3-> Hver skrá getur verið háð einhverjum öðrum skrám.
        \item<4-> Passa þarf að þegar allt verkefnið er þýtt þá þarf skrá $u$ að vera þýdd á undan skrá $v$ ef $v$ er háð $u$.
        \item<5-> Við getum þá búið til net, þar sem skrárnar eru hnútar og hæði er táknað með stefndum leggjum.
        \item<6-> Þetta verður þá stefnt net.
        \item<7-> Eðlilegt er að gera ráð fyrir að það sé órásað.
        \item<8-> Grannröðun gefur okkur þá röð til að þýða skrárnar.
    }
}

\section{Þessi glæra er viljandi auð}
\env{frame}
{
}

\end{document}
