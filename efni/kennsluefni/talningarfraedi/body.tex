\title{Talningarfræði}
\author{Bergur Snorrason}
\date{\today}

\begin{document}

\frame{\titlepage}

\section{Talningarfræði}
\subsection{Yfirlit}
\env{frame}
{
    \env{itemize}
    {
        \item<1-> Talningarfræði er sá angi strjállar stærðfræði sem fjallar um talningar á einhverjum fyrirbærum.
        \item<2-> Þegar við fjölluðum um tæmandi leit kom fram að fjöldi hlutmengja í $n$ staka mengi er $2^n$.
        \item<3-> Einnig kom fram að fjöldi umraðana á menginu $\{1, 2, ..., n\}$ er $n!$.
        \item<4-> Bæði eru þetta mikilvægar niðurstöður úr talningarfræði.
        \item<5-> Skerpum aðeins á grunnatriðum.
    }
}

\subsection{Grunnatriði}
\env{frame}
{
    \env{itemize}
    {
        \item<1-> Ef við erum með $n$ hluti af einni gerð og $m$ hluti af annari gerð þá getum við valið einn hlut af hvorri gerð á $n \cdot m$ vegu.
        \item<2-> Við þurfum í raun ekki meira en þetta.
        \item<3-> Við notuðum þessa reglu til að sanna niðurstöðurnar á glærunni á undan.
        \item<4-> Í talningarfræði er oft þægilegt að hugsa um endanleg mengi og fjöldatölur þeirra.
        \item<5-> Ef $A$ er mengi þá táknar $|A|$ fjölda staka í $A$.
        \item<6-> Við getum þá umorðað efsta punktinn sem: Ef $A$ og $B$ eru mengi þá er $|A \times B| = |A| \cdot |B|$.
    }
}

\subsection{Fjöldi hlutmengja}
\env{frame}
{
    \env{itemize}
    {
        \item<1-> Við vitum að það eru $2^n$ hlutmengi í $n$ staka mengi, en hvað eru mörg hlutmengi af stærð $k$?
        \item<2-> Þegar við veljum fyrsta stakið höfum við um $n$ stök að velja, síðan $n - 1$ stak og svo framvegis.
        \item<3-> Við fáum því $n \cdot (n - 1) \cdot ... \cdot (n - k + 1) = \dfrac{n!}{k!}$ mengi.
        \item<4-> Hvert mengi er þó talið $(n - k)!$ sinnum, svo loka talan er
        \[
            \frac{n!}{(n - k)!k!}.
        \]
        \item<5-> Þessi tala er táknuð með ${n \choose k}$.
    }
}

\subsection{Stærðir sammengja}
\env{frame}
{
    \env{itemize}
    {
        \item<1-> Tökum eftir að $|A \cup B| \neq |A| + |B|$ því það gætu verið stök í bæði $A$ og $B$.
        \item<2-> Ef svo er getum við einfaldlega fjarlægt þau stök sem eru tvítalin, og fáum
        \[
            |A \cup B| = |A| + |B| - |A \cap B|.
        \]
        \item<3-> Gerum nú ráð fyrir að
        \[
            |A_1 \cup ... \cup A_n|
            =
            \sum_{\substack{J \subset \{1, ..., n\} \\ J \neq \emptyset}} (-1)^{|J| - 1} \left | \bigcap_{j \in J} A_j \right |.
        \]
        \item<4-> Þá fæst...
    }
}

\subsection{Þrepunarglæra}
\env{frame}
{
    \scriptsize
    \env{align*}
    {
        |A_1 \cup ... \cup A_{n + 1}|
        &= |A_1 \cup ... \cup A_n| + |A_{n + 1}| - |(A_1 \cup ... \cup A_n) \cap A_{n + 1}|\\
        &= \sum_{\substack{J \subset \{1, ..., n\} \\ J \neq \emptyset}} (-1)^{|J| - 1} \left | \bigcap_{j \in J} A_j \right | + 
        |A_{n + 1}| -
        |(A_1 \cap A_{n + 1}) \cup ... \cup (A_n \cap A_{n + 1})|\\
        &= \sum_{\substack{J \subset \{1, ..., n\} \\ J \neq \emptyset}} (-1)^{|J| - 1} \left | \bigcap_{j \in J} A_j \right | + 
        |A_{n + 1}| +
        \sum_{\substack{J \subset \{1, ..., n\} \\ J \neq \emptyset}} (-1)^{|J|} \left | \bigcap_{j \in J} (A_j \cap A_{n + 1}) \right |\\
        &= \sum_{\substack{J \subset \{1, ..., n\} \\ J \neq \emptyset}} (-1)^{|J| - 1} \left | \bigcap_{j \in J} A_j \right | + 
        |A_{n + 1}| +
        \sum_{\substack{J \subset \{1, ..., n\} \\ J \neq \emptyset}} (-1)^{|J|} \left | \bigcap_{j \in J \cup \{n + 1\}} A_j \right |\\
        &= \sum_{\substack{J \subset \{1, ..., n\} \\ J \neq \emptyset}} (-1)^{|J| - 1} \left | \bigcap_{j \in J} A_j \right | + 
        \sum_{\substack{J \subset \{1, ..., n\}}} (-1)^{|J|} \left | \bigcap_{j \in J \cup \{n + 1\}} A_j \right |\\
        &= \sum_{\substack{J \subset \{1, ..., n + 1\} \\ J \neq \emptyset \\ n + 1 \not \in J}} (-1)^{|J| - 1} \left | \bigcap_{j \in J} A_j \right |
        + \sum_{\substack{J \subset \{1, ..., n + 1\} \\ n + 1 \in J}} (-1)^{|J| - 1} \left | \bigcap_{j \in J} A_j \right |\\
        &= \sum_{\substack{J \subset \{1, ..., n, n + 1\} \\ J \neq \emptyset}} (-1)^{|J| - 1} \left | \bigcap_{j \in J} A_j \right |
    }
}

\subsection{Lögmál um fjöldatölu sammengja}
\env{frame}
{
    \env{itemize}
    {
        \item<1-> Við höfum því sýnt með þrepun að
        \[
            |A_1 \cup ... \cup A_n|
            =
            \sum_{\substack{J \subset \{1, ..., n\} \\ J \neq \emptyset}} (-1)^{|J| - 1} \left | \bigcap_{j \in J} A_j \right |.
        \]
        \item<2-> Vera má að þessi jafna komi spánkst fyrir sjónir en í raun lýsir hún hvernig við fjarlægjum stök sem eru tvítekin,
                    bætum aftur við stökum sem eru þrítekin,
                    fjarlægjum aftur stök sem eru fjórtekin
                    og svo framvegis.
        \item<3-> Þessi jafna er kölluð \emph{lögmálið um fjöldatölu sammengja} (e. \emph{Inclusion-Exclusion principle}).
    }
}

\section{Talningardæmi: Umraðanir á fastapunkt}
\subsection{Lýsing}
\env{frame}
{
    \env{itemize}
    {
        \item<1-> Sjáum nú hagnýtingu á þessari jöfnu.
        \item<2-> Munum að gagntæk vörpun $\sigma \colon \{1, 2, ... n\} \rightarrow \{1, 2, ..., n\}$ kallast \emph{umröðun} (e. \emph{permutation}).
        \item<3-> Ef við festum $n$ þá höfum við sýnt að til séu $n!$ umraðanir.
        \item<4-> Næst segjum við að $k \in \{1, 2, ..., n\}$ sé \emph{fastapunktur $\sigma$} (e. \emph{fixed point of $\sigma$}) ef $\sigma(k) = k$.
        \item<5-> Með öðrum orðum hefur umröðunin ekki áhrif á þennan punkt.
        \item<6-> Hversu margar umraðanir hafa engan fastapunkt?
    }
}

\defverbatim{\derA}
{ \begin{verbatim}
1 2 3 4         2 1 4 3         3 1 2 4         4 1 2 3

1 2 4 3         2 1 3 4         3 1 4 2         4 1 3 2

1 3 2 4         2 3 1 4         3 2 1 4         4 2 1 3

1 3 4 2         2 3 4 1         3 2 4 1         4 2 3 1

1 4 2 3         2 4 1 3         3 4 1 2         4 3 1 2

1 4 3 2         2 4 3 1         3 4 2 1         4 3 2 1

\end{verbatim}}
\defverbatim{\derB}
{ \begin{verbatim}
1 2 3 4         2 1 4 3         3 1 2 4         4 1 2 3
^ ^ ^ ^                               ^                
1 2 4 3         2 1 3 4         3 1 4 2         4 1 3 2
^ ^                 ^ ^                             ^  
1 3 2 4         2 3 1 4         3 2 1 4         4 2 1 3
^     ^               ^           ^   ^           ^    
1 3 4 2         2 3 4 1         3 2 4 1         4 2 3 1
^                                 ^               ^ ^  
1 4 2 3         2 4 1 3         3 4 1 2         4 3 1 2
^                                                      
1 4 3 2         2 4 3 1         3 4 2 1         4 3 2 1
^   ^               ^                                  
\end{verbatim}}
\defverbatim{\derC}
{ \begin{verbatim}
                2 1 4 3                         4 1 2 3

                                3 1 4 2                



                2 3 4 1                                

                2 4 1 3         3 4 1 2         4 3 1 2

                                3 4 2 1         4 3 2 1

\end{verbatim}}
\subsection{Sýnidæmi}
\env{frame}
{
    \env{itemize}
    {
        \item<1-> Skoðum fyrst þegar $n = 4$:
    }
    \only<all:1>{\derA}
    \only<all:2>{\derB}
    \only<all:3>{\derC}
}

\subsection{Lausn}
\env{frame}
{
    \env{itemize}
    {
        \item<1-> Við munum frekar telja hversu margar umraðanir hafa fastapunkt.
        \item<2-> Þetta er algengt að gera í talningarfræði.
        \item<3-> Látum nú $A_j$ tákna mengi þeirra umraðana þar sem $j$ er fastapunktur.
        \item<4-> Þá er $\bigcap_{j \in J} A_j$ mengi þeirra umraðana þar sem allir punktar $J$ eru fastapunktar.
        \item<5-> Ef við festum $k$ punkta í umröðuninni getum við raðað restinni á $(n - k)!$ marga vegu.
        \item<6-> Svo $\left | \bigcap_{j \in J} A \right | = (n - |J|)!$.
        \item<7-> Takið eftir að seinni stærðin er bara háð fjölda staka í menginu $J$.
        \item<8-> Við vitum einnig að fjöldi hlutmengja $\{1, ..., n\}$ með $k$ stök er ${n \choose k}$.
    }
}

\subsection{Útleiðing á formúlu}
\env{frame}
{
    \env{itemize}
    {
        \item<1-> Við fáum loks að fjöldi umraðana með einhvern fastapunkt er
    }
    \env{align*}
    {
        |A_1 \cup ... \cup A_n|
        &= \sum_{\substack{J \subset \{1, ..., n\} \\ J \neq \emptyset}} (-1)^{|J| - 1} \left | \bigcap_{j \in J} A_j \right |\\
        &= \sum_{j = 1}^n (-1)^{j + 1} {n \choose j} (n - j)!\\
        &= \sum_{j = 1}^n (-1)^{j + 1} \frac{n!}{(n - j)!j!} (n - j)!\\
        &= n!\sum_{j = 1}^n \frac{(-1)^{j + 1}}{j!}.
    }
}

\subsection{Niðurlag}
\env{frame}
{
    \env{itemize}
    {
        \item<1-> Fjöldi umraðana með engan fastapunkt er því
    }
    \env{align*}
    {
        n! - n! \sum_{j = 1}^n \frac{(-1)^{j + 1}}{j!} 
        &= n! \left (1 + \sum_{j = 1}^n \frac{(-1)^j}{j!} \right )\\
        &= n! \left (\frac{(-1)^0}{0!} + \sum_{j = 1}^n \frac{(-1)^j}{j!} \right )\\
        &= n! \sum_{j = 0}^n \frac{(-1)^j}{j!}.
    }
    \env{itemize}
    {
        \item<2-> Algengt er að kalla þessa tölu $!n$.
        \item<3-> Takið eftir að $!n/n!$ er $n$-ta hlutsumma veldaraðar $e^x$, fyrir $x = -1$.
        \item<4-> Svo $!n/n!$ er að stefna á $e^{-1}$, þegar $n$ stefnir á $\infty$.
    }
}

\section{Dæmi: Hvernig reiknum við fjölda hlutmengja af gefinni stærð}
\subsection{Yfirlit og lausn}
\env{frame}
{
    \env{itemize}
    {
        \item<1-> Oft er hentugt að geta reiknað
        \[
            {n \choose k} \mod m.
        \]
        fyrir jákvæðar heiltölur $k \leq n < m$.
        \item<2-> Munið að
        \[
            {n \choose k} = \frac{n!}{(n - k)!k!}.
        \]
        \item<3-> Ein leið til að gera þetta er að finna fyrst margföldunarandhverfur $(n - k)!$ og $k!$ með tilliti til $m$.
        \item<4-> Við reiknum svo $n! \cdot ((n - k)!)^{-1} \cdot (k!)^{-1} \mod m$.
        \item<5-> Hér þarf að passa að margföldunarandhverfan sé til.
        \item<6-> Helst þarf $m$ að vera frumtala.
    }
}

\subsection{Útfærsla}
\env{frame}
{
    \selectcode{code/nck-mulinv.c}{26}{37}
}

\subsection{Tímaflækjur}
\env{frame}
{
    \env{itemize}
    {
        \item<1-> Ef við erum með fast $n$ og $m$ og viljum reikna fyrir $q$ mismunandi gildi á $k$ þá er tímaflækjan á þessari að ferð
                    $\mathcal{O}($\onslide<2->{$n \log m + q$}$)$.
    }
}

\subsection{Lausn með kvikri bestun}
\env{frame}
{
    \env{itemize}
    {
        \item<1-> Önnur leið til að reikna ${n \choose k} \mod m$ byggir á kvikri bestun.
        \item<2-> Sjáum fyrst að ef $n > 1$ og $1 < k < n$ þá
    }
    \onslide<3->
    {
        \env{align*}
        {
            {n - 1 \choose k - 1} + {n - 1 \choose k}
            &= \frac{(n - 1)!}{(n - k)!(k - 1)!} + \frac{(n - 1)!}{(n - k - 1)!k!}\\
            &= \frac{k(n - 1)!}{(n - k)!k!} + \frac{(n - k)(n - 1)!}{(n - k)!k!}\\
            &= \frac{k(n - 1)! + n(n - 1)! - k(n - 1)!}{(n - k)!k!}\\
            &= \frac{n(n - 1)!}{(n - k)!k!}\\
            &= \frac{n!}{(n - k)!k!}\\
            &= {n \choose k}.
        }
    }
}

\subsection{Rakningarformúla}
\env{frame}
{
    \env{itemize}
    {
        \item<1-> Látum því
        \[
            f(n, k) = \left \{
            \env{array}
            { {l l}
                0, & \text{ef $n < 1$}\\
                0, & \text{ef $k < 0$}\\
                0, & \text{ef $k > n$}\\
                1, & \text{ef $k = 0$ eða $k = n$}\\
                f(n - 1, k - 1) + f(n - 1, k) & \text{annars.}
            }
            \right .
        \]
        \item<2-> Við höfum svo að $f(n, k) = {n \choose k}$.
        \item<3-> Útfærslan notar ofansækna kvika bestun.
    }
}

\subsection{Útfærsla}
\env{frame}
{
    \selectcode{code/nck-dp.c}{5}{12}
}

\subsection{Tímaflækjur}
\env{frame}
{
    \env{itemize}
    {
        \item<1-> Sjáum að það eru $n^2$ stöður.
        \item<2-> Við reiknum hverja stöðu í $\mathcal{O}($\onslide<3->{$\,1\,$}$)$ tíma, svo við getum svarað $q$ fyrirspurnum í
                    $\mathcal{O}($\onslide<4->{$n^2 + q$}$)$ tíma.
        \item<5-> Það sem forritið okkar er í rauninni að gera er að reikna gildin í þríhyrningin Pascals.
        \item<6-> Þekkt er að $k$-ta talan í $n$-tu línu þríhyrnings Pascals er ${n \choose k}$.
    }
}

\defverbatim{\pascal}
{ \begin{verbatim}
                          1
                        1   1
                      1   2   1
                    1   3   3   1
                  1   4   6   4   1
                1   5  10  10   5   1
              1   6  15  20  15   6   1
            1   7  21  35  35  21   7   1
          1   8  28  56  70  56  28   8   1





\end{verbatim}}
\subsection{Mynd af Þríhyrning Pascals}
\env{frame}
{
    \frametitle{Þríhyrningur Pascals}
    \pascal
}

\section{Dæmi: Fjöldi umhverfinga í umröðun}
\subsection{Yfirlit}
\env{frame}
{
    \frametitle{Fjöldi umhverfinga í umröðun}
    \env{itemize}
    {
        \item<1-> Látum $\sigma$ vera umröðun á $\{1, ..., n\}$.
        \item<2-> Þá kallast heiltölupar $(i, j)$ þannig að $1 \leq i < j \leq n$ og $\sigma(i) > \sigma(j)$,
            \emph{umhverfing} (e. \emph{inversion}) í $\sigma$.
        \item<3-> Látum \ilcode{a} tákna $n$ staka lista þannig að $j$-ta stak listans sé $\sigma(j)$.
        \item<4-> Svona táknum við iðulega umraðanir þegar við útfærum þær í tölvu.
        \item<5-> Gerum nú ráð fyrir að eina leiðin okkar til að breyta \ilcode{a} er að skipta á aðlægum stökum.
        \item<6-> Hvað tekur það minnst margar aðgerðir að raða listanum?
        \item<7-> Það vill svo til að fjöldi umhverfinga í umröðununni er einmitt fjöldi aðgerða sem þarf til að raða listanum.
        \item<8-> Við getum notað okkur þetta til að finna fjölda umhverfinga.
    }
}

\subsection{Lausn}
\env{frame}
{
    \env{itemize}
    {
        \item<1-> Ein leið til að gera þetta er að færa fyrst minnsta stakið fremst, síðan næst minnsta stakið næst fremst og svo framvegis.
        \item<2-> Þetta tekur $\mathcal{O}($\onslide<3->{$n^2$}$)$ tíma.
        \item<4-> Þetta er því of hægt ef $n > 10^4$.
        \item<5-> Við getum þó notað biltré til að gera þetta hraðara.
        \item<6-> Tökum eftir að þegar við höfum sett stak á sinn stað getum við hætt að hugsa um það.
        \item<7-> Við þurfum í raun að geta sagt til um hversu mörg stök í listanum eru fyrir framan það stak sem við viljum færa
                (við teljum ekki þau stök sem eru komin á sinn stað).
        \item<8-> Við byrjum því með $1$ í hverju staki í biltrénu okkar.
        \item<9-> Þegar við höfum fært stak á sinn stað setjum við tilheyrandi gildi sem $0$.
        \item<10-> Summa fyrstu $j - 1$ stakanna í trénu er því fjöldi staka sem $j$-ta stakið í listanum þarf að sipta á til að komast á sinn stað.
    }
}
\defverbatim{\slowinva}
{ \begin{verbatim}
                4 1 6 5 7 3 2

                Svar:  0
\end{verbatim}}
\defverbatim{\slowinvb}
{ \begin{verbatim}
                4 1 6 5 7 3 2
                  ^
                Svar:  0
\end{verbatim}}
\defverbatim{\slowinvc}
{ \begin{verbatim}
                1 4 6 5 7 3 2
                ^
                Svar:  1
\end{verbatim}}
\defverbatim{\slowinvd}
{ \begin{verbatim}
                1 4 6 5 7 3 2
                            ^
                Svar:  1
\end{verbatim}}
\defverbatim{\slowinve}
{ \begin{verbatim}
                1 4 6 5 7 2 3
                          ^
                Svar:  2
\end{verbatim}}
\defverbatim{\slowinvf}
{ \begin{verbatim}
                1 4 6 5 2 7 3
                        ^
                Svar:  3
\end{verbatim}}
\defverbatim{\slowinvg}
{ \begin{verbatim}
                1 4 6 2 5 7 3
                      ^
                Svar:  4
\end{verbatim}}
\defverbatim{\slowinvh}
{ \begin{verbatim}
                1 4 2 6 5 7 3
                    ^
                Svar:  5
\end{verbatim}}
\defverbatim{\slowinvi}
{ \begin{verbatim}
                1 2 4 6 5 7 3
                  ^
                Svar:  6
\end{verbatim}}
\defverbatim{\slowinvj}
{ \begin{verbatim}
                1 2 4 6 5 7 3
                            ^
                Svar:  6
\end{verbatim}}
\defverbatim{\slowinvk}
{ \begin{verbatim}
                1 2 4 6 5 3 7
                          ^
                Svar:  7
\end{verbatim}}
\defverbatim{\slowinvl}
{ \begin{verbatim}
                1 2 4 6 3 5 7
                        ^
                Svar:  8
\end{verbatim}}
\defverbatim{\slowinvm}
{ \begin{verbatim}
                1 2 4 3 6 5 7
                      ^
                Svar:  9
\end{verbatim}}
\defverbatim{\slowinvn}
{ \begin{verbatim}
                1 2 3 4 6 5 7
                    ^
                Svar: 10
\end{verbatim}}
\defverbatim{\slowinvo}
{ \begin{verbatim}
                1 2 3 4 6 5 7
                      ^
                Svar: 10
\end{verbatim}}
\defverbatim{\slowinvp}
{ \begin{verbatim}
                1 2 3 4 6 5 7
                          ^
                Svar: 10
\end{verbatim}}
\defverbatim{\slowinvq}
{ \begin{verbatim}
                1 2 3 4 5 6 7
                        ^
                Svar: 11
\end{verbatim}}
\defverbatim{\slowinvr}
{ \begin{verbatim}
                1 2 3 4 5 6 7
                          ^
                Svar: 11
\end{verbatim}}
\defverbatim{\slowinvs}
{ \begin{verbatim}
                1 2 3 4 5 6 7
                            ^
                Svar: 11
\end{verbatim}}
\defverbatim{\slowinvt}
{ \begin{verbatim}
                1 2 3 4 5 6 7

                Svar: 11
\end{verbatim}}
\subsection{Sýnidæmi án biltrés}
\env{frame}
{
    \frametitle{Hæga aðferðin}
    \only<all:1>{\slowinva}
    \only<all:2>{\slowinvb}
    \only<all:3>{\slowinvc}
    \only<all:4>{\slowinvd}
    \only<all:5>{\slowinve}
    \only<all:6>{\slowinvf}
    \only<all:7>{\slowinvg}
    \only<all:8>{\slowinvh}
    \only<all:9>{\slowinvi}
    \only<all:10>{\slowinvj}
    \only<all:11>{\slowinvk}
    \only<all:12>{\slowinvl}
    \only<all:13>{\slowinvm}
    \only<all:14>{\slowinvn}
    \only<all:15>{\slowinvo}
    \only<all:16>{\slowinvp}
    \only<all:17>{\slowinvq}
    \only<all:18>{\slowinvr}
    \only<all:19>{\slowinvs}
    \only<all:20>{\slowinvt}
}

\defverbatim{\fastinva}
{ \begin{verbatim}
       Listinn: 4 1 6 5 7 3 2

       Biltréð: 1 1 1 1 1 1 1

                Svar:  0
\end{verbatim}}
\defverbatim{\fastinvb}
{ \begin{verbatim}
       Listinn: 4 1 6 5 7 3 2
                  ^
       Biltréð: 1 1 1 1 1 1 1

                Svar:  0
\end{verbatim}}
\defverbatim{\fastinvc}
{ \begin{verbatim}
       Listinn: 4 1 6 5 7 3 2
                  ^
       Biltréð: 1 1 1 1 1 1 1
               | |
                Svar:  0
\end{verbatim}}
\defverbatim{\fastinvd}
{ \begin{verbatim}
       Listinn: 4 x 6 5 7 3 2

       Biltréð: 1 0 1 1 1 1 1

                Svar:  1
\end{verbatim}}
\defverbatim{\fastinve}
{ \begin{verbatim}
       Listinn: 4 x 6 5 7 3 2
                            ^
       Biltréð: 1 0 1 1 1 1 1

                Svar:  1
\end{verbatim}}
\defverbatim{\fastinvf}
{ \begin{verbatim}
       Listinn: 4 x 6 5 7 3 2
                            ^
       Biltréð: 1 0 1 1 1 1 1
               |           |
                Svar:  1
\end{verbatim}}
\defverbatim{\fastinvg}
{ \begin{verbatim}
       Listinn: 4 x 6 5 7 3 x

       Biltréð: 1 0 1 1 1 1 0

                Svar:  6
\end{verbatim}}
\defverbatim{\fastinvh}
{ \begin{verbatim}
       Listinn: 4 x 6 5 7 3 x
                          ^
       Biltréð: 1 0 1 1 1 1 0

                Svar:  6
\end{verbatim}}
\defverbatim{\fastinvi}
{ \begin{verbatim}
       Listinn: 4 x 6 5 7 3 x
                          ^
       Biltréð: 1 0 1 1 1 1 0
               |         |
                Svar:  6
\end{verbatim}}
\defverbatim{\fastinvj}
{ \begin{verbatim}
       Listinn: 4 x 6 5 7 x x

       Biltréð: 1 0 1 1 1 0 0

                Svar: 10
\end{verbatim}}
\defverbatim{\fastinvk}
{ \begin{verbatim}
       Listinn: 4 x 6 5 7 x x
                ^
       Biltréð: 1 0 1 1 1 0 0

                Svar: 10
\end{verbatim}}
\defverbatim{\fastinvl}
{ \begin{verbatim}
       Listinn: 4 x 6 5 7 x x
                ^
       Biltréð: 1 0 1 1 1 0 0
               |
                Svar: 10
\end{verbatim}}
\defverbatim{\fastinvm}
{ \begin{verbatim}
       Listinn: x x 6 5 7 x x

       Biltréð: 0 0 1 1 1 0 0

                Svar: 10
\end{verbatim}}
\defverbatim{\fastinvn}
{ \begin{verbatim}
       Listinn: x x 6 5 7 x x
                      ^
       Biltréð: 0 0 1 1 1 0 0

                Svar: 10
\end{verbatim}}
\defverbatim{\fastinvo}
{ \begin{verbatim}
       Listinn: x x 6 5 7 x x
                      ^
       Biltréð: 0 0 1 1 1 0 0
               |     |
                Svar: 10
\end{verbatim}}
\defverbatim{\fastinvp}
{ \begin{verbatim}
       Listinn: x x 6 x 7 x x

       Biltréð: 0 0 1 0 1 0 0

                Svar: 11
\end{verbatim}}
\defverbatim{\fastinvq}
{ \begin{verbatim}
       Listinn: x x 6 x 7 x x
                    ^
       Biltréð: 0 0 1 0 1 0 0

                Svar: 11
\end{verbatim}}
\defverbatim{\fastinvr}
{ \begin{verbatim}
       Listinn: x x 6 x 7 x x
                    ^
       Biltréð: 0 0 1 0 1 0 0
               |   |
                Svar: 11
\end{verbatim}}
\defverbatim{\fastinvs}
{ \begin{verbatim}
       Listinn: x x x x 7 x x

       Biltréð: 0 0 0 0 1 0 0

                Svar: 11
\end{verbatim}}
\defverbatim{\fastinvt}
{ \begin{verbatim}
       Listinn: x x x x 7 x x
                        ^
       Biltréð: 0 0 0 0 1 0 0

                Svar: 11
\end{verbatim}}
\defverbatim{\fastinvu}
{ \begin{verbatim}
       Listinn: x x x x 7 x x
                        ^
       Biltréð: 0 0 0 0 1 0 0
               |       |
                Svar: 11
\end{verbatim}}
\defverbatim{\fastinvv}
{ \begin{verbatim}
       Listinn: x x x x x x x

       Biltréð: 0 0 0 0 0 0 0

                Svar: 11
\end{verbatim}}
\subsection{Sýnidæmi með biltréi}
\env{frame}
{
    \frametitle{Hraða aðferðin}
    \only<all:1>{\fastinva}
    \only<all:2>{\fastinvb}
    \only<all:3>{\fastinvc}
    \only<all:4>{\fastinvd}
    \only<all:5>{\fastinve}
    \only<all:6>{\fastinvf}
    \only<all:7>{\fastinvg}
    \only<all:8>{\fastinvh}
    \only<all:9>{\fastinvi}
    \only<all:10>{\fastinvj}
    \only<all:11>{\fastinvk}
    \only<all:12>{\fastinvl}
    \only<all:13>{\fastinvm}
    \only<all:14>{\fastinvn}
    \only<all:15>{\fastinvo}
    \only<all:16>{\fastinvp}
    \only<all:17>{\fastinvq}
    \only<all:18>{\fastinvr}
    \only<all:19>{\fastinvs}
    \only<all:20>{\fastinvt}
    \only<all:21>{\fastinvu}
    \only<all:22>{\fastinvv}
}

\subsection{Hvað ef tölurnar eru of stórar?}
\env{frame}
{
    \env{itemize}
    {
        \item<1-> Hvað gerum við þó ef tölurnar sem eru gefnar eru ekki í $\{1, 2, ..., n\}$ (líkt og í dæminu \emph{Ultra-QuickSort})?
        \item<2-> Þá virkar oft að skipta minnstu tölunni út fyrir $1$, næst minnstu fyrir $2$ og svo framvegis.
        \item<3-> Ef það eru endurtekningar þarf að passa að öllum eins tölum sé breytt í eins tölur.
        \item<4-> Þetta má gera með því að raða.
        \item<5-> Við röðum fyrst tvenndunum $(a_j, j)$, þar sem $a_j$ táknar $j$-ta stakið í listanum okkar, eftir fyrsta stakinu.
        \item<6-> Við getum þá labbað í gegn og breytt öllum tölunum.
        \item<7-> Að lokum röðum tvenndunum aftur eftir seinna stakinu.
    }
}

\subsection{Útfærsla}
\env{frame}
{
    \selectcode{code/invnum-segtree.c}{33}{58}
}

\subsection{Tímaflækjur}
\env{frame}
{
    \env{itemize}
    {
        \item<1-> Það tekur $\mathcal{O}($\onslide<2->{$n \log n$}$)$ tíma að smækka tölurnar.
        \item<3-> Hver aðgerð á biltrénu er framkvæmd í $\mathcal{O}($\onslide<4->{$\log n$}$)$ tíma.
        \item<5-> Við finnum því fjölda umhverfinga í $\mathcal{O}($\onslide<6->{$n \log n$}$)$ tíma.
        \item<7-> Svo við getum auðveldlega fundið fjölda umhverfinga fyrir $n < 10^6$.
        \item<8-> Takið þó eftir að fjöldinn gæti orðið $n \cdot (n - 1)/2$, svo það þarf alltaf að nota \ilcode{long long}.
    }
}

\subsection{Talning umhverfinga með því að deila og drottna}
\env{frame}
{
    \env{itemize}
    {
        \item<1-> Önnur leið til að telja umhverfingar er að deila og drottna.
        \item<2-> Við getum raðað tölunum með \ilcode{mergesort} og talið í leiðinni hvað við þurfum að færa tölurnar um mörg sæti.
        \item<3-> Þegar við erum að sameina erum við með tvo raðaða lista.
        \item<4-> Við fjarlægjum síðan minnsta stakið sem er í listunum.
        \item<5-> Ef við fjarlægjum stak úr vinstri listanum færist það um ekkert sæti.
        \item<6-> Ef við fjarlægjum stak úr hægri listanum færist það um eitt sæti fyrir hvert stak sem er eftir í vinstri listanum.
        \item<7-> Við getum útfærst þetta neðansækið og ofansækið.
    }
}

\subsection{Ofansækin útfærsla}
\env{frame}
{
    \selectcode{code/invnum-mergesort-td.c}{5}{26}
}

\subsection{Neðansækin útfærsla}
\env{frame}
{
    \selectcode{code/invnum-mergesort-bu.c}{6}{19}
}

\subsection{Tímaflækjur}
\env{frame}
{
    \env{itemize}
    {
        \item<1-> Bæði reikniritin hafa sömu tímaflækju og \ilcode{mergesort}, það er að segja $\mathcal{O}($\onslide<2->{$n \log n$}$)$.
    }
}

\section{Kvik bestun í talningarfræði}
\subsection{Yfirlit}
\env{frame}
{
    \env{itemize}
    {
        \item<1-> Mörg talningarfræði dæmi má smækka á þægilegan máta.
        \item<2-> Ef við látum, til dæmis, $f(n)$ tákna fjölda hlutmengja í mengin $\{1, 2, ..., n\}$ þá höfum við að
        \[
            f(n) = \left \{
            \env{array}
            { {l l}
                1, & \text{ef $n = 0$}\\
                2 \cdot f(n - 1), & \text{annars}.
            }
            \right .
        \]
        \item<3-> Þetta gildir því það eru $f(n - 1)$ hlutmengi sem innihalda $n$ og $f(n - 1)$ hlutmengi sem innihalda ekki $n$.
        \item<4-> Ef við getum smækkað dæmin á þennan máta má svo nota \onslide<5->{kvika bestun} til að leysa þau.
        \item<6-> Oft þarf að bæta við vídd til að halda utan um önnur gögn.
        \item<7-> Tökum dæmi.
    }
}

\subsection{Dæmi}
\env{frame}
{
    \env{itemize}
    {
        \item<1-> Gerum ráð fyrir að þú sért með $n$ spilastokka.
        \item<2-> Hver stokkur inniheldur $k$ spil, númeruð frá $1$ og upp í $k$.
        \item<3-> Á hversu marga vegu getur þú valið eitt spil úr hverjum stokk þannig að summa spilanna sé nákvæmlega $m$?
        \item<4-> Þar sem þessi tala getur verið mjög stór svo reikna skal hana $\mod 10^9 + 7$.
    }
}

\subsection{Rakningarformúla}
\env{frame}
{
    \env{itemize}
    {
    \item<1-> Festum fyrsta spilið sem $x$.
    \item<2-> Við eigum þá eftir $n - 1$ stokk og viljum fá summuna $m - x$ úr þeim.
    \item<3-> Með öðrum orðum höfum við smækkað dæmið.
    \item<4-> Við skilgreinum því
        \[
            f(x, y) = \left \{
                \env{array}
                { {l l}
                1, & \text{ef $n = 0$ og $y = 0$}\\
                0, & \text{ef $n = 0$ og $y \neq 0$}\\
                \sum_{j = 1}^k f(x - 1, y - j), & \text{annars}.
                }
                \right .
            \]
        \item<5-> Nú gildir að $f(x, y)$ er fjöldi leiða til að fá summuna $y$ með $x$ stokkum.
        \item<6-> Við getum svo útfært þetta eins of við höfum verið að útfæra kvikva bestun.
    }
}

\subsection{Útfærsla}
\env{frame}
{
    \selectcode{code/daemi-um-spil.c}{6}{15}
}

\subsection{Tímaflækja}
\env{frame}
{
    \env{itemize}
    {
        \item<1-> Við höfum $n \cdot m$ stöður og hverja stöðu má reikna í $\mathcal{O}($\onslide<2->{$\,k\,$}$)$ tíma.
        \item<3-> Svo tímaflækjan er $\mathcal{O}($\onslide<4->{$n \cdot m \cdot k$}$)$.
        \item<5-> Nú getur $m$ ekki verið stærra en $n \cdot k$ svo við fáum $\mathcal{O}($\onslide<6->{$n^2 \cdot k^2$}$)$.
    }
}

% TODO: Hér myndi líkindafræðin koma

\section{Fylkjareikningur}
\subsection{Yfirlit}
\env{frame}
{
    \env{itemize}
    {
        \item<1-> Í stærðfræði þýðir ,,fylki'' annað en í tölvunarfræði.
        \item<2-> Í stærðfræði er \emph{fylki} (e. \emph{matrix}) tvívíð uppröðun á tölum.
        \item<3-> Fylkið er sagt vera $n \times m$ ef það hefur $n$ línur og $m$ dálka.
        \item<4-> Dæmi um $2 \times 3$ fylki er
        \[
            \left (
            \env{array}
            { {l l l}
                2 & -1 & 0\\
                0 & 14 & 0
            }
            \right ).
        \]
        \item<5-> Við táknum yfirleitt stakið í línu $j$ og dálki $k$ í fylki $A$ með $A_{jk}$.
        \item<6-> Flyki í stærfræði er því eins og tvívítt fylki í tölvunarfærði.
    }
}

\subsection{Gagngrindur fyrir fylki}
\env{frame}
{
    \env{itemize}
    {
        \item<1-> Þegar við viljum geyma stærðfræði fylki í tölvu notum við oftast tvívítt tölvunarfræði fylki.
        \item<2-> Við höfum þá að $A_{jk}$ svarar til \ilcode{a[j][k]}.
        \item<3-> Ef $A$ er $n \times m$ fylki getum við líka geymt það með einvíðu tölvunarfræði fylki með samsvöruninni $A_{jk}$ við
                    \ilcode{a[j*m + k]}.
    }
}

\subsection{Aðgerðir á fylkjum}
\env{frame}
{
    \env{itemize}
    {
        \item<1-> Við getum lagt saman fylki af sömu stærð stakvíst, það er að segja $(A + B)_{jk} = A_{jk} + B_{jk}$.
        \item<2-> Frádráttur virkar eins.
        \item<3-> Við margföldum saman fylki af stærð $n \times m$ og $m \times r$ með
        \[
            (A \cdot B)_{jk} = \sum_{l = 1}^m A_{jl} \cdot B_{lk}.
        \]
        \item<4-> Oftast erum við að vinna með fylki sem eru af stærð $n \times n$.
        \item<5-> Slík fylki kallast \emph{ferningsfylki}.
        \item<6-> Takið eftir að ferningsfylkið $I$, gefið með $I_{jk} = 0$ ef $j \neq k$ og $I_{jk} = 1$ annars, er margföldunarhlutleysa.
    }
}

\subsection{Útfærslur}
\env{frame}
{
    \selectcode{code/matrix.c}{4}{23}
}

\subsection{Tímaflækjur}
\env{frame}
{
    \env{itemize}
    {
        \item<1-> Takið eftir að \ilcode{multo(...)} hefur tímaflækju $\mathcal{O}($\onslide<2->{$n^3$}$)$.
        \item<3-> Ef $A$ er $n \times n$ ferningsfylki getum við reinkað $A^p$.
        \item<4-> Með því að nota deila og drottna aðferð, líkt og við gerðum í síðustu viku, getum við reiknað $A^p$ í
            $\mathcal{O}($\onslide<5->{$n^3 \log p$}$)$ tíma.
        \item<6-> Þetta má nýta mikið í talningarfræði.
        \item<7-> Tökum dæmi.
    }
}

\subsection{Veldishafning flykja í netafræði}
\env{frame}
{
    \env{itemize}
    {
        \item<1-> Látum $G = (V, E)$ vera net og $A$ vera nágrannflyki $G$.
        \item<2-> Við munum nú leyfa netinu að hafa fleiri en einn veg á milli tveggja hnúta.
        \item<3-> Þá segir $A_{uv}$ hversu margir vegir liggja frá hnúta $u$ til hnúts $v$.
        \item<4-> Sýna má með þrepuna að $(A^p)_{uv}$ segir okkur þá hversu margir vegir liggja á milli hnúts $u$ og hnúts $v$,
                sem eru af lengd nákvæmlega $p$.
        \item<5-> Takið eftir að við getum leyst þetta dæmi fyrir mjög stór $p$.
        \item<6-> Til dæmis væri lausnin okkar leifturhröð fyrir $n = 50$ og $p = 10^{18}$.
    }
}

\section{Úrvinnsla lúnulegra rakningarvensla}
\subsection{Yfirlit}
\env{frame}
{
    \frametitle{Úrvinnsla línulegra rakningarvensl}
    \env{itemize}
    {
        \item<1-> Runa $(a_n)_{n \in \mathbb{N}}$ kallast \emph{$k$-ta stigs línulega rakningarvensl} ef til eru $c_1, ..., c_k$ þannig að
        \[
            a_n = \sum_{j = 1}^k c_j \cdot a_{n - j},
        \]
        fyrir öll $n$ þannig að $n - k \in \mathbb{N}$.
        \item<2-> Munið, til dæmis, að Fiboncci tölurnar eru gefnar með línulegu rakningarvenslunum þar sem $a_1 = a_2 = c_1 = c_2 = 1$.
        \item<3-> Látum $k$-ta stigs línuleg rakningarvensl vera gefnin, líkt og að ofan.
    }
}

\subsection{Útreikingur línulegra rakningarvensla með margföldun fylkja}
\env{frame}
{
    \env{itemize}
    {
        \item<1-> Skilgreinum nú flykið
        \[
            M = \left (
            \env{array}
            { {l l l l l}
                c_1 & c_2 & ... & c_{k - 1} & c_k\\
                1 & 0 & ... & 0 & 0\\
                0 & 1 & ... & 0 & 0\\
                \vdots & \vdots & \ddots & \vdots & \vdots\\
                0 & 0 & ... & 1 & 0
            }
            \right ).
        \]
        \item<2-> Takið eftir að
        \[
            M
            \cdot
            \left (
            \env{array}
            { {l}
                a_{n - 1}\\
                a_{n - 2}\\
                \vdots\\
                a_{n - k}
            }
            \right )
            =
            \left (
            \env{array}
            { {l}
                \sum_{j = 1}^k c_j \cdot a_{n - j}\\
                a_{n - 1}\\
                a_{n - 2}\\
                \vdots\\
                a_{n - k + 1}
            }
            \right )
            =
            \left (
            \env{array}
            { {l}
                a_n\\
                a_{n - 1}\\
                a_{n - 2}\\
                \vdots\\
                a_{n - k + 1}
            }
            \right ).
        \]
        \item<3-> Svo við getum notað fylkið $M$ til að fá næstu tölu í rakningarvenslunum.
    }
}

\subsection{Útreikingur línulegra rakningarvensla með veldishafningu fylkja}
\env{frame}
{
    \env{itemize}
    {
        \item<1-> Við þurfum þó ekki að hætta þar.
        \item<2-> Við getum notað $M^p$ til að fá $a_{k + p}$.
        \item<3-> Þetta getum við reinkað í $\mathcal{O}($\onslide<4->{$k^3 \log p$}$)$ tíma.
        \item<5-> Við getum, til dæmis, notað þetta til að reikna $n$-tu Fibonacci töluna í $\mathcal{O}($\onslide<6->{$\log n$}$)$ tíma.
        \item<7-> Það er meira en nógu hratt fyrir $n < 10^{18}$.
    }
}

\subsection{Útfærsla}
\env{frame}
{
    \selectcode{code/fib.c}{14}{34}
}

\subsection{Kerfisbundin leið til að ákvarða rakningarvenslin}
\env{frame}
{
    \env{itemize}
    {
        \item<1-> Í dæmum sem beita má þessari aðferð er oft erfitt að finna rakningarvenslin (þegar þau eru ekki gefin beint).
        \item<2-> Þá má nota netafræðina í staðinn.
        \item<3-> Tökum dæmi.
    }
}

\section{Dæmi}
\subsection{Yfirlit}
\env{frame}
{
    \env{itemize}
    {
        \item<1-> Þú vilt leggja flísar á ganginn þinn þannig að hver flís er annað hvort svört eða hvít,
                    og engar tvær aðliggjandi flísar mega vera hvítar.
        \item<2-> Gerum ráð fyrir að gangurinn sé þrjár flísar á breidd og $n$ flísar á lengd.
        \item<3-> Á hversu marga vegu getur þú lagt flísarnar?
        \item<4-> Flísar sem snertast horn í horn teljast ekki aðliggjandi.
    }
}

\defverbatim{\gangar}
{ \begin{verbatim}
     Hnútur: 0   1   2   3   4   5   6   7

             0   0   0   0   1   1   1   1
     Gangur: 0   0   1   1   0   0   1   1
             0   1   0   1   0   1   0   1
\end{verbatim}}
\subsection{Breytum í netfræði}
\env{frame}
{
    \env{itemize}
    {
        \item<1-> Hægt er að setja þessa talningu fram með fjórða stigs línulegum rakningarvenslum.
        \item<2-> Það er þó ekki auðséð hverjir stuðlarnir í þessum venslum erum.
        \item<3-> En hvað ef við breytum þessu í net.
        \item<4-> Látum hnútana í netinu vera mögulegir dálkar á ganginum:
        \item<5->[] \gangar
        \item<6-> Við bætum svo við legg á milli hnútanna $u$ og $v$ ef dálkarnir mega liggja hliðina á hvorum öðrum á ganginum.
        \item<7-> Látum \ilcode{1} tákna hvítar flísar.
        \item<8-> Takið eftir að stöður \ilcode{3}, \ilcode{6} og \ilcode{7} eru alfarið ólöglegar.
    }
}

\defverbatim{\gangarfylki}
{ \begin{verbatim}
        0  1  2  3  4  5  6  7
      +------------------------
    0 | 1  1  1  0  1  1  0  0
    1 | 1  0  1  0  1  0  0  0
    2 | 1  1  0  0  1  1  0  0
    3 | 0  0  0  0  0  0  0  0
    4 | 1  1  1  0  0  0  0  0
    5 | 1  0  1  0  0  0  0  0
    6 | 0  0  0  0  0  0  0  0
    7 | 0  0  0  0  0  0  0  0
\end{verbatim}}
\subsection{Nágrannfylkið}
\env{frame}
{
    \env{itemize}
    {
        \item<1-> Við fáum þá nágrannafylkið:
        \item<2->[] \gangarfylki
        \item<3-> Köllum þetta fylki $A$.
        \item<4-> Við þurfum síðan að leggja saman $(A^{p - 1})_{uv}$ fyrir öll $u$ og $v$ þar sem $u$ er löglegur dálkur.
        \item<5-> Þá eru við að leggja saman alla vegi í netinu af lengd $p$ sem byrja í löglegum dálki.
        \item<6-> Við þurfum ekki að passa að síðasti hnúturinn sé löglegur því það liggur enginn leggur í ólöglegan hnút.
    }
}

\subsection{Útfærsla}
\env{frame}
{
    \selectcode{code/daemi-um-flisar.c}{28}{48}
}

\section{Gauss eyðing}
\subsection{Yfirlit}
\env{frame}
{
    \frametitle{Gauss-eyðing}
    \env{itemize}
    {
        \item<1-> Umræða um fylkjaaðgerðir er ekki fullkláruð fyrr en minnst er á Gauss-eyðingu.
        \item<2-> Þið munið eflaust eftir henni úr línulegri algebru.
        \item<3-> Hún er nytsamlega til að, til dæmis, leysa jöfnuhneppi eða finna andhverfur fylkja.
    }
}

\subsection{Útfærlsa}
\env{frame}
{
    \selectcode{code/gauss.c}{5}{17}
}

\section{Þessi glæra er viljandi auð}
\env{frame}
{
}

\end{document}

