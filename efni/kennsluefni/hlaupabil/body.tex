\title{Hlaupabil}
\author{Bergur Snorrason}
\date{\today}

\begin{document}

\frame{\titlepage}

\env{frame}
{
	\frametitle{Hlaupabil}
	\env{itemize}
	{
		\item<1-> Aðferð \emph{hlaupabila} (e. \emph{sliding window}) er stundum hægt að nota til að taka
				dæmi sem hafa augljósa $\mathcal{O}(n^2)$ lausn og gera þau $\mathcal{O}(n)$ eða $\mathcal{O}(n\log n)$.
	}
}

\env{frame}
{
	\env{itemize}
	{
		\item<1-> Skoðum dæmi:
		\item<2-> Gefið $n$, $k$ og svo $n$ tölur $a_i$, þ.a. $a_i \in \{0, 1\}$ finndu
			lengd lengstu bilanna í rununni $(a_n)_{n \in \mathbb{N}}$ sem innihelda bara $1$ ef þú mátt breyta allt að $k$ tölum.
		\item<3-> Sjáum strax að maður vill alltaf breyta $0$ í $1$ og aldrei öfugt.
		\item<4-> Sjáum því að við erum að leita að lengstu bilunum í $(a_n)_{n \in \mathbb{N}}$ sem hefur í mesta lagi $k$ stök jöfn $0$.
		\item<5-> Gefum okkur nú hlaupabil. Það byrjar tómt.
		\item<6-> Við löbbum svo í gegnum $(a_n)_{n \in \mathbb{N}}$ og lengjum bilið að aftan.
		\item<7-> Ef það eru einhvern tímann fleiri en $k$ stök í bilinu sem eru $0$ þá minnkum við bilið að framan þar til svo er ekki lengur.
	}
}

\defverbatim{\hbilA}
{ \begin{verbatim}
        k = 2
        l = 0
        [0 1 1 0 1 0 0 0 1 1 1 1 0 0 1 1]
        |
\end{verbatim} }

\defverbatim{\hbilB}
{ \begin{verbatim}
        k = 2
        l = 1
        [0 1 1 0 1 0 0 0 1 1 1 1 0 0 1 1]
        | |
\end{verbatim} }

\defverbatim{\hbilC}
{ \begin{verbatim}
        k = 2
        l = 2
        [0 1 1 0 1 0 0 0 1 1 1 1 0 0 1 1]
        |   |
\end{verbatim} }

\defverbatim{\hbilD}
{ \begin{verbatim}
        k = 2
        l = 3
        [0 1 1 0 1 0 0 0 1 1 1 1 0 0 1 1]
        |     |
\end{verbatim} }

\defverbatim{\hbilE}
{ \begin{verbatim}
        k = 2
        l = 4
        [0 1 1 0 1 0 0 0 1 1 1 1 0 0 1 1]
        |       |
\end{verbatim} }

\defverbatim{\hbilF}
{ \begin{verbatim}
        k = 2
        l = 5
        [0 1 1 0 1 0 0 0 1 1 1 1 0 0 1 1]
        |         |
\end{verbatim} }

\defverbatim{\hbilG}
{ \begin{verbatim}
        k = 2
        l = 4
        [0 1 1 0 1 0 0 0 1 1 1 1 0 0 1 1]
          |       |
\end{verbatim} }

\defverbatim{\hbilH}
{ \begin{verbatim}
        k = 2
        l = 5
        [0 1 1 0 1 0 0 0 1 1 1 1 0 0 1 1]
          |         |
\end{verbatim} }

\defverbatim{\hbilI}
{ \begin{verbatim}
        k = 2
        l = 4
        [0 1 1 0 1 0 0 0 1 1 1 1 0 0 1 1]
            |       |
\end{verbatim} }

\defverbatim{\hbilJ}
{ \begin{verbatim}
        k = 2
        l = 3
        [0 1 1 0 1 0 0 0 1 1 1 1 0 0 1 1]
              |     |
\end{verbatim} }

\defverbatim{\hbilK}
{ \begin{verbatim}
        k = 2
        l = 2
        [0 1 1 0 1 0 0 0 1 1 1 1 0 0 1 1]
                |   |
\end{verbatim} }

\defverbatim{\hbilL}
{ \begin{verbatim}
        k = 2
        l = 3
        [0 1 1 0 1 0 0 0 1 1 1 1 0 0 1 1]
                |     |
\end{verbatim} }

\defverbatim{\hbilM}
{ \begin{verbatim}
        k = 2
        l = 2
        [0 1 1 0 1 0 0 0 1 1 1 1 0 0 1 1]
                  |   |
\end{verbatim} }

\defverbatim{\hbilN}
{ \begin{verbatim}
        k = 2
        l = 1
        [0 1 1 0 1 0 0 0 1 1 1 1 0 0 1 1]
                    | |
\end{verbatim} }

\defverbatim{\hbilO}
{ \begin{verbatim}
        k = 2
        l = 2
        [0 1 1 0 1 0 0 0 1 1 1 1 0 0 1 1]
                    |   |
\end{verbatim} }

\defverbatim{\hbilP}
{ \begin{verbatim}
        k = 2
        l = 3
        [0 1 1 0 1 0 0 0 1 1 1 1 0 0 1 1]
                    |     |
\end{verbatim} }

\defverbatim{\hbilQ}
{ \begin{verbatim}
        k = 2
        l = 4
        [0 1 1 0 1 0 0 0 1 1 1 1 0 0 1 1]
                    |       |
\end{verbatim} }

\defverbatim{\hbilR}
{ \begin{verbatim}
        k = 2
        l = 5
        [0 1 1 0 1 0 0 0 1 1 1 1 0 0 1 1]
                    |         |
\end{verbatim} }

\defverbatim{\hbilS}
{ \begin{verbatim}
        k = 2
        l = 6
        [0 1 1 0 1 0 0 0 1 1 1 1 0 0 1 1]
                    |           |
\end{verbatim} }

\defverbatim{\hbilT}
{ \begin{verbatim}
        k = 2
        l = 5
        [0 1 1 0 1 0 0 0 1 1 1 1 0 0 1 1]
                      |         |
\end{verbatim} }

\defverbatim{\hbilU}
{ \begin{verbatim}
        k = 2
        l = 6
        [0 1 1 0 1 0 0 0 1 1 1 1 0 0 1 1]
                      |           |
\end{verbatim} }

\defverbatim{\hbilV}
{ \begin{verbatim}
        k = 2
        l = 5
        [0 1 1 0 1 0 0 0 1 1 1 1 0 0 1 1]
                        |         |
\end{verbatim} }

\defverbatim{\hbilW}
{ \begin{verbatim}
        k = 2
        l = 6
        [0 1 1 0 1 0 0 0 1 1 1 1 0 0 1 1]
                        |           |
\end{verbatim} }

\defverbatim{\hbilX}
{ \begin{verbatim}
        k = 2
        l = 7
        [0 1 1 0 1 0 0 0 1 1 1 1 0 0 1 1]
                        |             |
\end{verbatim} }

\defverbatim{\hbilY}
{ \begin{verbatim}
        k = 2
        l = 8
        [0 1 1 0 1 0 0 0 1 1 1 1 0 0 1 1]
                        |               |
\end{verbatim} }

\env{frame}
{
	\only<all:1>{\hbilA}
	\only<all:2>{\hbilB}
	\only<all:3>{\hbilC}
	\only<all:4>{\hbilD}
	\only<all:5>{\hbilE}
	\only<all:6>{\hbilF}
	\only<all:7>{\hbilG}
	\only<all:8>{\hbilH}
	\only<all:9>{\hbilI}
	\only<all:10>{\hbilJ}
	\only<all:11>{\hbilK}
	\only<all:12>{\hbilL}
	\only<all:13>{\hbilM}
	\only<all:14>{\hbilN}
	\only<all:15>{\hbilO}
	\only<all:16>{\hbilP}
	\only<all:17>{\hbilQ}
	\only<all:18>{\hbilR}
	\only<all:19>{\hbilS}
	\only<all:20>{\hbilT}
	\only<all:21>{\hbilU}
	\only<all:22>{\hbilV}
	\only<all:23>{\hbilW}
	\only<all:24>{\hbilX}
	\only<all:25>{\hbilY}
}

\env{frame}
{
	\selectcode{code/hlaupabil-daemi1.c}{3}{22}
}

\env{frame}
{
	\env{itemize}
	{
		\item<1-> Hver tala í rununni er sett einu sinni í hlaupabilið og mögulega fjarlægð úr því.
		\item<2-> Svo tímaflækjan er $\mathcal{O}($\onslide<3->{$\,n\,$}$)$.
	}
}

\env{frame}
{
	\env{itemize}
	{
		\item<1-> Þetta dæmi var í auðveldari kantinum.
		\item<2-> Skoðum annað dæmi:
		\item<3-> Byjrum á nokkrum undirstöðu atriðum.
		\item<4-> Tvö bil kallast \emph{næstum sundurlæg} ef sniðmengi þeirra er tómt eða bara einn punktur.
		\item<5-> Sammengi bila má skrifa sem sammengi næstu sundurlægra bila.
		\item<6-> \emph{Lengd bilsins} $[a, b]$ er $b - a$.
		\item<7-> Til að finna \emph{lengd sammengis bila} skrifum við sammengið sem sammengi næstum sundurlægra bila
			og tökum summu lengda þeirra.
		\item<8-> Til dæmis eru bilin $[1, 2]$ og $[2, 3]$ næstum sundurlæg (en þó ekki sundurlæg) en 
			$[1, 3]$ og $[2, 4]$ eru það ekki. Nú $[1, 3] \cup [2, 4] = [1, 4]$ svo lengd 
			$[1, 3] \cup [2, 4]$ er $3$.
	}
}

\env{frame}
{
	\env{itemize}
	{
		\item<1-> Gefið $n$ bil hver er lengd sammengis þeirra.
	}
}

\env{frame}
{
	\env{itemize}
	{
		\item<1-> Geymum í lista tvenndir þar sem fyrra stakið er endapunktur bils og seinna stakið segir hvaða bili punkturinn tilleyrir.
		\item<2-> Röðum þessum punktum svo í vaxandi röð.
		\item<3-> Við löbbum í gegnum þennan raðaða lista og höldum utan um hlaupabil þannig að
			við bætum við bili í hlaupabilið þegar við rekumst á vinstri endapunkt þess og fjarlægjum það 
			þegar við rekumst á hægri endapunkt þess. 
		\item<4-> Við skoðum svo sérstaklega tilfellin þegar við erum ekki með nein bil í hlaupabilinu okkar.
		\item<5-> Sammengi þeirra bila sem við höfum farið í gegnum þá síðan hlaupabilið var síðast tómt er nú næstum
			sundurlægt öllum öðrum bilum sem okkur var gefið í byrjun.
		\item<6-> Við skilum því summu lengda þessara sammengja.
	}
}

\defverbatim{\lineAA}
{ \begin{verbatim}
   |                                  
 1:  x------------x
 2:     x----x
 3:  x----x                    
 4:          x----------x
 5:                         x------x
 6:                                                x--x
 7:                                            x------x
 8:                           x--x
 9:                                      x------------x
10:                         x------x
   |                    
[]
r = 0
\end{verbatim} }

\defverbatim{\lineAB}
{ \begin{verbatim}
     |                                  
 1:  x------------x
 2:     x----x
 3:  x----x                    
 4:          x----------x
 5:                         x------x
 6:                                                x--x
 7:                                            x------x
 8:                           x--x
 9:                                      x------------x
10:                         x------x
     |                    
[]
r = 0
\end{verbatim} }

\defverbatim{\lineAC}
{ \begin{verbatim}
     |                                  
 1:  x------------x
 2:     x----x
 3:  x----x                    
 4:          x----------x
 5:                         x------x
 6:                                                x--x
 7:                                            x------x
 8:                           x--x
 9:                                      x------------x
10:                         x------x
     |                    
[1]
r = 0
\end{verbatim} }

\defverbatim{\lineAD}
{ \begin{verbatim}
     |                                  
 1:  x------------x
 2:     x----x
 3:  x----x                    
 4:          x----------x
 5:                         x------x
 6:                                                x--x
 7:                                            x------x
 8:                           x--x
 9:                                      x------------x
10:                         x------x
     |                    
[1, 3]
r = 0
\end{verbatim} }

\defverbatim{\lineAE}
{ \begin{verbatim}
        |                                  
 1:  x------------x
 2:     x----x
 3:  x----x                    
 4:          x----------x
 5:                         x------x
 6:                                                x--x
 7:                                            x------x
 8:                           x--x
 9:                                      x------------x
10:                         x------x
        |                    
[1, 3]
r = 0
\end{verbatim} }

\defverbatim{\lineAF}
{ \begin{verbatim}
        |                                  
 1:  x------------x
 2:     x----x
 3:  x----x                    
 4:          x----------x
 5:                         x------x
 6:                                                x--x
 7:                                            x------x
 8:                           x--x
 9:                                      x------------x
10:                         x------x
        |                    
[1, 2, 3]
r = 0
\end{verbatim} }

\defverbatim{\lineAG}
{ \begin{verbatim}
          |                                  
 1:  x------------x
 2:     x----x
 3:  x----x                    
 4:          x----------x
 5:                         x------x
 6:                                                x--x
 7:                                            x------x
 8:                           x--x
 9:                                      x------------x
10:                         x------x
          |                    
[1, 2, 3]
r = 0
\end{verbatim} }

\defverbatim{\lineAH}
{ \begin{verbatim}
          |                                  
 1:  x------------x
 2:     x----x
 3:  x----x                    
 4:          x----------x
 5:                         x------x
 6:                                                x--x
 7:                                            x------x
 8:                           x--x
 9:                                      x------------x
10:                         x------x
          |                    
[1, 2]
r = 0
\end{verbatim} }

\defverbatim{\lineAI}
{ \begin{verbatim}
             |                                      
 1:  x------------x
 2:     x----x
 3:  x----x                    
 4:          x----------x
 5:                         x------x
 6:                                                x--x
 7:                                            x------x
 8:                           x--x
 9:                                      x------------x
10:                         x------x
             |                        
[1, 2]
r = 0
\end{verbatim} }

\defverbatim{\lineAJ}
{ \begin{verbatim}
             |                                      
 1:  x------------x
 2:     x----x
 3:  x----x                    
 4:          x----------x
 5:                         x------x
 6:                                                x--x
 7:                                            x------x
 8:                           x--x
 9:                                      x------------x
10:                         x------x
             |                        
[1, 2, 4]
r = 0
\end{verbatim} }

\defverbatim{\lineAK}
{ \begin{verbatim}
             |
 1:  x------------x
 2:     x----x
 3:  x----x                    
 4:          x----------x
 5:                         x------x
 6:                                                x--x
 7:                                            x------x
 8:                           x--x
 9:                                      x------------x
10:                         x------x
             |
[1, 4]
r = 0
\end{verbatim} }

\defverbatim{\lineAL}
{ \begin{verbatim}
                  |
 1:  x------------x
 2:     x----x
 3:  x----x                    
 4:          x----------x
 5:                         x------x
 6:                                                x--x
 7:                                            x------x
 8:                           x--x
 9:                                      x------------x
10:                         x------x
                  |
[1, 4]
r = 0
\end{verbatim} }

\defverbatim{\lineAM}
{ \begin{verbatim}
                  |
 1:  x------------x
 2:     x----x
 3:  x----x                    
 4:          x----------x
 5:                         x------x
 6:                                                x--x
 7:                                            x------x
 8:                           x--x
 9:                                      x------------x
10:                         x------x
                  |
[4]
r = 0
\end{verbatim} }

\defverbatim{\lineAN}
{ \begin{verbatim}
                        |
 1:  x------------x
 2:     x----x
 3:  x----x                    
 4:          x----------x
 5:                         x------x
 6:                                                x--x
 7:                                            x------x
 8:                           x--x
 9:                                      x------------x
10:                         x------x
                        |
[4]
r = 0
\end{verbatim} }

\defverbatim{\lineAO}
{ \begin{verbatim}
                        |
 1:  x------------x
 2:     x----x
 3:  x----x                    
 4:          x----------x
 5:                         x------x
 6:                                                x--x
 7:                                            x------x
 8:                           x--x
 9:                                      x------------x
10:                         x------x
                        |
[]
r = 0
\end{verbatim} }

\defverbatim{\lineAP}
{ \begin{verbatim}
                        |
 1:  x------------x
 2:     x----x
 3:  x----x                    
 4:          x----------x
 5:                         x------x
 6:                                                x--x
 7:                                            x------x
 8:                           x--x
 9:                                      x------------x
10:                         x------x
                        |
[]
r = 20
\end{verbatim} }

\defverbatim{\lineAQ}
{ \begin{verbatim}
                            |
 1:  x------------x
 2:     x----x
 3:  x----x                    
 4:          x----------x
 5:                         x------x
 6:                                                x--x
 7:                                            x------x
 8:                           x--x
 9:                                      x------------x
10:                         x------x
                            |
[]
r = 20
\end{verbatim} }

\defverbatim{\lineAR}
{ \begin{verbatim}
                            |
 1:  x------------x
 2:     x----x
 3:  x----x                    
 4:          x----------x
 5:                         x------x
 6:                                                x--x
 7:                                            x------x
 8:                           x--x
 9:                                      x------------x
10:                         x------x
                            |
[5]
r = 20
\end{verbatim} }

\defverbatim{\lineAS}
{ \begin{verbatim}
                            |
 1:  x------------x
 2:     x----x
 3:  x----x                    
 4:          x----------x
 5:                         x------x
 6:                                                x--x
 7:                                            x------x
 8:                           x--x
 9:                                      x------------x
10:                         x------x
                            |
[5, 10]
r = 20
\end{verbatim} }

\defverbatim{\lineAU}
{ \begin{verbatim}
                              |
 1:  x------------x
 2:     x----x
 3:  x----x                    
 4:          x----------x
 5:                         x------x
 6:                                                x--x
 7:                                            x------x
 8:                           x--x
 9:                                      x------------x
10:                         x------x
                              |
[5, 10]
r = 20
\end{verbatim} }

\defverbatim{\lineAV}
{ \begin{verbatim}
                              |
 1:  x------------x
 2:     x----x
 3:  x----x                    
 4:          x----------x
 5:                         x------x
 6:                                                x--x
 7:                                            x------x
 8:                           x--x
 9:                                      x------------x
10:                         x------x
                              |
[5, 8, 10]
r = 20
\end{verbatim} }

\defverbatim{\lineAW}
{ \begin{verbatim}
                                 |
 1:  x------------x
 2:     x----x
 3:  x----x                    
 4:          x----------x
 5:                         x------x
 6:                                                x--x
 7:                                            x------x
 8:                           x--x
 9:                                      x------------x
10:                         x------x
                                 |
[5, 8, 10]
r = 20
\end{verbatim} }

\defverbatim{\lineAX}
{ \begin{verbatim}
                                 |
 1:  x------------x
 2:     x----x
 3:  x----x                    
 4:          x----------x
 5:                         x------x
 6:                                                x--x
 7:                                            x------x
 8:                           x--x
 9:                                      x------------x
10:                         x------x
                                 |
[5, 10]
r = 20
\end{verbatim} }

\defverbatim{\lineAY}
{ \begin{verbatim}
                                   |
 1:  x------------x
 2:     x----x
 3:  x----x                    
 4:          x----------x
 5:                         x------x
 6:                                                x--x
 7:                                            x------x
 8:                           x--x
 9:                                      x------------x
10:                         x------x
                                   |
[5, 10]
r = 20
\end{verbatim} }

\defverbatim{\lineAZ}
{ \begin{verbatim}
                                   |
 1:  x------------x
 2:     x----x
 3:  x----x                    
 4:          x----------x
 5:                         x------x
 6:                                                x--x
 7:                                            x------x
 8:                           x--x
 9:                                      x------------x
10:                         x------x
                                   |
[5]
r = 20
\end{verbatim} }

\defverbatim{\lineBA}
{ \begin{verbatim}
                                   |
 1:  x------------x
 2:     x----x
 3:  x----x                    
 4:          x----------x
 5:                         x------x
 6:                                                x--x
 7:                                            x------x
 8:                           x--x
 9:                                      x------------x
10:                         x------x
                                   |
[]
r = 20
\end{verbatim} }

\defverbatim{\lineBB}
{ \begin{verbatim}
                                   |
 1:  x------------x
 2:     x----x
 3:  x----x                    
 4:          x----------x
 5:                         x------x
 6:                                                x--x
 7:                                            x------x
 8:                           x--x
 9:                                      x------------x
10:                         x------x
                                   |
[]
r = 28
\end{verbatim} }

\defverbatim{\lineBC}
{ \begin{verbatim}
                                         |
 1:  x------------x
 2:     x----x
 3:  x----x                    
 4:          x----------x
 5:                         x------x
 6:                                                x--x
 7:                                            x------x
 8:                           x--x
 9:                                      x------------x
10:                         x------x
                                         |
[]
r = 28
\end{verbatim} }

\defverbatim{\lineBD}
{ \begin{verbatim}
                                         |
 1:  x------------x
 2:     x----x
 3:  x----x                    
 4:          x----------x
 5:                         x------x
 6:                                                x--x
 7:                                            x------x
 8:                           x--x
 9:                                      x------------x
10:                         x------x
                                         |
[9]
r = 28
\end{verbatim} }

\defverbatim{\lineBDD}
{ \begin{verbatim}
                                               |
 1:  x------------x
 2:     x----x
 3:  x----x                    
 4:          x----------x
 5:                         x------x
 6:                                                x--x
 7:                                            x------x
 8:                           x--x
 9:                                      x------------x
10:                         x------x
                                               |
[9]
r = 28
\end{verbatim} }

\defverbatim{\lineBE}
{ \begin{verbatim}
                                               |
 1:  x------------x
 2:     x----x
 3:  x----x                    
 4:          x----------x
 5:                         x------x
 6:                                                x--x
 7:                                            x------x
 8:                           x--x
 9:                                      x------------x
10:                         x------x
                                               |
[7, 9]
r = 28
\end{verbatim} }

\defverbatim{\lineBF}
{ \begin{verbatim}
                                                   |
 1:  x------------x
 2:     x----x
 3:  x----x                    
 4:          x----------x
 5:                         x------x
 6:                                                x--x
 7:                                            x------x
 8:                           x--x
 9:                                      x------------x
10:                         x------x
                                                   |
[7, 9]
r = 28
\end{verbatim} }

\defverbatim{\lineBG}
{ \begin{verbatim}
                                                   |
 1:  x------------x
 2:     x----x
 3:  x----x                    
 4:          x----------x
 5:                         x------x
 6:                                                x--x
 7:                                            x------x
 8:                           x--x
 9:                                      x------------x
10:                         x------x
                                                   |
[6, 7, 9]
r = 28
\end{verbatim} }

\defverbatim{\lineBH}
{ \begin{verbatim}
                                                      |
 1:  x------------x
 2:     x----x
 3:  x----x                    
 4:          x----------x
 5:                         x------x
 6:                                                x--x
 7:                                            x------x
 8:                           x--x
 9:                                      x------------x
10:                         x------x
                                                      |
[6, 7, 9]
r = 28
\end{verbatim} }

\defverbatim{\lineBI}
{ \begin{verbatim}
                                                      |
 1:  x------------x
 2:     x----x
 3:  x----x                    
 4:          x----------x
 5:                         x------x
 6:                                                x--x
 7:                                            x------x
 8:                           x--x
 9:                                      x------------x
10:                         x------x
                                                      |
[7, 9]
r = 28
\end{verbatim} }

\defverbatim{\lineBJ}
{ \begin{verbatim}
                                                      |
 1:  x------------x
 2:     x----x
 3:  x----x                    
 4:          x----------x
 5:                         x------x
 6:                                                x--x
 7:                                            x------x
 8:                           x--x
 9:                                      x------------x
10:                         x------x
                                                      |
[9]
r = 28
\end{verbatim} }

\defverbatim{\lineBK}
{ \begin{verbatim}
                                                      |
 1:  x------------x
 2:     x----x
 3:  x----x                    
 4:          x----------x
 5:                         x------x
 6:                                                x--x
 7:                                            x------x
 8:                           x--x
 9:                                      x------------x
10:                         x------x
                                                      |
[]
r = 28
\end{verbatim} }

\defverbatim{\lineBL}
{ \begin{verbatim}
                                                      |
 1:  x------------x
 2:     x----x
 3:  x----x                    
 4:          x----------x
 5:                         x------x
 6:                                                x--x
 7:                                            x------x
 8:                           x--x
 9:                                      x------------x
10:                         x------x
                                                      |
[]
r = 42
\end{verbatim} }

\env{frame}
{
	\only<all:1>{\lineAA}
	\only<all:2>{\lineAB}
	\only<all:3>{\lineAC}
	\only<all:4>{\lineAD}
	\only<all:5>{\lineAE}
	\only<all:6>{\lineAF}
	\only<all:7>{\lineAG}
	\only<all:8>{\lineAH}
	\only<all:9>{\lineAI}
	\only<all:10>{\lineAJ}
	\only<all:11>{\lineAK}
	\only<all:12>{\lineAL}
	\only<all:13>{\lineAM}
	\only<all:14>{\lineAN}
	\only<all:15>{\lineAO}
	\only<all:16>{\lineAP}
	\only<all:17>{\lineAQ}
	\only<all:18>{\lineAR}
	\only<all:19>{\lineAS}
	\only<all:20>{\lineAU}
	\only<all:21>{\lineAV}
	\only<all:22>{\lineAW}
	\only<all:23>{\lineAX}
	\only<all:24>{\lineAY}
	\only<all:25>{\lineAZ}
	\only<all:26>{\lineBA}
	\only<all:27>{\lineBB}
	\only<all:28>{\lineBC}
	\only<all:29>{\lineBD}
	\only<all:30>{\lineBDD}
	\only<all:31>{\lineBE}
	\only<all:32>{\lineBF}
	\only<all:33>{\lineBG}
	\only<all:34>{\lineBH}
	\only<all:35>{\lineBI}
	\only<all:36>{\lineBJ}
	\only<all:37>{\lineBK}
	\only<all:38>{\lineBL}
}

\env{frame}
{
	\selectcode{code/hlaupabil-daemi2.c}{3}{32}
}

\env{frame}
{
	\env{itemize}
	{
		\item<1-> Við byrjum á að raða í $\mathcal{O}($\onslide<2->{$n \log n$}$)$ tíma.
		\item<3-> Síðan ítrum við í gegnum alla endapunktana sem tekur $\mathcal{O}($\onslide<4->{$\,n\,$}$)$ tíma.
		\item<5-> Svo lausnin hefur tímaflækjuna $\mathcal{O}($\onslide<6->{$n \log n$}$)$.
	}
}

\env{frame}
{
}

\end{document}
