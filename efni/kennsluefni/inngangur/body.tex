\title{Inngangur}
\author{Bergur Snorrason}
\date{\today}

\begin{document}

\frame{\titlepage}

\env{frame}
{
	\frametitle{Námsefni}
	\env{itemize}
	{
		\item<1-> Við munum styðjast lauslega við þriðju útgáfu af bókinni \emph{Competitive Programming}, eftir Steven og Felix Halim.
		\item<2-> Ég er með nokkur eintök af bókinni sem ég get selt ykkur.
		\item<3-> Bókin er frábær og fer í meira efni en við náum í námskeiðinu.
		\item<4-> Í grófum dráttum munum við taka eitt tiltekið efni í hverri viku.
		\item<5-> Vikan byrjar á fyrirlestri og seinni tíminn verður annað hvort fyrirestur eða dæmatími.
		\item<6-> Í hverri viku eru einnig vikuskil (fyrir utan þær vikur sem við höfum keppnir).
		\item<7-> Samtals verða í boð $15$ vikuskil.
		\item<8-> Námskeiðið er staðið/fallið og til að standa þarf $10$ vikuskil.
	}
}

\env{frame}
{
	\frametitle{Námsáætlun}
	\begin{tabular}{l l}
		Dags. & Efni\\
		9. janúar - 11. janúar & Inngangur\\
		16. janúar - 18. janúar & Tímaflækjur, mál og ad hoc\\
		23. janúar - 25. janúar & Tæmandi leit og gráðugar lausnir\\
		30. janúar - 1. febrúar & Deila og drottna og kvik bestun\\
		6. febrúar - 8. febrúar & Kvik bestun\\
		13. febrúar - 15. febrúar & Gagnagrindur\\
		20. febrúar - 18. febrúar & Miðmisseriskeppni\\
		27. febrúar - 1. mars & Netafræði\\
		6. mars - 8. mars & Netfræði\\
		13. mars - 15. mars & Talnafræði\\
		20. mars - 22. mars & Fléttufræði\\
		27. mars - 29. apríl & Rúmfræði\\
		3. apríl & Samansóp og Páskar\\
		12. apríl & Páskar og Lokakeppni\\
	\end{tabular}
}

\env{frame}
{
	\frametitle{Hvað er keppnisforritun?}
	\env{itemize}
	{
		\item<1-> Námskeiðið snýst um að undirbúa ykkur fyrir forritunarkeppnir.
		\item<2-> Að sjálfsögðu má setja fram forritunarkeppnir á marga vegu.
		\item<3-> Algengt er að keppnirnar snúist um að leysa sem flest forritunardæmi á sem stystum tíma.
		\item<4-> Hvert dæmi snýst um að nota þekkt reiknirit og lausnaraðferðir til að leysa almennt verkefni.
		\item<5-> Í námskeiðinu munum við kynnast þessum reikniritum og lausnaraðferðum.
	}
}

\env{frame}
{
	\frametitle{Hvernig eru vikuskilin}
	\env{itemize}
	{
		\item<1-> Í hverri viku verður lagður fyrir dæmalisti.
		\item<2-> Dæmin í listanum munu öll tengjast efni vikunnar.
		\item<3-> Dæmin munum einnig bara byggja á efni sem námsekiðið hefur snert á.
		\item<4-> Hvert dæmi mun hafa tiltekinn stigafjölda (erfiðari dæmi gefa fleiri stig).
		\item<5-> Til að standast vikuskilin þarf að ná vissum stigafjölda.
		\item<6-> Þið þurfið því aldrei að leysa öll dæmin á listanum.
		\item<7-> Síðustu misseri hafa yfirleitt verið átta til tíu dæmi sett fyrir í hverri viku og leysa þurft fjögur til sex léttust til að ná.
	}
}

\env{frame}
{
	\env{itemize}
	{
		\item<1-> Tvær vikur verða engin vikudæmi.
		\item<2-> Þeirra í stað koma keppnir.
		\item<3-> Keppnirnar eru í grófum dráttum svipaðar og vikuskilin.
		\item<4-> Sett eru fyrir nokkur dæmi sem þið eigið að leysa.
		\item<5-> Í stað þessa að hafa viku til að leysa þau, þá hafið þið þrjár til fimm klukkustundir.
		\item<6-> Kröfurnar til að fá skil í keppnunum eru ekki miklar, en einnig verður boðið upp á aukaskil fyrir þá nemendur sem leysa nokkur dæmi.
		\item<7-> Nánari smáatriði verða svo kynnt þegar nær dregur.
	}
}

\env{frame}
{
	\env{itemize}
	{
		\item<1-> Verkefnin eru einstaklingsverkefni.
		\item<2-> Það má ekki deila eða afrita lausnir.
		\item<3-> Ef nemandi er gripinn við slíkt fást ekki skil þá vikuna.
		\item<4-> Ítrekuð brot geta leitt til falls í námskeiðinu.
	}
}

\env{frame}
{
	\frametitle{Skil og yfirferð}
	\env{itemize}
	{
		\item<1-> Til eru mörg dæmasöfn á netinu (til dæmis \texttt{open.kattis.com} og \texttt{codeforces.com}).
		\item<2-> Við munum nýta okkur slík söfn.
		\item<3-> Öll vikudæmin munu koma frá dæmasafninu Kattis.
		\item<4-> Þið munið svo nálgast dæmin á \texttt{hi.kattis.com}.
		\item<5-> Þar skilið þið líka lausnunum ykkar.
		\item<6-> Ég á eftir að klára uppsetninguna og mun senda tilkynningu þegar það er tilbúið.
	}
}

\env{frame}
{
	\env{itemize}
	{
		\item<1-> Lausnir ykkar á dæmunum munu þurfa að lesa af \emph{staðalinntaki} og skrifa á \emph{staðalúttak}.
		\pause

		\begin{tabular}{l l l}
			Forritunarmál & Inntak & Úttak\\
			\hline
			\texttt{C} & \texttt{scanf(...)} & \texttt{printf(...)}\\
			\texttt{C++} & \texttt{cin} & \texttt{cout}\\
			\texttt{Python} & \texttt{input()} & \texttt{print(...)}
		\end{tabular}

		\item<3-> Þetta eru þau forritunarmál sem eru mest notuð í keppnisforritun.
		\item<4-> Í þessu námskeiði munum við, að mestu, útfæra í \texttt{C}/\texttt{C++}.
		\item<5-> Leysum nú saman eitt dæmi.
	}
}

\env{frame}
{
	\env{itemize}
	{
		\item<1-> Tökum dæmið \emph{R2}.
		\item<2-> \href{https://hi.kattis.com/problems/r2}{Það má finna hér}.
		\item<3-> Í grófum dráttum segir dæmið: Þér eru gefnar tvær heiltölur $R_1$ og $S$.
		\item<4-> Einnig er gefið að $S$ er meðaltal $R_1$ og $R_2$, þar sem $R_2$ er einhver önnur heiltala.
		\item<5-> Einnig er gefið að $-1000 \leq R_1, S \leq 1000$.
		\item<6-> Þið eigið svo að finna $R_2$.
	}
}

\env{frame}
{
	\env{itemize}
	{
		\item<1-> Við vitum að
		\[
			S = \frac{R_1 + R_2}{2}.
		\]
		\item<2-> Einangrum og fáum
		\[
			R_2 = 2 \cdot S - R_1.
		\]
		\item<3-> Þetta er þá svarið, en hvernig myndum við forrita þetta?
	}
}

\env{frame}
{
	\frametitle{Útfærsla í \texttt{C}}
	\code{code/r2.c}
}

\env{frame}
{
	\frametitle{Útfærsla í \texttt{C++}}
	\code{code/r2.cpp}
}

\env{frame}
{
	\frametitle{Útfærsla í \texttt{Python}}
	\code{code/r2.py}
}

\env{frame}
{
	\frametitle{Skoðum nú hvernig við sendum þetta inn á Kattis}
}

\env{frame}
{
	\frametitle{Hverju svarar Kattis?}
	\env{itemize}
	{
		\item<1-> Hvað gerist ef lausnin mín er röng?
		\item<2-> Kattis getur gefið nokkur mismunandi svör:
		\item<3-> \emph{Accepted}: Lausnin sé rétt.
		\item<4-> \emph{Compile Error}: Kattis náði ekki að þýða lausnina.
		\item<5-> \emph{Run Time Error}: Lausn kláraði ekki keyrslu eðlilega (krassaði).
		\item<6-> \emph{Time Limit Exceeded}: Lausn kláraði ekki keyrslu nógu hratt.
		\item<7-> \emph{Wrong Answer}: Lausnin svaraði röngu svari.
		\item<8-> Lausnin telst eingöngu rétt ef hún fær svarið \emph{Accepted} frá Kattis.
	}
}

\env{frame}
{
	\env{itemize}
	{
		\item<1-> Takið þó eftir að Kattis gefur ykkur engar frekari upplýsingar.
		\item<2-> Það er upp á ykkur komið að finna út úr því hvað er að lausninni ykkar.
	}
}

\env{frame}
{
	\env{itemize}
	{
		\item<1-> Skoðum eftirfarandi dæmi á \emph{open.kattis.com}:
		\item<2->[]
		\env{tabular}
		{
			{l l}
			Nafn & ID\\
			\hline
			Hello World & hello\\
			Sort Two Numbers & sorttwonumbers\\
			Quadrant Selection & quadrant\\
			Cold-puter Science & cold\\
			Baby Bites & babybites\\
			Guess the Number & guess\\
		}
	}
}

\env{frame}
{
}

\end{document}

