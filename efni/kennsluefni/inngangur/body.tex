\title{Inngangur}
\author{Bergur Snorrason}
\date{\today}

\begin{document}

\frame{\titlepage}

\env{frame}
{
    \frametitle{Námsefni}
    \env{itemize}
    {
        \item<1-> Við munum styðjast við bókina \emph{Competitive Programming}, eftir Steven og Felix Halim.
        %\item<2-> Ég er með nokkur eintök af bókinni sem ég get selt ykkur.
        \item<2-> Bókin er frábær og fer í meira efni en við náum í námskeiðinu.
        \item<3-> Í hverri viku verða heimadæmi og æfingakeppnir.
        \item<4-> Námskeiðið er staðið/fallið.
        \item<5-> Til að fá staðið þarf að standast kröfur í $10$ heimadæmum og $8$ æfingakeppnum,
                    ásamt því að ná lokaprófi.
    }
}

\env{frame}
{
    \frametitle{Námsáætlun}
    \begin{tabular}{l l}
        Dags. & Efni\\
        8. janúar, 10. janúar & Inngangur\\
        15. janúar, 17. janúar & Tímaflækjur, mál og ad hoc\\
        22. janúar, 24. janúar & Tæmandi leit og gráðugar lausnir\\
        29. janúar, 31. janúar & Deila og drottna\\
        5. febrúar, 7. febrúar & Kvik bestun\\
        12. febrúar, 14. febrúar & Kvik bestun\\
        19. febrúar, 21. febrúar & Gagnagrindur\\
        26. febrúar, 28. febrúar & Netafræði\\
        4. mars, 6. mars & Netfræði\\
        11. mars, 13. mars & Talnafræði\\
        18. mars, 20. mars & Fléttufræði\\
        25. mars & Rúmfræði og Páskar\\
        3. apríl & Páskar og Rúmfræði\\
        8. apríl, 10 apríl & Samansóp\\
        15. apríl, 17 apríl & Upprifjun\\
    \end{tabular}
}

\env{frame}
{
    \frametitle{Hvað er keppnisforritun?}
    \env{itemize}
    {
        \item<1-> Námskeiðið snýst um að undirbúa ykkur fyrir forritunarkeppnir.
        \item<2-> Forritunarkeppnir má setja fram á marga vegu.
        \item<3-> Keppnirnar sem við einblínu á snúast um að leysa sem flest forritunardæmi, á sem stystum tíma.
        \item<4-> Hvert dæmi snýst um að nota þekkt reiknirit og lausnaraðferðir til að leysa almennt verkefni.
        \item<5-> Í námskeiðinu munum við kynnast þessum reikniritum og lausnaraðferðum.
    }
}

\env{frame}
{
    \frametitle{Hvernig verða heimadæmin}
    \env{itemize}
    {
        \item<1-> Í hverri viku verður lagður fyrir dæmalisti.
        \item<2-> Dæmin í dæmalistanum munu öll tengjast efni vikunnar.
        \item<3-> Dæmin munum einnig bara byggja á efni sem námsekiðið hefur snert á.
        \item<4-> Hvert dæmi mun hafa tiltekinn stigafjölda (erfiðari dæmi gefa fleiri stig).
        \item<5-> Til að standast vikuskilin þarf að ná vissum stigafjölda.
        \item<6-> Þið þurfið því aldrei að leysa öll dæmin á listanum.
        \item<7-> Síðustu misseri hafa yfirleitt verið átta til tíu dæmi sett fyrir í hverri viku og leysa þurft þrjú til fimm léttust til að ná.
    }
}

\env{frame}
{
    \frametitle{Hvernig verða æfingakeppnirnar}
    \env{itemize}
    {
        \item<1-> Í hverri viku verður einnig haldin æfingakeppni í kennslustund.
        \item<2-> Dæmin í keppnunum verða úr efni vikunnar á undan (fyrir utan fyrstu vikuna).
        \item<3-> Í keppninni verða þrjú dæmi
        \env{itemize}
        {
            \item<4-> Létt dæmi.
            \item<5-> Miðlungs erfitt dæmi, líkt einhverju dæmi sem þið hafið séð áður.
            \item<6-> Erfitt dæmi.
        }
        \item<7-> Þið eigið að leysa tvö þessara dæma og hafið til þess $40$ mínútur.
        \item<8-> Fyrirkomulag keppninnar mun sennilega breytast þegar á líður.
        \item<9-> Takið eftir að keppnirnar eru haldnar í kennslustund og í kennslustofu.
    }
}

\env{frame}
{
    \frametitle{Skil og yfirferð}
    \env{itemize}
    {
        \item<1-> Til eru mörg dæmasöfn á netinu (til dæmis \texttt{open.kattis.com} og \texttt{codeforces.com}).
        \item<2-> Við munum nýta okkur slík söfn.
        \item<3-> Öll vikudæmin munu koma frá dæmasafninu Kattis.
        \item<4-> Þið munið nálgast dæmin á \texttt{hi.kattis.com}.
        \item<5-> Þar skilið þið líka lausnunum ykkar.
        %\item<6-> Ég á eftir að klára uppsetninguna og mun senda tilkynningu þegar það er tilbúið.
    }
}

\env{frame}
{
    \env{itemize}
    {
        \item<1-> Lausnirnar ykkar á dæmunum munu þurfa að lesa af \emph{staðalinntaki} (e. \emph{standard in})
            og skrifa á \emph{staðalúttak} (e. \emph{standard out}).
        \item<2->[]
        \begin{tabular}{l l l}
            Forritunarmál & Inntak & Úttak\\
            \hline
            \texttt{C} & \texttt{scanf(...)} & \texttt{printf(...)}\\
            \texttt{C++} & \texttt{cin} & \texttt{cout}\\
            \texttt{Python} & \texttt{input()} & \texttt{print(...)}
        \end{tabular}

        \item<3-> Þetta eru þau forritunarmál sem eru mest notuð í keppnisforritun.
        \item<4-> Í þessu námskeiði munum við, að mestu, útfæra í \texttt{C}/\texttt{C++}.
        \item<5-> Leysum nú saman eitt dæmi.
    }
}

\env{frame}
{
    \env{itemize}
    {
        \item<1-> Tökum dæmið \emph{R2}.
        \item<2-> {\color{blue} \href{https://hi.kattis.com/problems/r2}{Það má finna hér}}.
        \item<3-> Í grófum dráttum segir dæmið: Þér eru gefnar tvær heiltölur $R_1$ og $S$.
        \item<4-> Einnig er gefið að $S$ er meðaltal $R_1$ og $R_2$, þar sem $R_2$ er einhver önnur heiltala.
        \item<5-> Einnig er gefið að $-1000 \leq R_1, S \leq 1000$.
        \item<6-> Þið eigið svo að finna $R_2$.
    }
}

\env{frame}
{
    \env{itemize}
    {
        \item<1-> Við vitum að
        \[
            S = \frac{R_1 + R_2}{2}.
        \]
        \item<2-> Einangrum og fáum
        \[
            R_2 = 2 \cdot S - R_1.
        \]
        \item<3-> Þetta er þá svarið, en hvernig myndum við forrita þetta?
    }
}

\env{frame}
{
    \frametitle{Útfærsla í \texttt{C}}
    \code{code/r2.c}
}

\env{frame}
{
    \frametitle{Útfærsla í \texttt{C++}}
    \code{code/r2.cpp}
}

\env{frame}
{
    \frametitle{Útfærsla í \texttt{Python}}
    \code{code/r2.py}
}

\env{frame}
{
    \frametitle{Skoðum nú hvernig við sendum þetta inn á Kattis}
}

\env{frame}
{
    \frametitle{Hverju svarar Kattis?}
    \env{itemize}
    {
        \item<1-> Hvað gerist ef lausnin er röng?
        \item<2-> Kattis getur gefið nokkur mismunandi svör:
        \item<3-> \emph{Accepted}: Lausnin sé rétt.
        \item<4-> \emph{Compile Error}: Kattis náði ekki að þýða lausnina.
        \item<5-> \emph{Run Time Error}: Lausnin kláraði ekki keyrslu eðlilega (krassaði).
        \item<6-> \emph{Time Limit Exceeded}: Lausnin kláraði ekki keyrslu nógu hratt.
        \item<7-> \emph{Wrong Answer}: Lausnin svaraði röngu svari.
        \item<8-> Lausnin telst eingöngu rétt ef hún fær svarið \emph{Accepted} frá Kattis.
    }
}

\env{frame}
{
    \env{itemize}
    {
        \item<1-> Takið þó eftir að Kattis gefur ykkur engar frekari upplýsingar.
        \item<2-> Það er upp á ykkur komið að finna út úr því hvað er að lausninni ykkar.
    }
}

\env{frame}
{
    \env{itemize}
    {
        \item<1-> Skoðum nú fyrsta dæmaskamtinn á \texttt{hi.kattis.com}.
    }
}

\env{frame}
{
}

\end{document}

