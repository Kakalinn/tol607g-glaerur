\title{Reiknirit Prims ($1957$)}
\author{Bergur Snorrason}
\date{\today}

\begin{document}

\frame{\titlepage}

\env{frame}
{
	\env{itemize}
	{
		\item<1-> Gerum ráð fyrir að við séum með óstefnt vegið samanhangandi net $G = (V, E)$.
		\item<2-> Ef við viljum finna spannandi tré nægir að framkvæma leit í trénu (til dæmis breiddarleit eða dýptarleit).
		\item<3-> Þar sem við komum aðeins við í hverjum hnút einu sinni ferðumst við aðeins eftir $|V| - 1$ legg.
		\item<4-> Ef við viljum slembið spannandi tré getum við skipt biðröðinni í breiddarleit út fyrir einhverja gagnagrind
					sem skilar alltaf slembnu staki.
	}
}

\env{frame}
{
	\env{itemize}
	{
		\item<1-> Við getum líka beitt aðferð svipaðri reikniriti Dijkstras til að finna minnsta spannandi tré með því að ferðast í netinu.
		\item<2-> Við byrjum á að velja upphafshnút og merkjum hann sem ,,séðan''.
		\item<3-> Þar sem allar hnútirnar munu vera í spannandi trénu skiptir ekki máli hvaða hnút við veljum.
		\item<4-> Við ferðumst svo alltaf eftir þeim legg sem hefur minnsta vigt og tengist nákvæmlega einum ,,séðum'' hnút.
		\item<5-> Við merkjum svo hnútinn sem við ferðuðumst í sem ,,séðan''.
		\item<6-> Þetta er gert þangað til allir hnútar eru ,,séðir''.
	}
}

\env{frame}
{
	\env{center}
	{
		\env{tikzpicture}
		{
			\only<all:1> { \node[draw, circle, thick] (1) at (2,0) {}; }
			\only<all:2-> { \node[draw, circle, thick, blue] (1) at (2,0) {}; }

			\only<all:1-2> { \node[draw, circle, thick] (2) at (2,2) {}; }
			\only<all:3-> { \node[draw, circle, thick, blue] (2) at (2,2) {}; }

			\only<all:1-5> { \node[draw, circle, thick] (3) at (2,-2) {}; }
			\only<all:6-> { \node[draw, circle, thick, blue] (3) at (2,-2) {}; }

			\only<all:1-3> { \node[draw, circle, thick] (4) at (4,1) {}; }
			\only<all:4-> { \node[draw, circle, thick, blue] (4) at (4,1) {}; }

			\only<all:1-4> { \node[draw, circle, thick] (5) at (4,-1) {}; }
			\only<all:5-> { \node[draw, circle, thick, blue] (5) at (4,-1) {}; }

			\only<all:1-6> { \node[draw, circle, thick] (6) at (6,0) {}; }
			\only<all:7-> { \node[draw, circle, thick, blue] (6) at (6,0) {}; }

			\only<all:1-9> { \node[draw, circle, thick] (7) at (6,2) {}; }
			\only<all:10-> { \node[draw, circle, thick, blue] (7) at (6,2) {}; }

			\only<all:1-7> { \node[draw, circle, thick] (8) at (6,-2) {}; }
			\only<all:8-> { \node[draw, circle, thick, blue] (8) at (6,-2) {}; }

			\only<all:1-8> { \node[draw, circle, thick] (9) at (8,0) {}; }
			\only<all:9-> { \node[draw, circle, thick, blue] (9) at (8,0) {}; }




			\only<all:1-2> { \path[draw, thick] (1) -- (2); }
			\only<all:3-11> { \path[draw, thick, blue] (1) -- (2); }
			\node[fill = white] at (2,1) {$1$};

			\only<all:1-3> { \path[draw, thick] (2) -- (4); }
			\only<all:4-> { \path[draw, thick, blue] (2) -- (4); }
			\node[fill = white] at (3,1.5) {$3$};

			\only<all:1-4> { \path[draw, thick] (4) -- (5); }
			\only<all:5-> { \path[draw, thick, blue] (4) -- (5); }
			\node[fill = white] at (4,0) {$1$};

			\only<all:1-5> { \path[draw, thick] (3) -- (5); }
			\only<all:6-> { \path[draw, thick, blue] (3) -- (5); }
			\node[fill = white] at (3,-1.5) {$2$};

			\only<all:1-10> { \path[draw, thick] (1) -- (3); }
			\only<all:1-10> { \node[fill = white] at (2,-1) {$4$}; }

			\only<all:1-6> { \path[draw, thick] (4) -- (6); }
			\only<all:7-> { \path[draw, thick, blue] (4) -- (6); }
			\node[fill = white] at (5,0.5) {$9$};

			\only<all:1-7> { \path[draw, thick] (6) -- (8); }
			\only<all:8-> { \path[draw, thick, blue] (6) -- (8); }
			\node[fill = white] at (6,-1) {$5$};

			\only<all:1-8> { \path[draw, thick] (8) -- (9); }
			\only<all:9-> { \path[draw, thick, blue] (8) -- (9); }
			\node[fill = white] at (7,-1) {$1$};

			\only<all:1-9> { \path[draw, thick] (7) -- (9); }
			\only<all:10-> { \path[draw, thick, blue] (7) -- (9); }
			\node[fill = white] at (7,1) {$5$};

			\only<all:1-10> { \path[draw, thick] (6) -- (9); }
			\only<all:1-10> { \node[fill = white] at (7,0) {$7$}; }

			\only<all:1-10> { \path[draw, thick] (1) -- (5); }
			\only<all:1-10> { \node[fill = white] at (3,-0.5) {$3$}; }
		}
	}
}

\env{frame}
{
	\selectcode{code/prim.cpp}{11}{30}
}

\env{frame}
{
	\env{itemize}
	{
		\item<1-> Burt séð frá nokkrum smáatriðum er þetta reiknirit að gera það sama og reiknirit Dijkstras.
		\item<2-> Svo tímaflækjan er $\mathcal{O}($\onslide<3->{$(V + E) \log E$}$)$.
		\item<4-> Það er algengt að kenna reiknirit Prims, frekar en reiknirit Kruskals, þar sem það notast við forgangsbiðröð.
		\item<5-> Það þekkja mun fleiri forgangsbiðraðir en sammengisleit.
	}
}

\env{frame}
{
}

\end{document}
