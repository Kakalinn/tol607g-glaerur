\title{Hrúgur}
\author{Bergur Snorrason}
\date{\today}

\begin{document}

\frame{\titlepage}

\env{frame}
{
	\frametitle{Hrúgur}
	\env{itemize}
	{
		\item<1-> Rótartvíundatré sem uppfyllir að sérhver nóða er stærri en börnin sín er sagt uppfylla \emph{hrúguskilyrðið}.
		\item<2-> Við köllum slík tré \emph{hrúgur} (e. \emph{heap}).
		\item<3-> Hrúgur eru heppilega auðveldar í útfærslu.
		\item<4-> Við geymum tréð sem fylki og eina erfiðið er að viðhalda hrúguskilyrðinu.
	}
}

\env{frame}
{
	\env{itemize}
	{
		\item<1-> Þegar við geymum tréð sem fylki notum við eina af tveimur aðferðum.
		\item<2-> Sú fyrri:
		\env{itemize}
		{
			\item<3-> Rótin er í staki $1$ í fylkinu.
			\item<4-> Vinstra barn staks $i$ er stak $2 \cdot i$.
			\item<5-> Hægra barn staks $i$ er stak $2 \cdot i + 1$.
			\item<6-> Foreldri staks $i$ er stakið $\lfloor i/2 \rfloor$.
		}
		\item<7-> Sú seinni:
		\env{itemize}
		{
			\item<8-> Rótin er í staki $0$ í fylkinu.
			\item<9-> Vinstra barn staks $i$ er stak $2\cdot i + 1$.
			\item<10-> Hægra barn staks $i$ er stak $2\cdot i + 2$.
			\item<11-> Foreldri staks $i$ er stakið $\lfloor (i - 1)/2 \rfloor$.
		}
		\item<12-> Takið eftir í fyrri aðferðinni notum við ekki stak $0$ í fylkinu.
		\item<13-> Þetta má sjá sem bæði kost og galla.
		\item<14-> Það stak má nota til að geyma, til dæmis, stærðina á trénu.
	}
}

\env{frame}
{
	\env{itemize}
	{
		\item<1-> Bein afleiðing af hrúguskilyrðinu er að rótin er stærsta stakið í trénu.
		\item<2-> Við getum því alltaf fengið skjótan aðgang að stærsta stakinu í trénu.
		\item<3-> Algengt er að \emph{forgangsbiðraðir} (e. \emph{priority queues}) séu útfærðar með hrúgum.
	}
}

\env{frame}
{
	\selectcode{code/heap.c}{8}{41}
}

\env{frame}
{
	\env{itemize}
	{
		\item<1-> Gerum ráð fyrir að við séum með $n$ stök í hrúgunni.
		\item<2-> Þá er hæð trésins $\mathcal{O}($\onslide<3->{$\log n$}$)$.
		\item<4-> Þar sem \texttt{pop()} þarf aðeins að ferðast einu sinni niður að laufi er tímaflækjan $\mathcal{O}($\onslide<5->{$\log n$}$)$.
		\item<6-> Þar sem \texttt{push(...)} þarf aðeins að ferðast einu sinni upp að rót er tímaflækjan $\mathcal{O}($\onslide<7->{$\log n$}$)$.
		\item<8-> Nú þarf \texttt{peek()} ekki að gera annað en að lesa fremsta stakið í fylki svo tímaflækjan er $\mathcal{O}($\onslide<9->{$\,1\,$}$)$.
	}
}

\env{frame}
{
}

\end{document}
