\title{Reiknirit Knuths, Morrisar og Pratts (\texttt{KMP} $1970$)}
\author{Bergur Snorrason}
\date{\today}

\begin{document}

\frame{\titlepage}

\env{frame}
{
	\frametitle{Strengjaleit}
	\env{itemize}
	{
		\item<1-> Gefum okkur langan streng $s$ og styttri streng $p$.
		\item<2-> Hvernig getum við fundið alla hlutstrengi $s$ sem eru jafnir $p$.
		\item<3-> Fyrsta sem manni dettur í hug er að bera $p$ saman við alla hlutstrengi $s$ af sömu lengd og $p$.
	}
}

\env{frame}
{
	\selectcode{code/naive-strengjaleit.c}{5}{14}
}

\env{frame}
{
	\env{itemize}
	{
		\item<1-> Gerum ráð fyrir að strengurinn $s$ sé af lengd $n$ og strengurinn $p$ sé af lengd $m$.
		\item<2-> Fjöldi hlutstrengja í $s$ að lengd $m$ er $n - m + 1$.
		\item<3-> Strengjasamanburðurinn tekur línulegan tíma.
		\item<4-> Svo tímaflækja leitarinnar er $\mathcal{O}($\onslide<5->{$nm - m^2$}$)$.
		\item<6-> Ef $m = n/2$ þá er $nm - m^2 = n^2/2 - n^2/4 = n^2/4$ tímaflækjan er í raun $\mathcal{O}($\onslide<7->{$n^2$}$)$.
		\item<8-> Dæmi um leiðinlega strengi væri $s = \text{,,}aaaaaaaaaaaaaaaa\text{''}$ og $p = \text{,,}aaaaaaab\text{''}$.
		\item<9-> Þessi aðferð virkar þó sæmilega ef strengirnir eru nógu óreglulegir.
		\item<10-> Dæmi um það hvenær þessi aðferð er góð er ef maður er að leita að orði í skáldsögu.
	}
}

\env{frame}
{
	\env{itemize}
	{
		\item<1-> Aðferðin er líka nógu góð ef $\mathcal{O}(n^2)$ er ekki of hægt.
		\item<2-> Það er þó óþarfi að útfæra hana því hún fylgir með flestum forritunarmálum, til dæmis:
		\env{itemize}
		{
			\item<3-> Í \texttt{string.h} í \texttt{C} er \texttt{strstr(..)}.
			\item<4-> Í \texttt{string} í \texttt{C++} er \texttt{find(..)}.
			\item<5-> Í \texttt{String} í \texttt{Java} er \texttt{indexOf(..)}.
		}
		\item<6-> Munið bara að ef $n > 10^4$ er þetta yfirleitt of hægt.
	}
}

\env{frame}
{
	\env{itemize}
	{
		\item<1-> Er einhver leið til að bæta strengjaleitina úr fyrri glærum?
		\item<2-> Skoðum betur sértilfellið $p = \text{,,}aaaabbbb\text{''}$.
		\item<3-> Ef strengjasamanburðurinn misheppnast í $p_3$ þá myndi einfalda strengjaleitin okkar hliðra $p$ um einn og reyna aftur.
		\item<4-> En við vitum að fyrstu þrír stafnirnir í næsta hlutstreng stemma, svo við getum byrjað í $p_2$.
		\item<5-> Reiknirit Knuths, Morrisar og Pratts notar sér þessa hugmynd til að framkvæma strengjaleit.
		\item<6-> Reikniritið byrjar á að forreikna hversu mikið maður veit eftir misheppnaðan samanburð.
		\item<7-> Svo þurfum við einfaldlega að labba í gegnum $s$ og hliðra eins og á við.
	}
}

\env{frame}
{
	\env{itemize}
	{
		\item<1-> Til að finna hversu mikið á að hliðra hverju sinni þurfum við að reikna \emph{forstrengsfall} (e. \emph{prefix function})
					strengsins $p$.
		\item<2-> Við látum $f(j)$, $1 \leq j \leq |p|$, vera gefið með $f(j) = \max\{k \colon s[1,k] = s[j - k + 1, j]\}$.
		\item<3-> Sjáum fyrst að þetta fall uppfyllir $f(j + 1) \leq f(j) + 1$.
		\item<4-> Látum $k = f(j)$ og sjáum að ef $s[j + 1] = s[k]$ þá er $f(j + 1) = k + 1$.
		\item<5-> Ef $s[j + 1] \neq s[k]$ þá þurfum við að minnka $k$ þangað til við fáum jöfnuð.
		\item<6-> Við minnkum $k$ með því að láta $k' = f(k - 1)$.
		\item<7-> Það tekur $\mathcal{O}($\onslide<8->{$\,n\,$}$)$ tíma að reikna öll þessi gildi,
					því $f(j + 1) \leq f(j) + 1$, svo við munum ekki þurfa að minnka $k$ oftar en $n$ sinnum.
	}
}

\env{frame}
{
	\selectcode{code/kmp.c}{12}{30}
}

\env{frame}
{
	\env{itemize}
	{
		\item<1-> Takið eftir að hver ítrun innri lykkjanna svarar til einnar ítrunar ytri lykkjanna.
		\item<2-> Svo innri lykkjan keyrir, í heildina, ekki oftar en ytri lykkjan.
		\item<3-> Því er tímaflækjan í heildina $\mathcal{O}($\onslide<4->{$n + m$}$)$.
	}
}

\env{frame}
{
	\frametitle{Reiknirit Ahos og Corasicks ($1975$)}
	\env{itemize}
	{
		\item<1-> Til er önnur aðferð, svipuð og \texttt{KMP}, sem finnur staðsetningar margra orða í einu í streng.
		\item<2-> Hún er kennd við Aho og Corasick.
		\item<3-> Ég fer ekki í hana hér en hún byggir á því að gera \emph{forstrengstré} (e. \emph{prefix tree}), stundum kallað \emph{trie}, og
					nota kvika bestun til að finna \emph{bakstrengs hlekk} (e. \emph{suffix link}) fyrir hvern hnút.
		\item<4-> Reikniritið keyrir í línulegum tíma í lengd allra strengjanna, ásamt fjölda heppnaðra samanburða,
					að því gefnu að stafrófið sé takmarkað.
	}
}

\env{frame}
{
}

\end{document}
