\title{Reiknirit Knuths, Morrisar og Pratts ($1970$)}
\author{Bergur Snorrason}
\date{\today}

\begin{document}

\frame{\titlepage}

\section{Yfirlit}
\env{frame}
{
    \env{itemize}
    {
        \item<1-> Gefum okkur langan streng $s$ og styttri streng $p$.
        \item<2-> Hvernig getum við fundið alla hlutstrengi $s$ sem eru jafnir $p$.
        \item<3-> Fyrsta sem manni dettur í hug er að bera $p$ saman við alla hlutstrengi $s$ af sömu lengd og $p$.
    }
}

\section{Útfærsla á frumstæðri strengjaleit}
\env{frame}
{
    \selectcode{code/frumstaed-strengjaleit.c}{5}{14}
}

\section{Tímaflækja}
\env{frame}
{
    \env{itemize}
    {
        \item<1-> Gerum ráð fyrir að strengurinn $s$ sé af lengd $n$ og strengurinn $p$ sé af lengd $m$.
        \item<2-> Fjöldi hlutstrengja í $s$ að lengd $m$ er $n - m + 1$.
        \item<3-> Strengjasamanburðurinn tekur línulegan tíma.
        \item<4-> Svo tímaflækja leitarinnar er $\mathcal{O}($\onslide<5->{$nm - m^2$}$)$.
        \item<6-> Ef $m = n/2$ þá er $nm - m^2 = n^2/2 - n^2/4 = n^2/4$ tímaflækjan er í raun $\mathcal{O}($\onslide<7->{$n^2$}$)$.
        \item<8-> Dæmi um leiðinlega strengi væri $s = \text{,,}aaaaaaaaaaaaaaaa\text{''}$ og $p = \text{,,}aaaaaaab\text{''}$.
        \item<9-> Þessi aðferð virkar þó sæmilega ef strengirnir eru nógu óreglulegir.
        \item<10-> Dæmi um það hvenær þessi aðferð er góð er ef maður er að leita að orði í skáldsögu.
    }
}

\section{Innbyggðar útfærslur}
\env{frame}
{
    \env{itemize}
    {
        \item<1-> Aðferðin er líka nógu góð ef $\mathcal{O}(n^2)$ er ekki of hægt.
        \item<2-> Það er þó óþarfi að útfæra hana því hún fylgir með flestum forritunarmálum, til dæmis:
        \env{itemize}
        {
            \item<3-> Í \texttt{string.h} í \texttt{C} er \texttt{strstr(..)}.
            \item<4-> Í \texttt{string} í \texttt{C++} er \texttt{find(..)}.
            \item<5-> Í \texttt{String} í \texttt{Java} er \texttt{indexOf(..)}.
        }
        \item<6-> Munið bara að ef $n > 10^4$ er þetta yfirleitt of hægt.
    }
}

\section{Forstrengsfallið}
\env{frame}
{
    \env{itemize}
    {
        \item<1-> Við getum bætt tímaflækjun með því að nota svokallað \emph{forstrengsfall} (e. \emph{prefix function}) strengsins $a$.
        \item<2-> Forstrengsfallið $f$ er gefið með $f(j) = \max\{k \in \mathbb{N} \colon k < |a|, a[1,k] = a[j - k + 1, j]\}$.
        \item<3-> Sjáum fyrst að þetta fall uppfyllir $f(j + 1) \leq f(j) + 1$.
        \item<4-> Látum $k = f(j)$ og sjáum að ef $a[j + 1] = a[k]$ þá er $f(j + 1) = k + 1$.
        \item<5-> Ef $a[j + 1] \neq a[k]$ þá þurfum við að minnka $k$ þangað til við fáum jöfnuð.
        \item<6-> Við minnkum $k$ með því að láta $k' = f(k - 1)$.
    }
}

\section{Útfærsla}
\env{frame}
{
    \selectcode{code/kmp.c}{12}{17}
}

\section{Tímaflækja}
\env{frame}
{
    \env{itemize}
    {
        \item<1-> Takið eftir að hver ítrun innri lykkjanna svarar til einnar ítrunar ytri lykkjanna.
        \item<2-> Svo innri lykkjan keyrir, í heildina, ekki oftar en ytri lykkjan.
        \item<3-> Því er tímaflækjan í heildina $\mathcal{O}(\onslide<4->{\,n\,})$ fyrir streng af lengd $n$.
    }
}

\section{Strengjaleit með forstrengsfallinu}
\env{frame}
{
    \env{itemize}
    {
        \item<1-> En hvernig getum við notað forstrengsfallið til að framkvæma strengjaleit?
        \item<2-> Gerum ráð fyrir að við séum að leit að streng $p$ í stærri streng $s$.
        \item<3-> Látum þá $a = p + ``\alpha" + s$, þar sem $+$ táknar samskeytingu strengja og $\alpha$ er stafur sem er hvorki í $p$ né $s$.
        \item<4-> Við látum svo $f$ vera forstrengsfall strengsins $a$.
        \item<5-> Við höfum þá að $f \leq |p|$ og $f(j) = |p|$ þá og því aðeins að $j > |p|$ og hlutstrengurinn í $s$ sem byrjar á vísi $j - |p| - 1$
                    og er $|p|$ af lengd er sá sami og $p$.
    }
}

\section{Útfærsla}
\env{frame}
{
    \selectcode{code/kmp.c}{12}{26}
}

\section{Tímaflækja}
\env{frame}
{
    \env{itemize}
    {
        \item<1-> Gerum ráð fyrir að $|s| = n$ og $|p| = m$.
        \item<2-> Þá er $|a| = n + m + 1$, svo tímaflækjan er $\mathcal{O}(\onslide<3->{n + m})$.
    }
}

\section{Þessi glæra er viljandi auð}
\env{frame}
{
}

\end{document}
