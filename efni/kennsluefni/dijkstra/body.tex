\title{Reiknirit Dijkstras ($1959$)}
\author{Bergur Snorrason}
\date{\today}

\begin{document}

\frame{\titlepage}

\section{Lengdir vega í vegnum netum}
\env{frame}
{
    \env{itemize}
    {
        \item<1-> Gerum ráð fyrir að við séum með vegið net $G = (V, E, w)$.
        \item<2-> Látum $u_1, u_2, ..., u_n$ vera vega í netinu $(V, E)$.
        \item<3-> Við segjum þá að \emph{lengdin} á veginum sé
        \[
            \sum_{j = 1}^{n - 1} w(u_j, u_{j + 1}).
        \]
        \item<4-> Við höfum nú áhuga á að vita hvernig við finnum stysta veg milli tiltekinna hnúta.
        \item<5-> Tökum þó eftir einu.
        \item<6-> Þó svo að til sé vegur á milli hnúta þá þarf ekki að vera til stysti vegur.
        \item<7-> Tökum dæmi.
    }
}

\section{Dæmi um net með neikvæðri rás}
\env{frame}
{
    \env{center}
    {
        \env{tikzpicture}
        {
            \node[draw, circle, thick] (1) at (2,0) {};
            \node[draw, circle, thick] (2) at (2,2) {};
            \node[draw, circle, thick] (3) at (2,-2) {};
            \node[draw, circle, thick] (4) at (4,1) {};
            \node[draw, circle, thick] (5) at (4,-1) {};
            \node at (1,1) {$1\ 000\ 000$};
            \node at (1.5,-1) {$-1$};
            \node at (3,-0.2) {$-1$};
            \node at (3,-1.8) {$-1$};
            \node at (3,2) {$1$};
            \node at (4.5,0) {$1$};

            \path[draw] (1) -- (2);
            \path[draw] (2) -- (4);
            \path[draw] (4) -- (5);
            \path[draw] (3) -- (5);
            \path[draw] (1) -- (3);
            \path[draw] (1) -- (5);
        }
    }
}

\section{Neikvæðar rásir}
\env{frame}
{
    \env{itemize}
    {
        \item<1-> Vandamálin myndast þegar við getum gert rásir af neikvæðri lengd.
        \item<2-> Við munum sníða tímabundið framhjá þessu með því að gera ráð fyrir að $w(e) > 0$ gildi fyrir öll $e$ í $E$.
        \item<3-> Algengasta leiðin til að leysa þetta vandamál er með reikniriti Dijkstras.
        \item<4-> Það er ekki ósvipað breiddarleit.
    }
}

\section{Lausn}
\env{frame}
{
    \env{itemize}
    {
        \item<1-> Við merkjum alla hnúta ,,óséða'', nema einn sem við merkjum ,,séðan''.
        \item<2-> Sá hnútur er kallaður \emph{upphafsnúturinn}.
        \item<3-> Við gefum svo öllum hnútum gildi sem upphafstillt er sem $\infty$, nema upphafshnúturinn hefur gildi $0$.
        \item<4-> Þetta gildi svara í raun til stysta vegur sem við höfum fundið hingað til.
        \item<5-> Við endurtökum svo eftirfarandi skref þangað til engir hnútar eru ,,séðir'':
        \env{itemize}
        {
            \item<6-> Tökum þann ,,séða'' hnút $u$ sem hefur minnsta gildi.
            \item<7-> Táknum gildi $u$ með $g$.
            \item<8-> Fyrir alla leggi af gerðinni $e_v = (u, v)$ þá uppfærum við gildið hjá $v$ ef það er stærra en $g + w(e_v)$.
            \item<9-> Þetta þýðir í raun að til sé styttri vegur til $v$ í gegnum $u$.
            \item<10-> Síðan merkjum við $u$ sem ,,kláraðann''.
        }
    }
}

\env{frame}
{
    \env{itemize}
    {
        \item<1-> Tökum eftir að ef $w(e) = 1$ fyrir alla leggi $e \in E$ þá er þetta breiddarleit.
        \item<2-> Þetta reiknirit er gráðugt og við munum ekki sanna að það skili alltaf réttum gildum.
        \item<3-> Reikniritið skilar í raun stysta veg frá upphafshnútnum í alla hnúta.
    }
}

\section{Sýnidæmi}
\env{frame}
{
    \env{center}
    {
        \env{tikzpicture}
        {
            \onslide<all:1>{\node[draw, circle, thick, minimum size = 1cm] (1) at (2,0) {$\infty$};}
            \onslide<all:2>{\node[draw, circle, thick, minimum size = 1cm, yellow] (1) at (2,0) {$0$};}
            \onslide<all:3-8>{\node[draw, circle, thick, minimum size = 1cm, red] (1) at (2,0) {$0$};}
            \onslide<all:9->{\node[draw, circle, thick, minimum size = 1cm, blue] (1) at (2,0) {$0$};}

            \onslide<all:1-3>{\node[draw, circle, thick, minimum size = 1cm] (2) at (2,2) {$\infty$};}
            \onslide<all:4>{\node[draw, circle, thick, minimum size = 1cm, green] (2) at (2,2) {$\infty$};}
            \onslide<all:5-7>{\node[draw, circle, thick, minimum size = 1cm, green] (2) at (2,2) {$1$};}
            \onslide<all:8-9>{\node[draw, circle, thick, minimum size = 1cm, yellow] (2) at (2,2) {$1$};}
            \onslide<all:10-13>{\node[draw, circle, thick, minimum size = 1cm, red] (2) at (2,2) {$1$};}
            \onslide<all:14->{\node[draw, circle, thick, minimum size = 1cm, blue] (2) at (2,2) {$1$};}

            \onslide<all:1-3>{\node[draw, circle, thick, minimum size = 1cm] (3) at (2,-2) {$\infty$};}
            \onslide<all:4-6>{\node[draw, circle, thick, minimum size = 1cm, green] (3) at (2,-2) {$\infty$};}
            \onslide<all:7>{\node[draw, circle, thick, minimum size = 1cm, green] (3) at (2,-2) {$4$};}
            \onslide<all:8-15>{\node[draw, circle, thick, minimum size = 1cm, yellow] (3) at (2,-2) {$4$};}
            \onslide<all:16-17>{\node[draw, circle, thick, minimum size = 1cm, green] (3) at (2,-2) {$4$};}
            \onslide<all:18-24>{\node[draw, circle, thick, minimum size = 1cm, yellow] (3) at (2,-2) {$4$};}
            \onslide<all:25>{\node[draw, circle, thick, minimum size = 1cm, red] (3) at (2,-2) {$4$};}
            \onslide<all:26->{\node[draw, circle, thick, minimum size = 1cm, blue] (3) at (2,-2) {$4$};}

            \onslide<all:1-10>{\node[draw, circle, thick, minimum size = 1cm] (4) at (4,1) {$\infty$};}
            \onslide<all:11>{\node[draw, circle, thick, minimum size = 1cm, green] (4) at (4,1) {$\infty$};}
            \onslide<all:12>{\node[draw, circle, thick, minimum size = 1cm, green] (4) at (4,1) {$10$};}
            \onslide<all:13-15>{\node[draw, circle, thick, minimum size = 1cm, yellow] (4) at (4,1) {$10$};}
            \onslide<all:16>{\node[draw, circle, thick, minimum size = 1cm, green] (4) at (4,1) {$10$};}
            \onslide<all:17>{\node[draw, circle, thick, minimum size = 1cm, green] (4) at (4,1) {$4$};}
            \onslide<all:18-19>{\node[draw, circle, thick, minimum size = 1cm, yellow] (4) at (4,1) {$4$};}
            \onslide<all:20-23>{\node[draw, circle, thick, minimum size = 1cm, red] (4) at (4,1) {$4$};}
            \onslide<all:24->{\node[draw, circle, thick, minimum size = 1cm, blue] (4) at (4,1) {$4$};}

            \onslide<all:1-3>{\node[draw, circle, thick, minimum size = 1cm] (5) at (4,-1) {$\infty$};}
            \onslide<all:4-5>{\node[draw, circle, thick, minimum size = 1cm, green] (5) at (4,-1) {$\infty$};}
            \onslide<all:6-7>{\node[draw, circle, thick, minimum size = 1cm, green] (5) at (4,-1) {$3$};}
            \onslide<all:8-14>{\node[draw, circle, thick, minimum size = 1cm, yellow] (5) at (4,-1) {$3$};}
            \onslide<all:15-18>{\node[draw, circle, thick, minimum size = 1cm, red] (5) at (4,-1) {$3$};}
            \onslide<all:19->{\node[draw, circle, thick, minimum size = 1cm, blue] (5) at (4,-1) {$3$};}

            \onslide<all:1-20>{\node[draw, circle, thick, minimum size = 1cm] (6) at (6,0) {$\infty$};}
            \onslide<all:21>{\node[draw, circle, thick, minimum size = 1cm, green] (6) at (6,0) {$\infty$};}
            \onslide<all:22>{\node[draw, circle, thick, minimum size = 1cm, green] (6) at (6,0) {$7$};}
            \onslide<all:23-26>{\node[draw, circle, thick, minimum size = 1cm, yellow] (6) at (6,0) {$7$};}
            \onslide<all:27-31>{\node[draw, circle, thick, minimum size = 1cm, red] (6) at (6,0) {$7$};}
            \onslide<all:32->{\node[draw, circle, thick, minimum size = 1cm, blue] (6) at (6,0) {$7$};}

            \onslide<all:1-38>{\node[draw, circle, thick, minimum size = 1cm] (7) at (6,2) {$\infty$};}
            \onslide<all:39>{\node[draw, circle, thick, minimum size = 1cm, green] (7) at (6,2) {$\infty$};}
            \onslide<all:40>{\node[draw, circle, thick, minimum size = 1cm, green] (7) at (6,2) {$14$};}
            \onslide<all:41-42>{\node[draw, circle, thick, minimum size = 1cm, yellow] (7) at (6,2) {$14$};}
            \onslide<all:43>{\node[draw, circle, thick, minimum size = 1cm, red] (7) at (6,2) {$14$};}
            \onslide<all:44->{\node[draw, circle, thick, minimum size = 1cm, blue] (7) at (6,2) {$14$};}

            \onslide<all:1-27>{\node[draw, circle, thick, minimum size = 1cm] (8) at (6,-2) {$\infty$};}
            \onslide<all:28>{\node[draw, circle, thick, minimum size = 1cm, green] (8) at (6,-2) {$\infty$};}
            \onslide<all:29-30>{\node[draw, circle, thick, minimum size = 1cm, green] (8) at (6,-2) {$12$};}
            \onslide<all:31-32>{\node[draw, circle, thick, minimum size = 1cm, yellow] (8) at (6,-2) {$12$};}
            \onslide<all:33-36>{\node[draw, circle, thick, minimum size = 1cm, red] (8) at (6,-2) {$12$};}
            \onslide<all:37->{\node[draw, circle, thick, minimum size = 1cm, blue] (8) at (6,-2) {$12$};}

            \onslide<all:1-27>{\node[draw, circle, thick, minimum size = 1cm] (9) at (8,0) {$\infty$};}
            \onslide<all:28-29>{\node[draw, circle, thick, minimum size = 1cm, green] (9) at (8,0) {$\infty$};}
            \onslide<all:30>{\node[draw, circle, thick, minimum size = 1cm, green] (9) at (8,0) {$14$};}
            \onslide<all:31-33>{\node[draw, circle, thick, minimum size = 1cm, yellow] (9) at (8,0) {$14$};}
            \onslide<all:34>{\node[draw, circle, thick, minimum size = 1cm, green] (9) at (8,0) {$14$};}
            \onslide<all:35>{\node[draw, circle, thick, minimum size = 1cm, green] (9) at (8,0) {$13$};}
            \onslide<all:36-37>{\node[draw, circle, thick, minimum size = 1cm, yellow] (9) at (8,0) {$13$};}
            \onslide<all:38-41>{\node[draw, circle, thick, minimum size = 1cm, red] (9) at (8,0) {$13$};}
            \onslide<all:42->{\node[draw, circle, thick, minimum size = 1cm, blue] (9) at (8,0) {$13$};}

            \path[draw] (1) -- (2); \node[fill = white] at (2,1) {$1$};
            \path[draw] (2) -- (4); \node[fill = white] at (3,1.5) {$9$};
            \path[draw] (4) -- (5); \node[fill = white] at (4,0) {$1$};
            \path[draw] (3) -- (5); \node[fill = white] at (3,-1.5) {$2$};
            \path[draw] (1) -- (3); \node[fill = white] at (2,-1) {$4$};
            \path[draw] (4) -- (6); \node[fill = white] at (5,0.5) {$3$};
            \path[draw] (6) -- (8); \node[fill = white] at (6,-1) {$5$};
            \path[draw] (8) -- (9); \node[fill = white] at (7,-1) {$1$};
            \path[draw] (7) -- (9); \node[fill = white] at (7,1) {$1$};
            \path[draw] (6) -- (9); \node[fill = white] at (7,0) {$7$};
            \path[draw] (1) -- (5); \node[fill = white] at (3,-0.5) {$3$};
        }
    }
}

\section{Atriði um útfærslu}
\env{frame}
{
    \env{itemize}
    {
        \item<1-> Við útfærum þetta líkt og breiddarleit, nema í stað biðraðar notum við forgangsbiðröð.
        \item<2-> Við þurfum þó að passa okkur á einu.
        \item<3-> Þegar við uppfærum gildið á hnút $v$ bætum við nýja gildinu í forgangsbiðröð.
        \item<4-> Gamla gildið er þó ennþá í biðröðinni svo við þurfum að passa okkur á ítra þá ekki í gegnum alla nágranna $v$ aftur.
        \item<5-> Við gerum þetta með því að bera saman gildið sem er í forgangsbiðröðinni og besta gildið sem við höfum núþegar fundið.
        \item<6-> Forgangsbiðraðir í \texttt{C++} skila líka alltaf stærsta gildinu.
        \item<7-> Við höfum þó áhuga a minnsta gildinu, svo við skiptum um formerki á tölunum sem við látum inn í forgangsbiðröðina.
    }
}

\section{Útfærsla}
\env{frame}
{
    \selectcode{code/dijkstra.cpp}{9}{27}
}

\section{Tímaflækja}
\env{frame}
{
    \env{itemize}
    {
        \item<1-> Fyrir hvern legg í netinu gætum við þurft að bæta í forgangsbiðröðina.
        \item<2-> Við heimsækjum hver hnút að mestu einu sinni.
        \item<3-> Svo tímaflækjan er $\mathcal{O}($\onslide<4->{$(V + E) \log E$}$)$.
    }
}

\section{Þessi glæra er viljandi auð}
\env{frame}
{
}

\end{document}
