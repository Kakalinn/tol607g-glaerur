\title{Innbyggðar gagnagrindur}
\author{Bergur Snorrason}
\date{\today}

\begin{document}

\frame{\titlepage}

\section{Yfirlit}
\env{frame}
{
    \frametitle{Innbyggðar gagnagrindur í \ilcode{C++}}
    \env{itemize}
    {
        \item<1-> Grunnur \ilcode{C++} býr yfir mörgum öflugum gagnagrindum.
        \item<2-> Skoðum helstu slíku gagnagrindur og tímaflækjur mikilvægust aðgerða þeirra.
        \item<3-> Við munum bara fjalla um gagnagrindurnar í grófum dráttum.
        \item<4-> Það er hægt að finna ítarlegra efni og dæmi um notkun á netinu.
    }
}

\section{Fylki}
\env{frame}
{
    \frametitle{Fylki}
    \env{itemize}
    {
        \item<1-> Lýkt og í mörgum öðrum forritunarmálum eru fylki í \ilcode{C++}.
        \item<2-> Fylki geyma gögn og eru af fastri stærð.
        \item<3-> Þar sem þau eru af fastri stærð má gefa þeim tileinkað, aðliggjandi svæði í minni.
        \item<4-> Þetta leyfir manni að vísa í fylkið í $\mathcal{O}(1)$.
        \item<5->[]
        \env{tabular}
        {
            {l | l}
            Aðgerð & Tímaflækja\\
            \hline
            Lesa eða skrifa ótiltekið stak & $\mathcal{O}(1)$\\
            Bæta staki aftast & $\mathcal{O}(n)$\\
            Skeyta saman tveimur & $\mathcal{O}(n)$\\
        }
    }
}

\section{vector}
\env{frame}
{
    \frametitle{\ilcode{vector}}
    \env{itemize}
    {
        \item<1-> Gagnagrindin \ilcode{vector} er að mestu leiti eins og fylki.
        \item<2-> Það má þó bæta stökum aftan á \ilcode{vector} í $\mathcal{O}(1)$.
        \item<3-> Margir nota bara \ilcode{vector} og aldrei fylki sem slík.
        \item<4->[]
        \env{tabular}
        {
            {l | l}
            Aðgerð & Tímaflækja\\
            \hline
            Lesa eða skrifa ótiltekið stak & $\mathcal{O}(1)$\\
            Bæta staki aftast & $\mathcal{O}(1)$\\
            Skeyta saman tveimur & $\mathcal{O}(n)$\\
        }
    }
}

\section{vector kóði}
\env{frame}
{
    \code{code/vector.cpp}
    \code{code/vector.out}
}

\section{list}
\env{frame}
{
    \frametitle{\ilcode{list}}
    \env{itemize}
    {
        \item<1-> Gagnagrindin \ilcode{list} geymir gögn líkt og fylki gera, en stökin eru ekki aðliggjandi í minni.
        \item<2-> Því er uppfletting ekki hröð.
        \item<3-> Aftur á móti er hægt að gera smávægilegar breytingar á \ilcode{list} sem er ekki hægt að gera á fylkjum.
        \item<4->[]
        \env{tabular}
        {
            {l | l}
            Aðgerð & Tímaflækja\\
            \hline
            Finna stak & $\mathcal{O}(n)$\\
            Bæta staki aftast & $\mathcal{O}(1)$\\
            Bæta staki fremst & $\mathcal{O}(1)$\\
            Bæta staki fyrir aftan tiltekið stak & $\mathcal{O}(1)$\\
            Bæta staki fyrir framan tiltekið stak & $\mathcal{O}(1)$\\
            Skeyta saman tveimur & $\mathcal{O}(1)$\\
        }
    }
}

\section{list kóði}
\env{frame}
{
    \code{code/list.cpp}
    \code{code/list.out}
}

\section{stack}
\env{frame}
{
    \frametitle{\ilcode{stack}}
    \env{itemize}
    {
        \item<1-> Gagnagrindin \ilcode{stack} geymir gögn og leyfir aðgang að síðasta staki sem var bætt við.
        \item<2->[]
        \env{tabular}
        {
            {l | l}
            Aðgerð & Tímaflækja\\
            \hline
            Bæta við staki & $\mathcal{O}(1)$\\
            Lesa nýjasta stakið & $\mathcal{O}(1)$\\
            Fjarlægja nýjasta stakið  & $\mathcal{O}(1)$\\
        }
    }
}

\section{stack kóði}
\env{frame}
{
    \code{code/stack.cpp}
    \code{code/stack.out}
}

\section{queue}
\env{frame}
{
    \frametitle{\ilcode{queue}}
    \env{itemize}
    {
        \item<1-> Gagnagrindin \ilcode{queue} geymir gögn og leyfir aðgang að fyrsta stakinu sem var bætt við.
        \item<2->[]
        \env{tabular}
        {
            {l | l}
            Aðgerð & Tímaflækja\\
            \hline
            Bæta við staki & $\mathcal{O}(1)$\\
            Lesa elsta stakið & $\mathcal{O}(1)$\\
            Fjarlægja elsta stakið  & $\mathcal{O}(1)$\\
        }
    }
}

\section{queue kóði}
\env{frame}
{
    \code{code/queue.cpp}
    \code{code/queue.out}
}

\section{set}
\env{frame}
{
    \frametitle{\ilcode{set}}
    \env{itemize}
    {
        \item<1-> Gagnagrindin \ilcode{set} geymir gögn án endurtekninga og leyfir hraða uppflettingu.
        \item<2->[]
        \env{tabular}
        {
            {l | l}
            Aðgerð & Tímaflækja\\
            \hline
            Bæta við staki & $\mathcal{O}(\log n)$\\
            Fjarlægja stak & $\mathcal{O}(\log n)$\\
            Gá hvort staki hafi verið bætt við  & $\mathcal{O}(\log n)$\\
        }
    }
}

\section{set kóði}
\env{frame}
{
    \code{code/set.cpp}
    \code{code/set.out}
}

\section{Þessi glæra er viljandi auð}
\env{frame}
{
}

\end{document}
