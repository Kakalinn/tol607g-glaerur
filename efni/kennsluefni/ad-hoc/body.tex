\title{Ad hoc}
\author{Bergur Snorrason}
\date{\today}

\begin{document}

\frame{\titlepage}

\section{Yfirlit yfir lausnaraðferðir}
\env{frame}
{
    \frametitle{Lausnaraðferðir}
    \env{itemize}
    {
        \item<1-> Þegar við leysum dæmi í keppnisforritun notumst við oftast við eina af eftirfarandi aðferðum:
        \env{itemize}
        {
            \item<2-> \emph{Ad hoc},
            \item<3-> \emph{Tæmandi leit} eða \emph{ofbeldis aðferðin} (e. \emph{complete search, brute force}),
            \item<4-> \emph{Gráðug reiknirit} (e. \emph{greedy algorithms}),
            \item<5-> \emph{Deila og drottna} (e. \emph{divide and conquer}),
            \item<6-> \emph{Kvik bestun} (e. \emph{dynamic programming}).
        }
        \item<7-> Þessi skipting er ekki fullkomin, en hún hjálpar okkur að ræða dæmi.
        \item<8-> Við munum byrja á því að fjalla almennt um þessar aðferðir og fara svo í sértækara efni.
        \item<9-> Þá er oft gott að hafa í huga hvernig flokka megi reikniritin.
    }
}

\section{Hvað eru ad hoc dæmi?}
\env{frame}
{
    \frametitle{Ad hoc}
    \env{itemize}
    {
        \item<1-> Ef lausn dæmisins byggir ekki á sérþekkingu flokkast dæmið sem \emph{Ad hoc}.
        \item<2-> Þessi dæmi eru stundum flokkuð undir \emph{útfærsludæmi} (e. \emph{implementation}).
        \item<3-> Þetta er gert því flest Ad hoc dæmi snúast um að fylgja beint leiðbeiningum.
        \item<4-> Það eru þó undantekningar.
        \item<5-> Í NCPC $2020$ var Ad hoc dæmi sem mætti ekki flokkast sem útfærsludæmi.
        \item<6-> Léttari dæmin í keppnum eru oft Ad hoc dæmi.
        \item<7-> Áðurnefnt NCPC dæmi er þó aftur undanteking, því engin keppandi náði að leysa það í keppninni.
        \item<8-> Samkvæmt skilgreiningu getum við ekki rætt Ad hoc dæmi ítarlega. Tökum því nokkur dæmi.
    }
}

\section{Dæmi: Blandað brot}
\subsection{Lýsing}
\env{frame}
{
    \frametitle{Blandað brot}
    \env{itemize}
    {
        \item<1-> Þú átt að breyta almennu broti í blandað brot.
        \item<2-> Munið að almenna brotið $p/q$, og blandaða brotið $a\ b/c$ tákna sömu töluna ef $p/q = a + b/c$.
        \item<3-> Munið einnig að ef $a\ b/c$ er almennt brot þá gildir $b < c$.
        \item<4-> Blandaða brotið ykkar á að hafa sama nefnara og upprunarlega brotið.
        \item<5-> Inntakið inniheldur tvær heiltölur $1 \leq p, q \leq 10^9$.
        \item<6-> Úttakið skal innihalda blandaða brotið sem svarar til $p/q$.
        \item<7->[]
        \env{tabular}
        {
            {l | l | l}
            & Inntak & Úttak\\
            \hline
            Sýnidæmi 1 & \ilcode{27 12} & \ilcode{2 3 / 12}\\
            Sýnidæmi 2 & \ilcode{2460000 98400} & \ilcode{25 0 / 98400}\\
            Sýnidæmi 3 & \ilcode{3 4000} & \ilcode{0 3 / 4000}\\
        }
    }
}

\subsection{Lausn}
\env{frame}
{
    \frametitle{Lausn á blandað brot}
    \env{itemize}
    {
        \item<1-> Hér nægir okkur að reikna.
        \item<2-> Við getum aðeins stytt okkur leið með því að nota heiltöludeilingu.
        \item<3-> Við fáum þá að $a$ er heiltalan sem fæst með deilingunni $p/q$ og $b$ er afgangurinn.
    }
}

\subsection{Útfærsla}
\env{frame}
{
    \onslide<1->{\code{code/mixedfractions1.c}}
    \onslide<2->{\code{code/mixedfractions2.c}}
}

\section{Dæmi: Barnahjal}
\subsection{Lýsing}
\env{frame}
{
    \frametitle{Barnahjal}
    \env{itemize}
    {
        \item<1-> Þið eruð að reyna að kenna barni að telja.
        \item<2-> Það er þó ekki alltaf hægt að heyra hvað barnið segir.
        \item<3-> Þið viljið ákvarða hvort það sem barnið er að segja gæti mögulega verið rétt.
        \item<4-> Fyrsta lína inntaksins inniheldur heiltölu $1 \leq n \leq 10^3$.
        \item<5-> Síðan fylgir ein lína með $n$ strengjum.
        \item<6-> Hver strengur er annaðhvort heiltala á bilinu $[0, 10^4]$ eða strengurinn ``mumble''.
        \item<7-> Ef það er hægt að skipta út öllum ``mumble'' fyrir tölu þannig að talningin sé rétt skal prenta ``jebb''.
        \item<8-> Annars skal prenta ``neibb''.
    }
}

\subsection{Lausn}
\env{frame}
{
    \frametitle{Lausn á Barnahjal}
    \env{itemize}
    {
        \item<1-> Ef $i$-ti strengurinn inniheldur strenginn sem svarar til tölurnnar $i$ eða ``mumble'', fyrir öll $i$,
            þá er barnið kannski að telja rétt.
        \item<2-> Annars er barnið að telja rangt.
    }
}

\subsection{Útfærsla}
\env{frame}
{
    \frametitle{Útfærsla í Python}
    \code{code/babybites.py}
}

\section{Þessi glæra er viljandi auð}
\env{frame}
{
}

\end{document}
