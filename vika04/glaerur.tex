\usepackage[T1]{fontenc}
\usepackage[utf8]{inputenc}
%\usepackage[english]{babel}
\usepackage{xcolor}
\usepackage{amsmath}
\usepackage{amssymb}
\usepackage{amsthm}
\usepackage{mathtools}
\usepackage{listings}
\usepackage{hyperref}
\usepackage{color}
\usepackage{tikz}
\usetikzlibrary{angles, shapes}
\usepackage{array}

\setbeamertemplate{footline}{% 
  \hfill% 
  \insertframenumber%
  %\,/\,\inserttotalframenumber
  \kern2em\vskip4pt% 
}

\DeclareMathOperator{\lcm}{lcm}

\newcommand\env[2]
{
    \begin{#1}
    #2
    \end{#1}
}
\newcommand\varenv[3]
{
    \begin{#1}[#2]
    #3
    \end{#1}
}
\newcommand\slidewidth[1]
{
    \resizebox{\textwidth}{!}{#1}
}

% grá code blocks
\newcommand{\ilcode}[1]{\colorbox{gray!20}{\color{black} \texttt{#1}}}

\newcommand\code[1]{{\tiny\lstinputlisting[language = C, numbers = left]{#1}}}
\newcommand\selectcode[3]{{\tiny\lstinputlisting[language = C, numbers = left, firstnumber = #2, linerange = {#2 - #3}]{#1}}}

\definecolor{mygray}{rgb}{0.4,0.4,0.4}
\definecolor{mygreen}{rgb}{0, 0, 1}
\definecolor{myorange}{rgb}{1.0,0.4,0}

\lstset{
    keepspaces,
    breaklines       = false,
    commentstyle     = \color{mygray},
    numbersep        = 5pt,
    numberstyle      = \tiny\color{mygray},
    keywordstyle     = \color{mygreen},
    showspaces       = false,
    showstringspaces = false,
    stringstyle      = \color{myorange},
    tabsize          = 4
}
\lstset{literate=
{æ}{{\ae}}1
{í}{{\'{i}}}1
{ó}{{\'{o}}}1
{á}{{\'{a}}}1
{é}{{\'{e}}}1
{ú}{{\'{u}}}1
{ý}{{\'{y}}}1
{ð}{{\dh}}1
{þ}{{\th}}1
{ö}{{\"o}}1
{Á}{{\'{A}}}1
{Í}{{\'{I}}}1
{Ó}{{\'{O}}}1
{Ú}{{\'{U}}}1
{Æ}{{\AE}}1
{Ö}{{\"O}}1
{Ø}{{\O}}1
{Þ}{{\TH}}1
}




\title{Tæmandi leit og gráðug reiknirit}
\author{Bergur Snorrason}
\date{\today}

\begin{document}

\frame{\titlepage}

\env{frame}
{
	\frametitle{Almennar nálganir lausna}
	\env{itemize}
	{
		\item<1-> Þegar við leysum dæmi í keppnisforritun notumst við oftast við eina af eftirfarandi aðferðum:
		\env{itemize}
		{
			\item<2-> \emph{Ad hoc}.
			\item<3-> \emph{Tæmandi leit} eða \emph{ofbeldis aðferðin} (e. \emph{complete search, brute force}),
			\item<4-> \emph{Gráðug reiknirit} (e. \emph{greedy algorithms}),
			\item<5-> \emph{Deila og drottna} (e. \emph{divide and conquer}),
			\item<6-> \emph{Kvik bestun} (e. \emph{dynamic programming}).
		}
		\item<7-> Í síðustu vikum fjölluðum við um Ad hoc dæmi, tæmandi leit og gráðug reiknirit.
		\item<8-> Í þessari viku fjöllum við um deila og drottna reiknirit og kvika bestun.
	}
}

\env{frame}
{
	\frametitle{Deila og drottna}
	\env{itemize}
	{
		\item<1-> Sum dæmi má endurkvæmt skipta upp þangað til þau verða fáfengileg.
		\item<2-> Síðan má líma fáfengilegu lausnirnar saman í heildarlausn í lokinn.
		\item<3-> Slík reiknirit kallast \emph{deila og drottna} reiknirit.
		\item<4-> Þessi flokkur er sjaldgæfastur.
		\item<5-> Það eru þó mörg þekkt reiknirit sem nýta sér deila og drottna.
	}
}

\env{frame}
{
	\env{itemize}
	{
		\item<1-> 
	}
}


\end{document}


