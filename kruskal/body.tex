\title{Reiknirit Kruskals ($1956$)}
\author{Bergur Snorrason}
\date{\today}

\begin{document}

\frame{\titlepage}

\env{frame}
{
	\frametitle{Spannandi tré}
	\env{itemize}
	{
		\item<1-> Gerum ráð fyrir að við séum með samanhangandi óstenft net $G = (V, E)$.
		\item<2-> Munið að net kallast tré ef það er samanhanhandi og órásað.
		\item<3-> Auðvelt er að sýna óbeint að við getum gert $G$ að tréi með því að fjarlægja leggi.
		\item<4-> Einnig má sýna að tré uppfyllir alltaf $|E| = |V| - 1$.
		\item<5-> Ef $G'$ er tré sem fæst með því að fjarlægja leggi úr $G$ þá köllum við $G'$ \emph{spannandi tré $G$} (e. \emph{spanning tree}).
	}
}

\env{frame}
{
	\frametitle{Minnsta spannandi tré}
	\env{itemize}
	{
		\item<1-> Ef $G = (V, E, w)$ er vegið net og $G' = (V', E')$ er spannandi tré netsins $(V, E)$ þá segjum við að \emph{stærðin}
					á spannandi trénu sé
		\[
			S(G') = \sum_{e \in E'} w(e).
		\]
		\item<2-> Hvernig förum við að því að finna $G'$ þannig að $S(G')$ sé sem minnst.
		\item<3-> Slíkt $G'$ er kallað \emph{minnsta spannandi tré netsins $G$} (e. \emph{minimum spanning tree}),
					þó svo að það sé ekki ótvírætt ákvarðað.
		\item<4-> Við getum fundið minnsta spannandi tré gráðugt.
	}
}

\env{frame}
{
	\frametitle{Kruskal}
	\env{itemize}
	{
		\item<1-> Við getum lýst aðferðinni í einni málgrein.
		\item<2-> Við bætum alltaf við minnsta leggnum sem myndar ekki rás.
		\item<3-> Hvernig getum við gert þetta á hagkvæman hátt.
		\item<4-> Við byrjum með net með $|V|$ hnúta en enga leggi.
		\item<5-> Gerum ráð fyrir að við höfum bætt við nokkrum leggjum sem mynda ekki rás.
		\item<6-> Ef við viljum bæta við leggnum $(u, v)$ þá erum við í raun að sameina samhengisþættina sem hnútarnir $u$ og $v$ tilheyra.
		\item<7-> Ef þeir tilheyra sama samhengisþætti þá myndast rás við það að bæta við leggnum.
		\item<8-> Svo við getum notað \onslide<9->{sammengisleit} til að segja til um hvort leggur myndi rás.
	}
}

\env{frame}
{
	\env{center}
	{
		\env{tikzpicture}
		{
			\node[draw, circle, thick] (1) at (2,0) {};
			\node[draw, circle, thick] (2) at (2,2) {};
			\node[draw, circle, thick] (3) at (2,-2) {};
			\node[draw, circle, thick] (4) at (4,1) {};
			\node[draw, circle, thick] (5) at (4,-1) {};
			\node[draw, circle, thick] (6) at (6,0) {};
			\node[draw, circle, thick] (7) at (6,2) {};
			\node[draw, circle, thick] (8) at (6,-2) {};
			\node[draw, circle, thick] (9) at (8,0) {};

			\only<1> { \path[draw, thick] (1) -- (2); }
			\only<2> { \path[draw, thick, red] (1) -- (2); }
			\only<3-24> { \path[draw, thick, blue] (1) -- (2); }
			\node[fill = white] at (2,1) {$1$};

			\only<1-9> { \path[draw, thick] (2) -- (4); }
			\only<10> { \path[draw, thick, red] (2) -- (4); }
			\only<11-> { \path[draw, thick, blue] (2) -- (4); }
			\node[fill = white] at (3,1.5) {$3$};

			\only<1-3> { \path[draw, thick] (4) -- (5); }
			\only<4> { \path[draw, thick, red] (4) -- (5); }
			\only<5-> { \path[draw, thick, blue] (4) -- (5); }
			\node[fill = white] at (4,0) {$1$};

			\only<1-7> { \path[draw, thick] (3) -- (5); }
			\only<8> { \path[draw, thick, red] (3) -- (5); }
			\only<9-> { \path[draw, thick, blue] (3) -- (5); }
			\node[fill = white] at (3,-1.5) {$2$};

			\only<1-13> { \path[draw, thick] (1) -- (3); }
			\only<14> { \path[draw, thick, red] (1) -- (3); }
			\only<15-23> { \path[draw, thick, yellow] (1) -- (3); }
			\only<1-23> { \node[fill = white] at (2,-1) {$4$}; }

			\only<1-21> { \path[draw, thick] (4) -- (6); }
			\only<22> { \path[draw, thick, red] (4) -- (6); }
			\only<23-> { \path[draw, thick, blue] (4) -- (6); }
			\node[fill = white] at (5,0.5) {$9$};

			\only<1-15> { \path[draw, thick] (6) -- (8); }
			\only<16> { \path[draw, thick, red] (6) -- (8); }
			\only<17-> { \path[draw, thick, blue] (6) -- (8); }
			\node[fill = white] at (6,-1) {$5$};

			\only<1-5> { \path[draw, thick] (8) -- (9); }
			\only<6> { \path[draw, thick, red] (8) -- (9); }
			\only<7-> { \path[draw, thick, blue] (8) -- (9); }
			\node[fill = white] at (7,-1) {$1$};

			\only<1-17> { \path[draw, thick] (7) -- (9); }
			\only<18> { \path[draw, thick, red] (7) -- (9); }
			\only<19-> { \path[draw, thick, blue] (7) -- (9); }
			\node[fill = white] at (7,1) {$5$};

			\only<1-19> { \path[draw, thick] (6) -- (9); }
			\only<20> { \path[draw, thick, red] (6) -- (9); }
			\only<21-23> { \path[draw, thick, yellow] (6) -- (9); }
			\only<1-23> { \node[fill = white] at (7,0) {$7$}; }

			\only<1-11> { \path[draw, thick] (1) -- (5); }
			\only<12> { \path[draw, thick, red] (1) -- (5); }
			\only<13-23> { \path[draw, thick, yellow] (1) -- (5); }
			\only<1-23> { \node[fill = white] at (3,-0.5) {$3$}; }
		}
	}
}

\env{frame}
{
	\env{itemize}
	{
		\item<1-> Við höfum ekki áhuga á nágrönnum nóða heldur vigtum á leggjum svo við notum leggjalista í útfærslunni okkar.
		\item<2-> Við byrjum á að raða leggjalistanum eftir vigt leggjanna.
		\item<3-> Við göngum síðan á leggina og:
		\env{itemize}
		{
			\item<4-> Gerum ekkert ef leggurinn myndar rás (\texttt{find(u) == find(v)}).
			\item<5-> Bætum leggnum í spannandi tréð ef hann myndar ekki rás og sameinum í sammengisleitinn (\texttt{join(u, v)}).
		}
	}
}

\env{frame}
{
	\env{itemize}
	{
		\item<1-> Þetta reiknirit skilar alltaf spannandi tré, en það er meiri vinna að sýna að það sé ekki til minna spannandi tré.
		\item<2-> Við munum ekki týnast í slíkum smáatriðum hér.
	}
}

\env{frame}
{
	\selectcode{code/kruskal.c}{23}{35}
}

\env{frame}
{
	\env{itemize}
	{
		\item<1-> Það fyrsta sem við gerum er að raða leggjunum, sem við gerum í $\mathcal{O}($\onslide<2->{$E \log E$}$)$ tíma.
		\item<3-> Síðan ítrum við í gegn leggina og framkvæmum fastann fjölda af sammengisleitaraðgerðum fyrir hvern legg, sem tekur
					$\mathcal{O}($\onslide<4->{$E \alpha(V)$}$)$ tíma.
		\item<5-> Saman er þetta því $\mathcal{O}($\onslide<6->{$E \log E + E \alpha(V)$}$) = \mathcal{O}($\onslide<7->{$E \log E$}$)$.
	}
}

\env{frame}
{
}

\end{document}
