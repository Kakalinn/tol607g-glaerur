\title{Sniðmát}
\author{Bergur Snorrason}
\date{\today}

\begin{document}

\frame{\titlepage}

\env{frame}
{
	\frametitle{Sniðmát}
	\env{itemize}
	{
		\item<1-> Þessi mappa á að skila af sér sniðmát af því hvernig má setja upp glærur fyrir kennslu.
		\item<2-> Einn galli þess að láta punktana tínast inn, einn í einu, er að það er leiðinlegt að lesa yfir það eftir á.
		\item<3-> Þessi mappa sér um það.
		\item<4-> Nota má \texttt{./build.sh glaerur} til að búa til kennsluglærur.
		\item<5-> Nota má \texttt{./build.sh handout} til að búa til yfirlestrarglærur.
	}
}

\env{frame}
{
	\frametitle{Tína má inn á í \texttt{handout}}
	\env{itemize}
	{
		\item<all:1-> Stundum viljum við tína inn á glærur, jafnvel við yfirlestur.
		\item<all:2-> Þetta er aðalega þegar teikna á myndir, skref fyrir skref, fyrir nemendur.
		\item<all:3-> Eins og sést á þessari glæru, má þetta hæglega gera.
	}
}

\end{document}
