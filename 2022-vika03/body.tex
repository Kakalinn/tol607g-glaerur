\title{Lausnir á völdum dæmum úr viku þrjú}
\author{Bergur Snorrason}
\date{\today}

\begin{document}

\frame{\titlepage}

\env{frame}
{
	\env{itemize}
	{
		\item<1-> Ég mun leysa eftirfarandi dæmi:
		\env{itemize}
		{
			\item<2-> \emph{Veci},
			\item<3-> \emph{HKIO},
			\item<4-> \emph{Planetaris}.
		}
	}
}

\env{frame}
{
	\frametitle{Veci}
	\env{itemize}
	{
		\item<1-> Okkur er gefin tala $1 \leq n < 10^6$.
		\item<2-> Hver er minnsta talan stærri en $n$ sem inniheldur nákvæmlega sömu tölustafi?
		\item<3-> Ef engin slík tala er til er svarið \texttt{0}.
		\item<4-> \texttt{156 -> 165}.
		\item<5-> \texttt{330 -> 0}.
		\item<6-> \texttt{27711 -> 71127}.
	}
}

\env{frame}
{
	\frametitle{Veci}
	\env{itemize}
	{
		\item<1-> Það er freystandi að reyna að útfæra gráðuga lausn.
		\item<2-> En þess er ekki þörf.
		\item<3-> Tökum eftir að svarið hefur aldrei fleiri tölustafi en inntakið.
		\item<4-> Svo svarið er minna en $10^6$.
		\item<5-> Okkur nægir því að skoða allar heiltölur á bilinu $[n + 1, 10^6]$.
		\item<6-> En hvernig gáum við hvort tvær tölur hafi sömu tölustafi?
	}
}

\env{frame}
{
	\frametitle{Veci}
	\env{itemize}
	{
		\item<1-> Látum \texttt{x} vera heiltölu.
		\item<2-> Þá getum við fundið aftasta tölustafinn í \texttt{x} með \texttt{x\%10}.
		\item<3-> Við getum svo fjarlægt aftasta stafinn í \texttt{x} með \texttt{x/10}.
		\item<4-> Við fáum því eftirfarandi.
	}
}

\env{frame}
{
	\frametitle{Veci}
	\selectcode{code/veci.c}{3}{11}
}

\env{frame}
{
	\frametitle{Veci}
	\env{itemize}
	{
		\item<1-> Við þurfum nú bara að ítra í gegnum heiltölurnar á $[n + 1, 10^6]$.
	}
}

\env{frame}
{
	\frametitle{Veci}
	\selectcode{code/veci.c}{13}{18}
}

\env{frame}
{
	\frametitle{Veci}
	\env{itemize}
	{
		\item<1-> Í heildina verður þetta:
	}
}

\env{frame}
{
	\frametitle{Veci}
	\code{code/veci.c}
}

\env{frame}
{
	\frametitle{Veci}
	\env{itemize}
	{
		\item<1-> Tökum eftir að við ítrum í gegnum $\mathcal{O}($\onslide<2->{$\,n\,$}$)$ tölur.
		\item<3-> Tímaflækjan á samanburðinum er línulegur í lengd tölustafanna, það er að segja $\mathcal{O}($\onslide<4->{$\log n$}$)$.
		\item<5-> Svo tímaflækjan í heildina er $\mathcal{O}($\onslide<6->{$n \log n$}$)$.
	}
}

\env{frame}
{
	\frametitle{Veci}
	\env{itemize}
	{
		\item<1-> Þetta dæmi má útfæra á eftirfarandi hátt í \texttt{Python}.
	}
}

\env{frame}
{
	\frametitle{Veci}
	\code{code/veci.py}
}

\env{frame}
{
	\frametitle{HKIO}
	\env{itemize}
	{
		\item<1-> Gefnar eru $n$ heiltölur $a_1, \dots, a_n$.
		\item<2-> Finnið $j \leq k$ þannig að meðaltalið $\operatorname{avg}(a_j, a_{j + 1}, \dots, a_k)$ sé hámarkað.
		\item<3-> Gefið er að $1 \leq n \leq 10^5$.
	}
}

\env{frame}
{
	\frametitle{HKIO}
	\env{itemize}
	{
		\item<1-> Látum $m$ vera heiltölu þannig að $a_m = \max(a_1, \dots, a_n)$.
		\item<2-> Takið þá eftir að 
		\env{align*}
		{
				\operatorname{avg}(a_j, a_{j + 1}, \dots, a_k)
				&= \frac{a_j + a_{j + 1} + \dots + a_k}{k - j + 1}\\
				&\leq \frac{a_m + a_m + \dots + a_m}{k - j + 1}\\
				&= \frac{(k - j + 1) \cdot a_m}{k - j + 1}\\
				&= a_m.
		}
		\item<3-> Svo meðaltalið verður aldrei stærra en $a_m$.
		\item<4-> En einnig gildir að $\operatorname{avg}(a_m) = a_m$.
		\item<5-> Svo okkur nægir að finna stærstu töluna í listanum.
	}
}

\env{frame}
{
	\frametitle{HKIO}
	\code{code/hkio.c}
}

\env{frame}
{
	\frametitle{Planetaris}
	\env{itemize}
	{
		\item<1-> Atli og Finnur eru að spila tölvuleik sem snýst um að fanga sólkerfi.
		\item<2-> Í leiknum eru $1 \leq n \leq 10^5$ sólkerfi.
		\item<3-> Atli og Finnur senda einhvern fjölda skipa sinna á hvert sólkerfi.
		\item<4-> Atli fangar tiltekið sólkerfi ef hann sendir strangt fleiri skip þangað.
		\item<5-> Atli hefur $a$ skip og veit að Finnur mun senda $e_i$ skip á $i$-ta skólkerfið.
		\item<6-> Hver er mesti fjöldi sólkerfa sem Atli getur fangað?
	}
}

\env{frame}
{
	\frametitle{Planetaris}
	\env{itemize}
	{
		\item<1-> Við græðum jafn mikið að fanga hvert sólkerfi, svo það er best að fanga þau sólkerfi sem Finnur sendir fá skip á.
		\item<2-> Við föngum því einfaldlega sólkerfin í röð, byrjum á því sem Finnur sendir fæst skip á,
					svo næst það sem hann sendir næst fæst skip á, og svo framvegis.
		\item<3-> Þegar við föngum $i$-ta sólkerfið verðum við að passa að senda $e_i + 1$ skip, til að það verði ekki jafntefli.
		\item<4-> Við verðum líka að passa að hætta að fanga sólkerfi þegar við höfum ekki nóg af skipum.
	}
}

\env{frame}
{
	\frametitle{Planetaris}
	\selectcode{code/planetaris.c}{10}{19}
}

\env{frame}
{
	\frametitle{Planetaris}
	\env{itemize}
	{
		\item<1-> Tímaflækjan á þessari lausn er $\mathcal{O}($\onslide<2->{$n \log n$}$)$ sökum \onslide<2->{röðunar}.
		\item<3-> Takið eftir að það er mjög auðvelt að gera litlar villur sem gera lausnin ranga.
		\item<4-> Til dæmis fær eftirfarandi lausn rétt í sýnidæmum en rangt á fyrsta huldudæminu.
	}
}

\env{frame}
{
	\frametitle{Planetaris, röng lausn}
	\selectcode{code/planetaris1.c}{10}{23}
}

\env{frame}
{
}

\end{document}
