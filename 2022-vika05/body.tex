\title{Athugasemdir}
\author{Bergur Snorrason}
\date{\today}

\begin{document}

\frame{\titlepage}

\env{frame}
{
	\env{itemize}
	{
		\item<1-> Á mánudaginn verður ekki fyrirlestur.
		\item<2-> Þess í stað verður miðmisseriskeppnin.
		\item<3-> Vikuskilin verða ekki heldur af hefðbundnu sniði.
		\item<4-> Keppnin hefst klukkan $8:30$ og lýkur $13:30$.
		\item<5-> Við ætlumst ekki til þess að þið séuð að allan tímann.
		\item<6-> Til að fá vikuskil í næstu viku nægir heiðarleg tilraun við eitt dæmi.
		\item<7-> Síðan má fá aukaskil með því að leysa fjögur dæmi fyrir sunnudaginn.
		\item<8-> Næsta miðvikudag munum við fara yfir lausnir á dæmunum í keppninni.
	}
}

\env{frame}
{
	\env{itemize}
	{
		\item<1-> Ég er búinn að breyta kóðanum fyrir biltréin örlítið.
		\item<2-> Það var mjög auðvelt að gera lúmska villu.
	}
}

\env{frame}
{
	\frametitle{Gamla með villunni}
	\selectcode{code/bad-se.c}{8}{41}
}

\env{frame}
{
	\frametitle{Gamla án villunnar}
	\selectcode{code/old-se.c}{8}{41}
}

\env{frame}
{
	\frametitle{Nýja}
	\selectcode{code/new-se.c}{8}{41}
}

\env{frame}
{
	\env{itemize}
	{
		\item<1-> Breytingin er að geyma stærðina á trénu í staki \texttt{p[0]}.
		\item<2-> Við erum einmitt að nota, svo kallaða, ,,fyrri aðferð'' til að geyma tréð í fylki.
		\item<3-> Þá er það stak ekki notað.
	}
}

\env{frame}
{
	\frametitle{Nýja}
	\selectcode{code/new-se.c}{43}{54}
}

\env{frame}
{
}

\end{document}
