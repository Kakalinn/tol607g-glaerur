\usepackage[T1]{fontenc}
\usepackage[utf8]{inputenc}
\usepackage[english, icelandic]{babel}
\usepackage{amsmath}
\usepackage{amssymb}
\usepackage{amsthm}
\usepackage{mathtools}
\usepackage{listings}
\usepackage{hyperref}
\usepackage{color}
\usepackage{tikz}
\usetikzlibrary{angles, shapes}
\usepackage{array}

\DeclareMathOperator{\lcm}{lcm}


\newcommand\env[2]
{
	\begin{#1}
	#2
	\end{#1}
}
\newcommand\varenv[3]
{
	\begin{#1}[#2]
	#3
	\end{#1}
}

\newcommand\code[1]{{\tiny\lstinputlisting[language = C, numbers = left]{#1}}}
\newcommand\selectcode[3]{{\tiny\lstinputlisting[language = C, numbers = left, firstnumber = #2, linerange = {#2 - #3}]{#1}}}

\definecolor{mygray}{rgb}{0.4,0.4,0.4}
\definecolor{mygreen}{rgb}{0, 0, 1}
\definecolor{myorange}{rgb}{1.0,0.4,0}

\lstset{
	commentstyle=\color{mygray},
	numbersep=5pt,
	numberstyle=\tiny\color{mygray},
	keywordstyle=\color{mygreen},
	showspaces=false,
	showstringspaces=false,
	stringstyle=\color{myorange},
	tabsize=4
}
\lstset{literate=
{æ}{{\ae}}1
{í}{{\'{i}}}1
{ó}{{\'{o}}}1
{á}{{\'{a}}}1
{é}{{\'{e}}}1
{ú}{{\'{u}}}1
{ý}{{\'{y}}}1
{ð}{{\dh}}1
{þ}{{\th}}1
{ö}{{\"o}}1
{Á}{{\'{A}}}1
{Í}{{\'{I}}}1
{Ó}{{\'{O}}}1
{Ú}{{\'{U}}}1
{Æ}{{\AE}}1
{Ö}{{\"O}}1
{Ø}{{\O}}1
{Þ}{{\TH}}1
}


\title{Grunnatriði og Ad hoc}
\author{Bergur Snorrason}
\date{\today}

\begin{document}

\frame{\titlepage}

\env{frame}
{
	\frametitle{Grunntög og takmarkanir þeirra}
	\env{itemize}
	{
		\item<1-> Í grunninn snýst forritun um gögn.
		\item<2-> Þegar við forritum flokkum við gögnin okkar með \emph{tögum}.
		\item<3-> Dæmi um tög í \texttt{C/C++} eru \texttt{int} og \texttt{double}.
		\item<4-> Helstu tögin í \texttt{C/C++} eru (yfirleitt):
		\item<5->[]
		\env{tabular}
		{
			{l l l}
			Heiti & Lýsing & Skorður\\
			\texttt{int} & Heiltala & Á bilinu $[-2^{31}, 2^{31} - 1]$\\
			\texttt{unsigned int} & Heiltala & Á bilinu $[0, 2^{32} - 1]$\\
			\texttt{long long} & Heiltala & Á bilinu $[-2^{63}, 2^{63} - 1]$\\
			\texttt{unsigned long long} & Heiltala & Á bilinu $[0, 2^{64} - 1]$\\
			\texttt{double} & Fleytitala & Takmörkuð nákvæmni\\
			\texttt{char} & Heiltala & Á bilinu $[-128, 127]$\\
		}
	}
}

\env{frame}
{
	\frametitle{Hvað með tölur utan þessa bila?}
	\env{itemize}
	{
		\item<1-> Einn helsti kostur \texttt{Python} í keppnisforritun er að heiltölur geta verið eins stórar (eða litlar) og vera skal.
		\pause \code{fact.py}
		\pause \code{fact.out}
		\item<4-> Það er einnig hægt að nota \texttt{fractions} pakkann í \texttt{Python} til að vinna með fleytitölur án þess að tapa nákvæmni.
	}
}

\env{frame}
{
	\frametitle{Hvað með tölur utan þessa bila?}
	\env{itemize}
	{
		\item<1-> Sumir \texttt{C/C++} þýðendur bjóða upp á gagnatagið \texttt{\_\_int128} (til dæmis \texttt{gcc}).
		\item<2-> Þetta tag býður upp á að nota tölur á bilinu $[-2^{127}, 2^{127} - 1]$.
		\item<3-> Þetta þarf ekki að nota oft.
	}
}

\env{frame}
{
	\frametitle{Röðun}
	\env{itemize}
	{
		\item<1-> Við munum reglulega þurfa að raða gögnum í einhverja röð.
		\item<2->[]
		\env{tabular}
		{
			{l l}
			Forritunarmál & Röðun\\
			\hline
			\texttt{C} & \texttt{qsort(...)}\\
			\texttt{C++} & \texttt{sort(...)}\\
			\texttt{Python} & \texttt{this.sort()} eða \texttt{sorted(...)}\\
		}
		\item<3-> Skoðum nú hvert forritunarmál til að sjá nánar hvernig föllin eru notuð.
	}
}

\env{frame}
{
	\frametitle{Röðun í \texttt{C++}}
	\env{itemize}
	{
		\item<1-> Í grunninn tekur \texttt{sort(...)} við tveimur gildum.
		\item<2-> Fyrra gildið svarar til fyrsta staks þess sem við viljum raða og seinna gildið vísar á enda þess sem við viljum raða
			(ekki síðasta stakið)
		\item<3-> Ef við erum með $n$ staka fylki \texttt{a} þá röðum við því með \texttt{sort(a, a + n)}.
		\item<4-> Við getum raðað flest öllum ílátum með \texttt{sort}.
		\item<5-> Ef við erum með eitthva ílát (til dæmis \texttt{vector}) \texttt{a} má raða með \texttt{sort(a.begin(), a.end())}.
		\item<6-> Við getum líka bætt við okkar eigin samanburðarfalli sem þriðja inntak.
		\item<7-> Það kemur þá í stað ``minna eða samasem'' samanburðarins sem er sjálfgefinn.
	}
}

\env{frame}
{
	\frametitle{Röðun í \texttt{Python}}
	\env{itemize}
	{
		\item<1-> Til að raða lista í \texttt{Python} þá má nota annað hvort \texttt{this.sort()} eða \texttt{sorted(...)}.
		\item<2-> Gerum ráð fyrir að listinn okkar heiti \texttt{a}.
		\item<3-> Þá nægir að kalla á \texttt{a.sort()} og eftir það er \texttt{a} raðað.
		\item<4-> Hinsvegar skilar \texttt{sorted(a)} afriti af \texttt{a} sem hefur verið raðað.
		\item<5-> Til að raða \texttt{a} á þennan hátt þarf \texttt{a = sorted(a)}.
		\item<6-> Nota má inntakið \texttt{key} til að raða eftir öðrum samanburðum.
		\item<7-> Það er einnig inntak sem heitir \texttt{reverse} sem er Boole gildi sem leyfir auðveldlega að raða öfugt.
	}
}

\env{frame}
{
	\frametitle{Röðun í \texttt{C}}
	\env{itemize}
	{
		\item<1-> Í \texttt{C} er enginn sjálfgefinn samanburður, svo við þurfum alltaf að skrifa okkar eigið samanburðarfall.
		\item<2-> Til röðunar notum við fallið \texttt{qsort(...)}.
		\item<3-> Fallið tekur fjögur viðföng:
		\env{itemize}
		{
			\item<4-> \texttt{void* a}. Þetta er fylkið sem við viljum raða.
			\item<5-> \texttt{size\_t n}. Þetta er fjöldi staka í fylkinu sem \texttt{a} svarar til.
			\item<6-> \texttt{size\_t s}. Þetta er stærð hvers staks í fylkinu okkar (í bætum).
			\item<7-> \texttt{int (*cmp)(const void *, const void*)}. Þetta er samanburðarfallið okkar.
		}
		\item<8-> Síðasta inntakið er kannski flókið við fyrstu sýn en er einfalt fyrir okkur að nota.
		\item<9-> Þetta er \emph{fallabendir} (e. \emph{function pointer}) ef þið viljið kynna ykkur það frekar.
	}
}

\env{frame}
{
	\frametitle{Röðun í \emph{C}}
	\code{sort.c}
}

\env{frame}
{
	\frametitle{Uppsetning dæma}
	\env{itemize}
	{
		\item<1-> Dæmin sem við sjáum á Kattis eru (oftast) af stöðluðu sniði.
		\env{itemize}
		{
			\item<2-> Saga.
			\item<3-> Dæmið.
			\item<4-> Inntaks -og úttakslýsingar.
			\item<5-> Sýnidæmi.
		}
		\item<6-> Fyrstu tveir punktarnir geta verið blandaðir saman.
		\item<7-> Þeir eru líka lengsti hluti dæmisins.
	}
}
\env{frame}
{
	\includegraphics[scale = 0.38]{daemi}
}

\env{frame}
{
	\frametitle{Röng lausn. Hver er villan?}
	\code{differentint.cpp}
}

\env{frame}
{
	\frametitle{Rétt lausn}
	\code{different.cpp}
}

\env{frame}
{
	\frametitle{\texttt{long long}}
	\env{itemize}
	{
		\item<1-> Þurfum við þó alltaf að skrifa \texttt{long long}?
		\item<2-> Nei!
		\item<3-> Við getum notað \texttt{typedef}.
		\item<4-> Við notum einfaldlega \texttt{typedef <gamla> <nýja>;}.
		\item<5-> Venjan í keppnisforritun er að nota \texttt{typedef long long ll;}.
		\item<6-> Við munum nota \texttt{typedef} aftur.
	}
}

\env{frame}
{
	\frametitle{Rétt lausn með \texttt{typedef}}
	\code{differentll.cpp}
}

\env{frame}
{
	\frametitle{\texttt{Time Limit Exceede}}
	\env{itemize}
	{
		\item<1-> Hvernig vitum að lausnin okkar sé of hæg?
		\item<2-> Ein leið er að útfæra lausnina, senda hana inn og gá hvað Kattis segir.
		\item<3-> Það myndi þó spara mikla vinnu ef við gætum svarað spurningunni án þess að útfæra.
		\item<4-> Einnig gæti leynst önnur villa í útfærslunni okkar sem gefur okkur \texttt{Time Limit Exceeded} (\texttt{TLE}).
		\item<5-> Til að ákvarða hvort lausn sé nógu hröð þá notum við \emph{tímaflækjur}.
		\item<6-> Sum ykkar þekka tímaflækjur en önnur ekki.
		\item<7-> Skoðum fyrst hvað tímaflækjur eru í grófum dráttum.
	}
}

\env{frame}
{
	\frametitle{Tímaflækjur í grófum dráttum}
	\env{itemize}
	{
		\item<1-> Keyrslutími forrits er háður stærðinni á inntakinu.
		\item<2-> Tímaflækjan lýsir hvernig keyrslutími forritsins skalast með inntakinu (í versta falli).
		\item<3-> Ef forritið er með tímaflækju $\mathcal{O}(f(n))$ þýðir það að keyrslutíminn vex eins of $f$ þegar $n$ vex.
		\item<4-> Til dæmis ef forritið hefur tímaflækju $\mathcal{O}(n)$ þá tvöfaldast keyrslutími þegar inntakið tvöfaldast.
		\item<5-> Hér gerum við ráð fyrir að grunnaðgerðirnar okkar taki fastann tíma, eða séu með tímaflækju $\mathcal{O}(1)$.
	}
}

\env{frame}
{
	\env{itemize}
	{
		\item<1-> Ef forritið okkar þarf að framkvæma $\mathcal{O}(f(n))$ aðgerð $m$ sinnum þá er tímaflækjan $\mathcal{O}(m \cdot f(n))$.
		\item<2-> Þetta er reglan sem við notum oftast í keppnisforritun.
		\item<3-> Hún segir okkur til dæmis að tvöföld \texttt{for}-lykkja, þar sem hver \texttt{for}-lykkja er $n$ löng, er $\mathcal{O}(n^2)$.
		\item<4-> Ef við erum með tvær einfaldar \texttt{for}-lykkjur, báðar af lenged $n$, þá er forritið 
			$\mathcal{O}(n) + \mathcal{O}(n) = \mathcal{O}(n)$
		\item<5-> Einnig gildir að tímaflækja forritsins okkar takmarkast af hægasta hluta forritsins.
		\item<6-> Til dæmis er
			$\mathcal{O}(n + n + n + n + n^2) = \mathcal{O}(n^2)$.
	}
}

\env{frame}
{
	\frametitle{Stærðfræði}
	\env{itemize}
	{
		\item<1-> Við segjum að fall $g(x)$ sé í menginu $\mathcal{O}(f(x))$ ef til eru rauntölur $c$ og $x_0$ þannig að
		\[
			|g(x)| \leq c \cdot f(x)
		\]
		fyrir öll $x > x_0$.
		\item<2-> Þetta þýðir í raun að fallið $|g(x)|$ verður á endanum minna en $k \cdot f(x)$.
		\item<3-> Þessi lýsing undirstrikar betur að $f(x)$ er efra mat á $g(x)$, og er því að segja að $g(x)$ hagi sér ekki verr en $f(x)$.
	}
}

\env{frame}
{
	\frametitle{Þekktar tímaflækjur}
	\env{itemize}
	{
		\item<1-> Tímaflækjur algrengra aðgerða eru:
		\item<2->[]
		\scriptsize
		\env{tabular}
		{
			{l | l | l}
			Aðgerð & Lýsing & Tímaflækja\\
			\hline
			Línulega leit & Almenn leit í fylki & $\mathcal{O}(n)$\\
			Helmingunarleit & Leit í röðuðu fylki & $\mathcal{O}(\log n)$\\
			Röðun á heiltölum & Röðun á heiltalna fylki & $\mathcal{O}(n \log n)$\\
			Strengjasamanburður & Bera saman tvo strengi af lengd $n$ & $\mathcal{O}(n)$\\
			Almenn röðun & Röðun með $\mathcal{O}(T(m))$ samanburð & $\mathcal{O}(T(m) \cdot n \log n)$\\
		}
	}
}

\env{frame}
{
	\frametitle{$10^8$ reglan}
	\env{itemize}
	{
		\item<1-> Þegar við ræðum tímaflækjur er ``tími'' er ekki endilega rétt orðið.
		\item<2-> Við erum frekar að lýsa fjölda aðgerða sem forritið framkvæmir.
		\item<3-> Í keppnisforritun notum við \emph{$10^8$ regluna}:
		\env{itemize}
		{
			\item<4-> Tökum verstu tilfellin sem koma fyrir í inntakslýsingunni á dæminu,
						stingum því inn í tímaflækjuna okkar
						og deilum með $10^8$.
			\item<5-> Ef útkoman er minni en fjöldi sekúnda í tímamörkum dæmisins þá er lausnin okkar nógu hröð, annars er hún of hæg.
		}
		\item<6-> Þessa reglu mætti um orða sem: ``Við gerum ráð fyrir að forritið geti framkvæmt $10^8$ aðgerðir á sekúndu''.
		\item<7-> Þessi regla er gróf nálgun, en virkar mjög vel því þetta er það sem dæmahöfundar hafa í huga þegar þeir semja dæmi.
		\item<8-> Með þetta í huga fáum við eftirfarandi töflu.
	}
}

\env{frame}
{
	\env{tabular}
	{
		{l | l | l}
		Stærð $n$ & Versta tímaflækja & Dæmi\\
		\hline
		$\leq 10$ & $\mathcal{O}(n!)$ & TSP með tæmandi leit\\
		$\leq 15$ & $\mathcal{O}(n^22^n)$ & TSP með kvikri bestun\\
		$\leq 20$ & $\mathcal{O}(n2^n)$ & Kvik bestun yfir hlutmengi\\
		$\leq 100$ & $\mathcal{O}(n^4)$ & Almenn spyrðing\\
		$\leq 400$ & $\mathcal{O}(n^3)$ & Floyd-Warshall\\
		$\leq 10^4$ & $\mathcal{O}(n^2)$ & Lengsti sameiginlegi hlutstrengur\\
		$\leq 10^5$ & $\mathcal{O}(n \sqrt{n})$ & Reiknirit sem byggja á rótarþáttun\\
		$\leq 10^6$ & $\mathcal{O}(n \log n)$ & Of mikið til að þora að taka dæmi\\
		$\leq 10^7$ & $\mathcal{O}(n)$ & Næsta tala sem er stærri (\texttt{NGE})\\
		$\leq 2^{10^7}$ & $\mathcal{O}(\log n)$ & Helmingunarleit\\
		$> 2^{10^7}$ & $\mathcal{O}(1)$ & Ad hoc
	}
}

\env{frame}
{
	\frametitle{\texttt{TLE} trikk}
	\env{itemize}
	{
		\item<1-> Stundum fær maður \texttt{TLE} þótt maður sé viss um að lausnin sé nógu hröð.
		\item<2-> Ef forritið þarf að lesa eða skrifa mikið gæti það verið að hægja nóg á forritun til að gefa \texttt{TLE}.
		\item<3-> Þetta stafar af því að til að lesa eða skrifa þarf forritið að tala við stýrikerfið.
		\item<4-> Til að leysa þetta skrifa sum forrit í \emph{biðminni} (e. \emph{buffer}) og prenta bara þegar það fyllist.
		\item<5-> Svona er þetta gert í \texttt{C}.
	}
}

\env{frame}
{
	\frametitle{\texttt{TLE} trikk}
	\env{itemize}
	{
		\item<1-> Í \texttt{C++} er biðminnið tæmt þegar \texttt{std::endl} er prentað.
		\item<2-> Til að koma í veg fyrir þetta er hægt að prenta \texttt{\textbackslash n} í staðinn.
		\item<3-> Til dæmis \texttt{cout << x << '\textbackslash n'}.
		\item<4-> Það borgar sig einnig að setja \texttt{ios::sync\_with\_stdio(false)} fremst í \texttt{main()}.
		\item<5-> Ef þið eruð í \texttt{Java} mæli ég með \texttt{Kattio}.
		\item<6-> Það má finna á GitHub.
	}
}

\env{frame}
{
	\frametitle{Innbyggðar gagnagrindur í \texttt{C+}}
	\env{itemize}
	{
		\item<1-> Grunnur \texttt{C++} býr yfir mörgum sterkum gagnagrindum.
		\item<2-> Skoðum helstu slíku gagnagrindur og tímaflækjur mikilvægust aðgerða þeirra.
		\item<3-> Við munum bara fjalla um gagnagrindurnar í grófum dráttum.
		\item<4-> Það er hægt að finna ítarlegra efni og dæmi um notkun á netinu.
	}
}

\env{frame}
{
	\frametitle{Fylki}
	\env{itemize}
	{
		\item<1-> Lýkt og í mörgum öðrum forritunarmálum eru fylki í \texttt{C++}.
		\item<2-> Fylki geyma gögn og eru af fastri stærð.
		\item<3-> Þar sem þau eru af fastri stærð má gefa þeim tileinkað, aðliggjandi svæði í minni.
		\item<4-> Þetta leyfir manni að vísa í fylkið í $\mathcal{O}(1)$.
		\item<5->[]
		\env{tabular}
		{
			{l | l}
			Aðgerð & Tímaflækja\\
			\hline
			Lesa eða skrifa ótiltekið stak & $\mathcal{O}(1)$\\
			Bæta staki aftast & $\mathcal{O}(n)$\\
			Skeyta saman tveimur & $\mathcal{O}(n)$\\
		}
	}
}

\env{frame}
{
	\frametitle{\texttt{vector}}
	\env{itemize}
	{
		\item<1-> Gagnagrindin \texttt{vector} er að mestu leiti eins og fylki.
		\item<2-> Það má þó bæta stökum aftan á \texttt{vector} í $\mathcal{O}(1)$.
		\item<3-> Margir nota bara \texttt{vector} og aldrei fylki sem slík.
		\item<4->[]
		\env{tabular}
		{
			{l | l}
			Aðgerð & Tímaflækja\\
			\hline
			Lesa eða skrifa ótiltekið stak & $\mathcal{O}(1)$\\
			Bæta staki aftast & $\mathcal{O}(1)$\\
			Skeyta saman tveimur & $\mathcal{O}(n)$\\
		}
	}
}

\env{frame}
{
	\frametitle{\texttt{list}}
	\env{itemize}
	{
		\item<1-> Gagnagrindin \texttt{list} geymir gögn líkt og fylki gera, en stökin eru ekki aðliggjandi í minni.
		\item<2-> Því er uppfletting ekki hröð.
		\item<3-> Aftur á móti er hægt að gera smávægilegar breytingar á \texttt{list} sem er ekki hægt að gera á fylkjum.
		\item<4->[]
		\env{tabular}
		{
			{l | l}
			Aðgerð & Tímaflækja\\
			\hline
			Finna stak & $\mathcal{O}(n)$\\
			Bæta staki aftast & $\mathcal{O}(1)$\\
			Bæta staki fremst & $\mathcal{O}(1)$\\
			Bæta staki fyrir aftan tiltekið stak & $\mathcal{O}(1)$\\
			Bæta staki fyrir framan tiltekið stak & $\mathcal{O}(1)$\\
			Skeyta saman tveimur & $\mathcal{O}(1)$\\
		}
	}
}

\env{frame}
{
	\frametitle{\texttt{stack}}
	\env{itemize}
	{
		\item<1-> Gagnagrindin \texttt{stack} geymir gögn og leyfir aðgang að síðasta staki sem var bætt við.
		\item<2->[]
		\env{tabular}
		{
			{l | l}
			Aðgerð & Tímaflækja\\
			\hline
			Bæta við staki & $\mathcal{O}(1)$\\
			Lesa nýjasta stakið & $\mathcal{O}(1)$\\
			Fjarlægja nýjasta stakið  & $\mathcal{O}(1)$\\
		}
	}
}

\env{frame}
{
	\frametitle{\texttt{queue}}
	\env{itemize}
	{
		\item<1-> Gagnagrindin \texttt{queue} geymir gögn og leyfir aðgang að fyrsta stakinu sem var bætt við.
		\item<2->[]
		\env{tabular}
		{
			{l | l}
			Aðgerð & Tímaflækja\\
			\hline
			Bæta við staki & $\mathcal{O}(1)$\\
			Lesa elsta stakið & $\mathcal{O}(1)$\\
			Fjarlægja elsta stakið  & $\mathcal{O}(1)$\\
		}
	}
}

\env{frame}
{
	\frametitle{\texttt{set}}
	\env{itemize}
	{
		\item<1-> Gagnagrindin \texttt{set} geymir gögn án endurtekninga og leyfir hraða uppflettingu.
		\item<2->[]
		\env{tabular}
		{
			{l | l}
			Aðgerð & Tímaflækja\\
			\hline
			Bæta við staki & $\mathcal{O}(\log n)$\\
			Fjarlægja stak & $\mathcal{O}(\log n)$\\
			Gá hvort staki hafi verið bætt við  & $\mathcal{O}(\log n)$\\
		}
	}
}

\env{frame}
{
	\frametitle{Lausnar aðferðir}
	\env{itemize}
	{
		\item<1-> Þegar við leysum dæmi í keppnisforritun notumst við oftast við eina af eftirfarandi aðferðum:
		\env{itemize}
		{
			\item<2-> \emph{Ad hoc},
			\item<3-> \emph{Tæmandi leit} eða \emph{ofbeldis aðferðin} (e. \emph{complete search, brute force}),
			\item<4-> \emph{Gráðug reiknirit} (e. \emph{greedy algorithms}),
			\item<5-> \emph{Deila og drottna} (e. \emph{divide and conquer}),
			\item<6-> \emph{Kvik bestun} (e. \emph{dynamic programming}).
		}
		\item<7-> Þessi skipting er ekki fullkomin, en það er þó gott að hafa hana í huga.
		\item<8-> Til dæmis má færa rök fyrir því að gráðugar lausnir og \texttt{D\&C} séu sértilfelli af kvikri bestun.
		\item<9-> Við munum byrja á því að fjalla almennt um þessar aðferðir og fara svo í sértækara efni.
		\item<10-> Þá er oft gott að hafa í huga hvernig flokka megi reikniritin.
	}
}

\env{frame}
{
	\frametitle{Ad hoc}
	\env{itemize}
	{
		\item<1-> Ef lausn dæmisins byggir ekki á sérþekkingu flokkast dæmið sem \emph{Ad hoc}.
		\item<2-> Þessi dæmi eru stundum flokkuð undir ``implementation'', eða sem \emph{útfærsludæmi}.
		\item<3-> Þetta er gert því flest Ad hoc dæmi snúast um að fylgja beint leiðbeiningum.
		\item<4-> Það eru þó undantekningar.
		\item<5-> Í NCPC $2020$ var Ad hoc dæmi sem mætti ekki flokkast sem útfærsludæmi.
		\item<6-> Ad hoc dæmi flokkast oft til léttari dæma í keppnum.
		\item<7-> Áðurnefnt NCPC dæmi er þó aftur undanteking, því engin keppandi náði að leysa það dæmi.
		\item<8-> Samkvæmt skilgreiningu getum við ekki rætt Ad hoc dæmi ítarlega. Tökum því nokkur dæmi.
	}
}

\env{frame}
{
	\frametitle{Blandað brot}
	\env{itemize}
	{
		\item<1-> Þú átt að breyta almennu broti í blandað brot.
		\item<2-> Munið að almenna brotið $p/q$, og blandaða brotið $a\ b/c$ tákna sömu töluna ef $p/q = a + b/c$.
		\item<3-> Munið einnig að ef $a\ b/c$ er almennt brot þá gildir $b < c$.
		\item<4-> Blandaða brotið ykkar á að hafa sama nefnara og upprunarlega brotið.
		\item<5-> Inntakið inniheldur tvær heiltölur $1 \leq p, q \leq 10^9$.
		\item<6-> Úttakið skal innihalda blandaða brotið sem svarar til $p/q$.
		\item<7->[]
		\env{tabular}
		{
			{l | l | l}
			& Inntak & Úttak\\
			\hline
			Sýnidæmi 1 & \texttt{27 12} & \texttt{2 3 / 12}\\
			Sýnidæmi 2 & \texttt{2460000 98400} & \texttt{25 0 / 98400}\\
			Sýnidæmi 3 & \texttt{3 4000} & \texttt{0 3 / 4000}\\
		}
	}
}

\env{frame}
{
	\frametitle{Lausn á blandað brot}
	\env{itemize}
	{
		\item<1-> Hér nægir okkur að reikna.
		\item<2-> Við getum aðeins stytt okkur leið með því að nota heiltöludeilingu.
		\item<3-> Við fáum þá að $a$ er heiltalan sem fæst með deilingunni $p/q$ og $b$ er afgangurinn.
	}
}

\env{frame}
{
	\code{mixedfractions.c}
}

\env{frame}
{
	\frametitle{Barnahjal}
	\env{itemize}
	{
		\item<1-> Þið eruð að reyna að kenna barni að telja.
		\item<2-> Það er þó ekki alltaf hægt að heyra hvað barnið segir.
		\item<3-> Þið viljið ákvarða hvort það sem barnið er að segja gæti mögulega verið rétt.
		\item<4-> Fyrsta lína inntaksins inniheldur heiltölu $1 \leq n \leq 10^3$.
		\item<5-> Síðan fylgir ein lína með $n$ strengjum.
		\item<6-> Hver strengur er annaðhvort heiltala á bilinu $[0, 10^4]$ eða strengurinn ``mumble''.
		\item<7-> Ef það er hægt að skipta út öllum ``mumble'' fyrir tölu þannig að talningin sé rétt skal prenta ``jebb''.
		\item<8-> Annars skal prenta ``neibb''.
	}
}

\env{frame}
{
	\frametitle{Lausn á Barnahjal}
	\env{itemize}
	{
		\item<1-> Ef $i$-ti strengurinn inniheldur strenginn sem svarar til tölurnnar $i$ eða ``mumble'', fyrir öll $i$,
			þá er barnið kannski að telja rétt.
		\item<2-> Annars er barnið að telja rangt.
	}
}

\env{frame}
{
	\code{babybites.py}
}

%\env{frame}
%{
	%\frametitle{Test}
	%\env{itemize}
	%{
		%\item<1->
	%}
%}

\end{document}

