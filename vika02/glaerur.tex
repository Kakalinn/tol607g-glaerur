%\documentclass[handout]{beamer}
\documentclass{beamer}
\usefonttheme[onlymath]{serif}
\usepackage[T1]{fontenc}
\usepackage[utf8]{inputenc}
\usepackage[english, icelandic]{babel}
\usepackage{amsmath}
\usepackage{amssymb}
\usepackage{amsthm}
\usepackage{gensymb}
\usepackage{parskip}
\usepackage{mathtools}
\usepackage{listings}
\usepackage{xfrac}
\usepackage{graphicx}
\usepackage{xcolor}
\usepackage{tikz}
\usepackage{xifthen}
\usepackage{tkz-euclide}
\usepackage{hyperref}
\usetikzlibrary{calc}
\usepackage{multicol}

\DeclareMathOperator{\lcm}{lcm}
\DeclareMathOperator{\diam}{diam}
\DeclareMathOperator{\dist}{dist}
\DeclareMathOperator{\ord}{ord}
\DeclareMathOperator{\Aut}{Aut}
\DeclareMathOperator{\Inn}{Inn}
\DeclareMathOperator{\Ker}{Ker}
\DeclareMathOperator{\trace}{trace}
\DeclareMathOperator{\fix}{fix}
\DeclareMathOperator{\Log}{Log}
\renewcommand\O{\mathcal{O}}
\newcommand\floor[1]{\left\lfloor#1\right\rfloor}
\newcommand\ceil[1]{\left\lceil#1\right\rceil}
\newcommand\abs[1]{\left|#1\right|}
\newcommand\p[1]{\left(#1\right)}
\newcommand\sqp[1]{\left[#1\right]}
\newcommand\cp[1]{\left\{#1\right\}}
\newcommand\norm[1]{\left\lVert#1\right\rVert}
\renewcommand\qedsymbol{$\blacksquare$}
\renewcommand\Im{\operatorname{Im}}
\renewcommand\Re{\operatorname{Re}}
\usepackage{color}

\newcommand\env[2]
{
	\begin{#1}
	#2
	\end{#1}
}
\newcommand\varenv[3]
{
	\begin[#2]{#1}
	#3
	\end{#1}
}

\newcommand\code[1]{\tiny\lstinputlisting[language=C]{#1}}

\definecolor{mygray}{rgb}{0.4,0.4,0.4}
\definecolor{mygreen}{rgb}{0, 0, 1}
\definecolor{myorange}{rgb}{1.0,0.4,0}

\lstset{
	commentstyle=\color{mygray},
	numbersep=5pt,
	numberstyle=\tiny\color{mygray},
	keywordstyle=\color{mygreen},
	showspaces=false,
	showstringspaces=false,
	stringstyle=\color{myorange},
	tabsize=4
}
\lstset{literate=
{æ}{{\ae}}1
{í}{{\'{i}}}1
{ó}{{\'{o}}}1
{á}{{\'{a}}}1
{é}{{\'{e}}}1
{ú}{{\'{u}}}1
{ý}{{\'{y}}}1
{ð}{{\dh}}1
{þ}{{\th}}1
{ö}{{\"o}}1
{Á}{{\'{A}}}1
{Í}{{\'{I}}}1
{Ó}{{\'{O}}}1
{Ú}{{\'{U}}}1
{Æ}{{\AE}}1
{Ö}{{\"O}}1
{Ø}{{\O}}1
{Þ}{{\TH}}1
}

%\usetheme{Madrid}

\title{Grunnatriði og Ad Hoc}
\author{Bergur Snorrason}
\date{\today}

\begin{document}

\frame{\titlepage}

\env{frame}
{
	\frametitle{Grunntög og takmarkanir þeirra}
	\env{itemize}
	{
		\item<1-> Í grunninn snýst forritun um gögn.
		\item<2-> Þegar við forritum flokkum við gögnin okkar með \emph{tögum}.
		\item<3-> Dæmi um tög í \texttt{C/C++} eru \texttt{int} og \texttt{double}.
		\item<4-> Helstu tögin í \texttt{C/C++} eru (yfirleitt):
		\env{tabular}
		{
			{l l l}
			Heiti & Lýsing & Skorður\\
			\texttt{int} & Heiltala & Á bilinu $[-2^{31}, 2^{31} - 1]$\\
			\texttt{unsigned int} & Heiltala & Á bilinu $[0, 2^{32} - 1]$\\
			\texttt{long long} & Heiltala & Á bilinu $[-2^{63}, 2^{63} - 1]$\\
			\texttt{unsigned long long} & Heiltala & Á bilinu $[0, 2^{64} - 1]$\\
			\texttt{double} & Fleytitala & Takmörkuð nákvæmni\\
			\texttt{char} & Heiltala & Á bilinu $[-128, 127]$\\
		}
	}
}

\env{frame}
{
	\frametitle{Hvað með tölur utan þessa bila?}
	\env{itemize}
	{
		\item<1-> Einn helsti kostur \texttt{Python} í keppnisforritun er að heiltölur geta verið eins stórar (eða litlar) og vera skal.
		\pause \code{fact.py}
		\pause \code{fact.out}
		\item<4-> Það er einnig hægt að nota \texttt{fractions} pakkann í \texttt{Python} til að vinna með fleytitölur án þess að tapa nákvæmni.
	}
}

\env{frame}
{
	\frametitle{Hvað með tölur utan þessa bila?}
	\env{itemize}
	{
		\item<1-> Sumir \texttt{C/C++} þýðendur bjóða upp á gagnatagið \texttt{\_\_int128} (til dæmis \texttt{gcc}).
		\item<2-> Þetta tag býður upp á að nota tölur á bilinu $[-2^{127}, 2^{127} - 1]$.
		\item<3-> Þetta þarf ekki að nota oft.
	}
}

\env{frame}
{
	\frametitle{Röðun}
	\env{itemize}
	{
		\item<1-> Við munum reglulega þurfa að raða gögnum í einhverja röð.
		\item<2-> 
		\pause
		\begin{tabular}{l l}
			Forritunarmál & Röðun\\
			\hline
			\texttt{C} & \texttt{qsort(...)}\\
			\texttt{C++} & \texttt{sort()}\\
			\texttt{Python} & \texttt{this.sort()} eða \texttt{sorted(...)}\\
		\end{tabular}
		\item<3-> Skoðum nú hvert forritunarmál til að sjá nánar hvernig föllin eru notuð.
	}
}

\env{frame}
{
	\frametitle{Röðun í \texttt{C++}}
	\env{itemize}
	{
		\item<1-> Í grunninn tekur \texttt{sort(...)} við tveimur gildum.
		\item<2-> Fyrra gildið svarar til fyrsta staks þess sem við viljum raða og seinna gildið vísar á enda þess sem við viljum raða
			(ekki síðasta stakið)
		\item<3-> Ef við erum með $n$ staka fylki \texttt{a} þá röðum við því með \texttt{sort(a, a + n)}.
		\item<4-> Við getum raða nær öllum ílátum með \texttt{sort}.
		\item<5-> Ef við erum með eitthva ílát (til dæmis \texttt{vector}) \texttt{a} má raða með \texttt{sort(a.begin(), a.end())}.
		\item<6-> Við getum líka bætt við okkar eigin samanburðarfalli sem þriðja inntak.
		\item<7-> Það kemur þá í stað ``minna eða samasem'' samanburðarins sem er sjálfgefinn.
	}
}

\env{frame}
{
	\frametitle{Röðun í \texttt{Python}}
	\env{itemize}
	{
		\item<1-> Til að raða lista í \texttt{Python} þá má nota annað hvort \texttt{this.sort()} eða \texttt{sorted(...)}.
		\item<2-> Gerum ráð fyrir að listinn okkar heiti \texttt{a}.
		\item<3-> Þá nægir að kalla á \texttt{a.sort()} og eftir það er \texttt{a} raðað.
		\item<4-> Hinsvegar skilar \texttt{sorted(a)} afriti af \texttt{a} sem hefur verið raðað.
		\item<5-> Til að raða \texttt{a} á þennan hátt þarf \texttt{a = sorted(a)}.
		\item<6-> Nota má inntakið \texttt{key} til að raða eftir öðrum samanburðum.
		\item<7-> Það er einnig inntak sem heitir \texttt{reverse} sem er Boole gildi sem leyfir auðveldlega að raða öfugt.
	}
}

\env{frame}
{
	\frametitle{Röðun í \texttt{C}}
	\env{itemize}
	{
		\item<1-> Í \texttt{C} er enginn sjálfgefinn samanburður, svo við þurfum alltaf að skrifa okkar eigið samanburðarfall.
		\item<2-> Til röðunar notum við fallið \texttt{qsort(...)}.
		\item<3-> Fallið tekur fjögur viðföng:
		\env{itemize}
		{
			\item<4-> \texttt{void* a}. Þetta er fylkið sem við viljum raða.
			\item<5-> \texttt{size\_t n}. Þetta er fjöldi staka í fylkinu sem \texttt{a} svarar til.
			\item<6-> \texttt{size\_t s}. Þetta er stærð hvers staks í fylkinu okkar (í bætum).
			\item<7-> \texttt{int (*cmp)(const void *, const void*)}. Þetta er samanburðarfallið okkar.
		}
		\item<8-> Síðasta inntakið er kannski flókið við fyrstu sýn en er einfalt fyrir okkur að nota.
		\item<9-> Þetta er \emph{fallabendir} (e. \emph{function pointer}) ef þið viljið kynna ykkur það frekar.
	}
}

\env{frame}
{
	\frametitle{Röðun í \emph{C}}
	\code{sort.c}
}

\env{frame}
{
	\frametitle{Uppsetning dæma}
	\env{itemize}
	{
		\item<1-> Dæmin sem við sjáum á Kattis eru (oftast) af stöðluðu sniði.
		\env{itemize}
		{
			\item<2-> Fyrst er saga.
		}
	}
}

\end{document}

