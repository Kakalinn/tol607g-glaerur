\title{Samansóp}
\author{Bergur Snorrason}
\date{\today}

\begin{document}

\frame{\titlepage}

\env{frame}
{
	\frametitle{Lokakeppnin}
	\env{itemize}
	{
		\item<1-> Í næstu viku verður lokakeppnin, og þar með síðasti tíminn í þessu námskeiði.
		\item<2-> Mér lýst best á að hafa keppnina í miðvikudagstímanum okkur.
		\item<3-> Við höfum þá stoðtíma í mánudagstímanum.
		\item<4-> Í keppninni verða fimm dæmi.
		\item<5-> Skil fást fyrir að leysa eitt dæmi.
		\item<6-> Ef þið leysið þrjú þeirra fáið þið aukaskil.
	}
}

\env{frame}
{
}

\env{frame}
{
	\frametitle{Strengjaleit}
	\env{itemize}
	{
		\item<1-> Gefum okkur langan streng $s$ og styttri streng $p$.
		\item<2-> Hvernig getum við fundið alla hlutstrengi $s$ sem eru jafnir $p$.
		\item<3-> Fyrsta sem manni dettur í hug er að bera $p$ saman við alla hlutstrengi $s$ af sömu lengd og $p$.
	}
}

\env{frame}
{
	\selectcode{code/naive-strengjaleit.c}{5}{14}
}

\env{frame}
{
	\env{itemize}
	{
		\item<1-> Gerum ráð fyrir að strengurinn $s$ sé af lengd $n$ og strengurinn $p$ sé af lengd $m$.
		\item<2-> Fjöldi hlutstrengja í $s$ að lengd $m$ er $n - m + 1$.
		\item<3-> Strengjasamanburðurinn tekur línulegan tíma.
		\item<4-> Svo tímaflækja leitarinnar er $\mathcal{O}($\onslide<5->{$nm - m^2$}$)$.
		\item<6-> Ef $m = n/2$ þá er $nm - m^2 = n^2/2 - n^2/4 = n^2/4$ tímaflækjan er í raun $\mathcal{O}($\onslide<7->{$n^2$}$)$.
		\item<8-> Dæmi um leiðinlega strengi væri $s = \text{,,}aaaaaaaaaaaaaaaa\text{''}$ og $p = \text{,,}aaaaaaab\text{''}$.
		\item<9-> Þessi aðferð virkar þó sæmilega ef strengirnir eru nógu óreglulegir.
		\item<10-> Dæmi um það hvenær þessi aðferð er góð er ef maður er að leita að orði í skáldsögu.
	}
}

\env{frame}
{
	\env{itemize}
	{
		\item<1-> Aðferðin er líka nógu goð ef $\mathcal{O}(n^2)$ er ekki of hægt.
		\item<2-> Það er þó óþarfi að útfæra hana því hún fylgir með flestum forritunarmálum, til dæmis:
		\env{itemize}
		{
			\item<3-> Í \texttt{string.h} í \texttt{C} er \texttt{strstr(..)}.
			\item<4-> Í \texttt{string} í \texttt{C++} er \texttt{find(..)}.
			\item<5-> Í \texttt{String} í \texttt{Java} er \texttt{indexOf(..)}.
		}
		\item<6-> Munið bara að ef $n > 10^4$ er þetta yfirleitt of hægt.
	}
}

\env{frame}
{
	\frametitle{Reiknirit Knuth, Morrisar og Pratts (\texttt{KMP}) strengjaleit ($1970$)}
	\env{itemize}
	{
		\item<1-> Er einhver leið til að bæta strengjaleitina úr fyrri glærum?
		\item<2-> Skoðum betur sértilfellið $p = \text{,,}aaaabbbb\text{''}$.
		\item<3-> Ef strengjasamanburðurinn misheppnast í $p_3$ þá myndi einfalda strengjaleitin okkar hliðra $p$ um einn og reyna aftur.
		\item<4-> En við vitum að fyrstu þrír stafnirnir í næsta hlutstreng stemma, svo við getum byrjað í $p_2$.
		\item<5-> Reiknirit Knuths, Morrisar og Pratts notar sér þessa hugmynd til að framkvæma strengjaleit.
		\item<6-> Reikniritið byrjar á að forreikna hversu mikið maður veit eftir misheppnaðan samanburð.
		\item<7-> Svo þurfum við einfaldlega að labba í gegnum $s$ og hliðra eins og á við.
	}
}

\env{frame}
{
	\env{itemize}
	{
		\item<1-> Til að finna hversu mikið á að hliðra hverju sinni þurfum við að reikna \emph{forstrengs fall} (e. \emph{prefix function})
					strengsins $p$.
		\item<2-> Við látum $f(j)$, $1 \leq j \leq |p|$, vera gefið með $f(j) = \max\{k \colon s[1,k] = s[j - k + 1, j]\}$.
		\item<3-> Sjáum fyrst að þetta fall uppfyllir $f(j + 1) \leq f(j) + 1$.
		\item<4-> Látum $k = f(j)$ og sjáum að ef $s[j + 1] = s[k]$ þá er $f(j + 1) = k + 1$.
		\item<5-> Ef $s[j + 1] \neq s[k]$ þá þurfum við að minnka $k$ þangað til við fáum jöfnuð.
		\item<6-> Við minnkum $k$ með því að láta $k' = f(k - 1)$.
		\item<7-> Það tekur $\mathcal{O}($\onslide<8->{$\,n\,$}$)$ tíma að reikna öll þessi gildi,
					því $f(j + 1) \leq f(j) + 1$, svo við munum ekki þurfa að minnka $k$ oftar en $n$ sinnum.
	}
}

\env{frame}
{
	\selectcode{code/kmp.c}{12}{30}
}

\env{frame}
{
	\env{itemize}
	{
		\item<1-> Takið eftir að hver ítrun innri lykkjanna svarar til einnar ítrunar ytri lykkjanna.
		\item<2-> Svo innri lykkjan keyrir, í heildina, ekki oftar en ytri lykkjan.
		\item<3-> Því er tímaflækjan í heildina $\mathcal{O}($\onslide<8->{$n + m$}$)$.
	}
}

\env{frame}
{
	\frametitle{Reiknirit Ahos og Corasicks ($1975$)}
	\env{itemize}
	{
		\item<1-> Til er önnur aðferð, svipuð og \texttt{KMP}, sem finnur staðsetningar margra orða í einu í streng.
		\item<2-> Hún er kennd við Aho og Corasick.
		\item<3-> Ég fer ekki í hana hér en hún byggir á því að gera \emph{Trie} og
					nota kvika bestun til að finna \emph{bakstrengs hlekk} (e. \emph{suffix link}) fyrir hvern hnút.
		\item<4-> Reikniritið keyrir í línulegum tíma í lengd allra strengjanna, ásamt fjölda heppnaðra samanburða.
	}
}

\env{frame}
{
	\frametitle{Hlaupabil}
	\env{itemize}
	{
		\item<1-> Aðferð hlaupabila (e. sliding window) er stundum hægt að nota til að taka
				dæmi sem hafa augljósa $\mathcal{O}(n^2)$ og gera þau $\mathcal{O}(n)$ eða $\mathcal{O}(n\log n)$.
	}
}

\env{frame}
{
	\env{itemize}
	{
		\item<1-> Skoðum dæmi:
		\item<2-> Gefið $n$, $k$ og svo $n$ tölur $a_i$, þ.a. $a_i \in \{0, 1\}$ finndu
			lengd lengstu bilanna í rununni $(a_n)_{n \in \mathbb{N}}$ sem innihelda bara $1$ ef þú mátt breyta allt að $k$ tölum.
		\item<3-> Sjáum strax að maður vill alltaf breyta $0$ í $1$ og aldrei öfugt.
		\item<4-> Sjáum því að við erum að leita að lengstu bilunum í $(a_n)_{n \in \mathbb{N}}$ sem hefur í mesta lagi $k$ stök jöfn $0$.
		\item<5-> Gefum okkur nú hlaupabil. Það byrjar tómt.
		\item<6-> Við löbbum svo í gegnum $(a_n)_{n \in \mathbb{N}}$ og lengjum bilið að aftan.
		\item<7-> Ef það eru einhvern tímann fleiri en $k$ stök í bilinu sem eru $0$ þá minnkum við bilið að framan þar til svo er ekki lengur.
	}
}

\defverbatim{\hbilA}
{ \begin{verbatim}
        k = 2
        l = 0
        [0 1 1 0 1 0 0 0 1 1 1 1 0 0 1 1]
        |
\end{verbatim} }

\defverbatim{\hbilB}
{ \begin{verbatim}
        k = 2
        l = 1
        [0 1 1 0 1 0 0 0 1 1 1 1 0 0 1 1]
        | |
\end{verbatim} }

\defverbatim{\hbilC}
{ \begin{verbatim}
        k = 2
        l = 2
        [0 1 1 0 1 0 0 0 1 1 1 1 0 0 1 1]
        |   |
\end{verbatim} }

\defverbatim{\hbilD}
{ \begin{verbatim}
        k = 2
        l = 3
        [0 1 1 0 1 0 0 0 1 1 1 1 0 0 1 1]
        |     |
\end{verbatim} }

\defverbatim{\hbilE}
{ \begin{verbatim}
        k = 2
        l = 4
        [0 1 1 0 1 0 0 0 1 1 1 1 0 0 1 1]
        |       |
\end{verbatim} }

\defverbatim{\hbilF}
{ \begin{verbatim}
        k = 2
        l = 5
        [0 1 1 0 1 0 0 0 1 1 1 1 0 0 1 1]
        |         |
\end{verbatim} }

\defverbatim{\hbilG}
{ \begin{verbatim}
        k = 2
        l = 4
        [0 1 1 0 1 0 0 0 1 1 1 1 0 0 1 1]
          |       |
\end{verbatim} }

\defverbatim{\hbilH}
{ \begin{verbatim}
        k = 2
        l = 5
        [0 1 1 0 1 0 0 0 1 1 1 1 0 0 1 1]
          |         |
\end{verbatim} }

\defverbatim{\hbilI}
{ \begin{verbatim}
        k = 2
        l = 4
        [0 1 1 0 1 0 0 0 1 1 1 1 0 0 1 1]
            |       |
\end{verbatim} }

\defverbatim{\hbilJ}
{ \begin{verbatim}
        k = 2
        l = 3
        [0 1 1 0 1 0 0 0 1 1 1 1 0 0 1 1]
              |     |
\end{verbatim} }

\defverbatim{\hbilK}
{ \begin{verbatim}
        k = 2
        l = 2
        [0 1 1 0 1 0 0 0 1 1 1 1 0 0 1 1]
                |   |
\end{verbatim} }

\defverbatim{\hbilL}
{ \begin{verbatim}
        k = 2
        l = 3
        [0 1 1 0 1 0 0 0 1 1 1 1 0 0 1 1]
                |     |
\end{verbatim} }

\defverbatim{\hbilM}
{ \begin{verbatim}
        k = 2
        l = 2
        [0 1 1 0 1 0 0 0 1 1 1 1 0 0 1 1]
                  |   |
\end{verbatim} }

\defverbatim{\hbilN}
{ \begin{verbatim}
        k = 2
        l = 1
        [0 1 1 0 1 0 0 0 1 1 1 1 0 0 1 1]
                    | |
\end{verbatim} }

\defverbatim{\hbilO}
{ \begin{verbatim}
        k = 2
        l = 2
        [0 1 1 0 1 0 0 0 1 1 1 1 0 0 1 1]
                    |   |
\end{verbatim} }

\defverbatim{\hbilP}
{ \begin{verbatim}
        k = 2
        l = 3
        [0 1 1 0 1 0 0 0 1 1 1 1 0 0 1 1]
                    |     |
\end{verbatim} }

\defverbatim{\hbilQ}
{ \begin{verbatim}
        k = 2
        l = 4
        [0 1 1 0 1 0 0 0 1 1 1 1 0 0 1 1]
                    |       |
\end{verbatim} }

\defverbatim{\hbilR}
{ \begin{verbatim}
        k = 2
        l = 5
        [0 1 1 0 1 0 0 0 1 1 1 1 0 0 1 1]
                    |         |
\end{verbatim} }

\defverbatim{\hbilS}
{ \begin{verbatim}
        k = 2
        l = 6
        [0 1 1 0 1 0 0 0 1 1 1 1 0 0 1 1]
                    |           |
\end{verbatim} }

\defverbatim{\hbilT}
{ \begin{verbatim}
        k = 2
        l = 5
        [0 1 1 0 1 0 0 0 1 1 1 1 0 0 1 1]
                      |         |
\end{verbatim} }

\defverbatim{\hbilU}
{ \begin{verbatim}
        k = 2
        l = 6
        [0 1 1 0 1 0 0 0 1 1 1 1 0 0 1 1]
                      |           |
\end{verbatim} }

\defverbatim{\hbilV}
{ \begin{verbatim}
        k = 2
        l = 5
        [0 1 1 0 1 0 0 0 1 1 1 1 0 0 1 1]
                        |         |
\end{verbatim} }

\defverbatim{\hbilW}
{ \begin{verbatim}
        k = 2
        l = 6
        [0 1 1 0 1 0 0 0 1 1 1 1 0 0 1 1]
                        |           |
\end{verbatim} }

\defverbatim{\hbilX}
{ \begin{verbatim}
        k = 2
        l = 7
        [0 1 1 0 1 0 0 0 1 1 1 1 0 0 1 1]
                        |             |
\end{verbatim} }

\defverbatim{\hbilY}
{ \begin{verbatim}
        k = 2
        l = 8
        [0 1 1 0 1 0 0 0 1 1 1 1 0 0 1 1]
                        |               |
\end{verbatim} }

\env{frame}
{
	\only<all:1>{\hbilA}
	\only<all:2>{\hbilB}
	\only<all:3>{\hbilC}
	\only<all:4>{\hbilD}
	\only<all:5>{\hbilE}
	\only<all:6>{\hbilF}
	\only<all:7>{\hbilG}
	\only<all:8>{\hbilH}
	\only<all:9>{\hbilI}
	\only<all:10>{\hbilJ}
	\only<all:11>{\hbilK}
	\only<all:12>{\hbilL}
	\only<all:13>{\hbilM}
	\only<all:14>{\hbilN}
	\only<all:15>{\hbilO}
	\only<all:16>{\hbilP}
	\only<all:17>{\hbilQ}
	\only<all:18>{\hbilR}
	\only<all:19>{\hbilS}
	\only<all:20>{\hbilT}
	\only<all:21>{\hbilU}
	\only<all:22>{\hbilV}
	\only<all:23>{\hbilW}
	\only<all:24>{\hbilX}
	\only<all:25>{\hbilY}
}

\env{frame}
{
	\selectcode{code/hlaupabil-daemi1.c}{3}{22}
}

\env{frame}
{
	\env{itemize}
	{
		\item<1-> Hver tala í rununni er sett einu sinni í hlaupabilið og mögulega fjarlægð úr því.
		\item<2-> Svo tímaflækjan er $\mathcal{O}($\onslide<3->{$\,n\,$}$)$.
	}
}

\env{frame}
{
	\env{itemize}
	{
		\item<1-> Þetta dæmi var í auðveldari kantinum.
		\item<2-> Skoðum annað dæmi:
		\item<3-> Byjrum á nokkrum undirstöðu atriðum.
		\item<4-> Tvö bil kallast \emph{næstum sundurlæg} ef sniðmengi þeirra er tómt eða bara einn punktur.
		\item<5-> Sammengi bila má skrifa sem sammengi næstu sundurlægra bila.
		\item<6-> \emph{Lengd bilsins} $[a, b]$ er $b - a$.
		\item<7-> Til að finna \emph{lengd sammengis bila} skrifum við sammengið sem sammengi næstum sundurlægra bila
			og tökum summu lengda þeirra.
		\item<8-> Til dæmis eru bilin $[1, 2]$ og $[2, 3]$ næstum sundurlæg (en þó ekki sundurlæg) en 
			$[1, 3]$ og $[2, 4]$ eru það ekki. Nú $[1, 3] \cup [2, 4] = [1, 4]$ svo lengd 
			$[1, 3] \cup [2, 4]$ er $3$.
	}
}

\env{frame}
{
	\env{itemize}
	{
		\item<1-> Gefið $n$ bil hver er lengd sammengis þeirra.
	}
}

\env{frame}
{
	\env{itemize}
	{
		\item<1-> Geymum í lista tvenndir þar sem fyrra stakið er endapunktur bils og seinna stakið segir hvaða bili punkturinn tilleyrir.
		\item<2-> Röðum þessum punktum svo í vaxandi röð.
		\item<3-> Við löbbum í gegnum þennan raðaða lista og höldum utan um hlaupabil þannig að
			við bætum við bili í hlaupabilið þegar við rekumst á vinstri endapunkt þess og fjarlægjum það 
			þegar við rekumst á hægri endapunkt þess. 
		\item<4-> Við skoðum svo sérstaklega tilfellin þegar við erum ekki með nein bil í hlaupabilinu okkar.
		\item<5-> Sammengi þeirra bila sem við höfum farið í gegnum þá síðan hlaupabilið var síðast tómt er nú
			sundurlægt öllum öðrum bilum sem okkur var gefið í byrjun.
		\item<6-> Við skilum því summu lengda þessara sammengja.
	}
}

\defverbatim{\lineAA}
{ \begin{verbatim}
   |                                  
 1:  x------------x
 2:     x----x
 3:  x----x                    
 4:          x----------x
 5:                         x------x
 6:                                                x--x
 7:                                            x------x
 8:                           x--x
 9:                                      x------------x
10:                         x------x
   |                    
[]
r = 0
\end{verbatim} }

\defverbatim{\lineAB}
{ \begin{verbatim}
     |                                  
 1:  x------------x
 2:     x----x
 3:  x----x                    
 4:          x----------x
 5:                         x------x
 6:                                                x--x
 7:                                            x------x
 8:                           x--x
 9:                                      x------------x
10:                         x------x
     |                    
[]
r = 0
\end{verbatim} }

\defverbatim{\lineAC}
{ \begin{verbatim}
     |                                  
 1:  x------------x
 2:     x----x
 3:  x----x                    
 4:          x----------x
 5:                         x------x
 6:                                                x--x
 7:                                            x------x
 8:                           x--x
 9:                                      x------------x
10:                         x------x
     |                    
[1]
r = 0
\end{verbatim} }

\defverbatim{\lineAD}
{ \begin{verbatim}
     |                                  
 1:  x------------x
 2:     x----x
 3:  x----x                    
 4:          x----------x
 5:                         x------x
 6:                                                x--x
 7:                                            x------x
 8:                           x--x
 9:                                      x------------x
10:                         x------x
     |                    
[1, 3]
r = 0
\end{verbatim} }

\defverbatim{\lineAE}
{ \begin{verbatim}
        |                                  
 1:  x------------x
 2:     x----x
 3:  x----x                    
 4:          x----------x
 5:                         x------x
 6:                                                x--x
 7:                                            x------x
 8:                           x--x
 9:                                      x------------x
10:                         x------x
        |                    
[1, 3]
r = 0
\end{verbatim} }

\defverbatim{\lineAF}
{ \begin{verbatim}
        |                                  
 1:  x------------x
 2:     x----x
 3:  x----x                    
 4:          x----------x
 5:                         x------x
 6:                                                x--x
 7:                                            x------x
 8:                           x--x
 9:                                      x------------x
10:                         x------x
        |                    
[1, 2, 3]
r = 0
\end{verbatim} }

\defverbatim{\lineAG}
{ \begin{verbatim}
          |                                  
 1:  x------------x
 2:     x----x
 3:  x----x                    
 4:          x----------x
 5:                         x------x
 6:                                                x--x
 7:                                            x------x
 8:                           x--x
 9:                                      x------------x
10:                         x------x
          |                    
[1, 2, 3]
r = 0
\end{verbatim} }

\defverbatim{\lineAH}
{ \begin{verbatim}
          |                                  
 1:  x------------x
 2:     x----x
 3:  x----x                    
 4:          x----------x
 5:                         x------x
 6:                                                x--x
 7:                                            x------x
 8:                           x--x
 9:                                      x------------x
10:                         x------x
          |                    
[1, 2]
r = 0
\end{verbatim} }

\defverbatim{\lineAI}
{ \begin{verbatim}
             |                                      
 1:  x------------x
 2:     x----x
 3:  x----x                    
 4:          x----------x
 5:                         x------x
 6:                                                x--x
 7:                                            x------x
 8:                           x--x
 9:                                      x------------x
10:                         x------x
             |                        
[1, 2]
r = 0
\end{verbatim} }

\defverbatim{\lineAJ}
{ \begin{verbatim}
             |                                      
 1:  x------------x
 2:     x----x
 3:  x----x                    
 4:          x----------x
 5:                         x------x
 6:                                                x--x
 7:                                            x------x
 8:                           x--x
 9:                                      x------------x
10:                         x------x
             |                        
[1, 2, 4]
r = 0
\end{verbatim} }

\defverbatim{\lineAK}
{ \begin{verbatim}
             |
 1:  x------------x
 2:     x----x
 3:  x----x                    
 4:          x----------x
 5:                         x------x
 6:                                                x--x
 7:                                            x------x
 8:                           x--x
 9:                                      x------------x
10:                         x------x
             |
[1, 4]
r = 0
\end{verbatim} }

\defverbatim{\lineAL}
{ \begin{verbatim}
                  |
 1:  x------------x
 2:     x----x
 3:  x----x                    
 4:          x----------x
 5:                         x------x
 6:                                                x--x
 7:                                            x------x
 8:                           x--x
 9:                                      x------------x
10:                         x------x
                  |
[1, 4]
r = 0
\end{verbatim} }

\defverbatim{\lineAM}
{ \begin{verbatim}
                  |
 1:  x------------x
 2:     x----x
 3:  x----x                    
 4:          x----------x
 5:                         x------x
 6:                                                x--x
 7:                                            x------x
 8:                           x--x
 9:                                      x------------x
10:                         x------x
                  |
[4]
r = 0
\end{verbatim} }

\defverbatim{\lineAN}
{ \begin{verbatim}
                        |
 1:  x------------x
 2:     x----x
 3:  x----x                    
 4:          x----------x
 5:                         x------x
 6:                                                x--x
 7:                                            x------x
 8:                           x--x
 9:                                      x------------x
10:                         x------x
                        |
[4]
r = 0
\end{verbatim} }

\defverbatim{\lineAO}
{ \begin{verbatim}
                        |
 1:  x------------x
 2:     x----x
 3:  x----x                    
 4:          x----------x
 5:                         x------x
 6:                                                x--x
 7:                                            x------x
 8:                           x--x
 9:                                      x------------x
10:                         x------x
                        |
[]
r = 0
\end{verbatim} }

\defverbatim{\lineAP}
{ \begin{verbatim}
                        |
 1:  x------------x
 2:     x----x
 3:  x----x                    
 4:          x----------x
 5:                         x------x
 6:                                                x--x
 7:                                            x------x
 8:                           x--x
 9:                                      x------------x
10:                         x------x
                        |
[]
r = 20
\end{verbatim} }

\defverbatim{\lineAQ}
{ \begin{verbatim}
                            |
 1:  x------------x
 2:     x----x
 3:  x----x                    
 4:          x----------x
 5:                         x------x
 6:                                                x--x
 7:                                            x------x
 8:                           x--x
 9:                                      x------------x
10:                         x------x
                            |
[]
r = 20
\end{verbatim} }

\defverbatim{\lineAR}
{ \begin{verbatim}
                            |
 1:  x------------x
 2:     x----x
 3:  x----x                    
 4:          x----------x
 5:                         x------x
 6:                                                x--x
 7:                                            x------x
 8:                           x--x
 9:                                      x------------x
10:                         x------x
                            |
[5]
r = 20
\end{verbatim} }

\defverbatim{\lineAS}
{ \begin{verbatim}
                            |
 1:  x------------x
 2:     x----x
 3:  x----x                    
 4:          x----------x
 5:                         x------x
 6:                                                x--x
 7:                                            x------x
 8:                           x--x
 9:                                      x------------x
10:                         x------x
                            |
[5, 10]
r = 20
\end{verbatim} }

\defverbatim{\lineAU}
{ \begin{verbatim}
                              |
 1:  x------------x
 2:     x----x
 3:  x----x                    
 4:          x----------x
 5:                         x------x
 6:                                                x--x
 7:                                            x------x
 8:                           x--x
 9:                                      x------------x
10:                         x------x
                              |
[5, 10]
r = 20
\end{verbatim} }

\defverbatim{\lineAV}
{ \begin{verbatim}
                              |
 1:  x------------x
 2:     x----x
 3:  x----x                    
 4:          x----------x
 5:                         x------x
 6:                                                x--x
 7:                                            x------x
 8:                           x--x
 9:                                      x------------x
10:                         x------x
                              |
[5, 8, 10]
r = 20
\end{verbatim} }

\defverbatim{\lineAW}
{ \begin{verbatim}
                                 |
 1:  x------------x
 2:     x----x
 3:  x----x                    
 4:          x----------x
 5:                         x------x
 6:                                                x--x
 7:                                            x------x
 8:                           x--x
 9:                                      x------------x
10:                         x------x
                                 |
[5, 8, 10]
r = 20
\end{verbatim} }

\defverbatim{\lineAX}
{ \begin{verbatim}
                                 |
 1:  x------------x
 2:     x----x
 3:  x----x                    
 4:          x----------x
 5:                         x------x
 6:                                                x--x
 7:                                            x------x
 8:                           x--x
 9:                                      x------------x
10:                         x------x
                                 |
[5, 10]
r = 20
\end{verbatim} }

\defverbatim{\lineAY}
{ \begin{verbatim}
                                   |
 1:  x------------x
 2:     x----x
 3:  x----x                    
 4:          x----------x
 5:                         x------x
 6:                                                x--x
 7:                                            x------x
 8:                           x--x
 9:                                      x------------x
10:                         x------x
                                   |
[5, 10]
r = 20
\end{verbatim} }

\defverbatim{\lineAZ}
{ \begin{verbatim}
                                   |
 1:  x------------x
 2:     x----x
 3:  x----x                    
 4:          x----------x
 5:                         x------x
 6:                                                x--x
 7:                                            x------x
 8:                           x--x
 9:                                      x------------x
10:                         x------x
                                   |
[5]
r = 20
\end{verbatim} }

\defverbatim{\lineBA}
{ \begin{verbatim}
                                   |
 1:  x------------x
 2:     x----x
 3:  x----x                    
 4:          x----------x
 5:                         x------x
 6:                                                x--x
 7:                                            x------x
 8:                           x--x
 9:                                      x------------x
10:                         x------x
                                   |
[]
r = 20
\end{verbatim} }

\defverbatim{\lineBB}
{ \begin{verbatim}
                                   |
 1:  x------------x
 2:     x----x
 3:  x----x                    
 4:          x----------x
 5:                         x------x
 6:                                                x--x
 7:                                            x------x
 8:                           x--x
 9:                                      x------------x
10:                         x------x
                                   |
[]
r = 28
\end{verbatim} }

\defverbatim{\lineBC}
{ \begin{verbatim}
                                         |
 1:  x------------x
 2:     x----x
 3:  x----x                    
 4:          x----------x
 5:                         x------x
 6:                                                x--x
 7:                                            x------x
 8:                           x--x
 9:                                      x------------x
10:                         x------x
                                         |
[]
r = 28
\end{verbatim} }

\defverbatim{\lineBD}
{ \begin{verbatim}
                                         |
 1:  x------------x
 2:     x----x
 3:  x----x                    
 4:          x----------x
 5:                         x------x
 6:                                                x--x
 7:                                            x------x
 8:                           x--x
 9:                                      x------------x
10:                         x------x
                                         |
[9]
r = 28
\end{verbatim} }

\defverbatim{\lineBDD}
{ \begin{verbatim}
                                               |
 1:  x------------x
 2:     x----x
 3:  x----x                    
 4:          x----------x
 5:                         x------x
 6:                                                x--x
 7:                                            x------x
 8:                           x--x
 9:                                      x------------x
10:                         x------x
                                               |
[9]
r = 28
\end{verbatim} }

\defverbatim{\lineBE}
{ \begin{verbatim}
                                               |
 1:  x------------x
 2:     x----x
 3:  x----x                    
 4:          x----------x
 5:                         x------x
 6:                                                x--x
 7:                                            x------x
 8:                           x--x
 9:                                      x------------x
10:                         x------x
                                               |
[7, 9]
r = 28
\end{verbatim} }

\defverbatim{\lineBF}
{ \begin{verbatim}
                                                   |
 1:  x------------x
 2:     x----x
 3:  x----x                    
 4:          x----------x
 5:                         x------x
 6:                                                x--x
 7:                                            x------x
 8:                           x--x
 9:                                      x------------x
10:                         x------x
                                                   |
[7, 9]
r = 28
\end{verbatim} }

\defverbatim{\lineBG}
{ \begin{verbatim}
                                                   |
 1:  x------------x
 2:     x----x
 3:  x----x                    
 4:          x----------x
 5:                         x------x
 6:                                                x--x
 7:                                            x------x
 8:                           x--x
 9:                                      x------------x
10:                         x------x
                                                   |
[6, 7, 9]
r = 28
\end{verbatim} }

\defverbatim{\lineBH}
{ \begin{verbatim}
                                                      |
 1:  x------------x
 2:     x----x
 3:  x----x                    
 4:          x----------x
 5:                         x------x
 6:                                                x--x
 7:                                            x------x
 8:                           x--x
 9:                                      x------------x
10:                         x------x
                                                      |
[6, 7, 9]
r = 28
\end{verbatim} }

\defverbatim{\lineBI}
{ \begin{verbatim}
                                                      |
 1:  x------------x
 2:     x----x
 3:  x----x                    
 4:          x----------x
 5:                         x------x
 6:                                                x--x
 7:                                            x------x
 8:                           x--x
 9:                                      x------------x
10:                         x------x
                                                      |
[7, 9]
r = 28
\end{verbatim} }

\defverbatim{\lineBJ}
{ \begin{verbatim}
                                                      |
 1:  x------------x
 2:     x----x
 3:  x----x                    
 4:          x----------x
 5:                         x------x
 6:                                                x--x
 7:                                            x------x
 8:                           x--x
 9:                                      x------------x
10:                         x------x
                                                      |
[9]
r = 28
\end{verbatim} }

\defverbatim{\lineBK}
{ \begin{verbatim}
                                                      |
 1:  x------------x
 2:     x----x
 3:  x----x                    
 4:          x----------x
 5:                         x------x
 6:                                                x--x
 7:                                            x------x
 8:                           x--x
 9:                                      x------------x
10:                         x------x
                                                      |
[]
r = 28
\end{verbatim} }

\defverbatim{\lineBL}
{ \begin{verbatim}
                                                      |
 1:  x------------x
 2:     x----x
 3:  x----x                    
 4:          x----------x
 5:                         x------x
 6:                                                x--x
 7:                                            x------x
 8:                           x--x
 9:                                      x------------x
10:                         x------x
                                                      |
[]
r = 42
\end{verbatim} }

\env{frame}
{
	\only<all:1>{\lineAA}
	\only<all:2>{\lineAB}
	\only<all:3>{\lineAC}
	\only<all:4>{\lineAD}
	\only<all:5>{\lineAE}
	\only<all:6>{\lineAF}
	\only<all:7>{\lineAG}
	\only<all:8>{\lineAH}
	\only<all:9>{\lineAI}
	\only<all:10>{\lineAJ}
	\only<all:11>{\lineAK}
	\only<all:12>{\lineAL}
	\only<all:13>{\lineAM}
	\only<all:14>{\lineAN}
	\only<all:15>{\lineAO}
	\only<all:16>{\lineAP}
	\only<all:17>{\lineAQ}
	\only<all:18>{\lineAR}
	\only<all:19>{\lineAS}
	\only<all:20>{\lineAU}
	\only<all:21>{\lineAV}
	\only<all:22>{\lineAW}
	\only<all:23>{\lineAX}
	\only<all:24>{\lineAY}
	\only<all:25>{\lineAZ}
	\only<all:26>{\lineBA}
	\only<all:27>{\lineBB}
	\only<all:28>{\lineBC}
	\only<all:29>{\lineBD}
	\only<all:30>{\lineBDD}
	\only<all:31>{\lineBE}
	\only<all:32>{\lineBF}
	\only<all:33>{\lineBG}
	\only<all:34>{\lineBH}
	\only<all:35>{\lineBI}
	\only<all:36>{\lineBJ}
	\only<all:37>{\lineBK}
	\only<all:38>{\lineBL}
}

\env{frame}
{
	\selectcode{code/hlaupabil-daemi2.c}{3}{32}
}

\env{frame}
{
	\env{itemize}
	{
		\item<1-> Við byrjum á að raða í $\mathcal{O}($\onslide<2->{$n \log n$}$)$ tíma.
		\item<3-> Síðan ítrum við í gegnum alla endapunktana sem tekur $\mathcal{O}($\onslide<4->{$\,n\,$}$)$ tíma.
		\item<5-> Svo lausnin hefur tímaflækjuna $\mathcal{O}($\onslide<6->{$n \log n$}$)$.
	}
}

\env{frame}
{
	\frametitle{Næsta stærra stak (\texttt{NGE})}
	\env{itemize}
	{
		\item<1-> Látum $a$ vera lista af $n$ tölum.
		\item<2-> Okkar langar, fyrir hvert stak í listanum, að finna næst stak í listanum sem er stærra (e. \emph{next greater element} (\texttt{NGE})).
		\item<3-> Sem dæmi er \texttt{NGE} miðju stakins $4$ í listanum $(2, 3, 4, 8, 5)$ talan $8$.
		\item<4-> Til þæginda segjum við að \texttt{NGE} tölunnar $8$ í listanum $(2, 3, 4, 8, 5)$ sé $-1$.
		\item<5-> Það er auðséð að við getum reiknað \texttt{NGE} allra talnanna með tvöfaldri \texttt{for}-lykkju.
	}
}

\env{frame}
{
	\selectcode{code/slow-nge.c}{3}{11}
}

\env{frame}
{
	\env{itemize}
	{
		\item<1-> Þar sem þessi lausnir er tvöföld \texttt{for}-lykkja, hvor af lengd $n$, þá er lausnin $\mathcal{O}($\onslide<2->{$n^2$}$)$.
	}
}

\env{frame}
{
	\env{itemize}
	{
		\item<1-> En þetta má bæta.
		\item<2-> Gefum okkur hlaða $h$. 
		\item<3-> Löbbum í gegnum $a$ í réttri röð.
		\item<4-> Tökum nú tölur úr hlaðan og setjum \texttt{NGE} þeirra talna sem $a_i$ á meðan $a_i$ er stærri en toppurinn á hlaðanum.
		\item<5-> Þegar toppurinn á hlaðanum er stærri en $a_i$ þá látum við $a_i$ á hlaðann og höldum svo áfram.
		\item<6-> Bersýnilega er hlaðinn ávallt raðaður, svo þú færð allar tölur sem eiga að hafa $a_i$ sem \texttt{NGE}.
		\item<7-> Þegar búið er að fara í gegnum $a$ látum við \texttt{NGE} þeirra staka sem eftir eru í $h$ vera $-1$.
	}
}

\defverbatim{\ngeAA}
{ \begin{verbatim}
 0 1 2 3 4 5 6 7
[2 3 1 5 7 6 4 8]
|

 0 1 2 3 4 5 6 7
[x x x x x x x x]


h: []
\end{verbatim} }

\defverbatim{\ngeAB}
{ \begin{verbatim}
 0 1 2 3 4 5 6 7
[2 3 1 5 7 6 4 8]
 |

 0 1 2 3 4 5 6 7
[x x x x x x x x]


h: []
\end{verbatim} }

\defverbatim{\ngeAC}
{ \begin{verbatim}
 0 1 2 3 4 5 6 7
[2 3 1 5 7 6 4 8]
 |

 0 1 2 3 4 5 6 7
[x x x x x x x x]


h: [0]
\end{verbatim} }

\defverbatim{\ngeAD}
{ \begin{verbatim}
 0 1 2 3 4 5 6 7
[2 3 1 5 7 6 4 8]
   |

 0 1 2 3 4 5 6 7
[x x x x x x x x]


h: [0]
\end{verbatim} }

\defverbatim{\ngeAE}
{ \begin{verbatim}
 0 1 2 3 4 5 6 7
[2 3 1 5 7 6 4 8]
 ^ |

 0 1 2 3 4 5 6 7
[x x x x x x x x]


h: [0]
\end{verbatim} }

\defverbatim{\ngeAF}
{ \begin{verbatim}
 0 1 2 3 4 5 6 7
[2 3 1 5 7 6 4 8]
 ^ |

 0 1 2 3 4 5 6 7
[1 x x x x x x x]


h: [0]
\end{verbatim} }

\defverbatim{\ngeAG}
{ \begin{verbatim}
 0 1 2 3 4 5 6 7
[2 3 1 5 7 6 4 8]
   |

 0 1 2 3 4 5 6 7
[1 x x x x x x x]


h: []
\end{verbatim} }

\defverbatim{\ngeAH}
{ \begin{verbatim}
 0 1 2 3 4 5 6 7
[2 3 1 5 7 6 4 8]
   |

 0 1 2 3 4 5 6 7
[1 x x x x x x x]


h: [1]
\end{verbatim} }

\defverbatim{\ngeAI}
{ \begin{verbatim}
 0 1 2 3 4 5 6 7
[2 3 1 5 7 6 4 8]
     |

 0 1 2 3 4 5 6 7
[1 x x x x x x x]


h: [1]
\end{verbatim} }

\defverbatim{\ngeAJ}
{ \begin{verbatim}
 0 1 2 3 4 5 6 7
[2 3 1 5 7 6 4 8]
   ^ |

 0 1 2 3 4 5 6 7
[1 x x x x x x x]


h: [1]
\end{verbatim} }

\defverbatim{\ngeAK}
{ \begin{verbatim}
 0 1 2 3 4 5 6 7
[2 3 1 5 7 6 4 8]
     |

 0 1 2 3 4 5 6 7
[1 x x x x x x x]


h: [1]
\end{verbatim} }

\defverbatim{\ngeAL}
{ \begin{verbatim}
 0 1 2 3 4 5 6 7
[2 3 1 5 7 6 4 8]
     |

 0 1 2 3 4 5 6 7
[1 x x x x x x x]


h: [1 2]
\end{verbatim} }

\defverbatim{\ngeAM}
{ \begin{verbatim}
 0 1 2 3 4 5 6 7
[2 3 1 5 7 6 4 8]
       |

 0 1 2 3 4 5 6 7
[1 x x x x x x x]


h: [1 2]
\end{verbatim} }

\defverbatim{\ngeAN}
{ \begin{verbatim}
 0 1 2 3 4 5 6 7
[2 3 1 5 7 6 4 8]
     ^ |

 0 1 2 3 4 5 6 7
[1 x x x x x x x]


h: [1 2]
\end{verbatim} }

\defverbatim{\ngeAO}
{ \begin{verbatim}
 0 1 2 3 4 5 6 7
[2 3 1 5 7 6 4 8]
     ^ |

 0 1 2 3 4 5 6 7
[1 x 3 x x x x x]


h: [1 2]
\end{verbatim} }

\defverbatim{\ngeAP}
{ \begin{verbatim}
 0 1 2 3 4 5 6 7
[2 3 1 5 7 6 4 8]
   ^   |

 0 1 2 3 4 5 6 7
[1 x 3 x x x x x]


h: [1]
\end{verbatim} }

\defverbatim{\ngeAQ}
{ \begin{verbatim}
 0 1 2 3 4 5 6 7
[2 3 1 5 7 6 4 8]
   ^   |

 0 1 2 3 4 5 6 7
[1 3 3 x x x x x]


h: [1]
\end{verbatim} }

\defverbatim{\ngeAR}
{ \begin{verbatim}
 0 1 2 3 4 5 6 7
[2 3 1 5 7 6 4 8]
       |

 0 1 2 3 4 5 6 7
[1 3 3 x x x x x]


h: []
\end{verbatim} }

\defverbatim{\ngeAS}
{ \begin{verbatim}
 0 1 2 3 4 5 6 7
[2 3 1 5 7 6 4 8]
       |

 0 1 2 3 4 5 6 7
[1 3 3 x x x x x]


h: [3]
\end{verbatim} }

\defverbatim{\ngeAT}
{ \begin{verbatim}
 0 1 2 3 4 5 6 7
[2 3 1 5 7 6 4 8]
         |

 0 1 2 3 4 5 6 7
[1 3 3 x x x x x]


h: [3]
\end{verbatim} }

\defverbatim{\ngeAU}
{ \begin{verbatim}
 0 1 2 3 4 5 6 7
[2 3 1 5 7 6 4 8]
       ^ |

 0 1 2 3 4 5 6 7
[1 3 3 x x x x x]


h: [3]
\end{verbatim} }

\defverbatim{\ngeAV}
{ \begin{verbatim}
 0 1 2 3 4 5 6 7
[2 3 1 5 7 6 4 8]
       ^ |

 0 1 2 3 4 5 6 7
[1 3 3 4 x x x x]


h: [3]
\end{verbatim} }

\defverbatim{\ngeAW}
{ \begin{verbatim}
 0 1 2 3 4 5 6 7
[2 3 1 5 7 6 4 8]
         |

 0 1 2 3 4 5 6 7
[1 3 3 4 x x x x]


h: []
\end{verbatim} }

\defverbatim{\ngeAX}
{ \begin{verbatim}
 0 1 2 3 4 5 6 7
[2 3 1 5 7 6 4 8]
         |

 0 1 2 3 4 5 6 7
[1 3 3 4 x x x x]


h: [4]
\end{verbatim} }

\defverbatim{\ngeAY}
{ \begin{verbatim}
 0 1 2 3 4 5 6 7
[2 3 1 5 7 6 4 8]
           |

 0 1 2 3 4 5 6 7
[1 3 3 4 x x x x]


h: [4]
\end{verbatim} }

\defverbatim{\ngeAZ}
{ \begin{verbatim}
 0 1 2 3 4 5 6 7
[2 3 1 5 7 6 4 8]
         ^ |

 0 1 2 3 4 5 6 7
[1 3 3 4 x x x x]


h: [4]
\end{verbatim} }

\defverbatim{\ngeBA}
{ \begin{verbatim}
 0 1 2 3 4 5 6 7
[2 3 1 5 7 6 4 8]
           |

 0 1 2 3 4 5 6 7
[1 3 3 4 x x x x]


h: [4]
\end{verbatim} }

\defverbatim{\ngeBB}
{ \begin{verbatim}
 0 1 2 3 4 5 6 7
[2 3 1 5 7 6 4 8]
           |

 0 1 2 3 4 5 6 7
[1 3 3 4 x x x x]


h: [4 5]
\end{verbatim} }

\defverbatim{\ngeBC}
{ \begin{verbatim}
 0 1 2 3 4 5 6 7
[2 3 1 5 7 6 4 8]
             |

 0 1 2 3 4 5 6 7
[1 3 3 4 x x x x]


h: [4 5]
\end{verbatim} }

\defverbatim{\ngeBD}
{ \begin{verbatim}
 0 1 2 3 4 5 6 7
[2 3 1 5 7 6 4 8]
           ^ |

 0 1 2 3 4 5 6 7
[1 3 3 4 x x x x]


h: [4 5]
\end{verbatim} }

\defverbatim{\ngeBE}
{ \begin{verbatim}
 0 1 2 3 4 5 6 7
[2 3 1 5 7 6 4 8]
             |

 0 1 2 3 4 5 6 7
[1 3 3 4 x x x x]


h: [4 5]
\end{verbatim} }

\defverbatim{\ngeBF}
{ \begin{verbatim}
 0 1 2 3 4 5 6 7
[2 3 1 5 7 6 4 8]
             |

 0 1 2 3 4 5 6 7
[1 3 3 4 x x x x]


h: [4 5 6]
\end{verbatim} }

\defverbatim{\ngeBG}
{ \begin{verbatim}
 0 1 2 3 4 5 6 7
[2 3 1 5 7 6 4 8]
               |

 0 1 2 3 4 5 6 7
[1 3 3 4 x x x x]


h: [4 5 6]
\end{verbatim} }

\defverbatim{\ngeBH}
{ \begin{verbatim}
 0 1 2 3 4 5 6 7
[2 3 1 5 7 6 4 8]
             ^ |

 0 1 2 3 4 5 6 7
[1 3 3 4 x x x x]


h: [4 5 6]
\end{verbatim} }

\defverbatim{\ngeBI}
{ \begin{verbatim}
 0 1 2 3 4 5 6 7
[2 3 1 5 7 6 4 8]
             ^ |

 0 1 2 3 4 5 6 7
[1 3 3 4 x x 7 x]


h: [4 5 6]
\end{verbatim} }

\defverbatim{\ngeBJ}
{ \begin{verbatim}
 0 1 2 3 4 5 6 7
[2 3 1 5 7 6 4 8]
           ^   |

 0 1 2 3 4 5 6 7
[1 3 3 4 x x 7 x]


h: [4 5]
\end{verbatim} }

\defverbatim{\ngeBK}
{ \begin{verbatim}
 0 1 2 3 4 5 6 7
[2 3 1 5 7 6 4 8]
           ^   |

 0 1 2 3 4 5 6 7
[1 3 3 4 x 7 7 x]


h: [4 5]
\end{verbatim} }

\defverbatim{\ngeBL}
{ \begin{verbatim}
 0 1 2 3 4 5 6 7
[2 3 1 5 7 6 4 8]
         ^     |

 0 1 2 3 4 5 6 7
[1 3 3 4 x 7 7 x]


h: [4]
\end{verbatim} }

\defverbatim{\ngeBM}
{ \begin{verbatim}
 0 1 2 3 4 5 6 7
[2 3 1 5 7 6 4 8]
         ^     |

 0 1 2 3 4 5 6 7
[1 3 3 4 7 7 7 x]


h: [4]
\end{verbatim} }

\defverbatim{\ngeBN}
{ \begin{verbatim}
 0 1 2 3 4 5 6 7
[2 3 1 5 7 6 4 8]
               |

 0 1 2 3 4 5 6 7
[1 3 3 4 7 7 7 x]


h: []
\end{verbatim} }

\defverbatim{\ngeBO}
{ \begin{verbatim}
 0 1 2 3 4 5 6 7
[2 3 1 5 7 6 4 8]
               |

 0 1 2 3 4 5 6 7
[1 3 3 4 7 7 7 x]


h: [8]
\end{verbatim} }

\defverbatim{\ngeBP}
{ \begin{verbatim}
 0 1 2 3 4 5 6 7
[2 3 1 5 7 6 4 8]


 0 1 2 3 4 5 6 7
[1 3 3 4 7 7 7 x]


h: [8]
\end{verbatim} }

\env{frame}
{
	\only<all:1>{\ngeAA}
	\only<all:2>{\ngeAB}
	\only<all:3>{\ngeAC}
	\only<all:4>{\ngeAD}
	\only<all:5>{\ngeAE}
	\only<all:6>{\ngeAF}
	\only<all:7>{\ngeAG}
	\only<all:8>{\ngeAH}
	\only<all:9>{\ngeAI}
	\only<all:10>{\ngeAJ}
	\only<all:11>{\ngeAK}
	\only<all:12>{\ngeAL}
	\only<all:13>{\ngeAM}
	\only<all:14>{\ngeAN}
	\only<all:15>{\ngeAO}
	\only<all:16>{\ngeAP}
	\only<all:17>{\ngeAQ}
	\only<all:18>{\ngeAR}
	\only<all:19>{\ngeAS}
	\only<all:20>{\ngeAT}
	\only<all:21>{\ngeAU}
	\only<all:22>{\ngeAV}
	\only<all:23>{\ngeAW}
	\only<all:24>{\ngeAX}
	\only<all:25>{\ngeAY}
	\only<all:26>{\ngeAZ}
	\only<all:27>{\ngeBA}
	\only<all:28>{\ngeBB}
	\only<all:29>{\ngeBC}
	\only<all:30>{\ngeBD}
	\only<all:31>{\ngeBE}
	\only<all:32>{\ngeBF}
	\only<all:33>{\ngeBG}
	\only<all:34>{\ngeBH}
	\only<all:35>{\ngeBI}
	\only<all:36>{\ngeBJ}
	\only<all:37>{\ngeBK}
	\only<all:38>{\ngeBL}
	\only<all:39>{\ngeBM}
	\only<all:40>{\ngeBN}
	\only<all:41>{\ngeBO}
	\only<all:42>{\ngeBP}
}

\env{frame}
{
	\selectcode{code/nge.c}{3}{12}
}

\env{frame}
{

	\env{itemize}
	{
		\item<1-> Við setjum hverja tölu í hlaðann að mestu einu sinni og tökum hana svo úr hlaðanum.
		\item<2-> Svo tímaflækjan er $\mathcal{O}($\onslide<3->{$\,n\,$}$)$.
	}
}

\env{frame}
{
}

\end{document}
