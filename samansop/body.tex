\title{Samansóp}
\author{Bergur Snorrason}
\date{\today}

\begin{document}

\frame{\titlepage}

\env{frame}
{
	\frametitle{Lokakeppnin}
	\env{itemize}
	{
		\item<1-> Í næstu viku verður lokakeppnin, og þar með síðasti tíminn í þessu námskeiði.
		\item<2-> Mér lýst best á að hafa keppnina í miðvikudagstímanum okkur.
		\item<3-> Við höfum þá stoðtíma í mánudagstímanum.
		\item<4-> Í keppninni verða fimm dæmi.
		\item<5-> Skil fást fyrir að leysa eitt dæmi.
		\item<6-> Ef þið leysið þrjú þeirra fáið þið aukaskil.
	}
}

\env{frame}
{
}

\env{frame}
{
	\frametitle{Strengjaleit}
	\env{itemize}
	{
		\item<1-> Gefum okkur langan streng \texttt{s} og styttri streng \texttt{p}.
		\item<2-> Hvernig getum við fundið alla hlutstrengi \texttt{s} sem eru jafnir \texttt{p}.
		\item<3-> Fyrsta sem manni dettur í hug er að bera \texttt{p} saman við alla hlutstrengi \texttt{s} af sömu lengd og \texttt{p}.
	}
}

\env{frame}
{
	\code{code/naive-strengjaleit.c}
}

\env{frame}
{
	\env{itemize}
	{
		\item<1-> Gerum ráð fyrir að \texttt{s} sé af lengd $n$ og \texttt{p} sé af lengd $m$.
		\item<2-> Fjöldi hlutstrengja í \texttt{s} að lengd $m$ er $n - m + 1$.
		\item<3-> Strengja samanburðurinn tekur línulegan tíma.
		\item<4-> Svo tímaflækja leitarinnar er $\mathcal{O}($\onslide<5->{$nm - m^2$}$)$.
		\item<6-> Ef $m = n/2$ þá er $nm - m^2 = n^2/2 - n^2/4 = n^2/4$ tímaflækjan er í raun $\mathcal{O}($\onslide<7->{$n^2$}$)$.
		\item<8-> Dæmi um leiðinlega strengi væri \texttt{s = ``aaaaaaaaaaaaaaaa''} og \texttt{p = ``aaaaaaab''}.
	}
}

\env{frame}
{
	\env{itemize}
	{
		\item<1-> Þó þessi aðferð sé ekki góð þá er hún stundum nógu góð.
		\item<2-> Það er þó óþarfi að útfæra hana því hún fylgir með flestum forritunarmálum, til dæmis:
		\env{itemize}
		{
			\item<3-> Í \text{string.h} í \texttt{C} er \texttt{strstr(..)}.
			\item<4-> Í \text{string} í \texttt{C++} er \texttt{find(..)}.
			\item<5-> Í \text{String} í \texttt{Java} er \texttt{indexOf(..)}.
		}
		\item<6-> Munið bara að ef $n > 10^4$ er þetta yfirleitt of hægt.
	}
}

\env{frame}
{
	\frametitle{Reiknirit Knuth, Morrisar og Pratts (\texttt{KMP}) strengjaleit ($1970$)}
	\env{itemize}
	{
		\item<1-> Er einhver leið til að bæta strengjaleitina úr fyrri glærum?
		\item<2-> Skoðum betur sértilfellið \texttt{p = ``aaaabbbb''}.
		\item<3-> Ef strengja samanburðurinn misheppnast í \texttt{p[3]} þá myndi einfalda strengjaleitin okkar hliðra \texttt{p} um einn og reyna aftur.
		\item<4-> En við vitum að fyrstu þrír stafnirnir í næsta hlutstreng stemma, svo við getum byrjað í \texttt{p[2]}.
		\item<5-> Reiknirit Knuths, Morrisar og Pratts notar sér þessa hugmynd til að framkvæma strengjaleit.
		\item<6-> Reikniritið byrjar á að forreiknað hversu mikið maður veit eftir misheppnaðan samanburð.
	}
}

\env{frame}
{
	\selectcode{code/kmp.c}{12}{21}
}

\env{frame}
{
	\env{itemize}
	{
		\item<1-> Svo þurfum við einfaldlega að labba í gegnum $s$ og hliðra eins og á við.
	}
}

\env{frame}
{
	\selectcode{code/kmp.c}{23}{32}
}

\env{frame}
{
	\env{itemize}
	{
		\item<1-> Takið eftir að hver ítrun innri lykkjanna svarar til einnar ítrunar ytri lykkjanna.
		\item<2-> Svo innri lykkjan keyrir, í heildina, ekki oftar en ytri lykkjan.
		\item<3-> Því fæst að tímaflækja forreikninganna $\mathcal{O}($\onslide<4->{$\,m\,$}$)$.
		\item<5-> Eins er tímaflækja strengjaleitarinnar $\mathcal{O}($\onslide<6->{$\,n\,$}$)$.
		\item<7-> Sama er því tímaflækjan $\mathcal{O}($\onslide<8->{$n + m$}$)$.
	}
}

\env{frame}
{
	\frametitle{Reiknirit Ahos og Corasicks ($1975$)}
	\env{itemize}
	{
		\item<1-> Til er önnur aðferð, svipuð og \texttt{KMP}, sem finnur staðsetningar margra orða í einu í streng.
		\item<2-> Hún er kennd við Aho og Corasick.
		\item<3-> Ég fer ekki í hana hér en hún byggir á því að gera stöðuvél.
		\item<4-> Reikniritið keyrir í línulegum tíma í lengd allra strengjanna, ásamt fjölda heppnaðra samanburða.
	}
}

\env{frame}
{
	\frametitle{Hlaupabil}
	\env{itemize}
	{
		\item<1-> Aðferð hlaupabila (e. sliding window) er stundum hægt að nota til að taka
				dæmi sem hafa augljósa $\mathcal{O}(n^2)$ og gera þau $\mathcal{O}(n)$ eða $\mathcal{O}(n\log n)$.
	}
}

\env{frame}
{
	\env{itemize}
	{
		\item<1-> Skoðum dæmi:
		\item<2-> Gefið $n$, $k$ og svo $n$ tölur $a_i$, þ.a. $a_i \in \{0, 1\}$ finndu
			lengd lengsta bils í rununni $(a_n)_{n \in \mathbb{N}}$ sem inniheldur bara $1$ ef þú mátt breyta allt að $k$ tölum.
		\item<3-> Sjáum strax að maður vill alltaf breyta $0$ í $1$ og aldrei öfugt.
		\item<4-> Sjáum því að við erum að leita að lengsta bili í $(a_n)_{n \in \mathbb{N}}$ sem hefur mesta $k$ stök jöfn $0$.
		\item<5-> Gefum okkur nú hlaupabil. Það byrjar tómt.
		\item<6-> Við löbbum svo í gegnum $(a_n)_{n \in \mathbb{N}}$ og lengjum bilið að aftan.
		\item<7-> Ef það eru einhvern tímann fleiri en $k$ stök í bilinu sem eru $0$ þá minnkum við bilið að framan þar til svo er ekki lengur.
	}
}

\defverbatim{\hbilA}
{ \begin{verbatim}
        k = 2
        l = 0
        [0 1 1 0 1 0 0 0 1 1 1 1 0 0 1 1]
        |
\end{verbatim} }

\defverbatim{\hbilB}
{ \begin{verbatim}
        k = 2
        l = 1
        [0 1 1 0 1 0 0 0 1 1 1 1 0 0 1 1]
        | |
\end{verbatim} }

\defverbatim{\hbilC}
{ \begin{verbatim}
        k = 2
        l = 2
        [0 1 1 0 1 0 0 0 1 1 1 1 0 0 1 1]
        |   |
\end{verbatim} }

\defverbatim{\hbilD}
{ \begin{verbatim}
        k = 2
        l = 3
        [0 1 1 0 1 0 0 0 1 1 1 1 0 0 1 1]
        |     |
\end{verbatim} }

\defverbatim{\hbilE}
{ \begin{verbatim}
        k = 2
        l = 4
        [0 1 1 0 1 0 0 0 1 1 1 1 0 0 1 1]
        |       |
\end{verbatim} }

\defverbatim{\hbilF}
{ \begin{verbatim}
        k = 2
        l = 5
        [0 1 1 0 1 0 0 0 1 1 1 1 0 0 1 1]
        |         |
\end{verbatim} }

\defverbatim{\hbilG}
{ \begin{verbatim}
        k = 2
        l = 4
        [0 1 1 0 1 0 0 0 1 1 1 1 0 0 1 1]
          |       |
\end{verbatim} }

\defverbatim{\hbilH}
{ \begin{verbatim}
        k = 2
        l = 5
        [0 1 1 0 1 0 0 0 1 1 1 1 0 0 1 1]
          |         |
\end{verbatim} }

\defverbatim{\hbilI}
{ \begin{verbatim}
        k = 2
        l = 4
        [0 1 1 0 1 0 0 0 1 1 1 1 0 0 1 1]
            |       |
\end{verbatim} }

\defverbatim{\hbilJ}
{ \begin{verbatim}
        k = 2
        l = 3
        [0 1 1 0 1 0 0 0 1 1 1 1 0 0 1 1]
              |     |
\end{verbatim} }

\defverbatim{\hbilK}
{ \begin{verbatim}
        k = 2
        l = 2
        [0 1 1 0 1 0 0 0 1 1 1 1 0 0 1 1]
                |   |
\end{verbatim} }

\defverbatim{\hbilL}
{ \begin{verbatim}
        k = 2
        l = 3
        [0 1 1 0 1 0 0 0 1 1 1 1 0 0 1 1]
                |     |
\end{verbatim} }

\defverbatim{\hbilM}
{ \begin{verbatim}
        k = 2
        l = 2
        [0 1 1 0 1 0 0 0 1 1 1 1 0 0 1 1]
                  |   |
\end{verbatim} }

\defverbatim{\hbilN}
{ \begin{verbatim}
        k = 2
        l = 1
        [0 1 1 0 1 0 0 0 1 1 1 1 0 0 1 1]
                    | |
\end{verbatim} }

\defverbatim{\hbilO}
{ \begin{verbatim}
        k = 2
        l = 2
        [0 1 1 0 1 0 0 0 1 1 1 1 0 0 1 1]
                    |   |
\end{verbatim} }

\defverbatim{\hbilP}
{ \begin{verbatim}
        k = 2
        l = 3
        [0 1 1 0 1 0 0 0 1 1 1 1 0 0 1 1]
                    |     |
\end{verbatim} }

\defverbatim{\hbilQ}
{ \begin{verbatim}
        k = 2
        l = 4
        [0 1 1 0 1 0 0 0 1 1 1 1 0 0 1 1]
                    |       |
\end{verbatim} }

\defverbatim{\hbilR}
{ \begin{verbatim}
        k = 2
        l = 5
        [0 1 1 0 1 0 0 0 1 1 1 1 0 0 1 1]
                    |         |
\end{verbatim} }

\defverbatim{\hbilS}
{ \begin{verbatim}
        k = 2
        l = 6
        [0 1 1 0 1 0 0 0 1 1 1 1 0 0 1 1]
                    |           |
\end{verbatim} }

\defverbatim{\hbilT}
{ \begin{verbatim}
        k = 2
        l = 5
        [0 1 1 0 1 0 0 0 1 1 1 1 0 0 1 1]
                      |         |
\end{verbatim} }

\defverbatim{\hbilU}
{ \begin{verbatim}
        k = 2
        l = 6
        [0 1 1 0 1 0 0 0 1 1 1 1 0 0 1 1]
                      |           |
\end{verbatim} }

\defverbatim{\hbilV}
{ \begin{verbatim}
        k = 2
        l = 5
        [0 1 1 0 1 0 0 0 1 1 1 1 0 0 1 1]
                        |         |
\end{verbatim} }

\defverbatim{\hbilW}
{ \begin{verbatim}
        k = 2
        l = 6
        [0 1 1 0 1 0 0 0 1 1 1 1 0 0 1 1]
                        |           |
\end{verbatim} }

\defverbatim{\hbilX}
{ \begin{verbatim}
        k = 2
        l = 7
        [0 1 1 0 1 0 0 0 1 1 1 1 0 0 1 1]
                        |             |
\end{verbatim} }

\defverbatim{\hbilY}
{ \begin{verbatim}
        k = 2
        l = 8
        [0 1 1 0 1 0 0 0 1 1 1 1 0 0 1 1]
                        |               |
\end{verbatim} }

\env{frame}
{
	\only<all:1>{\hbilA}
	\only<all:2>{\hbilB}
	\only<all:3>{\hbilC}
	\only<all:4>{\hbilD}
	\only<all:5>{\hbilE}
	\only<all:6>{\hbilF}
	\only<all:7>{\hbilG}
	\only<all:8>{\hbilH}
	\only<all:9>{\hbilI}
	\only<all:10>{\hbilJ}
	\only<all:11>{\hbilK}
	\only<all:12>{\hbilL}
	\only<all:13>{\hbilM}
	\only<all:14>{\hbilN}
	\only<all:15>{\hbilO}
	\only<all:16>{\hbilP}
	\only<all:17>{\hbilQ}
	\only<all:18>{\hbilR}
	\only<all:19>{\hbilS}
	\only<all:20>{\hbilT}
	\only<all:21>{\hbilU}
	\only<all:22>{\hbilV}
	\only<all:23>{\hbilW}
	\only<all:24>{\hbilX}
	\only<all:25>{\hbilY}
}

\env{frame}
{
	\code{code/hlaupabil-daemi1.c}
}

\env{frame}
{
	\env{itemize}
	{
		\item<1-> Hver tala í rununni er sett einu sinni í hlaupabilið og mögulega fjarlægð úr því.
		\item<2-> Svo tímaflækjan er $\mathcal{O}($\onslide<3->{$\,n\,$}$)$.
	}
}

\env{frame}
{
	\env{itemize}
	{
		\item<1-> Þetta dæmi er nú í auðveldari kantinum.
		\item<2-> Skoðum annað dæmi:
		\item<3-> Byjrum á nokkrum undirstöðu atriðum.
		\item<4-> Tvö bil kallast \emph{næstum sundurlæg} ef sniðmengi þeirra er tómt eða bara einn punktur.
		\item<5-> Sammengi bila má skrifa sem sammengi næstu sundurlægra bila.
		\item<6-> \emph{Lengd bilsins} $[a, b]$ er $b - a$.
		\item<7-> Til að finna \emph{lengd sammengis bila} skrifum við sammengið sem sammengi næstum sundurlægra bila
			og tökum summu lengda þeirra.
		\item<8-> Til dæmis eru bilin $[1, 2]$ og $[2, 3]$ næstum sundurlæg (en þó ekki sundurlæg) en 
			$[1, 3]$ og $[2, 4]$ eru það ekki. Nú $[1, 3] \cup [2, 4] = [1, 4]$ svo lengd 
			$[1, 3] \cup [2, 4]$ er $3$.
	}
}

\env{frame}
{
	\env{itemize}
	{
		\item<1-> Gefið $n$ bil hver er lengd sammengis þeirra.
	}
}

\env{frame}
{
	\env{itemize}
	{
		\item<1-> Geymum í lista tvenndir þar sem fyrra stakið er endapunktur bils og seinna stakið segir hvaða bili punkturinn tilleyrir.
		\item<2-> Röðum þessum punktum svo í vaxandi röð.
		\item<3-> Við löbbum í gegnum þennan raðaða lista og höldum utan um hlaupabil þannig að
			við bætum við bili í hlaupabilið þegar við rekumst á vinstri endapunkt þess og fjarlægjum það 
			þegar við rekumst á hægri endapunkt þess. 
		\item<4-> Við skoðum svo sérstaklega tilfellin þegar við erum ekki með nein bil í hlaupabilinu okkur.
		\item<5-> Sammengi þeirra bila sem við höfum farið í gegnum þá síðan hlaupabilið var síðast tómt er nú
			sundurlægt öllum öðrum bilum sem okkur var gefið í byrjun.
		\item<6-> Við skilum því summu lengda þessara sammengja.
	}
}

\defverbatim{\lineAA}
{ \begin{verbatim}
   |                                  
 1:  x------------x
 2:     x----x
 3:  x----x                    
 4:          x----------x
 5:                         x------x
 6:                                                x--x
 7:                                            x------x
 8:                           x--x
 9:                                      x------------x
10:                         x------x
   |                    
[]
r = 0
\end{verbatim} }

\defverbatim{\lineAB}
{ \begin{verbatim}
     |                                  
 1:  x------------x
 2:     x----x
 3:  x----x                    
 4:          x----------x
 5:                         x------x
 6:                                                x--x
 7:                                            x------x
 8:                           x--x
 9:                                      x------------x
10:                         x------x
     |                    
[]
r = 0
\end{verbatim} }

\defverbatim{\lineAC}
{ \begin{verbatim}
     |                                  
 1:  x------------x
 2:     x----x
 3:  x----x                    
 4:          x----------x
 5:                         x------x
 6:                                                x--x
 7:                                            x------x
 8:                           x--x
 9:                                      x------------x
10:                         x------x
     |                    
[1]
r = 0
\end{verbatim} }

\defverbatim{\lineAD}
{ \begin{verbatim}
     |                                  
 1:  x------------x
 2:     x----x
 3:  x----x                    
 4:          x----------x
 5:                         x------x
 6:                                                x--x
 7:                                            x------x
 8:                           x--x
 9:                                      x------------x
10:                         x------x
     |                    
[1, 3]
r = 0
\end{verbatim} }

\defverbatim{\lineAE}
{ \begin{verbatim}
        |                                  
 1:  x------------x
 2:     x----x
 3:  x----x                    
 4:          x----------x
 5:                         x------x
 6:                                                x--x
 7:                                            x------x
 8:                           x--x
 9:                                      x------------x
10:                         x------x
        |                    
[1, 3]
r = 0
\end{verbatim} }

\defverbatim{\lineAF}
{ \begin{verbatim}
        |                                  
 1:  x------------x
 2:     x----x
 3:  x----x                    
 4:          x----------x
 5:                         x------x
 6:                                                x--x
 7:                                            x------x
 8:                           x--x
 9:                                      x------------x
10:                         x------x
        |                    
[1, 2, 3]
r = 0
\end{verbatim} }

\defverbatim{\lineAG}
{ \begin{verbatim}
          |                                  
 1:  x------------x
 2:     x----x
 3:  x----x                    
 4:          x----------x
 5:                         x------x
 6:                                                x--x
 7:                                            x------x
 8:                           x--x
 9:                                      x------------x
10:                         x------x
          |                    
[1, 2, 3]
r = 0
\end{verbatim} }

\defverbatim{\lineAH}
{ \begin{verbatim}
          |                                  
 1:  x------------x
 2:     x----x
 3:  x----x                    
 4:          x----------x
 5:                         x------x
 6:                                                x--x
 7:                                            x------x
 8:                           x--x
 9:                                      x------------x
10:                         x------x
          |                    
[1, 2]
r = 0
\end{verbatim} }

\defverbatim{\lineAI}
{ \begin{verbatim}
             |                                      
 1:  x------------x
 2:     x----x
 3:  x----x                    
 4:          x----------x
 5:                         x------x
 6:                                                x--x
 7:                                            x------x
 8:                           x--x
 9:                                      x------------x
10:                         x------x
             |                        
[1, 2]
r = 0
\end{verbatim} }

\defverbatim{\lineAJ}
{ \begin{verbatim}
             |                                      
 1:  x------------x
 2:     x----x
 3:  x----x                    
 4:          x----------x
 5:                         x------x
 6:                                                x--x
 7:                                            x------x
 8:                           x--x
 9:                                      x------------x
10:                         x------x
             |                        
[1, 2, 4]
r = 0
\end{verbatim} }

\defverbatim{\lineAK}
{ \begin{verbatim}
             |
 1:  x------------x
 2:     x----x
 3:  x----x                    
 4:          x----------x
 5:                         x------x
 6:                                                x--x
 7:                                            x------x
 8:                           x--x
 9:                                      x------------x
10:                         x------x
             |
[1, 4]
r = 0
\end{verbatim} }

\defverbatim{\lineAL}
{ \begin{verbatim}
                  |
 1:  x------------x
 2:     x----x
 3:  x----x                    
 4:          x----------x
 5:                         x------x
 6:                                                x--x
 7:                                            x------x
 8:                           x--x
 9:                                      x------------x
10:                         x------x
                  |
[1, 4]
r = 0
\end{verbatim} }

\defverbatim{\lineAM}
{ \begin{verbatim}
                  |
 1:  x------------x
 2:     x----x
 3:  x----x                    
 4:          x----------x
 5:                         x------x
 6:                                                x--x
 7:                                            x------x
 8:                           x--x
 9:                                      x------------x
10:                         x------x
                  |
[4]
r = 0
\end{verbatim} }

\defverbatim{\lineAN}
{ \begin{verbatim}
                        |
 1:  x------------x
 2:     x----x
 3:  x----x                    
 4:          x----------x
 5:                         x------x
 6:                                                x--x
 7:                                            x------x
 8:                           x--x
 9:                                      x------------x
10:                         x------x
                        |
[4]
r = 0
\end{verbatim} }

\defverbatim{\lineAO}
{ \begin{verbatim}
                        |
 1:  x------------x
 2:     x----x
 3:  x----x                    
 4:          x----------x
 5:                         x------x
 6:                                                x--x
 7:                                            x------x
 8:                           x--x
 9:                                      x------------x
10:                         x------x
                        |
[]
r = 0
\end{verbatim} }

\defverbatim{\lineAP}
{ \begin{verbatim}
                        |
 1:  x------------x
 2:     x----x
 3:  x----x                    
 4:          x----------x
 5:                         x------x
 6:                                                x--x
 7:                                            x------x
 8:                           x--x
 9:                                      x------------x
10:                         x------x
                        |
[]
r = 20
\end{verbatim} }

\defverbatim{\lineAQ}
{ \begin{verbatim}
                            |
 1:  x------------x
 2:     x----x
 3:  x----x                    
 4:          x----------x
 5:                         x------x
 6:                                                x--x
 7:                                            x------x
 8:                           x--x
 9:                                      x------------x
10:                         x------x
                            |
[]
r = 20
\end{verbatim} }

\defverbatim{\lineAR}
{ \begin{verbatim}
                            |
 1:  x------------x
 2:     x----x
 3:  x----x                    
 4:          x----------x
 5:                         x------x
 6:                                                x--x
 7:                                            x------x
 8:                           x--x
 9:                                      x------------x
10:                         x------x
                            |
[5]
r = 20
\end{verbatim} }

\defverbatim{\lineAS}
{ \begin{verbatim}
                            |
 1:  x------------x
 2:     x----x
 3:  x----x                    
 4:          x----------x
 5:                         x------x
 6:                                                x--x
 7:                                            x------x
 8:                           x--x
 9:                                      x------------x
10:                         x------x
                            |
[5, 10]
r = 20
\end{verbatim} }

\defverbatim{\lineAU}
{ \begin{verbatim}
                              |
 1:  x------------x
 2:     x----x
 3:  x----x                    
 4:          x----------x
 5:                         x------x
 6:                                                x--x
 7:                                            x------x
 8:                           x--x
 9:                                      x------------x
10:                         x------x
                              |
[5, 10]
r = 20
\end{verbatim} }

\defverbatim{\lineAV}
{ \begin{verbatim}
                              |
 1:  x------------x
 2:     x----x
 3:  x----x                    
 4:          x----------x
 5:                         x------x
 6:                                                x--x
 7:                                            x------x
 8:                           x--x
 9:                                      x------------x
10:                         x------x
                              |
[5, 8, 10]
r = 20
\end{verbatim} }

\defverbatim{\lineAW}
{ \begin{verbatim}
                                 |
 1:  x------------x
 2:     x----x
 3:  x----x                    
 4:          x----------x
 5:                         x------x
 6:                                                x--x
 7:                                            x------x
 8:                           x--x
 9:                                      x------------x
10:                         x------x
                                 |
[5, 8, 10]
r = 20
\end{verbatim} }

\defverbatim{\lineAX}
{ \begin{verbatim}
                                 |
 1:  x------------x
 2:     x----x
 3:  x----x                    
 4:          x----------x
 5:                         x------x
 6:                                                x--x
 7:                                            x------x
 8:                           x--x
 9:                                      x------------x
10:                         x------x
                                 |
[5, 10]
r = 20
\end{verbatim} }

\defverbatim{\lineAY}
{ \begin{verbatim}
                                   |
 1:  x------------x
 2:     x----x
 3:  x----x                    
 4:          x----------x
 5:                         x------x
 6:                                                x--x
 7:                                            x------x
 8:                           x--x
 9:                                      x------------x
10:                         x------x
                                   |
[5, 10]
r = 20
\end{verbatim} }

\defverbatim{\lineAZ}
{ \begin{verbatim}
                                   |
 1:  x------------x
 2:     x----x
 3:  x----x                    
 4:          x----------x
 5:                         x------x
 6:                                                x--x
 7:                                            x------x
 8:                           x--x
 9:                                      x------------x
10:                         x------x
                                   |
[5]
r = 20
\end{verbatim} }

\defverbatim{\lineBA}
{ \begin{verbatim}
                                   |
 1:  x------------x
 2:     x----x
 3:  x----x                    
 4:          x----------x
 5:                         x------x
 6:                                                x--x
 7:                                            x------x
 8:                           x--x
 9:                                      x------------x
10:                         x------x
                                   |
[]
r = 20
\end{verbatim} }

\defverbatim{\lineBB}
{ \begin{verbatim}
                                   |
 1:  x------------x
 2:     x----x
 3:  x----x                    
 4:          x----------x
 5:                         x------x
 6:                                                x--x
 7:                                            x------x
 8:                           x--x
 9:                                      x------------x
10:                         x------x
                                   |
[]
r = 28
\end{verbatim} }

\defverbatim{\lineBC}
{ \begin{verbatim}
                                         |
 1:  x------------x
 2:     x----x
 3:  x----x                    
 4:          x----------x
 5:                         x------x
 6:                                                x--x
 7:                                            x------x
 8:                           x--x
 9:                                      x------------x
10:                         x------x
                                         |
[]
r = 28
\end{verbatim} }

\defverbatim{\lineBD}
{ \begin{verbatim}
                                         |
 1:  x------------x
 2:     x----x
 3:  x----x                    
 4:          x----------x
 5:                         x------x
 6:                                                x--x
 7:                                            x------x
 8:                           x--x
 9:                                      x------------x
10:                         x------x
                                         |
[9]
r = 28
\end{verbatim} }

\defverbatim{\lineBDD}
{ \begin{verbatim}
                                               |
 1:  x------------x
 2:     x----x
 3:  x----x                    
 4:          x----------x
 5:                         x------x
 6:                                                x--x
 7:                                            x------x
 8:                           x--x
 9:                                      x------------x
10:                         x------x
                                               |
[9]
r = 28
\end{verbatim} }

\defverbatim{\lineBE}
{ \begin{verbatim}
                                               |
 1:  x------------x
 2:     x----x
 3:  x----x                    
 4:          x----------x
 5:                         x------x
 6:                                                x--x
 7:                                            x------x
 8:                           x--x
 9:                                      x------------x
10:                         x------x
                                               |
[7, 9]
r = 28
\end{verbatim} }

\defverbatim{\lineBF}
{ \begin{verbatim}
                                                   |
 1:  x------------x
 2:     x----x
 3:  x----x                    
 4:          x----------x
 5:                         x------x
 6:                                                x--x
 7:                                            x------x
 8:                           x--x
 9:                                      x------------x
10:                         x------x
                                                   |
[7, 9]
r = 28
\end{verbatim} }

\defverbatim{\lineBG}
{ \begin{verbatim}
                                                   |
 1:  x------------x
 2:     x----x
 3:  x----x                    
 4:          x----------x
 5:                         x------x
 6:                                                x--x
 7:                                            x------x
 8:                           x--x
 9:                                      x------------x
10:                         x------x
                                                   |
[6, 7, 9]
r = 28
\end{verbatim} }

\defverbatim{\lineBH}
{ \begin{verbatim}
                                                      |
 1:  x------------x
 2:     x----x
 3:  x----x                    
 4:          x----------x
 5:                         x------x
 6:                                                x--x
 7:                                            x------x
 8:                           x--x
 9:                                      x------------x
10:                         x------x
                                                      |
[6, 7, 9]
r = 28
\end{verbatim} }

\defverbatim{\lineBI}
{ \begin{verbatim}
                                                      |
 1:  x------------x
 2:     x----x
 3:  x----x                    
 4:          x----------x
 5:                         x------x
 6:                                                x--x
 7:                                            x------x
 8:                           x--x
 9:                                      x------------x
10:                         x------x
                                                      |
[7, 9]
r = 28
\end{verbatim} }

\defverbatim{\lineBJ}
{ \begin{verbatim}
                                                      |
 1:  x------------x
 2:     x----x
 3:  x----x                    
 4:          x----------x
 5:                         x------x
 6:                                                x--x
 7:                                            x------x
 8:                           x--x
 9:                                      x------------x
10:                         x------x
                                                      |
[9]
r = 28
\end{verbatim} }

\defverbatim{\lineBK}
{ \begin{verbatim}
                                                      |
 1:  x------------x
 2:     x----x
 3:  x----x                    
 4:          x----------x
 5:                         x------x
 6:                                                x--x
 7:                                            x------x
 8:                           x--x
 9:                                      x------------x
10:                         x------x
                                                      |
[]
r = 28
\end{verbatim} }

\defverbatim{\lineBL}
{ \begin{verbatim}
                                                      |
 1:  x------------x
 2:     x----x
 3:  x----x                    
 4:          x----------x
 5:                         x------x
 6:                                                x--x
 7:                                            x------x
 8:                           x--x
 9:                                      x------------x
10:                         x------x
                                                      |
[]
r = 42
\end{verbatim} }

\env{frame}
{
	\only<all:1>{\lineAA}
	\only<all:2>{\lineAB}
	\only<all:3>{\lineAC}
	\only<all:4>{\lineAD}
	\only<all:5>{\lineAE}
	\only<all:6>{\lineAF}
	\only<all:7>{\lineAG}
	\only<all:8>{\lineAH}
	\only<all:9>{\lineAI}
	\only<all:10>{\lineAJ}
	\only<all:11>{\lineAK}
	\only<all:12>{\lineAL}
	\only<all:13>{\lineAM}
	\only<all:14>{\lineAN}
	\only<all:15>{\lineAO}
	\only<all:16>{\lineAP}
	\only<all:17>{\lineAQ}
	\only<all:18>{\lineAR}
	\only<all:19>{\lineAS}
	\only<all:20>{\lineAU}
	\only<all:21>{\lineAV}
	\only<all:22>{\lineAW}
	\only<all:23>{\lineAX}
	\only<all:24>{\lineAY}
	\only<all:25>{\lineAZ}
	\only<all:26>{\lineBA}
	\only<all:27>{\lineBB}
	\only<all:28>{\lineBC}
	\only<all:29>{\lineBD}
	\only<all:30>{\lineBDD}
	\only<all:31>{\lineBE}
	\only<all:32>{\lineBF}
	\only<all:33>{\lineBG}
	\only<all:34>{\lineBH}
	\only<all:35>{\lineBI}
	\only<all:36>{\lineBJ}
	\only<all:37>{\lineBK}
	\only<all:38>{\lineBL}
}

\env{frame}
{
	\code{code/hlaupabil-daemi2.c}
}

\env{frame}
{
	\env{itemize}
	{
		\item<1-> Við byrjum á að raða í $\mathcal{O}($\onslide<2->{$n \log n$}$)$ tíma.
		\item<3-> Síðan ítrum við í gegnum alla endapunktana sem tekur $\mathcal{O}($\onslide<4->{$\,n\,$}$)$ tíma.
		\item<5-> Svo lausnin hefur tímaflækjuna $\mathcal{O}($\onslide<6->{$n \log n$}$)$.
	}
}

\env{frame}
{
	\frametitle{Lengsta vaxandi hlutruna (\texttt{LIS})}
	\env{itemize}
	{
		\item<1-> Hlutruna í talnarunu er runa af tölum, allar úr upprunalegu rununni, sem eru í sömu röð og í upprunalegu rununni.
		\item<2-> Hvernig getum við fundið \emph{lengtsu vaxandi hlutrunu} (e. \emph{longest increasing subsequence} (\texttt{LIS})) í gefinni runu?
		\item<3-> Sem dæmi er \texttt{[2 3 5 9]} ein ef lengstu vaxandi hlutrunum \texttt{[2 3 1 5 9 8 7]}.
	}
}

\env{frame}
{
	\env{itemize}
	{
		\item<1-> Gerum ráð fyrir að við höfum talnarunu af lengd $1 \leq n \leq 15$.
		\item<2-> Við getum þá prófað allar hlutrunur, skoðað hvort þær séu vaxandi og geymt þá lengstu.
	}
}

\env{frame}
{
	\code{code/lis-bf.c}
}

\env{frame}
{
	\env{itemize}
	{
		\item<1-> Við þurfum að skoða öll hlutmengi vísamengis rununnar og fyrir hvert þeirra þarf að ítra í gegnum allt vísamengið.
		\item<2-> Þar sem vísamengið inniheldur $n$ stök hefur þessi lausn tímaflækjuna $\mathcal{O}($\onslide<3->{$n \cdot 2^n$}$)$.
	}
}

\env{frame}
{
	\env{itemize}
	{
		\item<1-> Það má þó gera þetta hraðar.
		\item<2-> Við getum notað kvika bestun!
		\item<3-> Látum \texttt{a} vera runu af $n$ tölum.
		\item<4-> Látum $f(k)$ vera lengd lengstu hlutrunu sem endar í $k$-ta staki $a$.
		\item<5-> Við höfum þá að
		\[
			f(x) = \left \{
			\env{array}
			{ {l l}
				1, & \text{ef $x = 1$} \\
				\displaystyle \max_{\substack{1 \leq k < x\\a_k \leq a_x}} f(k) + 1 & \text{annars.}
			}
			\right .
		\]
	}
}

\env{frame}
{
	\selectcode{code/lis-dp.c}{6}{15}
}

\env{frame}
{
	\env{itemize}
	{
		\item<1-> Við höfum $n$ stöður og hverja stöðu má reikna í $\mathcal{O}($\onslide<2->{$\,n\,$}$)$ tíma.
		\item<3-> Svo þessi lausn hefur tímaflækjuna $\mathcal{O}($\onslide<4->{$n^2$}$)$ tíma.
	}
}

\env{frame}
{
	\env{itemize}
	{
		\item<1-> En er hægt að gera þetta ennþá hraðar?
		\item<2-> Heldur betur!
		\item<3-> Við getum notað helmingunarleit.
		\item<4-> Skoðum first reiknirit sem er ekki hentugt að útfæra.
	}
}

\env{frame}
{
	\env{itemize}
	{
		\item<1-> Höfum lista af listum.
		\item<2-> Skilgreinum röðun þannig að listar eru bornir saman eftir síðasta staki.
		\item<3-> Listalistinn okkar byrjar tómur.
		\item<4-> Löbbum í gegnum \texttt{a} í réttri röð og fyrir hvert stak \texttt{a[i]} finnum við
			þann lista sem hefur stærsta aftasta stakið sem er minna en \texttt{a[i]}.
		\item<5-> Við afritum nú listann sem við fundum, setjum hann fyrir aftan, bætum stakinu okkar við hann
			og fjarlægjum listann fyrir aftann nýja listann (ef það er einhver).
		\item<6-> Að þessu loknu er aftasti listinn í listalistanum einn af lengstu vaxandi hlutrununum.
		\item<7-> Rúllum í gegnum þetta fyrir listann \texttt{[0 8 4 12 2 10 6 14 1 9 5 13 3 11 7 15]}.
	}
}

\defverbatim{\lisAA}
{ \begin{verbatim}
[0, 8, 4, 12, 2, 10, 6, 14, 1, 9, 5, 13, 3, 11, 7, 15]
 |

   []






\end{verbatim} }

\defverbatim{\lisAB}
{ \begin{verbatim}
[0, 8, 4, 12, 2, 10, 6, 14, 1, 9, 5, 13, 3, 11, 7, 15]
 |

-> []






\end{verbatim} }

\defverbatim{\lisAC}
{ \begin{verbatim}
[0, 8, 4, 12, 2, 10, 6, 14, 1, 9, 5, 13, 3, 11, 7, 15]
 |

   []
   []





\end{verbatim} }

\defverbatim{\lisAD}
{ \begin{verbatim}
[0, 8, 4, 12, 2, 10, 6, 14, 1, 9, 5, 13, 3, 11, 7, 15]
 |

   []
   [0]





\end{verbatim} }

\defverbatim{\lisAE}
{ \begin{verbatim}
[0, 8, 4, 12, 2, 10, 6, 14, 1, 9, 5, 13, 3, 11, 7, 15]
    |

   []
   [0]





\end{verbatim} }

\defverbatim{\lisAF}
{ \begin{verbatim}
[0, 8, 4, 12, 2, 10, 6, 14, 1, 9, 5, 13, 3, 11, 7, 15]
    |

   []
-> [0]





\end{verbatim} }

\defverbatim{\lisAG}
{ \begin{verbatim}
[0, 8, 4, 12, 2, 10, 6, 14, 1, 9, 5, 13, 3, 11, 7, 15]
    |

   []
   [0]
   [0]




\end{verbatim} }

\defverbatim{\lisAH}
{ \begin{verbatim}
[0, 8, 4, 12, 2, 10, 6, 14, 1, 9, 5, 13, 3, 11, 7, 15]
    |

   []
   [0]
   [0, 8]




\end{verbatim} }

\defverbatim{\lisAI}
{ \begin{verbatim}
[0, 8, 4, 12, 2, 10, 6, 14, 1, 9, 5, 13, 3, 11, 7, 15]
       |

   []
   [0]
   [0, 8]




\end{verbatim} }

\defverbatim{\lisAJ}
{ \begin{verbatim}
[0, 8, 4, 12, 2, 10, 6, 14, 1, 9, 5, 13, 3, 11, 7, 15]
       |

   []
-> [0]
   [0, 8]




\end{verbatim} }

\defverbatim{\lisAK}
{ \begin{verbatim}
[0, 8, 4, 12, 2, 10, 6, 14, 1, 9, 5, 13, 3, 11, 7, 15]
       |

   []
   [0]
   [0]
   [0, 8]



\end{verbatim} }

\defverbatim{\lisAL}
{ \begin{verbatim}
[0, 8, 4, 12, 2, 10, 6, 14, 1, 9, 5, 13, 3, 11, 7, 15]
       |

   []
   [0]
   [0, 4]
   [0, 8]



\end{verbatim} }

\defverbatim{\lisAM}
{ \begin{verbatim}
[0, 8, 4, 12, 2, 10, 6, 14, 1, 9, 5, 13, 3, 11, 7, 15]
       |

   []
   [0]
   [0, 4]




\end{verbatim} }

\defverbatim{\lisAN}
{ \begin{verbatim}
[0, 8, 4, 12, 2, 10, 6, 14, 1, 9, 5, 13, 3, 11, 7, 15]
           |

   []
   [0]
   [0, 4]




\end{verbatim} }

\defverbatim{\lisAO}
{ \begin{verbatim}
[0, 8, 4, 12, 2, 10, 6, 14, 1, 9, 5, 13, 3, 11, 7, 15]
           |

   []
   [0]
-> [0, 4]




\end{verbatim} }

\defverbatim{\lisAP}
{ \begin{verbatim}
[0, 8, 4, 12, 2, 10, 6, 14, 1, 9, 5, 13, 3, 11, 7, 15]
           |

   []
   [0]
   [0, 4]
   [0, 4]



\end{verbatim} }

\defverbatim{\lisAQ}
{ \begin{verbatim}
[0, 8, 4, 12, 2, 10, 6, 14, 1, 9, 5, 13, 3, 11, 7, 15]
           |

   []
   [0]
   [0, 4]
   [0, 4, 12]



\end{verbatim} }

\defverbatim{\lisAR}
{ \begin{verbatim}
[0, 8, 4, 12, 2, 10, 6, 14, 1, 9, 5, 13, 3, 11, 7, 15]
              |

   []
   [0]
   [0, 4]
   [0, 4, 12]



\end{verbatim} }

\defverbatim{\lisAS}
{ \begin{verbatim}
[0, 8, 4, 12, 2, 10, 6, 14, 1, 9, 5, 13, 3, 11, 7, 15]
              |

   []
-> [0]
   [0, 4]
   [0, 4, 12]



\end{verbatim} }

\defverbatim{\lisAT}
{ \begin{verbatim}
[0, 8, 4, 12, 2, 10, 6, 14, 1, 9, 5, 13, 3, 11, 7, 15]
              |

   []
   [0]
   [0]
   [0, 4]
   [0, 4, 12]


\end{verbatim} }

\defverbatim{\lisAU}
{ \begin{verbatim}
[0, 8, 4, 12, 2, 10, 6, 14, 1, 9, 5, 13, 3, 11, 7, 15]
              |

   []
   [0]
   [0, 2]
   [0, 4]
   [0, 4, 12]


\end{verbatim} }

\defverbatim{\lisAV}
{ \begin{verbatim}
[0, 8, 4, 12, 2, 10, 6, 14, 1, 9, 5, 13, 3, 11, 7, 15]
              |

   []
   [0]
   [0, 2]
   [0, 4, 12]



\end{verbatim} }

\defverbatim{\lisAW}
{ \begin{verbatim}
[0, 8, 4, 12, 2, 10, 6, 14, 1, 9, 5, 13, 3, 11, 7, 15]
                  |

   []
   [0]
   [0, 2]
   [0, 4, 12]



\end{verbatim} }

\defverbatim{\lisAX}
{ \begin{verbatim}
[0, 8, 4, 12, 2, 10, 6, 14, 1, 9, 5, 13, 3, 11, 7, 15]
                  |

   []
   [0]
-> [0, 2]
   [0, 4, 12]



\end{verbatim} }

\defverbatim{\lisAY}
{ \begin{verbatim}
[0, 8, 4, 12, 2, 10, 6, 14, 1, 9, 5, 13, 3, 11, 7, 15]
                  |

   []
   [0]
   [0, 2]
   [0, 2]
   [0, 4, 12]


\end{verbatim} }

\defverbatim{\lisAZ}
{ \begin{verbatim}
[0, 8, 4, 12, 2, 10, 6, 14, 1, 9, 5, 13, 3, 11, 7, 15]
                  |

   []
   [0]
   [0, 2]
   [0, 2, 10]
   [0, 4, 12]


\end{verbatim} }

\defverbatim{\lisBA}
{ \begin{verbatim}
[0, 8, 4, 12, 2, 10, 6, 14, 1, 9, 5, 13, 3, 11, 7, 15]
                  |

   []
   [0]
   [0, 2]
   [0, 2, 10]



\end{verbatim} }

\defverbatim{\lisBB}
{ \begin{verbatim}
[0, 8, 4, 12, 2, 10, 6, 14, 1, 9, 5, 13, 3, 11, 7, 15]
                     |

   []
   [0]
   [0, 2]
   [0, 2, 10]



\end{verbatim} }

\defverbatim{\lisBC}
{ \begin{verbatim}
[0, 8, 4, 12, 2, 10, 6, 14, 1, 9, 5, 13, 3, 11, 7, 15]
                     |

   []
   [0]
-> [0, 2]
   [0, 2, 10]



\end{verbatim} }

\defverbatim{\lisBD}
{ \begin{verbatim}
[0, 8, 4, 12, 2, 10, 6, 14, 1, 9, 5, 13, 3, 11, 7, 15]
                     |

   []
   [0]
   [0, 2]
   [0, 2]
   [0, 2, 10]


\end{verbatim} }

\defverbatim{\lisBE}
{ \begin{verbatim}
[0, 8, 4, 12, 2, 10, 6, 14, 1, 9, 5, 13, 3, 11, 7, 15]
                     |

   []
   [0]
   [0, 2]
   [0, 2, 6]
   [0, 2, 10]


\end{verbatim} }

\defverbatim{\lisBF}
{ \begin{verbatim}
[0, 8, 4, 12, 2, 10, 6, 14, 1, 9, 5, 13, 3, 11, 7, 15]
                     |

   []
   [0]
   [0, 2]
   [0, 2, 6]



\end{verbatim} }

\defverbatim{\lisBG}
{ \begin{verbatim}
[0, 8, 4, 12, 2, 10, 6, 14, 1, 9, 5, 13, 3, 11, 7, 15]
                         |

   []
   [0]
   [0, 2]
   [0, 2, 6]



\end{verbatim} }

\defverbatim{\lisBH}
{ \begin{verbatim}
[0, 8, 4, 12, 2, 10, 6, 14, 1, 9, 5, 13, 3, 11, 7, 15]
                         |

   []
   [0]
   [0, 2]
-> [0, 2, 6]



\end{verbatim} }

\defverbatim{\lisBI}
{ \begin{verbatim}
[0, 8, 4, 12, 2, 10, 6, 14, 1, 9, 5, 13, 3, 11, 7, 15]
                         |

   []
   [0]
   [0, 2]
   [0, 2, 6]
   [0, 2, 6]


\end{verbatim} }

\defverbatim{\lisBJ}
{ \begin{verbatim}
[0, 8, 4, 12, 2, 10, 6, 14, 1, 9, 5, 13, 3, 11, 7, 15]
                         |

   []
   [0]
   [0, 2]
   [0, 2, 6]
   [0, 2, 6, 14]


\end{verbatim} }

\defverbatim{\lisBK}
{ \begin{verbatim}
[0, 8, 4, 12, 2, 10, 6, 14, 1, 9, 5, 13, 3, 11, 7, 15]
                            |

   []
   [0]
   [0, 2]
   [0, 2, 6]
   [0, 2, 6, 14]


\end{verbatim} }

\defverbatim{\lisBL}
{ \begin{verbatim}
[0, 8, 4, 12, 2, 10, 6, 14, 1, 9, 5, 13, 3, 11, 7, 15]
                            |

   []
-> [0]
   [0, 2]
   [0, 2, 6]
   [0, 2, 6, 14]


\end{verbatim} }

\defverbatim{\lisBM}
{ \begin{verbatim}
[0, 8, 4, 12, 2, 10, 6, 14, 1, 9, 5, 13, 3, 11, 7, 15]
                            |

   []
   [0]
   [0]
   [0, 2]
   [0, 2, 6]
   [0, 2, 6, 14]

\end{verbatim} }

\defverbatim{\lisBN}
{ \begin{verbatim}
[0, 8, 4, 12, 2, 10, 6, 14, 1, 9, 5, 13, 3, 11, 7, 15]
                            |

   []
   [0]
   [0, 1]
   [0, 2]
   [0, 2, 6]
   [0, 2, 6, 14]

\end{verbatim} }

\defverbatim{\lisBO}
{ \begin{verbatim}
[0, 8, 4, 12, 2, 10, 6, 14, 1, 9, 5, 13, 3, 11, 7, 15]
                            |

   []
   [0]
   [0, 1]
   [0, 2, 6]
   [0, 2, 6, 14]


\end{verbatim} }

\defverbatim{\lisBP}
{ \begin{verbatim}
[0, 8, 4, 12, 2, 10, 6, 14, 1, 9, 5, 13, 3, 11, 7, 15]
                               |

   []
   [0]
   [0, 1]
   [0, 2, 6]
   [0, 2, 6, 14]


\end{verbatim} }

\defverbatim{\lisBQ}
{ \begin{verbatim}
[0, 8, 4, 12, 2, 10, 6, 14, 1, 9, 5, 13, 3, 11, 7, 15]
                               |

   []
   [0]
   [0, 1]
-> [0, 2, 6]
   [0, 2, 6, 14]


\end{verbatim} }

\defverbatim{\lisBR}
{ \begin{verbatim}
[0, 8, 4, 12, 2, 10, 6, 14, 1, 9, 5, 13, 3, 11, 7, 15]
                               |

   []
   [0]
   [0, 1]
   [0, 2, 6]
   [0, 2, 6]
   [0, 2, 6, 14]

\end{verbatim} }

\defverbatim{\lisBS}
{ \begin{verbatim}
[0, 8, 4, 12, 2, 10, 6, 14, 1, 9, 5, 13, 3, 11, 7, 15]
                               |

   []
   [0]
   [0, 1]
   [0, 2, 6]
   [0, 2, 6, 9]
   [0, 2, 6, 14]

\end{verbatim} }

\defverbatim{\lisBT}
{ \begin{verbatim}
[0, 8, 4, 12, 2, 10, 6, 14, 1, 9, 5, 13, 3, 11, 7, 15]
                               |

   []
   [0]
   [0, 1]
   [0, 2, 6]
   [0, 2, 6, 9]


\end{verbatim} }

\defverbatim{\lisBU}
{ \begin{verbatim}
[0, 8, 4, 12, 2, 10, 6, 14, 1, 9, 5, 13, 3, 11, 7, 15]
                                  |

   []
   [0]
   [0, 1]
   [0, 2, 6]
   [0, 2, 6, 9]


\end{verbatim} }

\defverbatim{\lisBV}
{ \begin{verbatim}
[0, 8, 4, 12, 2, 10, 6, 14, 1, 9, 5, 13, 3, 11, 7, 15]
                                  |

   []
   [0]
-> [0, 1]
   [0, 2, 6]
   [0, 2, 6, 9]


\end{verbatim} }

\defverbatim{\lisBW}
{ \begin{verbatim}
[0, 8, 4, 12, 2, 10, 6, 14, 1, 9, 5, 13, 3, 11, 7, 15]
                                  |

   []
   [0]
   [0, 1]
   [0, 1]
   [0, 2, 6]
   [0, 2, 6, 9]

\end{verbatim} }

\defverbatim{\lisBX}
{ \begin{verbatim}
[0, 8, 4, 12, 2, 10, 6, 14, 1, 9, 5, 13, 3, 11, 7, 15]
                                  |

   []
   [0]
   [0, 1]
   [0, 1, 5]
   [0, 2, 6]
   [0, 2, 6, 9]

\end{verbatim} }

\defverbatim{\lisBY}
{ \begin{verbatim}
[0, 8, 4, 12, 2, 10, 6, 14, 1, 9, 5, 13, 3, 11, 7, 15]
                                  |

   []
   [0]
   [0, 1]
   [0, 1, 5]
   [0, 2, 6, 9]


\end{verbatim} }

\defverbatim{\lisBZ}
{ \begin{verbatim}
[0, 8, 4, 12, 2, 10, 6, 14, 1, 9, 5, 13, 3, 11, 7, 15]
                                      |

   []
   [0]
   [0, 1]
   [0, 1, 5]
   [0, 2, 6, 9]


\end{verbatim} }

\defverbatim{\lisBZ}
{ \begin{verbatim}
[0, 8, 4, 12, 2, 10, 6, 14, 1, 9, 5, 13, 3, 11, 7, 15]
                                      |

   []
   [0]
   [0, 1]
   [0, 1, 5]
-> [0, 2, 6, 9]


\end{verbatim} }

\defverbatim{\lisCA}
{ \begin{verbatim}
[0, 8, 4, 12, 2, 10, 6, 14, 1, 9, 5, 13, 3, 11, 7, 15]
                                      |

   []
   [0]
   [0, 1]
   [0, 1, 5]
   [0, 2, 6, 9]
   [0, 2, 6, 9]

\end{verbatim} }

\defverbatim{\lisCB}
{ \begin{verbatim}
[0, 8, 4, 12, 2, 10, 6, 14, 1, 9, 5, 13, 3, 11, 7, 15]
                                      |

   []
   [0]
   [0, 1]
   [0, 1, 5]
   [0, 2, 6, 9]
   [0, 2, 6, 9, 14]

\end{verbatim} }

\defverbatim{\lisCC}
{ \begin{verbatim}
[0, 8, 4, 12, 2, 10, 6, 14, 1, 9, 5, 13, 3, 11, 7, 15]
                                         |

   []
   [0]
   [0, 1]
   [0, 1, 5]
   [0, 2, 6, 9]
   [0, 2, 6, 9, 14]

\end{verbatim} }

\defverbatim{\lisCD}
{ \begin{verbatim}
[0, 8, 4, 12, 2, 10, 6, 14, 1, 9, 5, 13, 3, 11, 7, 15]
                                         |

   []
   [0]
-> [0, 1]
   [0, 1, 5]
   [0, 2, 6, 9]
   [0, 2, 6, 9, 14]

\end{verbatim} }

\defverbatim{\lisCE}
{ \begin{verbatim}
[0, 8, 4, 12, 2, 10, 6, 14, 1, 9, 5, 13, 3, 11, 7, 15]
                                         |

   []
   [0]
   [0, 1]
   [0, 1]
   [0, 1, 5]
   [0, 2, 6, 9]
   [0, 2, 6, 9, 14]
\end{verbatim} }

\defverbatim{\lisCF}
{ \begin{verbatim}
[0, 8, 4, 12, 2, 10, 6, 14, 1, 9, 5, 13, 3, 11, 7, 15]
                                         |

   []
   [0]
   [0, 1]
   [0, 1, 3]
   [0, 1, 5]
   [0, 2, 6, 9]
   [0, 2, 6, 9, 14]
\end{verbatim} }

\defverbatim{\lisCG}
{ \begin{verbatim}
[0, 8, 4, 12, 2, 10, 6, 14, 1, 9, 5, 13, 3, 11, 7, 15]
                                         |

   []
   [0]
   [0, 1]
   [0, 1, 3]
   [0, 2, 6, 9]
   [0, 2, 6, 9, 14]

\end{verbatim} }

\defverbatim{\lisCH}
{ \begin{verbatim}
[0, 8, 4, 12, 2, 10, 6, 14, 1, 9, 5, 13, 3, 11, 7, 15]
                                             |

   []
   [0]
   [0, 1]
   [0, 1, 3]
   [0, 2, 6, 9]
   [0, 2, 6, 9, 14]

\end{verbatim} }

\defverbatim{\lisCI}
{ \begin{verbatim}
[0, 8, 4, 12, 2, 10, 6, 14, 1, 9, 5, 13, 3, 11, 7, 15]
                                             |

   []
   [0]
   [0, 1]
   [0, 1, 3]
-> [0, 2, 6, 9]
   [0, 2, 6, 9, 14]

\end{verbatim} }

\defverbatim{\lisCJ}
{ \begin{verbatim}
[0, 8, 4, 12, 2, 10, 6, 14, 1, 9, 5, 13, 3, 11, 7, 15]
                                             |

   []
   [0]
   [0, 1]
   [0, 1, 3]
   [0, 2, 6, 9]
   [0, 2, 6, 9]
   [0, 2, 6, 9, 14]
\end{verbatim} }

\defverbatim{\lisCK}
{ \begin{verbatim}
[0, 8, 4, 12, 2, 10, 6, 14, 1, 9, 5, 13, 3, 11, 7, 15]
                                             |

   []
   [0]
   [0, 1]
   [0, 1, 3]
   [0, 2, 6, 9]
   [0, 2, 6, 9, 11]
   [0, 2, 6, 9, 14]
\end{verbatim} }

\defverbatim{\lisCL}
{ \begin{verbatim}
[0, 8, 4, 12, 2, 10, 6, 14, 1, 9, 5, 13, 3, 11, 7, 15]
                                             |

   []
   [0]
   [0, 1]
   [0, 1, 3]
   [0, 2, 6, 9]
   [0, 2, 6, 9, 11]

\end{verbatim} }

\defverbatim{\lisCM}
{ \begin{verbatim}
[0, 8, 4, 12, 2, 10, 6, 14, 1, 9, 5, 13, 3, 11, 7, 15]
                                                |

   []
   [0]
   [0, 1]
   [0, 1, 3]
   [0, 2, 6, 9]
   [0, 2, 6, 9, 11]

\end{verbatim} }

\defverbatim{\lisCN}
{ \begin{verbatim}
[0, 8, 4, 12, 2, 10, 6, 14, 1, 9, 5, 13, 3, 11, 7, 15]
                                                |

   []
   [0]
   [0, 1]
-> [0, 1, 3]
   [0, 2, 6, 9]
   [0, 2, 6, 9, 11]

\end{verbatim} }

\defverbatim{\lisCO}
{ \begin{verbatim}
[0, 8, 4, 12, 2, 10, 6, 14, 1, 9, 5, 13, 3, 11, 7, 15]
                                                |

   []
   [0]
   [0, 1]
   [0, 1, 3]
   [0, 1, 3]
   [0, 2, 6, 9]
   [0, 2, 6, 9, 11]
\end{verbatim} }

\defverbatim{\lisCP}
{ \begin{verbatim}
[0, 8, 4, 12, 2, 10, 6, 14, 1, 9, 5, 13, 3, 11, 7, 15]
                                                |

   []
   [0]
   [0, 1]
   [0, 1, 3]
   [0, 1, 3, 7]
   [0, 2, 6, 9]
   [0, 2, 6, 9, 11]
\end{verbatim} }

\defverbatim{\lisCQ}
{ \begin{verbatim}
[0, 8, 4, 12, 2, 10, 6, 14, 1, 9, 5, 13, 3, 11, 7, 15]
                                                |

   []
   [0]
   [0, 1]
   [0, 1, 3]
   [0, 1, 3, 7]
   [0, 2, 6, 9, 11]

\end{verbatim} }

\defverbatim{\lisCR}
{ \begin{verbatim}
[0, 8, 4, 12, 2, 10, 6, 14, 1, 9, 5, 13, 3, 11, 7, 15]
                                                    |

   []
   [0]
   [0, 1]
   [0, 1, 3]
   [0, 1, 3, 7]
   [0, 2, 6, 9, 11]

\end{verbatim} }

\defverbatim{\lisCS}
{ \begin{verbatim}
[0, 8, 4, 12, 2, 10, 6, 14, 1, 9, 5, 13, 3, 11, 7, 15]
                                                    |

   []
   [0]
   [0, 1]
   [0, 1, 3]
   [0, 1, 3, 7]
-> [0, 2, 6, 9, 11]

\end{verbatim} }

\defverbatim{\lisCT}
{ \begin{verbatim}
[0, 8, 4, 12, 2, 10, 6, 14, 1, 9, 5, 13, 3, 11, 7, 15]
                                                    |

   []
   [0]
   [0, 1]
   [0, 1, 3]
   [0, 1, 3, 7]
   [0, 2, 6, 9, 11]
   [0, 2, 6, 9, 11]
\end{verbatim} }

\defverbatim{\lisCU}
{ \begin{verbatim}
[0, 8, 4, 12, 2, 10, 6, 14, 1, 9, 5, 13, 3, 11, 7, 15]
                                                    |

   []
   [0]
   [0, 1]
   [0, 1, 3]
   [0, 1, 3, 7]
   [0, 2, 6, 9, 11]
   [0, 2, 6, 9, 11, 15]
\end{verbatim} }

\defverbatim{\lisCV}
{ \begin{verbatim}
[0, 8, 4, 12, 2, 10, 6, 14, 1, 9, 5, 13, 3, 11, 7, 15]


   []
   [0]
   [0, 1]
   [0, 1, 3]
   [0, 1, 3, 7]
   [0, 2, 6, 9, 11]
   [0, 2, 6, 9, 11, 15]
\end{verbatim} }

% find, dupe, add, dump, bump
\env{frame}
{
	\only<all:1>{\lisAA}
	\only<all:2>{\lisAB}
	\only<all:3>{\lisAC}
	\only<all:4>{\lisAD}
	\only<all:5>{\lisAE}
	\only<all:6>{\lisAF}
	\only<all:7>{\lisAG}
	\only<all:8>{\lisAH}
	\only<all:9>{\lisAI}
	\only<all:10>{\lisAJ}
	\only<all:11>{\lisAK}
	\only<all:12>{\lisAL}
	\only<all:13>{\lisAM}
	\only<all:14>{\lisAN}
	\only<all:15>{\lisAO}
	\only<all:16>{\lisAP}
	\only<all:17>{\lisAQ}
	\only<all:18>{\lisAR}
	\only<all:19>{\lisAS}
	\only<all:20>{\lisAT}
	\only<all:21>{\lisAU}
	\only<all:22>{\lisAV}
	\only<all:23>{\lisAW}
	\only<all:24>{\lisAX}
	\only<all:25>{\lisAY}
	\only<all:26>{\lisAZ}
	\only<all:27>{\lisBA}
	\only<all:28>{\lisBB}
	\only<all:29>{\lisBC}
	\only<all:30>{\lisBD}
	\only<all:31>{\lisBE}
	\only<all:32>{\lisBF}
	\only<all:33>{\lisBG}
	\only<all:34>{\lisBH}
	\only<all:35>{\lisBI}
	\only<all:36>{\lisBJ}
	\only<all:37>{\lisBK}
	\only<all:38>{\lisBL}
	\only<all:39>{\lisBM}
	\only<all:40>{\lisBN}
	\only<all:41>{\lisBO}
	\only<all:42>{\lisBP}
	\only<all:43>{\lisBQ}
	\only<all:44>{\lisBR}
	\only<all:45>{\lisBS}
	\only<all:46>{\lisBT}
	\only<all:47>{\lisBU}
	\only<all:48>{\lisBV}
	\only<all:49>{\lisBW}
	\only<all:50>{\lisBX}
	\only<all:51>{\lisBY}
	\only<all:52>{\lisBZ}
	\only<all:53>{\lisCA}
	\only<all:54>{\lisCB}
	\only<all:55>{\lisCC}
	\only<all:56>{\lisCD}
	\only<all:57>{\lisCE}
	\only<all:58>{\lisCF}
	\only<all:59>{\lisCG}
	\only<all:60>{\lisCH}
	\only<all:61>{\lisCI}
	\only<all:62>{\lisCJ}
	\only<all:63>{\lisCK}
	\only<all:64>{\lisCL}
	\only<all:65>{\lisCM}
	\only<all:66>{\lisCN}
	\only<all:67>{\lisCO}
	\only<all:68>{\lisCP}
	\only<all:69>{\lisCQ}
	\only<all:70>{\lisCR}
	\only<all:71>{\lisCS}
	\only<all:72>{\lisCT}
	\only<all:73>{\lisCU}
	\only<all:74>{\lisCV}
}

\env{frame}
{
	\env{itemize}
	{
		\item<1-> Hvernig útfærum við þetta?
		\item<2-> Við nýtum okkur það að listarnir sem við vorum með eru að mestu óþarfir því við þurfum bara aftasta stakið í hverjum þeirra.
		\item<3-> Við erum þá með lista af tölum, sem gerir allt mun auðveldara.
	}
}

\env{frame}
{
	\selectcode{code/lis.c}{12}{21}
}

\env{frame}
{
	\env{itemize}
	{
		\item<1-> Takið eftir að í hverju skrefi notum við helmingunarleit.
		\item<2-> Svo tímaflækjan er $\mathcal{O}($\onslide<3->{$n \log n$}$)$.
	}
}

\env{frame}
{
	\frametitle{Næsta stærra stak (\texttt{NGE})}
	\env{itemize}
	{
		\item<1-> Látum \texttt{a} vera lista af $n$ tölum.
		\item<2-> Við segjum að \emph{næsta stak stærra en \texttt{a[i]}} (e. \emph{next greater element} (\texttt{NGE}))
			sé minnsta stak \texttt{a[j]} þ.a. $j > i$.
		\item<3-> Sem dæmi er \texttt{NGE} miðju stakins $4$ í listanum \texttt{[2, 3, 4, 8, 5]} talan $8$.
		\item<4-> Til þæginda segjum við að \texttt{NGE} tölunnar $8$ í listanum \texttt{[2, 3, 4, 8, 5]} sé $-1$.
		\item<5-> Það er auðséð að við getum reiknað \texttt{NGE} allra talnanna með tvöfaldri \texttt{for}-lykkju.
	}
}

\env{frame}
{
	\selectcode{code/slow-nge.c}{4}{12}
}

\env{frame}
{
	\env{itemize}
	{
		\item<1-> Þar sem þessi lausnir er tvöföld \texttt{for}-lykkja, hvor af lengd $n$, þá er lausnin $\mathcal{O}($\onslide<2->{$n^2$}$)$.
	}
}

\env{frame}
{
	\env{itemize}
	{
		\item<1-> En þetta má bæta.
		\item<2-> Gefum okkur hlaða \texttt{h}. 
		\item<3-> Löbbum í gegnum \texttt{a} í réttri röð.
		\item<4-> Tökum nú tölur úr hlaðan og setjum \texttt{NGE} þeirra talna sem \texttt{a[i]} á meðan \texttt{a[i]} er stærri en toppurinn á hlaðanum.
		\item<5-> Þegar toppurinn á hlaðanum er stærri en \texttt{a[i]} þá látum við \texttt{a[i]} á hlaðann og höldum svo áfram.
		\item<6-> Bersýnilega er hlaðinn ávallt raðaður, svo þú færð allar tölur sem eiga að hafa \texttt{a[i]} sem \texttt{NGE}.
		\item<7-> Þegar búið er að fara í gegnum \texttt{a} látum við \texttt{NGE} þeirra staka sem eftir eru í \texttt{h} vera $-1$.
	}
}

\defverbatim{\ngeAA}
{ \begin{verbatim}
 0 1 2 3 4 5 6 7
[2 3 1 5 7 6 4 8]
|

 0 1 2 3 4 5 6 7
[x x x x x x x x]


h: []
\end{verbatim} }

\defverbatim{\ngeAB}
{ \begin{verbatim}
 0 1 2 3 4 5 6 7
[2 3 1 5 7 6 4 8]
 |

 0 1 2 3 4 5 6 7
[x x x x x x x x]


h: []
\end{verbatim} }

\defverbatim{\ngeAC}
{ \begin{verbatim}
 0 1 2 3 4 5 6 7
[2 3 1 5 7 6 4 8]
 |

 0 1 2 3 4 5 6 7
[x x x x x x x x]


h: [0]
\end{verbatim} }

\defverbatim{\ngeAD}
{ \begin{verbatim}
 0 1 2 3 4 5 6 7
[2 3 1 5 7 6 4 8]
   |

 0 1 2 3 4 5 6 7
[x x x x x x x x]


h: [0]
\end{verbatim} }

\defverbatim{\ngeAE}
{ \begin{verbatim}
 0 1 2 3 4 5 6 7
[2 3 1 5 7 6 4 8]
 ^ |

 0 1 2 3 4 5 6 7
[x x x x x x x x]


h: [0]
\end{verbatim} }

\defverbatim{\ngeAF}
{ \begin{verbatim}
 0 1 2 3 4 5 6 7
[2 3 1 5 7 6 4 8]
 ^ |

 0 1 2 3 4 5 6 7
[1 x x x x x x x]


h: [0]
\end{verbatim} }

\defverbatim{\ngeAG}
{ \begin{verbatim}
 0 1 2 3 4 5 6 7
[2 3 1 5 7 6 4 8]
   |

 0 1 2 3 4 5 6 7
[1 x x x x x x x]


h: []
\end{verbatim} }

\defverbatim{\ngeAH}
{ \begin{verbatim}
 0 1 2 3 4 5 6 7
[2 3 1 5 7 6 4 8]
   |

 0 1 2 3 4 5 6 7
[1 x x x x x x x]


h: [1]
\end{verbatim} }

\defverbatim{\ngeAI}
{ \begin{verbatim}
 0 1 2 3 4 5 6 7
[2 3 1 5 7 6 4 8]
     |

 0 1 2 3 4 5 6 7
[1 x x x x x x x]


h: [1]
\end{verbatim} }

\defverbatim{\ngeAJ}
{ \begin{verbatim}
 0 1 2 3 4 5 6 7
[2 3 1 5 7 6 4 8]
   ^ |

 0 1 2 3 4 5 6 7
[1 x x x x x x x]


h: [1]
\end{verbatim} }

\defverbatim{\ngeAK}
{ \begin{verbatim}
 0 1 2 3 4 5 6 7
[2 3 1 5 7 6 4 8]
     |

 0 1 2 3 4 5 6 7
[1 x x x x x x x]


h: [1]
\end{verbatim} }

\defverbatim{\ngeAL}
{ \begin{verbatim}
 0 1 2 3 4 5 6 7
[2 3 1 5 7 6 4 8]
     |

 0 1 2 3 4 5 6 7
[1 x x x x x x x]


h: [1 2]
\end{verbatim} }

\defverbatim{\ngeAM}
{ \begin{verbatim}
 0 1 2 3 4 5 6 7
[2 3 1 5 7 6 4 8]
       |

 0 1 2 3 4 5 6 7
[1 x x x x x x x]


h: [1 2]
\end{verbatim} }

\defverbatim{\ngeAN}
{ \begin{verbatim}
 0 1 2 3 4 5 6 7
[2 3 1 5 7 6 4 8]
     ^ |

 0 1 2 3 4 5 6 7
[1 x x x x x x x]


h: [1 2]
\end{verbatim} }

\defverbatim{\ngeAO}
{ \begin{verbatim}
 0 1 2 3 4 5 6 7
[2 3 1 5 7 6 4 8]
     ^ |

 0 1 2 3 4 5 6 7
[1 x 3 x x x x x]


h: [1 2]
\end{verbatim} }

\defverbatim{\ngeAP}
{ \begin{verbatim}
 0 1 2 3 4 5 6 7
[2 3 1 5 7 6 4 8]
   ^   |

 0 1 2 3 4 5 6 7
[1 x 3 x x x x x]


h: [1]
\end{verbatim} }

\defverbatim{\ngeAQ}
{ \begin{verbatim}
 0 1 2 3 4 5 6 7
[2 3 1 5 7 6 4 8]
   ^   |

 0 1 2 3 4 5 6 7
[1 3 3 x x x x x]


h: [1]
\end{verbatim} }

\defverbatim{\ngeAR}
{ \begin{verbatim}
 0 1 2 3 4 5 6 7
[2 3 1 5 7 6 4 8]
       |

 0 1 2 3 4 5 6 7
[1 3 3 x x x x x]


h: []
\end{verbatim} }

\defverbatim{\ngeAS}
{ \begin{verbatim}
 0 1 2 3 4 5 6 7
[2 3 1 5 7 6 4 8]
       |

 0 1 2 3 4 5 6 7
[1 3 3 x x x x x]


h: [3]
\end{verbatim} }

\defverbatim{\ngeAT}
{ \begin{verbatim}
 0 1 2 3 4 5 6 7
[2 3 1 5 7 6 4 8]
         |

 0 1 2 3 4 5 6 7
[1 3 3 x x x x x]


h: [3]
\end{verbatim} }

\defverbatim{\ngeAU}
{ \begin{verbatim}
 0 1 2 3 4 5 6 7
[2 3 1 5 7 6 4 8]
       ^ |

 0 1 2 3 4 5 6 7
[1 3 3 x x x x x]


h: [3]
\end{verbatim} }

\defverbatim{\ngeAV}
{ \begin{verbatim}
 0 1 2 3 4 5 6 7
[2 3 1 5 7 6 4 8]
       ^ |

 0 1 2 3 4 5 6 7
[1 3 3 4 x x x x]


h: [3]
\end{verbatim} }

\defverbatim{\ngeAW}
{ \begin{verbatim}
 0 1 2 3 4 5 6 7
[2 3 1 5 7 6 4 8]
         |

 0 1 2 3 4 5 6 7
[1 3 3 4 x x x x]


h: []
\end{verbatim} }

\defverbatim{\ngeAX}
{ \begin{verbatim}
 0 1 2 3 4 5 6 7
[2 3 1 5 7 6 4 8]
         |

 0 1 2 3 4 5 6 7
[1 3 3 4 x x x x]


h: [4]
\end{verbatim} }

\defverbatim{\ngeAY}
{ \begin{verbatim}
 0 1 2 3 4 5 6 7
[2 3 1 5 7 6 4 8]
           |

 0 1 2 3 4 5 6 7
[1 3 3 4 x x x x]


h: [4]
\end{verbatim} }

\defverbatim{\ngeAZ}
{ \begin{verbatim}
 0 1 2 3 4 5 6 7
[2 3 1 5 7 6 4 8]
         ^ |

 0 1 2 3 4 5 6 7
[1 3 3 4 x x x x]


h: [4]
\end{verbatim} }

\defverbatim{\ngeBA}
{ \begin{verbatim}
 0 1 2 3 4 5 6 7
[2 3 1 5 7 6 4 8]
           |

 0 1 2 3 4 5 6 7
[1 3 3 4 x x x x]


h: [4]
\end{verbatim} }

\defverbatim{\ngeBB}
{ \begin{verbatim}
 0 1 2 3 4 5 6 7
[2 3 1 5 7 6 4 8]
           |

 0 1 2 3 4 5 6 7
[1 3 3 4 x x x x]


h: [4 5]
\end{verbatim} }

\defverbatim{\ngeBC}
{ \begin{verbatim}
 0 1 2 3 4 5 6 7
[2 3 1 5 7 6 4 8]
             |

 0 1 2 3 4 5 6 7
[1 3 3 4 x x x x]


h: [4 5]
\end{verbatim} }

\defverbatim{\ngeBD}
{ \begin{verbatim}
 0 1 2 3 4 5 6 7
[2 3 1 5 7 6 4 8]
           ^ |

 0 1 2 3 4 5 6 7
[1 3 3 4 x x x x]


h: [4 5]
\end{verbatim} }

\defverbatim{\ngeBE}
{ \begin{verbatim}
 0 1 2 3 4 5 6 7
[2 3 1 5 7 6 4 8]
             |

 0 1 2 3 4 5 6 7
[1 3 3 4 x x x x]


h: [4 5]
\end{verbatim} }

\defverbatim{\ngeBF}
{ \begin{verbatim}
 0 1 2 3 4 5 6 7
[2 3 1 5 7 6 4 8]
             |

 0 1 2 3 4 5 6 7
[1 3 3 4 x x x x]


h: [4 5 6]
\end{verbatim} }

\defverbatim{\ngeBG}
{ \begin{verbatim}
 0 1 2 3 4 5 6 7
[2 3 1 5 7 6 4 8]
               |

 0 1 2 3 4 5 6 7
[1 3 3 4 x x x x]


h: [4 5 6]
\end{verbatim} }

\defverbatim{\ngeBH}
{ \begin{verbatim}
 0 1 2 3 4 5 6 7
[2 3 1 5 7 6 4 8]
             ^ |

 0 1 2 3 4 5 6 7
[1 3 3 4 x x x x]


h: [4 5 6]
\end{verbatim} }

\defverbatim{\ngeBI}
{ \begin{verbatim}
 0 1 2 3 4 5 6 7
[2 3 1 5 7 6 4 8]
             ^ |

 0 1 2 3 4 5 6 7
[1 3 3 4 x x 7 x]


h: [4 5 6]
\end{verbatim} }

\defverbatim{\ngeBJ}
{ \begin{verbatim}
 0 1 2 3 4 5 6 7
[2 3 1 5 7 6 4 8]
           ^   |

 0 1 2 3 4 5 6 7
[1 3 3 4 x x 7 x]


h: [4 5]
\end{verbatim} }

\defverbatim{\ngeBK}
{ \begin{verbatim}
 0 1 2 3 4 5 6 7
[2 3 1 5 7 6 4 8]
           ^   |

 0 1 2 3 4 5 6 7
[1 3 3 4 x 7 7 x]


h: [4 5]
\end{verbatim} }

\defverbatim{\ngeBL}
{ \begin{verbatim}
 0 1 2 3 4 5 6 7
[2 3 1 5 7 6 4 8]
         ^     |

 0 1 2 3 4 5 6 7
[1 3 3 4 x 7 7 x]


h: [4]
\end{verbatim} }

\defverbatim{\ngeBM}
{ \begin{verbatim}
 0 1 2 3 4 5 6 7
[2 3 1 5 7 6 4 8]
         ^     |

 0 1 2 3 4 5 6 7
[1 3 3 4 7 7 7 x]


h: [4]
\end{verbatim} }

\defverbatim{\ngeBN}
{ \begin{verbatim}
 0 1 2 3 4 5 6 7
[2 3 1 5 7 6 4 8]
               |

 0 1 2 3 4 5 6 7
[1 3 3 4 7 7 7 x]


h: []
\end{verbatim} }

\defverbatim{\ngeBO}
{ \begin{verbatim}
 0 1 2 3 4 5 6 7
[2 3 1 5 7 6 4 8]
               |

 0 1 2 3 4 5 6 7
[1 3 3 4 7 7 7 x]


h: [8]
\end{verbatim} }

\defverbatim{\ngeBP}
{ \begin{verbatim}
 0 1 2 3 4 5 6 7
[2 3 1 5 7 6 4 8]


 0 1 2 3 4 5 6 7
[1 3 3 4 7 7 7 x]


h: [8]
\end{verbatim} }

\env{frame}
{
	\only<all:1>{\ngeAA}
	\only<all:2>{\ngeAB}
	\only<all:3>{\ngeAC}
	\only<all:4>{\ngeAD}
	\only<all:5>{\ngeAE}
	\only<all:6>{\ngeAF}
	\only<all:7>{\ngeAG}
	\only<all:8>{\ngeAH}
	\only<all:9>{\ngeAI}
	\only<all:10>{\ngeAJ}
	\only<all:11>{\ngeAK}
	\only<all:12>{\ngeAL}
	\only<all:13>{\ngeAM}
	\only<all:14>{\ngeAN}
	\only<all:15>{\ngeAO}
	\only<all:16>{\ngeAP}
	\only<all:17>{\ngeAQ}
	\only<all:18>{\ngeAR}
	\only<all:19>{\ngeAS}
	\only<all:20>{\ngeAT}
	\only<all:21>{\ngeAU}
	\only<all:22>{\ngeAV}
	\only<all:23>{\ngeAW}
	\only<all:24>{\ngeAX}
	\only<all:25>{\ngeAY}
	\only<all:26>{\ngeAZ}
	\only<all:27>{\ngeBA}
	\only<all:28>{\ngeBB}
	\only<all:29>{\ngeBC}
	\only<all:30>{\ngeBD}
	\only<all:31>{\ngeBE}
	\only<all:32>{\ngeBF}
	\only<all:33>{\ngeBG}
	\only<all:34>{\ngeBH}
	\only<all:35>{\ngeBI}
	\only<all:36>{\ngeBJ}
	\only<all:37>{\ngeBK}
	\only<all:38>{\ngeBL}
	\only<all:39>{\ngeBM}
	\only<all:40>{\ngeBN}
	\only<all:41>{\ngeBO}
	\only<all:42>{\ngeBP}
}

\env{frame}
{
	\selectcode{code/nge.c}{4}{13}
}

\env{frame}
{

	\env{itemize}
	{
		\item<1-> Við setjum hverja tölu í hlaðann að mestu einu sinni og tökum hana svo úr hlaðanum.
		\item<2-> Svo tímaflækjan er $\mathcal{O}($\onslide<3->{$\,n\,$}$)$.
	}
}

\env{frame}
{
}

\end{document}
