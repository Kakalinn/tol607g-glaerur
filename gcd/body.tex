\title{Talnafræði}
\subtitle{Stærsti samdeilir og minnsta samfeldi}
\author{Bergur Snorrason}
\date{\today}

\begin{document}

\frame{\titlepage}

\env{frame}
{
	\env{itemize}
	{
		\item<1-> Látum $a$, $b$, $g$ og $h$ vera jákvæðar heiltölur.
		\item<2-> Við segjum að talan $g$ sé \emph{samdeilir} $a$ og $b$ ef $g$ deilir bæði $a$ og $b$.
		\item<3-> Til dæmis ef $a = 2$ og $b = 4$ þá gæti $g$ verið annað hvort $1$ eða $2$.
		\item<4-> Við segjum að talan $h$ sé \emph{samfeldi} $a$ og $b$ ef bæði $a$ og $b$ deila $h$.
		\item<5-> Til dæmis ef $a = 2$ og $b = 4$ þá gæti $h$ verið $4$, $8$ eða margar aðrar tölur.
		\item<6-> \emph{Stærsta samdeili} $a$ og $b$ táknum við með $\gcd(a, b)$.
		\item<7-> \emph{Minnsta samfeldi} $a$ og $b$ táknum við með $\lcm(a, b)$.
		\item<8-> Við munum einblína á að reikna stærsta samdeili því $\lcm(a, b) \cdot \gcd(a, b) = a \cdot b$.
	}
}

\env{frame}
{
	\env{itemize}
	{
		\item<1-> Hvernig finnum við stærsta samdeili tveggja talna?
		\item<2-> Látum $a$ og $b$ vera jákvæðar heiltölur og $g$ vera stærsta samdeilir þeirra.
		\item<3-> Gerum einnig ráð fyrir að $a < b$ (ef $a = b$ þá er $g = a$).
		\item<4-> Tökum eftir að $g$ deilir líka $b - a$.
		\item<5-> Svo okkur nægir að finna stærsta samdeili $a$ og $b - a$.
		\item<6->[] \selectcode{code/gcd.c}{12}{17}
	}
}

\env{frame}
{
	\env{itemize}
	{
		\item<1-> Tökum eftir að ef $a = 2$ hefur þetta fall tímaflækjuna $\mathcal{O}($\onslide<2->{$\,b\,$}$)$.
		\item<3-> Svo þetta verður almennt aldrei betra en $\mathcal{O}($\onslide<4->{$\max(a, b)$}$)$.
		\item<5-> En við getum bætt þetta.
		\item<6-> Skoðum eitt einfalt dæmi.
	}
}

\defverbatim{\slowgcdA}
{ \begin{verbatim}
                 26 101
              
              
              
              
              
              
              
              
              
              
              
              
              
              
              
              
              
\end{verbatim}}
\defverbatim{\slowgcdB}
{ \begin{verbatim}
                 26 101
              -> 26  75
              
              
              
              
              
              
              
              
              
              
              
              
              
              
              
              
\end{verbatim}}
\defverbatim{\slowgcdC}
{ \begin{verbatim}
                 26 101
              -> 26  75
              -> 26  49
              
              
              
              
              
              
              
              
              
              
              
              
              
              
              
\end{verbatim}}
\defverbatim{\slowgcdD}
{ \begin{verbatim}
                 26 101
              -> 26  75
              -> 26  49
              -> 26  23
              
              
              
              
              
              
              
              
              
              
              
              
              
              
\end{verbatim}}
\defverbatim{\slowgcdE}
{ \begin{verbatim}
                 26 101
              -> 26  75
              -> 26  49
              -> 26  23
              -> 23  26
              
              
              
              
              
              
              
              
              
              
              
              
              
\end{verbatim}}
\defverbatim{\slowgcdF}
{ \begin{verbatim}
                 26 101
              -> 26  75
              -> 26  49
              -> 26  23
              -> 23  26
              -> 23   3
              
              
              
              
              
              
              
              
              
              
              
              
\end{verbatim}}
\defverbatim{\slowgcdG}
{ \begin{verbatim}
                 26 101
              -> 26  75
              -> 26  49
              -> 26  23
              -> 23  26
              -> 23   3
              ->  3  23
              
              
              
              
              
              
              
              
              
              
              
\end{verbatim}}
\defverbatim{\slowgcdH}
{ \begin{verbatim}
                 26 101
              -> 26  75
              -> 26  49
              -> 26  23
              -> 23  26
              -> 23   3
              ->  3  23
              ->  3  20
              
              
              
              
              
              
              
              
              
              
\end{verbatim}}
\defverbatim{\slowgcdI}
{ \begin{verbatim}
                 26 101
              -> 26  75
              -> 26  49
              -> 26  23
              -> 23  26
              -> 23   3
              ->  3  23
              ->  3  20
              ->  3  17
              
              
              
              
              
              
              
              
              
\end{verbatim}}
\defverbatim{\slowgcdJ}
{ \begin{verbatim}
                 26 101
              -> 26  75
              -> 26  49
              -> 26  23
              -> 23  26
              -> 23   3
              ->  3  23
              ->  3  20
              ->  3  17
              ->  3  14
              
              
              
              
              
              
              
              
\end{verbatim}}
\defverbatim{\slowgcdK}
{ \begin{verbatim}
                 26 101
              -> 26  75
              -> 26  49
              -> 26  23
              -> 23  26
              -> 23   3
              ->  3  23
              ->  3  20
              ->  3  17
              ->  3  14
              ->  3  11
              
              
              
              
              
              
              
\end{verbatim}}
\defverbatim{\slowgcdL}
{ \begin{verbatim}
                 26 101
              -> 26  75
              -> 26  49
              -> 26  23
              -> 23  26
              -> 23   3
              ->  3  23
              ->  3  20
              ->  3  17
              ->  3  14
              ->  3  11
              ->  3   8
              
              
              
              
              
              
\end{verbatim}}
\defverbatim{\slowgcdM}
{ \begin{verbatim}
                 26 101
              -> 26  75
              -> 26  49
              -> 26  23
              -> 23  26
              -> 23   3
              ->  3  23
              ->  3  20
              ->  3  17
              ->  3  14
              ->  3  11
              ->  3   8
              ->  3   5
              
              
              
              
              
\end{verbatim}}
\defverbatim{\slowgcdN}
{ \begin{verbatim}
                 26 101
              -> 26  75
              -> 26  49
              -> 26  23
              -> 23  26
              -> 23   3
              ->  3  23
              ->  3  20
              ->  3  17
              ->  3  14
              ->  3  11
              ->  3   8
              ->  3   5
              ->  3   2
              
              
              
              
\end{verbatim}}
\defverbatim{\slowgcdO}
{ \begin{verbatim}
                 26 101
              -> 26  75
              -> 26  49
              -> 26  23
              -> 23  26
              -> 23   3
              ->  3  23
              ->  3  20
              ->  3  17
              ->  3  14
              ->  3  11
              ->  3   8
              ->  3   5
              ->  3   2
              ->  2   3
              
              
              
\end{verbatim}}
\defverbatim{\slowgcdP}
{ \begin{verbatim}
                 26 101
              -> 26  75
              -> 26  49
              -> 26  23
              -> 23  26
              -> 23   3
              ->  3  23
              ->  3  20
              ->  3  17
              ->  3  14
              ->  3  11
              ->  3   8
              ->  3   5
              ->  3   2
              ->  2   3
              ->  2   1
              
              
\end{verbatim}}
\defverbatim{\slowgcdQ}
{ \begin{verbatim}
                 26 101
              -> 26  75
              -> 26  49
              -> 26  23
              -> 23  26
              -> 23   3
              ->  3  23
              ->  3  20
              ->  3  17
              ->  3  14
              ->  3  11
              ->  3   8
              ->  3   5
              ->  3   2
              ->  2   3
              ->  2   1
              ->  1   2
              
\end{verbatim}}
\defverbatim{\slowgcdR}
{ \begin{verbatim}
                 26 101
              -> 26  75
              -> 26  49
              -> 26  23
              -> 23  26
              -> 23   3
              ->  3  23
              ->  3  20
              ->  3  17
              ->  3  14
              ->  3  11
              ->  3   8
              ->  3   5
              ->  3   2
              ->  2   3
              ->  2   1
              ->  1   2
              ->  1   1
\end{verbatim}}

\env{frame}
{
	\only<1>{\slowgcdA}
	\only<2>{\slowgcdB}
	\only<3>{\slowgcdC}
	\only<4>{\slowgcdD}
	\only<5>{\slowgcdE}
	\only<6>{\slowgcdF}
	\only<7>{\slowgcdG}
	\only<8>{\slowgcdH}
	\only<9>{\slowgcdI}
	\only<10>{\slowgcdJ}
	\only<11>{\slowgcdK}
	\only<12>{\slowgcdL}
	\only<13>{\slowgcdM}
	\only<14>{\slowgcdN}
	\only<15>{\slowgcdO}
	\only<16>{\slowgcdP}
	\only<17>{\slowgcdQ}
	\only<18>{\slowgcdR}
}

\env{frame}
{
	\env{itemize}
	{
		\item<1-> Við getum tekið saman þau skref sem eiga sér stað þangað til $a > b$.
		\item<2-> Við finnum $q$ þannig að $b - a \cdot q$ sé jákvætt og minna en $a$.
		\item<4-> En við getum fundið þessa tölu með \onslide<4->{leifareikningi}.
		\item<5-> Við notum því $\gcd(a, b) = \gcd(r, a)$ í staðinn fyrir $\gcd(a, b) = \gcd(a, b - a)$, þar sem $r$ er leif $b$ með tilliti til $a$.
	}
}

\defverbatim{\gcdA}
{ \begin{verbatim}
                 26 101





\end{verbatim}}
\defverbatim{\gcdB}
{ \begin{verbatim}
                 26 101
              -> 23  26




\end{verbatim}}
\defverbatim{\gcdC}
{ \begin{verbatim}
                 26 101
              -> 23  26
              ->  3  23



\end{verbatim}}
\defverbatim{\gcdD}
{ \begin{verbatim}
                 26 101
              -> 23  26
              ->  3  23
              ->  2   3


\end{verbatim}}
\defverbatim{\gcdE}
{ \begin{verbatim}
                 26 101
              -> 23  26
              ->  3  23
              ->  2   3
              ->  1   2

\end{verbatim}}
\defverbatim{\gcdF}
{ \begin{verbatim}
                 26 101
              -> 23  26
              ->  3  23
              ->  2   3
              ->  1   2
              ->  0   1
\end{verbatim}}

\env{frame}
{
	\only<1>{\gcdA}
	\only<2>{\gcdB}
	\only<3>{\gcdC}
	\only<4>{\gcdD}
	\only<5>{\gcdE}
	\only<6>{\gcdF}
}

\env{frame}
{
	\env{itemize}
	{
		\item<1-> Útfærslan einfaldast töluvert með þessari bætingu.
		\item<2->[] \selectcode{code/gcd.c}{7}{10}
		\item<3-> Þessi útfærsla verður $\mathcal{O}($\onslide<4->{$\log \max(a, b)$}$)$.
		\item<5-> Ástæðan fyrir þessari bætingu er að ef \texttt{a} minnkar lítið eftir eitt skref
					þá verður lítill munur á \texttt{a} og \texttt{b}, svo næst minnkar \texttt{a} meira.
		\item<6-> Við kennum þetta reiknirit við Evklíð.
		\item<7-> Þetta ferli er einnig kallað \emph{keðjudeiling}.
	}
}

\env{frame}
{
	\frametitle{Jafna Bézouts}
	\env{itemize}
	{
		\item<1-> Algengt er að nota keðjudeilingu til að leysa jöfnu Bézouts.
		\item<2-> Látum $a$ og $b$ vera jákvæðar heiltölur.
		\item<3-> Þá eru til heiltölur $x$ og $y$ þannig að $a \cdot x + b \cdot y = \gcd(a, b)$.
		\item<4-> Þessi jafna kallast \emph{jafna Bézouts}.
		\item<5-> Við notum svo kallaða \emph{útvíkkaða keðjudeilingu} til að finna tölurnar $x$ og $y$.
		\item<6->[] \selectcode{code/egcd.c}{7}{19}
	}
}

\env{frame}
{
	\env{itemize}
	{
		\item<1-> Algengasta hagnýting jöfn Bézouts er til að finna margföldunarandhverfur.
		\item<2-> Við höfum áður gert það með litlu setningu Fermats.
		\item<3-> Gerum ráð fyrir að $a$ og $m$ séu jákvæðar heiltölur þannig að $\gcd(a, m) = 1$.
		\item<4-> Látum svo heiltölurnar $x$ og $y$ leysa Bézout jöfnuna $a \cdot x + m \cdot y = 1$.
		\item<5-> Þá fæst að $x$ er margföldunarandhverfa $a$ með tilliti til $m$.
		\item<6-> Ef $\gcd(a, m) \neq 1$ þá er margföldunarandhverfa $a$ ekki til.
		\item<7->[] \selectcode{code/mulinv.c}{19}{25}
		\item<8-> Takið eftir að $x$ getur verið neikvæð.
		\item<9-> Til að koma í veg fyrir það má breyta skilagildinu í \texttt{(x\%m + m)\%m}.
	}
}

\env{frame}
{
	\env{itemize}
	{
		\item<1-> Takið eftir að þetta reiknirit virkar stundum þegar $m$ er ekki frumtala en litla setning Fermats virkar bara þegar $m$ er frumtala.
	}
}

\env{frame}
{
}

\end{document}
