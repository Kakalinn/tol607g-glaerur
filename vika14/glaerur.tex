\documentclass{beamer}
\usefonttheme[onlymath]{serif}
\usepackage[T1]{fontenc}
\usepackage[utf8]{inputenc}
\usepackage[english, icelandic]{babel}
\usepackage{amsmath}
\usepackage{amssymb}
\usepackage{amsthm}
\usepackage{gensymb}
\usepackage{parskip}
\usepackage{mathtools}
\usepackage{listings}
\usepackage{xfrac}
\usepackage{graphicx}
\usepackage{xcolor}
\usepackage{tikz}
\usepackage{xifthen}
\usepackage{tkz-euclide}
\usetkzobj{all}
\usetikzlibrary{calc}
\usepackage{multicol}

\DeclareMathOperator{\lcm}{lcm}
\DeclareMathOperator{\diam}{diam}
\DeclareMathOperator{\dist}{dist}
\DeclareMathOperator{\ord}{ord}
\DeclareMathOperator{\Aut}{Aut}
\DeclareMathOperator{\Inn}{Inn}
\DeclareMathOperator{\Ker}{Ker}
\DeclareMathOperator{\trace}{trace}
\DeclareMathOperator{\fix}{fix}
\DeclareMathOperator{\Log}{Log}
\renewcommand\O{\mathcal{O}}
\newcommand\floor[1]{\left\lfloor#1\right\rfloor}
\newcommand\ceil[1]{\left\lceil#1\right\rceil}
\newcommand\abs[1]{\left|#1\right|}
\newcommand\p[1]{\left(#1\right)}
\newcommand\sqp[1]{\left[#1\right]}
\newcommand\cp[1]{\left\{#1\right\}}
\newcommand\norm[1]{\left\lVert#1\right\rVert}
\renewcommand\qedsymbol{$\blacksquare$}
\renewcommand\Im{\operatorname{Im}}
\renewcommand\Re{\operatorname{Re}}
\usepackage{color}

\newcommand\env[2]
{
	\begin{#1}
	#2
	\end{#1}
}
\newcommand\varenv[3]
{
	\begin[#2]{#1}
	#3
	\end{#1}
}

\newcommand\code[1]{\tiny\lstinputlisting[language=C]{#1}}

\definecolor{mygray}{rgb}{0.4,0.4,0.4}
\definecolor{mygreen}{rgb}{0, 0, 1}
\definecolor{myorange}{rgb}{1.0,0.4,0}

\lstset{
	commentstyle=\color{mygray},
	numbersep=5pt,
	numberstyle=\tiny\color{mygray},
	keywordstyle=\color{mygreen},
	showspaces=false,
	showstringspaces=false,
	stringstyle=\color{myorange},
	tabsize=4
}
\lstset{literate=
{æ}{{\ae}}1
{í}{{\'{i}}}1
{ó}{{\'{o}}}1
{á}{{\'{a}}}1
{é}{{\'{e}}}1
{ú}{{\'{u}}}1
{ý}{{\'{y}}}1
{ð}{{\dh}}1
{þ}{{\th}}1
{ö}{{\"o}}1
{Á}{{\'{A}}}1
{Í}{{\'{I}}}1
{Ó}{{\'{O}}}1
{Ú}{{\'{U}}}1
{Æ}{{\AE}}1
{Ö}{{\"O}}1
{Ø}{{\O}}1
{Þ}{{\TH}}1
}

\usetheme{Madrid}

\title{Lokakeppni TÖL607G 2019}
\author{Atli Fannar Franklín \& Bergur Snorrason}
\date{\today}

\begin{document}

\frame{\titlepage}

\env{frame}
{
	\frametitle{Dómnefnd}
	\small
	\env{itemize}
	{
		\item<1-> Atli Fannar Franklín.
		\item<2-> Bergur Snorrason.
	}
}

\env{frame}
{
	\frametitle{Planetaris}
	\small
	\env{block}
	{
		{Dæmi}
		Þér er sagt til hvaða svæða mótspilarinn þinn í tölvuleik mun senda heri sína og hversu stór herinn þinn er.
		Finndu hversu mörg svæði þú getur mest unnið.
	}
}

\env{frame}
{
	\frametitle{Planetaris}
	\small
	\env{block}
	{
		{Lausn}
		Til að vinna sem flest svæði viltu forgangsraða þau svæði sem mótspilarinn sýnir minni áhuga.
		Þið raðir því svæðunum eftir því hversu stóra heri mótspilarinn þinn ætlar að senda þangað (í vaxandi röð).
		Þið gangið svo á röðina á meðan herinn ykkar er nógu stór til að vinna svæði.
	}
	\pause
	\env{block}
	{
		{Gryfja}
		Það er ekki nóg að jafna mótspilarann þinn, þú þarft að sendi einum fleiri.
	}
	\pause
	\env{block}
	{
		{Tölfræði}
		Fyrsta lausn: Alexander Guðmundsson ($00:21$).\\
		Fjöldi lausna (tilraunir) : 11 (44).
	}
}

\env{frame}
{
	\frametitle{Association of Myths}
	\small
	\env{block}
	{
		{Dæmi}
		Þið eruð beðin um að reikna romsu.
	}
}

\env{frame}
{
	\frametitle{Association of Myths}
	\small
	\env{block}
	{
		{Lausn}
		Nálægt enda romsunnar á að deilda $k$-ta stigs margliðu $k + 1$ sinni, en það er núll.
		Svo það þarf bara að reikna og prenta
		\[
			\frac{l^2}{\pi e} + \frac{1}{l + 1},
		\]
		þ.e.a.s. eftir innlestur nægir\\
		\texttt{printf("\%.2f\textbackslash n", l*l/(M\_PI*M\_E) + (1.0)/(l + 1.0));}.
	}
	\pause
	\env{block}
	{
		{Gryfja}
		Svarið er ekki alltaf $9.59$.\\
		Það þurfa að vera {\bf nákvæmlega} tveir aukastafir.
	}
	\pause
	\env{block}
	{
		{Tölfræði}
		Fyrsta lausn: Helgi Sigtryggsson ($00:42$)
		Fjöldi lausna (tilraunir) : 7 (17).
	}
}

\env{frame}
{
	\frametitle{Tildes}
	\small
	\env{block}
	{
		{Dæmi}
		Þið eruð að fylgjast með veislu. Í veislunni er fólk að tala saman. Samnræðurnar eru þannig að þegar fólk
		byrjar að tala saman þá hættir það því ekki. Því myndast sundirlægir samræðuhópar og þið eigið að halda
		utan um hversu margir eru í hverju hópi.
	}
}

\env{frame}
{
	\frametitle{Tildes}
	\small
	\env{block}
	{
		{Lausn}
		Hóparnir mynda sundurlæg mengi sem þið þurfið að geta sameinað. 
		\pause
		Við getum notað sammengjaleit (e. union-find) til að gera þetta nógu hratt.
		Við þurfum síðan einnig að geyma hversu stórt hvert sammengi er. Það má gera
		með union-by-rank.
	}
	\pause
	\env{block}
	{
		{Tölfræði}
		Fyrsta lausn: Eyleifur Bjarkason ($00:04$).\\
		Fjöldi lausna (tilraunir) : 9 (34).
	}
}

\env{frame}
{
	\frametitle{Geezer Scripts}
	\small
	\env{block}
	{
		{Dæmi}
		Þið eigið að finna leiðina í gegnum hellakerfið sem leifir spilaranum að taka sem
		minnstan skaða, eða segja að ekki sé til leið sem hleypir spilaranum í gegn á lífi.
	}
}

\env{frame}
{
	\frametitle{Geezer Scripts}
	\small
	\env{block}
	{
		{Lausn}
		Maður lítur á hellakerfið sem stefnt vigtað net þar sem vigtir leggjanna er skaðinn
		sem spilari verður fyrir ef hann ferðast eftir þeim legg. Þetta er þá bara spurning 
		um að keyra reiknirit Dijkstra til að finna ódýrustu leiðina á leiðarenda.
	}
	\pause
	\env{block}
	{
		{Gryfja}
		Ekki er hægt að herma bardagana til að komast að því hvað spilarinn missir mikið líf
		í gefnum bardaga. Ef spilarinn hefur skaðagildi $A$, andstæðingurinn skaðagildi $a$ og
		$h$ heilsustig þá mun spilarinn missa $a \floor{\frac{h - 1}{A}}$ heilsustig í þeim bardaga.
	}
	\pause
	\env{block}
	{
		{Tölfræði}
		Fyrsta lausn: -\\
		Fjöldi lausna (tilraunir) : 0 (22).
	}
}

\env{frame}
{
	\frametitle{Strikercount}
	\small
	\env{block}
	{
		{Dæmi}
		Þið eruð að spila fyrstu persónu skotleik og viljið vita hversu marga óvini þú getur hitt
		í einu skoti. Óvinir eru gefnir sem hringir í plani. Þú ert í núllpunkti plansins.
	}
}

\env{frame}
{
	\frametitle{Strikercount}
	\small
	\env{block}
	{
		{Lausn}
		Við getum í raun skoðað hringina í planinu sem bil á einingahringskífunni. Við erum þá að reyna að finna þann
		punkt á einingaskífunni sem er innihaldinn í öllum þessu bilum. Fyrir slíkan punkt gildir augljóslega að honum sé
		hægt að hliðra þangað til hann er endapunktur bils svo okkur nægir að að skoða endapunktana. Nánar tiltekið
		getum við notað hlaupabil (e. sliding window) til að finna þann punkt. Fyrir nánari umfjöllun getið þið
		skoðað glærurnar úr viku $12$, nánar tiltekið sýnidæmið um stærð sammengis bila. Þetta dæmi er eins að miklu leiti
		nema hvað að þið haldið utan um hversu stórt hlaupabilið verður.
	}
	\pause
	\env{block}
	{
		{Gryfja}
		Ólíkt dæminu í viku $12$ erum við núna á hring, svo við þurfum að labba einu sinni í gegnum bilin til að sjá
		hversu mörg bilin eru í upphafi.
	}
	\pause
	\env{block}
	{
		{Tölfræði}
		Fyrsta lausn: -\\
		Fjöldi lausna (tilraunir) : 0 (1).
	}
}

\env{frame}
{
	\frametitle{Digbuild}
	\small
	\env{block}
	{
		{Dæmi}
		Við viljum ákvarða hvað má velja hlutmengi í $3 \times n$ reitum þ.a. við veljum enga tvo aðlæga reiti
		(aðlægir reitir eru bara þeir sem deila hlið) og skila svarinu modulo $10^9 + 7$.
	}
}

\env{frame}
{
	\frametitle{Digbuild}
	\small
	\env{block}
	{
		{Lausn}
		Við búum til rakningavensl úr þessu. Við táknum með $a_{n, b_1b_2b_3}$ fjölda leiða til þess að gera
		þetta á $3 \times n$ reitum þar sem síðasti dálkurinn er $b_1b_2b_3$ þar sem $b_i$ er $0$ eða $1$ og
		táknar þá hvort við tökum reit $i$ með eða ekki. Við sjáum að $a_{n,111}, a_{n,110}$ og $a_{n,011}$
		eru alltaf núll því við megum ekki velja tvo hlið við hlið svo við hunsum þau gildi. Fyrir hin valin
		á $b_i$-in verður þá $a_{1,b_1b_2b_3} = 1$. Nú setjum við upp rakningavensl fyrir þessar stærðir og
		fáum eftirfarandi jöfnur
		
		\begin{align*}
		a_{n,000} &= a_{n-1,000} + a_{n-1,001} + a_{n-1,010} + a_{n-1,100} + a_{n-1,101} \\
		a_{n,001} &= a_{n-1,000} + a_{n-1,010} + a_{n-1,100} \\
		a_{n,010} &= a_{n-1,000} + a_{n-1,001} + a_{n-1,100} + a_{n-1,101} \\
		a_{n,100} &= a_{n-1,000} + a_{n-1,001} + a_{n-1,010} \\
		a_{n,101} &= a_{n-1,000} + a_{n-1,010} \\
		\end{align*}
	}
}

\env{frame}
{
	\frametitle{Digbuild}
	\small
	\env{block}
	{
		{Lausn}
		Nú má gera ýmislegt til að leysa þetta. Meðal annars má leysa upp úr rakningavenslahneppinu á 
		síðustu glæru til að fá línunlegt rakningavensl af fimmta stigi og reikna upp úr því með
		fylkjaveldishafningu. Það sem má líka gera er að skilgreina vigurinn $v = (1, 1, 1, 1, 1)^T$
		og breytingarfylkið
		
		\begin{scriptsize}
		\[m = \p{\begin{matrix}1&1&1&1&1\\1&0&1&1&0\\1&1&0&1&1\\1&1&1&0&0\\1&0&1&0&0\end{matrix}}\]
		\end{scriptsize}
		
		Þá þar sem $v = (a_{1,000},\dots,a_{1,101})$ fæst útfrá jöfnuhneppinu á síðustu glæru að
		
		\[m^{n-1} v = (a_{n,000},a_{n,001},a_{n,010},a_{n,100},a_{n,101})^T\]
		
		Og þá má reikna svarið sem summuna af þessum fimm stökum eða sem bara fremsta stakið, það
		þarf þá að hliðra $n$ til um einn eftir tilvikum.
	}
}

\env{frame}
{
	\frametitle{Digbuild}
	\small
	\env{block}
	{
		{Lausn}
		Þar sem reikna má $n$-ta veldi fylkis í $\mathcal{O}(\log(n))$ margföldunum og fylkjamargföldun
		tekur teningstíma í stærð þess fæst að þetta keyri í $\mathcal{O}(k^3\log(n))$ þar sem $k$ er
		fast sem 5 og $n \leq 10^{18}$ svo $\log(n) < 60$. Ef reiknað er upp úr jöfnuhneppinu í staðinn
		verður $k = 4$ sem gefur ómarktækan mun í keyrsluhraða.
	}
	\pause
	\env{block}
	{
		{Gryfja}
		Ekki er nóg að reikna út jöfnuhneppið að ofan og leysa þetta með kvikri bestun í línulegum tíma
		því $n$ er of stórt til þess. Því þarf að leysa þetta með fylkjaveldahafningu.
	}
	\pause
	\env{block}
	{
		{Tölfræði}
		Fyrsta lausn: -\\
		Fjöldi lausna (tilraunir) : 0 (0).
	}
}

\end{document}
