\title{Reiknirit Bellmans og Fords ($1958$ og $1956$)}
\author{Bergur Snorrason}
\date{\today}

\begin{document}

\frame{\titlepage}

\env{frame}
{
	\env{itemize}
	{
		\item<1-> Hvað gerum við ef við viljum nota reinkirit Dijkstras en það mega vera neikvæðar vigtir á leggjunum.
		\item<2-> Við getum þá notað reiknirit sem er kennt við Bellman og Ford.
		\item<3-> Við þurfum þó að fórna keyrslutíma.
	}
}

\env{frame}
{
	\env{itemize}
	{
		\item<1-> Þetta reiknirit er að vissu leiti einfaldara en reiknirit Dijkstras.
		\item<2-> Við notum kvika bestun og svörum spurningunni ,,Hver er stysta leiðin frá $u$ til $v$ sem fer að mestu í $k$ hnúta?''.
		\item<3-> Hér táknar $u$ upphafshnútinn á meðan $v$ og $k$ eru frjálsar breytur.
		\item<5-> Látum þá $f(v, k)$ tákna systa veg frá hnútnum $u$ til hnútsins $v$ sem fer ekki í fleiri en $k$ hnúta.
		\item<6-> Til að einfalda skriftir þá skilgreinum við
					\[
						E_u = \{v \in V:\, (u, v) \in E\}
					\]
					og
					\[
						E^v = \{u \in V:\, (u, v) \in E\}.
					\]
		\item<7-> Við fáum að
	}
	\onslide<7->
	{
					\[
						f(v, k) = \left \{
						\env{array}
						{ {l l}
							0, & \text{ef $u = v$ og $k = 0$}\\
							\infty, & \text{ef $u \neq v$ og $k = 0$}\\
							\min(f(v, k - 1), & \\
							\ \min_{u \in E^v} w((u, v)) + f(u, k - 1)), & \text{ef $u \neq v$ og $k = 0$}\\
						}
						\right .
					\]
	}
}

\env{frame}
{
	\env{itemize}
	{
		\item<1-> Við munum leysa þetta með neðansækinni kvikri bestun.
		\item<2-> Gerum ráð fyrir að taflan sem við notum fyrir minnun hafi dálk sem svari til $k$ breytunnar.
		\item<3-> Þá er hver staða aðeins háð stöðum í röðinni fyrir ofan sig.
		\item<4-> Við notum því aðeins síðustu línu fylkisins þegar við fyllum inn í töfluna.
		\item<5-> Því má geyma tvívíða fylkið sem einvítt fylki.
	}
}

\env{frame}
{
	\env{itemize}
	{
		\item<1-> Við erum ekki búin þegar við höfum reiknað öll gildin á $f(v, k)$.
		\item<2-> Hvað með neikvæðar rásir?
		\item<3-> Takið fyrst eftir að ef það er ekki neikvæð rás í netinu þá heimsækir systi vegur milli hnúta engan hnúta tvisvar.
		\item<4-> Einnig er ekki nóg að það sé neikvæð rás í netinu heldur þarf að vera hægt að komast í hana frá upphafshnútnum
					og svo má vera að það sé ekki hægt að komast frá rásinni í alla aðra hnúta.
		\item<5-> Við getum einfaldlega prófað að lengja vegina um $|V| - 1$ hnúta í viðbót.
		\item<6-> Ef vegalengdin styttist einhverntíman þá er betra að heimsækja einhvern hnút oftar en einu sinni,
					sem þýðir að það sé neikvæð rás á leiðinni.
	}
}

\env{frame}
{
	\env{center}
	{
		\env{tikzpicture}
		{ [scale = 0.75]
			\only<all:1->{\node[draw, circle] (1) at (2,0) {\tiny $1$};}
			\only<all:1->{\node[draw, circle] (2) at (2,2) {\tiny $2$};}
			\only<all:1->{\node[draw, circle] (3) at (2,-2) {\tiny $3$};}
			\only<all:1->{\node[draw, circle] (4) at (4,1) {\tiny $4$};}
			\only<all:1->{\node[draw, circle] (5) at (4,-1) {\tiny $5$};}
			\only<all:1->{\node[draw, circle] (6) at (6,0) {\tiny $6$};}
			\only<all:1->{\node[draw, circle] (7) at (6,2) {\tiny $7$};}
			\only<all:1->{\node[draw, circle] (8) at (6,-2) {\tiny $8$};}
			\only<all:1->{\node[draw, circle] (9) at (8,0) {\tiny $9$};}

			\path[draw, thick, <-] (1) -- (2); \node[fill = white] at (2,1) {\tiny $1$};
			\path[draw, thick, <-] (2) -- (4); \node[fill = white] at (3,1.5) {\tiny $9$};
			\path[draw, thick, <->] (4) -- (5); \node[fill = white] at (4,0) {\tiny $1$};
			\path[draw, thick, ->] (3) -- (5); \node[fill = white] at (3,-1.5) {\tiny $2$};
			\path[draw, thick, ->] (1) -- (3); \node[fill = white] at (2,-1) {\tiny $-4$};
			\path[draw, thick, ->] (4) -- (6); \node[fill = white] at (5,0.5) {\tiny $3$};
			\path[draw, thick, ->] (6) -- (8); \node[fill = white] at (6,-1) {\tiny $5$};
			\path[draw, thick, ->] (8) -- (9); \node[fill = white] at (7,-1) {\tiny $1$};
			\path[draw, thick, <-] (7) -- (9); \node[fill = white] at (7,1) {\tiny $1$};
			\path[draw, thick, <-] (6) -- (9); \node[fill = white] at (7,0) {\tiny $-7$};
			\path[draw, thick, <->] (1) -- (5); \node[fill = white] at (3,-0.5) {\tiny $3$};
			\path[draw, thick, ->] (4) -- (7); \node[fill = white] at (5,1.5) {\tiny $8$};
		}
% view this array in fullscreen.
		\[
		\env{array}
		{ {r | r r r r r r r r r}
			               k & 1 &      2 &      3 &      4 &      5 &      6 &      7 &             8 &             9\\
			\hline
			\onslide<all:1->  {0 & \color{blue} 0 &          \infty &          \infty &          \infty &          \infty &         \infty &          \infty &          \infty &          \infty}\\
			\onslide<all:2->  {1 &              0 &          \infty & \color{blue} -4 &          \infty & \color{blue}  3 &         \infty &          \infty &          \infty &          \infty}\\
			\onslide<all:3->  {2 &              0 &          \infty &              -4 & \color{blue}  4 & \color{blue} -2 &         \infty &          \infty &          \infty &          \infty}\\
			\onslide<all:4->  {3 &              0 & \color{blue} 13 &              -4 & \color{blue} -1 &              -2 & \color{blue} 7 & \color{blue} 12 &          \infty &          \infty}\\
			\onslide<all:5->  {4 &              0 & \color{blue}  8 &              -4 &              -1 &              -2 & \color{blue} 2 & \color{blue}  7 & \color{blue} 12 &          \infty}\\
			\onslide<all:6->  {5 &              0 &               8 &              -4 &              -1 &              -2 &              2 &               7 & \color{blue}  7 & \color{blue} 13}\\
			\onslide<all:7->  {6 &              0 &               8 &              -4 &              -1 &              -2 &              2 &               7 &               7 & \color{blue}  8}\\
			\onslide<all:8->  {7 &              0 &               8 &              -4 &              -1 &              -2 & \color{blue} 1 &               7 &               7 &               8}\\
			\onslide<all:9->  {8 &              0 &               8 &              -4 &              -1 &              -2 &              1 &               7 &  \color{blue} 6 &               8}\\
			\onslide<all:10-> {9 &              0 &               8 &              -4 &              -1 &              -2 &              1 &               7 &               6 &   \color{red} 7}\\
		}
		\]
	}
}

\env{frame}
{
	\selectcode{code/bellman-ford.cpp}{10}{23}
}

\env{frame}
{
	\env{itemize}
	{
		\item<1-> Sjáum að í fyrri hluta reikniritsins ýtrum við í gegnum alla leggi og allar nóður $(|V| - 1)$-sinnum.
		\item<2-> Tímaflækjan á þeim hluta er því $\mathcal{O}($\onslide<3->{$E \cdot V$}$)$.
		\item<4-> Seinni hlutinn er svo að ítra yfir nákvæmlega það sama, svo tímaflækja þar er eins.
		\item<5-> Því fæst að reikniritið er í heildina $\mathcal{O}($\onslide<6->{$E \cdot V$}$)$.
		\item<7-> Þetta er töluvert verra en reiknirit Dijkstras (svipað og að fara úr $\mathcal{O}(n \cdot \log n)$ í $\mathcal{O}(n^2)$).
	}
}

\env{frame}
{
}

\end{document}
