\documentclass{beamer}
\usefonttheme[onlymath]{serif}
\usepackage[T1]{fontenc}
\usepackage[utf8]{inputenc}
\usepackage[english, icelandic]{babel}
\usepackage{amsmath}
\usepackage{amssymb}
\usepackage{amsthm}
\usepackage{gensymb}
\usepackage{parskip}
\usepackage{mathtools}
\usepackage{listings}
\usepackage{xfrac}
\usepackage{graphicx}
\usepackage{xcolor}
\usepackage{tikz}
\usepackage{xifthen}
\usepackage{tkz-euclide}
\usetkzobj{all}
\usetikzlibrary{calc}
\usepackage{multicol}

\DeclareMathOperator{\lcm}{lcm}
\DeclareMathOperator{\diam}{diam}
\DeclareMathOperator{\dist}{dist}
\DeclareMathOperator{\ord}{ord}
\DeclareMathOperator{\Aut}{Aut}
\DeclareMathOperator{\Inn}{Inn}
\DeclareMathOperator{\Ker}{Ker}
\DeclareMathOperator{\trace}{trace}
\DeclareMathOperator{\fix}{fix}
\DeclareMathOperator{\Log}{Log}
\renewcommand\O{\mathcal{O}}
\newcommand\floor[1]{\left\lfloor#1\right\rfloor}
\newcommand\ceil[1]{\left\lceil#1\right\rceil}
\newcommand\abs[1]{\left|#1\right|}
\newcommand\p[1]{\left(#1\right)}
\newcommand\sqp[1]{\left[#1\right]}
\newcommand\cp[1]{\left\{#1\right\}}
\newcommand\norm[1]{\left\lVert#1\right\rVert}
\renewcommand\qedsymbol{$\blacksquare$}
\renewcommand\Im{\operatorname{Im}}
\renewcommand\Re{\operatorname{Re}}
\usepackage{color}

\newcommand\env[2]
{
	\begin{#1}
	#2
	\end{#1}
}
\newcommand\varenv[3]
{
	\begin[#2]{#1}
	#3
	\end{#1}
}

\newcommand\code[1]{\tiny\lstinputlisting[language=C]{#1}}

\definecolor{mygray}{rgb}{0.4,0.4,0.4}
\definecolor{mygreen}{rgb}{0, 0, 1}
\definecolor{myorange}{rgb}{1.0,0.4,0}

\lstset{
	commentstyle=\color{mygray},
	numbersep=5pt,
	numberstyle=\tiny\color{mygray},
	keywordstyle=\color{mygreen},
	showspaces=false,
	showstringspaces=false,
	stringstyle=\color{myorange},
	tabsize=4
}
\lstset{literate=
{æ}{{\ae}}1
{í}{{\'{i}}}1
{ó}{{\'{o}}}1
{á}{{\'{a}}}1
{é}{{\'{e}}}1
{ú}{{\'{u}}}1
{ý}{{\'{y}}}1
{ð}{{\dh}}1
{þ}{{\th}}1
{ö}{{\"o}}1
{Á}{{\'{A}}}1
{Í}{{\'{I}}}1
{Ó}{{\'{O}}}1
{Ú}{{\'{U}}}1
{Æ}{{\AE}}1
{Ö}{{\"O}}1
{Ø}{{\O}}1
{Þ}{{\TH}}1
}

\usetheme{Madrid}

\title{Rúmfræði}
\author{Bergur Snorrason}
\date{\today}

\AtBeginSection[] {
	\begin{frame}
		\frametitle{Efnisyfirlit}
		\tableofcontents[currentsection]
	\end{frame}
}

\begin{document}

\frame{\titlepage}

\section[Inngangur]{Inngangur}

\env{frame}
{
	\frametitle{Google Code Jam}
	\env{itemize}
	{
		\item<1-> Aðfaranótt 6. apríl hefst fyrsta umferð í Google Code Jam.
		\item<2-> Hún stendur yfir í 24 tíma.
		\item<3-> Ég mæli með.
	}
}

\env{frame}
{
	\frametitle{Lokakeppnin}
	\env{itemize}
	{
		\item<1-> Eftir viku verður lokakeppnin okkar.
		\item<2-> Dæmin verða sex talsnins:
		\env{itemize}
		{
			\item<3-> \emph{Assosiation of Myths}
			\item<4-> \emph{Digbuild}
			\item<5-> \emph{Geezer Scripts}
			\item<6-> \emph{Planetaris}
			\item<7-> \emph{Strikercount}
			\item<8-> \emph{Tildes}
		}
		\item<9-> Keppnin verður með svipuðu sniði og síðast.
		\item<10-> Til að fá skil þarf að leysa eitt dæmi.
	}
}

\section[Rúmfræði]{Rúmfræði}

\env{frame}
{
	\frametitle{Hornaföll}
	\env{columns}
	{
		\env{column}
		{
			{0.7\textwidth}
			\env{itemize}
			{
				\item<1-> Þessi glæra ætti að vera upprifjun fyrir flest ykkar.
				\item<2-> Þríhyrningur er sagður rétthyrndur ef eitt horna hans er $90^{\circ}$.
				\item<3-> Fyrir rétthyrnda þríhyrninga gildir:
				\env{itemize}
				{
				\item<4-> $\dfrac{A}{L} = \cos\theta$.
				\item<5-> $\dfrac{M}{L} = \sin\theta$.
				\item<6-> $\dfrac{M}{A} = \dfrac{\sin\theta}{\cos\theta} = \tan\theta$.
				}
				\item<7-> Einnig gildir regla Pýthagorasar, $L^2 = A^2 + M^2$.
			}
		}
		\env{column}
		{
			{0.3\textwidth}
			\scalebox{0.75}
			{
				\env{tikzpicture}
				{
					\tkzInit[xmin=0,xmax=4,ymin=0,ymax=3]
					\tkzDefPoint(0,0){A}
					\tkzDefPoint(4,0){B}
					\tkzDefPoint(0,3){C}
					\tkzMarkRightAngle[size = 0.4](B,A,C)
					\tkzMarkAngle[label=$\theta$, size = 0.8](C,B,A)
					\tkzDrawSegments(A,B B,C C,A)

					\tkzLabelSegment[below=1pt](A,B){$A$}  
					\tkzLabelSegment[above=2pt](B,C){$L$}  
					\tkzLabelSegment[left=1pt](A,C){$M$}  
				}
			}
		}
	}
}

\env{frame}
{
	\frametitle{Hornaföll}
	\env{columns}
	{
		\env{column}
		{
			{0.7\textwidth}
			\env{itemize}
			{
				\item<1-> Almennar gildir um þríhyrninga:
				\env{itemize}
				{
				\item<2-> $\dfrac{\sin a}{A} = \dfrac{\sin b}{B} = \dfrac{\sin c}{C}$ (sínus reglan).
				\item<3-> $A^2 = B^2 + C^2 - 2BC\cos a$ (kósínus reglan)
				}
				\item<4-> {\bf Æfing:} Sannið reglu Pýthagorasar með kósínus reglunni.
			}
		}
		\env{column}
		{
			{0.3\textwidth}
			\scalebox{0.8}
			{
				\env{tikzpicture}
				{
					\tkzInit[xmin=0,xmax=4,ymin=0,ymax=3]
					\tkzDefPoint(0,0){A}
					\tkzDefPoint(4,0){B}
					\tkzDefPoint(2,3){C}
					\tkzMarkAngle[label=$b$, size = 0.8](B,A,C)
					\tkzMarkAngle[label=$c$, size = 0.8](C,B,A)
					\tkzMarkAngle[label=$a$, size = 0.8](A,C,B)
					\tkzDrawSegments(A,B B,C C,A)

					\tkzLabelSegment[below=1pt](A,B){$A$}  
					\tkzLabelSegment[above=1pt](B,C){$B$}  
					\tkzLabelSegment[left=1pt](A,C){$C$}  
				}
			}
		}
	}
}

\env{frame}
{
	\frametitle{Sýnidæmi: NN and the Optical Illusion - Codeforces}
	Þér er gefið heiltölu $n$ og rauntölu $r$. Þú teiknar hring á blað með geilsa $r$.
	Þú vilt teikna $n$ jafn stóra hringi í kringum hringinn þinn þannig að þeir skeri
	hringinn þinn og aðlæga hringi í nákvæmlega einum punkti.
	Hver þarf geilsi ytri hringjanna að vera.\\
	https://codeforces.com/problemset/problem/1100/C
}

\env{frame}
{
	\frametitle{Sýnidæmi: NN and the Optical Illusion - Codeforces}
	Ef $n = 6$ fæst eftirfarandi mynd, þar sem $R$ er svarið.
	\center
	\env{tikzpicture}
	{
		\tkzInit[xmin=0,xmax=4,ymin=0,ymax=3]
		\tkzDefPoint(0,0){O}
		\tkzDefPoint(1,0){A}
		\tkzDefPoint(-1,0){B}
		\tkzDefPoint(0.5, 0.8660254038){C}
		\tkzDefPoint(0.5, -0.8660254038){D}
		\tkzDefPoint(-0.5, -0.8660254038){E}
		\tkzDefPoint(-0.5, 0.8660254038){F}

		\tkzDefPoint(2,0){AA}
		\tkzDefPoint(-2,0){BB}
		\tkzDefPoint(1,1.732050808){CC}
		\tkzDefPoint(1,-1.732050808){DD}
		\tkzDefPoint(-1,-1.732050808){EE}
		\tkzDefPoint(-1,1.732050808){FF}

		\tkzDrawSegments(O,A EE,E)
		\tkzDrawCircle(O,A)
		\tkzDrawCircle[dashed](AA,A)
		\tkzDrawCircle[dashed](BB,B)
		\tkzDrawCircle[dashed](CC,C)
		\tkzDrawCircle[dashed](DD,D)
		\tkzDrawCircle[dashed](EE,E)
		\tkzDrawCircle[dashed](FF,F)

		\tkzLabelSegment[above=1pt](O,A){$r$}  
		\tkzLabelSegment[left=1pt](E,EE){$R$}
	}
}

\env{frame}
{
	\frametitle{Sýnidæmi: NN and the Optical Illusion - Codeforces}
	Sjáum að fjarlægðin frá miðjum myndarinnar að miðju ytri hringjanna er
	$r + R$. Við fáum því eftirfarandi jafnarma þríhyrning.
	\center
	\env{tikzpicture}
	{
		\tkzInit[xmin=0,xmax=4,ymin=0,ymax=3]
		\tkzDefPoint(0,0){O}
		\tkzDefPoint(1,0){A}
		\tkzDefPoint(-1,0){B}
		\tkzDefPoint(0.5, 0.8660254038){C}
		\tkzDefPoint(0.5, -0.8660254038){D}
		\tkzDefPoint(-0.5, -0.8660254038){E}
		\tkzDefPoint(-0.5, 0.8660254038){F}

		\tkzDefPoint(2,0){AA}
		\tkzDefPoint(-2,0){BB}
		\tkzDefPoint(1,1.732050808){CC}
		\tkzDefPoint(1,-1.732050808){DD}
		\tkzDefPoint(-1,-1.732050808){EE}
		\tkzDefPoint(-1,1.732050808){FF}
		\tkzLabelAngle[pos = 0.6](DD,O,AA){$\theta$}
		\tkzMarkAngle[size = 0.3, pos = 0.1](DD,O,AA)

		\tkzDrawSegments(O,AA O,DD AA,DD)
		\tkzDrawCircle(O,A)
		\tkzDrawCircle[dashed](AA,A)
		\tkzDrawCircle[dashed](BB,B)
		\tkzDrawCircle[dashed](CC,C)
		\tkzDrawCircle[dashed](DD,D)
		\tkzDrawCircle[dashed](EE,E)
		\tkzDrawCircle[dashed](FF,F)

		\tkzLabelSegment[above=1pt](O,AA){$r + R$}  
		\tkzLabelSegment[left=1pt](O,DD){$r + R$}
		\tkzLabelSegment[right=1pt](AA,DD){$2R$}
	}
}

\env{frame}
{
	\frametitle{Sýnidæmi: NN and the Optical Illusion - Codeforces}
	\env{itemize}
	{
	\item<1-> Hornið $\theta$ af síðustu glæru er það eina (ásamt $R$, að sjálfsögðu)
		háð $n$ á myndinn, svo almennt þurfum við að finna $R$ út frá eftirfarandi mynd.\\
	\item<2-> Nú er 
		$\theta = \dfrac{360^{\circ}}{n}$
		og
		$\omega = \dfrac{180^{\circ} - \theta}{2}$.
	\item<3-> Sínus reglan gefur okkur svo að
		\[
			\frac{2R}{\sin \theta} = \frac{r + R}{\sin \omega}
			\Rightarrow
			2R\sin\omega - R\sin\theta = r\sin\theta
			\Rightarrow
			R = \frac{r\sin\theta}{2\sin\omega - \sin\theta}
		\]
	}
	\center
	\env{tikzpicture}
	{
		\tkzInit[xmin=0,xmax=4,ymin=0,ymax=3]
		\tkzDefPoint(0,0){O}
		\tkzDefPoint(2,0){AA}
		\tkzDefPoint(1,-1.732050808){DD}
		\tkzLabelAngle[pos = 0.6](DD,O,AA){$\theta$}
		\tkzMarkAngle[size = 0.3, pos = 0.1](DD,O,AA)

		\tkzLabelAngle[pos = -0.6](O,AA,DD){$\omega$}
		\tkzMarkAngle[size = 0.3, pos = 0.1](O,AA,DD)

		\tkzLabelAngle[pos = 0.6](AA,DD,O){$\omega$}
		\tkzMarkAngle[size = 0.3, pos = 0.1](AA,DD,O)

		\tkzDrawSegments(O,AA O,DD AA,DD)

		\tkzLabelSegment[above=1pt](O,AA){$r + R$}  
		\tkzLabelSegment[left=1pt](O,DD){$r + R$}
		\tkzLabelSegment[right=1pt](AA,DD){$2R$}
	}
}

\env{frame}
{
	\frametitle{Tvinntölur}
	\env{itemize}
	{
		\item<1-> Skilgreinum mengið $\mathbb{C} := \mathbb{R} \times \mathbb{R}$.
		\item<2-> Skilgreinum svo margföldun á $\mathbb{C}$ þannig að fyrir $(a, b), (c, d) \in \mathbb{C}$ þ.a.
			\[
				(a, b) \cdot (c, d) = (ac - bd, ad + bc)
			\]
		\item<3-> Við táknum iðulega $(0, 1) \in \mathbb{C}$ með $i$ og $(x, y) \in \mathbb{C}$ með $x + yi$.
		\item<4-> Stök $\mathbb{C}$ kallast \emph{tvinntölur}.
		\item<5-> Ef $z = x + yi \in \mathbb{C}$ þá köllum við $x$ \emph{raunhluta} $z$ og $y$ \emph{þverhluta} $z$.
		\item<6-> Ef $z = x + yi$ þá er \emph{lengd} $z$ gefin með $|z| = \sqrt{x^2 + y^2}$.
		\item<6-> Ef $z = x + yi$ köllum við $x - yi$ \emph{samoka} $z$, táknað $\overline{z}$.
	}
}

\section[Rúmfræði í bitum]{Rúmfræði í bitum}

\env{frame}
{
	\frametitle{Inngangur}
	\env{itemize}
	{
		\item<1-> Við munum aðallega fjalla tvívíða rúmfræði í þessum fyrirlestri.
		\item<2-> Þegar leysa þarf þrívíð rúmfræði dæmi er oft góð hugmynd að byrja
			á að leysa dæmin í tveimur víddum (ef unnt er) og reyna svo að yfirfæra
			tvívíðu lausnina í þriðju víddina.
		\item<3-> Hingað til í námskeiðinu höfum við að mestu fengist við heiltölur
			og stöku sinnum þurft að vinna með fleytitölur.
		\item<4-> Í rúmfræði er þetta þó öfugt, við vinnum aðallega með fleytitölur og
			stöku sinnum heiltölur.
		\item<5-> Þegar við notum fleytitölur er mikilvægt að passa að samanburðir er
			ekki fullkomnir.
		\item<6-> Við látum því duga að tvær tölur sé \emph{nógu} líkar, í vissum skilningi.
	}
}

\begin{frame}[fragile]
	\frametitle{Fleytitölu samanburðir}
	\code{comp.h}
\end{frame}

\env{frame}
{
	\frametitle{Punktar}
	\env{itemize}
	{
		\item<1-> Það eru tvær algengar leiðir til að útfæra punkta í plani.
		\item<2-> Augljósari aðferðin er að skilgreina gagnagrind (\texttt{struct}) sem geymir
			tvær fleytitölur.
		\item<3-> Hin aðferðin er að nota innbyggða (í flestum málum) tvinntölu gagnatagið.
		\item<4-> Þó þessi aðgerð gæti verið nokkuð heimulleg þá er hún þægileg og fljótleg í útfærslu.
	}
}

\env{frame}
{
	\frametitle{Notkun á \texttt{complex} úr \texttt{C++} í rúmfræði}
	\env{itemize}
	{
		\item<1-> Fallið \texttt{real(p)} skilar ofanvarpi $p$ á $x$-ás.
		\item<2-> Fallið \texttt{imag(p)} skilar ofanvarpi $p$ á $y$-ás.
		\item<3-> Fallið \texttt{abs(p)} skilar fjarlægð $p$ frá $(0, 0)$.
		\item<4-> Fallið \texttt{abs(p - q)} skilar fjarlægð milli $p$ og $q$.
		\item<5-> Fallið \texttt{arg(p)} skilar horninu sem $p$ myndar við jákvæða hluta $x$-ás.
		\item<6-> Fallið \texttt{norm(p)} skilar sama og \texttt{abs(p)*abs(p)}.
		\item<7-> Fallið \texttt{conj(p)} speglar $p$ um $x$-ás.
	}
}

\begin{frame}[fragile]
	\frametitle{Punktar}
	\code{point.h}
\end{frame}

\env{frame}
{
	\frametitle{Sýnidæmi}
	Þú byrjar í $(0, 0)$ og færð gefnar skipanir. Skapnirnar eru allar einn bókstafur og ein tala.
	Ef skipunin er \texttt{'f' x} gengur þú áfram um $x$ metra, \texttt{'b' x} gengur þú aftur á bak
	um $x$ metra, \texttt{'r' x} snýrð þú þér um $x$ gráður til hægri og \texttt{'l' x} snýrð þú
	þér um $x$ gráður til vinstri. Eftir að fylgja öllum þessum skipunum, hversu langt ertu frá
	$(0, 0)$.
}

\env{frame}
{
	\frametitle{Sýnidæmi}
	\env{itemize}
	{
	\item<1-> Ef við erum í $p \in \mathbb{C}$ og viljum taka $r$ metra skref í stefnu $\theta$ getum
		við einfaldlega lagt $r\cdot e^{i\theta}$ við $p$.
	\item<2-> Hvert við snúum í upphafi skiptir ekki mál því það hefur ekki áhrif á fjarlægðinni til $(0, 0)$.
	}
}

\begin{frame}[fragile]
	\frametitle{Sýnidæmi}
	\code{kindalogo.cpp}
\end{frame}

\env{frame}
{
	\frametitle{Línur}
	\env{itemize}
	{
		\item<1-> Samkvæmt fyrstu frumsendu rúmfræðinnar skilgreina tveir ólíkir punktar nákvæmlega eina línu.
		\item<2-> Svo við getum einfaldlega sagt að lína sé tveir punktar.
		\item<3-> Helsti ókostur þessa aðferðar er að sama línan getur verið skilgreint með mismunandi pörum af puntkum.
		\item<4-> Stundum hentar betur að skilgreina línu með skurðpunkt við $y$-ás og hallatölu. 
		\item<5-> Þá er einfaldara að bera saman línur en það þarf að höndla sérstaklega línuna í gegnum $(0, 0)$ og $(0, 1)$.
	}
}

\env{frame}
{
	\frametitle{Skurðpunktur lína}
	\env{itemize}
	{
		\item<1-> Ef tvær línur eru ekki samsíða þá skerast þær í nákvæmlega einum punkti.
		\item<2-> Við þurfum oft að finna þennan punkt.
		\item<3-> Gefum okkur tvær línur $\{(x, y) : ax+by=c\}$ og $\{(x, y) : dx+ey=f\}$.
		\item<4-> Gerum ráð fyrir að línurnar séu ekki samsíða.
		\item<5-> Skurðpunkturinn fæst þá greinilega með því að leysa jöfnuhneppið
			\[
				\left (
				\begin{array}{c c | c}
					a & b & c\\
					d & e & f
				\end{array}
				\right )
			\]
	}
}

\env{frame}
{
	\frametitle{Skurðpunktur lína}
		\[
			\left (
			\begin{array}{c c}
				a & b\\
				d & e
			\end{array}
			\right )
			\left (
			\begin{array}{c}
				x\\
				y
			\end{array}
			\right )
			=
			\left (
			\begin{array}{c}
				c\\
				f
			\end{array}
			\right )
		\]
			\pause 
		\[
			\Rightarrow
			\left (
			\begin{array}{c}
				x\\
				y
			\end{array}
			\right )
			=
			\left (
			\begin{array}{c c}
				a & b\\
				d & e
			\end{array}
			\right )^{-1}
			\left (
			\begin{array}{c}
				c\\
				f
			\end{array}
			\right )
		\]
			\pause 
		\[
			\Rightarrow
			\left (
			\begin{array}{c}
				x\\
				y
			\end{array}
			\right )
			=
			\frac{1}{ae - bd}
			\left (
			\begin{array}{c c}
				e & -b\\
				-d & a
			\end{array}
			\right )
			\left (
			\begin{array}{c}
				c\\
				f
			\end{array}
			\right )
		\]
			\pause 
		\[
			\Rightarrow
			\left (
			\begin{array}{c}
				x\\
				y
			\end{array}
			\right )
			=
			\frac{1}{ae - bd}
			\left (
			\begin{array}{c}
				ce - bf\\
				af - cd
			\end{array}
			\right )
		\]
			\pause 
		\[
			\Rightarrow
			\left (
			\begin{array}{c}
				x\\
				y
			\end{array}
			\right )
			=
			\left (
			\begin{array}{c}
			\frac{ce - bf}{ae - bd}\\
			\frac{af - cd}{ae - bd}
			\end{array}
			\right )
		\]
}

\env{frame}
{
	\frametitle{Línustrik}
	\env{itemize}
	{
		\item<1-> Af augljósum ástæðum sést að best er að skilgreina línustrik sem par punkta, nánar tiltekið endapunkta línustriksins.
		\item<2-> Markmið okkar í þessum hluta af fyrirlestrinu verður að útfæra fall sem finnur fjarlægð milli línustrika.
		\item<3-> Við munum byrja á að útfæra góð hjálparföll sem nýtast meðal annars í að finna þessa fjarlægð, en eru einnig
			hentug í öðrum dæmum.
		\item<4-> Við munum útfæra:
		\env{itemize}
		{
			\item<5-> Fall sem skoðar hvort tvö bil skerist (\texttt{bxb}).
			\item<6-> Fall sem skoðar hvort línustrik skerist (\texttt{lxl}).
			\item<7-> Fall sem finnur stystu fjarlægð punkts og línustriks (\texttt{p2l}).
			\item<8-> Fall sem finnur stystu fjarlægð tveggja línustrika (\texttt{l2l}).
		}
	}
}

\env{frame}
{
	\frametitle{Skurður bila}
	\env{itemize}
	{
		\item<1-> Þegar við viljum skoða skurð tveggja bila nægir okkur að skoða hvort annar endapunktur bils er í hinu bilinu.
		\item<2-> \code{bxb.h}
	}
}

\env{frame}
{
	\frametitle{Skurður tveggja línustrika}
	\env{itemize}
	{
		\small
		\item<1-> Þessi glæra gæti verið nokkuð dularfull en hún mun verða skýrari seinna í þessum fyrirlestri.
		\item<2-> Við munum skoða áttunina á þríhyrningunum sem við getum myndað með endapunktum línustrikanna.
		\item<3-> Látum $a, b, c, d$ vera endapunkta línustrikanna.
		\item<4-> Ef þríhyningurinn stikaður með $\langle a, b, c, a \rangle$ hefur öfuga áttun miðað við $\langle a, b, d, a \rangle$ þá liggja
			punktarnir $c$ og $d$ sitthvoru megin við línustrikið $\langle a, b \rangle$.
		\item<5-> Við þurfum líka að ganga úr skugga um ferhyrningarnir sem endapunktarnir mynda skerast.
		\item<6-> Það eru leiðinleg sértilfelli þegar þrír af endapunktunum liggja á sömulínunni.
		\item<7-> Ég mun eftirláta ykkur að laga þetta sértilfelli, því þar sem við höfum aðallega
			áhuga á að finna fjarlægð línubila nægir að segja að línustrikin skerist ekki í þessu sértilfelli.
		\item<8-> Þetta mun skýrast betur á eftir.
	}
}

\env{frame}
{
	\frametitle{Skurður tveggja línustrika}
	\code{lxl.h}
}

\env{frame}
{
	\frametitle{Fjarlægð punkts og línustriks}
	\env{itemize}
	{
		\item<1-> Hér hefst fjörið. 
		\item<2-> Látum línustrikið vera $\langle (x_0, y_0), (x_1, y_1) \rangle$ og punktinn $(x, y)$.
		\item<3-> Við getum þá stikað línustrikið með $l(t) = (x_0 + t\cdot(x_1 - x_0), y_0 + t\cdot(y_1 - y_0)), t \in [0, 1].$
		\item<4-> Látum $d$ tákna Evklíðsku firðina. Við munum nú lágmarka $d^2$ til að auðvelda deildunina.
		\item<5-> Látum $f(t) = d(l(t), (x, y))^2$.
		\item<6-> Við fáum enn fremur að
		\env{align*}
		{
			f(t)
			=& (x_0 + t\cdot(x_1 - x_0) - x)^2 + (y_0 + t\cdot(y_1 - y_0) - y)^2\\
			=& x_0^2 + t^2(x_1 - x_0)^2 + x^2 + 2x_0t(x_1 - x_0) - 2xx_0 - 2tx(x_1 - x_0)\\
			+& y_0^2 + t^2(y_1 - y_0)^2 + y^2 + 2y_0t(y_1 - y_0) - 2yy_0 - 2ty(y_1 - y_0).
		}
	}
}

\env{frame}
{
	\frametitle{Fjarlægð punkts og línustriks}
	\env{itemize}
	{
		\item<1-> Deildum nú $f$ og fáum 
		\env{align*}
		{
			f'(t)
			=& 2t(x_1 - x_0)^2 + 2x_0(x_1 - x_0) - 2x(x_1 - x_0)\\
			+& 2t(y_1 - y_0)^2 + 2y_0(y_1 - y_0) - 2y(y_1 - y_0).
		}
		\item<2-> Metum nú í $0$ óg fáum
		\env{align*}
		{
			& f'(t_0) = 0\\
			\Rightarrow & t_0(x_1 - x_0)^2 + x_0(x_1 - x_0) - x(x_1 - x_0)\\
			+& t_0(y_1 - y_0)^2 + y_0(y_1 - y_0) - y(y_1 - y_0) = 0\\
			\Rightarrow & t_0 = \frac{x(x_1 - x_0) + y(y_1 - y_0) - x_0(x_1 - x_0) - y_0(y_1 - y_0)}{(y_1 - y_0)^2 + (x_1 - x_0)^2}\\
			\Rightarrow & t_0 = \frac{(x - x_0)(x_1 - x_0) + (y - y_0)(y_1 - y_0)}{(y_1 - y_0)^2 + (x_1 - x_0)^2}.\\
		}
	}
}

\env{frame}
{
	\frametitle{Fjarlægð punkts og línustriks}
	\env{itemize}
	{
		\item<1-> Við erum nú búin að finna þann punkt á línunni gegnum $(x_0, y_0)$ og $(x_1, y_1)$ sem er
			næstur $(x, y)$. 
		\item<2-> Þar sem línustrikið er stikað af $t$ þegar $t \in [0, 1]$ þá er þessi punktur
			á línustrikinu ef $t_0 \in [0, 1]$.
		\item<3-> Ef svo er ekki nægir okkur að skoða fjarlægð $(x, y)$ til endapunktana $(x_0, y_0)$ og $(x_1, y_1)$.
		\item<4-> \code{p2l.h}
	}
}

\env{frame}
{
	\frametitle{Fjarlægð milli tveggja línustrika}
	\env{itemize}
	{
		\item<1-> Gefum okkur línustrikin $\langle a, b \rangle$ og $\langle c, d \rangle$.
		\item<2-> Ef þau skerast þá er fjarlægðin á milli þeirra bersýnilega $0$.
		\item<3-> Gerum því ráð fyrir að þau skerist ekki.
		\item<4-> Þá er bersýnilega fjarlægð línustrikana minnst á milli 
			$a$ og $\langle c, d \rangle$, 
			$b$ og $\langle c, d \rangle$, 
			$c$ og $\langle a, b \rangle$ eða 
			$d$ og $\langle a, b \rangle$.
		\item<5-> Ég hef ekki minnst á það hingað til, en við höfum gert ráð fyrir að endapunktar 
			línustrikana séu mismunandi.
		\item<6-> Ef $a = b$ og $c = d$ þá er fjarlægð línustrikana fjarlægðin milli $a$ og $c$.
		\item<7-> Ef $a = b$ og $c \neq d$ þá er fjarlægð línustrikana fjarlægðin milli $a$ og $\langle c, d \rangle$.
		\item<8-> Ef $a \neq b$ og $c = d$ þá er fjarlægð línustrikana fjarlægðin milli $c$ og $\langle a, b \rangle$.
	}
}

\env{frame}
{
	\frametitle{Fjarlægð milli tveggja línustrika}
	\code{l2l.h}
}

\env{frame}
{
	\frametitle{Hringir}
	\env{itemize}
	{
		\item<1-> Til að útfæra hringi geymum við iðulega miðpunkt hringsins og geisla hans.
		\item<2-> Það er gott að vita að unnt er að ákvarð hring útfrá:
		\env{itemize}
		{
			\item<3-> Miðju og geisla.
			\item<4-> Miðju og og punkti á jaðri hringsins.
			\item<5-> Þremur punktum á jaðri hringsins.
			\item<6-> Tveimur punktum á jaðri hringsins og geisla (hér er þó tveir mögulegir hringir).
		}
	}
}

\env{frame}
{
	\frametitle{Marghyrningar}
	\env{itemize}
	{
		\item<1-> \emph{Marghyrningur} er samfelldur, lokaður ferill í plani 
			sem samanstendur af beinum línustrikum.
		\item<2-> Ef ferillinn er einfaldur þá kallast marghyrningurinn \emph{einfaldur}.
		\item<3-> Marghyrningur eru sagður vera \emph{kúptur} ef sérhver beina lína dregin gegnum
			marghyrninginn sker mest tvo punkta á marghyrningnum.
	}
	\scalebox{0.6}
	{
		\env{tikzpicture}
		{
			\tkzInit[xmin=0,xmax=10,ymin=0,ymax=5]
			\tkzDefPoint(0,2){A1}
			\tkzDefPoint(4,0){B1}
			\tkzDefPoint(4,3){C1}
			\tkzDefPoint(3,4){D1}
			\tkzDefPoint(2,4){E1}
			\tkzDefPoint(1,5){F1}
			\tkzDrawSegments(A1,B1 B1,C1 C1,D1 D1,E1 E1,F1 F1,A1)

			\tkzDefPoint(7,2){A2}
			\tkzDefPoint(11,0){B2}
			\tkzDefPoint(7,3){C2}
			\tkzDefPoint(10,4){D2}
			\tkzDefPoint(9,4){E2}
			\tkzDefPoint(8,5){F2}
			\tkzDrawSegments(A2,B2 B2,C2 C2,D2 D2,E2 E2,F2 F2,A2)

			\tkzDefPoint(13,4){A3}
			\tkzDefPoint(13,0){B3}
			\tkzDefPoint(19,0){C3}
			\tkzDefPoint(19,4){D3}
			\tkzDefPoint(16,5){E3}
			\tkzDrawSegments(A3,B3 B3,C3 C3,D3 D3,E3 E3,A3)
		}
	}
}

\env{frame}
{
	\frametitle{Framsetning marghyrninga}
	\env{itemize}
	{
		\item<1-> Þegar við viljum tákna marghyrning í tölvu notum við einfaldlega hornpunkta hans.
		\item<2-> Röð punktanna skiptir máli.
			\scalebox{0.3}
			{
				\env{tikzpicture}
				{
					\tkzInit[xmin=0,xmax=10,ymin=0,ymax=5]
					\tkzDefPoint(0,2){A1}
					\tkzDefPoint(2,2){B1}
					\tkzDefPoint(2,0){C1}
					\tkzDefPoint(0,0){D1}
					\tkzDrawSegments(A1,B1 B1,C1 C1,D1 D1,A1)

					\tkzDefPoint(4,2){A2}
					\tkzDefPoint(6,0){B2}
					\tkzDefPoint(6,2){C2}
					\tkzDefPoint(4,0){D2}
					\tkzDrawSegments(A2,B2 B2,C2 C2,D2 D2,A2)
				}
			}
		\item<3-> Til þæginda geymum við einn punkt tvisvar, nánar tiltekið er fremsti og aftasti punkturinn eins.
		\item<4-> Þetta er því við höfum oft meira áhuga á línustrikunum milli hornpuntkanna heldur en hornpunktunum sjálfum.
		\item<5-> \code{polygon.h}
	}
}

\env{frame}
{
	\frametitle{Nokkur atriði um marghyrninga}
	\env{itemize}
	{
		\item<1-> Þar sem marghyrningar er mjög vinsælir í keppnum munum við fara í nokkur atriði sem er gott að kunna.
	}
}

\env{frame}
{
	\frametitle{Ummál marghyrnings}
	\env{itemize}
	{
		\item<1-> Ummála marghyrnings er einfalt að reikna í línulegum tíma.
		\item<2-> Maður leggur einfaldlega saman allar hliðarlengdirnar.
	}
}

\begin{frame}[fragile]
	\frametitle{Ummál marghyrnings}
	\code{ummal.h}
\end{frame}

\env{frame}
{
	\frametitle{Flatarmál marghyrnings}
	\env{itemize}
	{
		\item<1-> Ummál marghyrninga er þó ekki jafnt algengt í keppnum og flatarmál marghyrninga.
		\item<2-> Það er einnig auðvelt að reikna flatarmálið í línulegum tíma, þó það sé ekki
			endilega augljóst að þetta skili flatarmálinu.
		\item<3-> Fyrir áhugasama er hægt að nota setningu Green til að leiða út eftirfarandi forritsbút.
	}
}

\env{frame}
{
	\frametitle{Flatarmál marghyrnings}
	\env{itemize}
	{
		\item<1-> \code{flatarmal.h}
		\item<2-> Við þurfum að skila tölugildinu því við getum fengið neikvætt flatarmál.
		\item<3-> Neikvætt flatarmál hljómar kannski furðulega en það fellur eðlilega úr sönnun summunar
			ef notast er við setningu Green.
		\item<4-> Til að nota hana þarf að reikna ferilheildi og útkoman úr ferilheildum skiptir
			um formerki þegar breytt er um átt stikunar ferilsins.
		\item<5-> Þetta þýðir að formerki \texttt{r} eftir forlykkjuna er jákvætt ef punktar \texttt{p} 
			eru gefnir rangsælis og neikvætt ef þeir eru gefnir réttsælis.
	}
}

\env{frame}
{
	\frametitle{Punktur í marghyrning}
	\env{itemize}
	{
		\item<1-> Að ákvarða hvort punktur sé inni í marghyrning (e. the point in polygon problem) er algent undirvandamál
			í rúmfræði dæmum.
		\item<2-> Aðallega er gengist við tvær aðferðir til að leysa slík dæmi, sú fyrri er að nota geislarakningu 
			(e. raytracing) og hin er að reikna summu aðliggjandi horna marghyrningsins miðað við punktinn.
	}
}

\env{frame}
{
	\frametitle{Geislarakning}
	\env{itemize}
	{
		\item<1-> Við getum dregið geisla frá punktinum sem við viljum skoða í einhverja átt og talið hversu oft
			við skerum jaðar marghyrningsins. 
		\item<2-> Ef við erum fyrir utan marghyrninginn og skerum jaðarinn erum við inni í honum,
			en ef við erum fyrir utan og skerum jaðarinn erum við fyrir innan (þetta er í raun skilgreining
			á því hvenær geislinn \emph{sker} marghyrninginn).
		\item<3-> Svo ef við skerum jaðarinn slétt tölu sinnum er punkturinn fyrir innan, og annars fyrir utan (Setning Jordan).
		\item<4-> Við getum látið geislann vera línustrik, nógu langt til að vera út fyrir marghyrninginn, og
			notað síðan \texttt{lxl} til að ákvarða í línulegum tíma hversu oft geislinn sker marghyrninginn.
	}
}

\env{frame}
{
	\frametitle{Geislarakning}
	\center
	\scalebox{1.0}
	{
		\env{tikzpicture}
		{
			\tkzInit[xmin=-10,xmax=10,ymin=-10,ymax=10]
			\tkzDefPoint(-1,-1){A}
			\tkzDefPoint(-1,1){B}
			\tkzDefPoint(3,1){C}
			\tkzDefPoint(4,0){D}
			\tkzDefPoint(5,1){E}
			\tkzDefPoint(6,1){F}
			\tkzDefPoint(7,0){G}
			\tkzDefPoint(6,-1){H}
			\tkzDrawSegments(A,B B,C C,D D,E E,F F,G G,H H,A)

			\tkzDefPoint(0,-0.5){P1}
			\tkzDefPoint(0,0){P2}
			\tkzDefPoint(-2,0.5){P3}
			\tkzDefPoint(10,-0.5){Q1}
			\tkzDefPoint(10,0){Q2}
			\tkzDefPoint(10,0.5){Q3}
			\tkzDrawPoint(P1)
			\tkzDrawPoint(P2)
			\tkzDrawPoint(P3)
			\tkzDrawSegments[dashed](P1,Q1 P2,Q2 P3,Q3)
		}
	}
}

\env{frame}
{
	\frametitle{Geislarakning}
	\env{itemize}
	{
		\item<1-> Þessi aðferð er með aragrúa af sértilfellum sem gerir þessa aðferð nokkuð óþægilega í útfærslu.
		\item<2-> Öll sértilfellin eiga það sameiginlegt að vera þegar geislinn sker endapunkta línustrika marghyrningsins.
		\item<3-> Ef marghyrningurinn er kúptur er nokkuð auðvelt að eiga við þessi sértilfelli, en það
			gildir ekki í flestum dæmum.
		\item<4-> Þessi aðferð verður því ekki úrfærð hér.
	}
}

\env{frame}
{
	\frametitle{Afstæð hornasumma}
	\env{itemize}
	{
		\item<1-> Látum $p_i$, $i < n$ tákna hornpunkta marghyrnings, $p$ einhvern punkt, $\alpha(a, b, c)$ vera
			hornið milli sem punktarnir $a$, $b$ og $c$ mynda og $\beta(a, b, c)$ vera $1$ ef brotna línustrikið
			$\langle a, b, c \rangle$ ,,beygir'' til vinstri en $-1$ annars.
		\item<2-> \emph{Afstæð hornsumma marghyrnings með tilliti til punkts $q$} er
			$\sum_{i = 0}^n \beta(q, p_i, p_{i + 1})\alpha(p_i, q, p_{i + 1})$.
		\item<3-> Ef $q$ er inni í marghyrningnum þá er þessi summa bersýnilega $2\pi$.
		\item<4-> Ef $q$ er fyrir utan marghyrninginn þá verður summan hins vegar $0$.
	}
	\scalebox{0.7}
	{
		\env{tikzpicture}
		{
			\tkzInit[xmin=0,xmax=4,ymin=0,ymax=3]
			\tkzDefPoint(1,3){A}
			\tkzDefPoint(0,0){B}
			\tkzDefPoint(3,0){C}
			\tkzDrawSegments(A,B B,C)
			\tkzMarkAngle[label=$\alpha$, size = 0.8](C,B,A)

			\tkzDrawPoint(A)
			\tkzDrawPoint(B)
			\tkzDrawPoint(C)
			\tkzLabelPoint(A){$a$}
			\tkzLabelPoint(B){$b$}
			\tkzLabelPoint(C){$c$}
		}
	}
}

\env{frame}
{
	\frametitle{Afstæð hornasumma}
	\scalebox{1.0}
	{
		\env{tikzpicture}
		{
			\tkzInit[xmin=-10,xmax=10,ymin=-10,ymax=10]
			\tkzDefPoint(-1,-3){A}
			\tkzDefPoint(-1,3){B}
			\tkzDefPoint(3,3){C}
			\tkzDefPoint(4,0){D}
			\tkzDefPoint(5,3){E}
			\tkzDefPoint(6,3){F}
			\tkzDefPoint(7,0){G}
			\tkzDefPoint(6,-3){H}
			\tkzDrawSegments(A,B B,C C,D D,E E,F F,G G,H H,A)

			\tkzDefPoint(0,-0.5){P1}
			\tkzDefPoint(-2,0.5){P2}
			\tkzDrawSegments[dashed](P1,A P1,B P1,C P1,D P1,E P1,F P1,G P1,H)
			\tkzDrawPoint(P1)
		}
	}
}

\env{frame}
{
	\frametitle{Afstæð hornasumma}
	\scalebox{1.0}
	{
		\env{tikzpicture}
		{
			\tkzInit[xmin=-10,xmax=10,ymin=-10,ymax=10]
			\tkzDefPoint(-1,-3){A}
			\tkzDefPoint(-1,3){B}
			\tkzDefPoint(3,3){C}
			\tkzDefPoint(4,0){D}
			\tkzDefPoint(5,3){E}
			\tkzDefPoint(6,3){F}
			\tkzDefPoint(7,0){G}
			\tkzDefPoint(6,-3){H}
			\tkzDrawSegments(A,B B,C C,D D,E E,F F,G G,H H,A)

			\tkzDefPoint(-2,0.5){P1}
			\tkzDrawSegments[dashed](P1,A P1,B P1,C P1,D P1,E P1,F P1,G P1,H)
			\tkzDrawPoint(P1)
		}
	}
}

\env{frame}
{
	\frametitle{Afstæð hornasumma}
	\scalebox{1.0}
	{
		\env{tikzpicture}
		{
			\tkzInit[xmin=-10,xmax=10,ymin=-10,ymax=10]
			\tkzDefPoint(-1,-3){A}
			\tkzDefPoint(-1,3){B}
			\tkzDefPoint(3,3){C}
			\tkzDefPoint(4,0){D}
			\tkzDefPoint(5,3){E}
			\tkzDefPoint(6,3){F}
			\tkzDefPoint(7,0){G}
			\tkzDefPoint(6,-3){H}
			\tkzDrawSegments(A,B B,C C,D D,E E,F F,G G,H H,A)

			\tkzDefPoint(4,2){P1}
			\tkzDefPoint(-2,0.5){P2}
			\tkzDrawSegments[dashed](P1,A P1,B P1,C P1,D P1,E P1,F P1,G P1,H)
			\tkzDrawPoint(P1)
		}
	}
}

\env{frame}
{
	\frametitle{Afstæð hornasumma}
	\code{is_in.h}
}

\env{frame}
{
	\frametitle{Kútpur hjúpur punktasafns}
	\env{itemize}
	{
		\item<1-> \emph{Kúptur hjúpur punktasafns} er minnsti kúpti marghyrningur sem
			inniheldur all punkta punktasafnsins.
		\item<2-> Maður getur ímyndað sér að maður taki teygju og strekki hana yfir punkta safnið og
			sleppi henni svo.
		\item<3-> Við munum nota aðferð sem kallast \emph{Graham's scan} til að finna kúpta hjúp punktasafns.
	}
	\center
	\scalebox{0.3}
	{
		\env{tikzpicture}
		{
			\tkzInit[xmin=-10,xmax=10,ymin=-10,ymax=10]
			\tkzDefPoint(-2.4,-7.4){A}
			\tkzDefPoint(-5.1,-0.6){B}
			\tkzDefPoint(7.3,1.7){C}

			\tkzDefPoint(4.1,-0.6){P1}
			\tkzDefPoint(0,0){P2}
			\tkzDefPoint(-3,-3.4){P3}
			\tkzDefPoint(-2,-4){P4}
			\tkzDefPoint(-1,-2){P5}
		}
	}
}

\env{frame}
{
	\frametitle{Kútpur hjúpur punktasafns}
	\env{itemize}
	{
		\item<1-> \emph{Kúptur hjúpur punktasafns} er minnsti kúpti marghyrningur sem
			inniheldur all punkta punktasafnsins.
		\item<1-> Maður getur ímyndað sér að maður taki teygju og strekki hana yfir punkta safnið og
			sleppi henni svo.
		\item<1-> Við munum nota aðferð sem kallast \emph{Graham's scan} til að finna kúpta hjúp punktasafns.
	}
	\center
	\scalebox{0.3}
	{
		\env{tikzpicture}
		{
			\tkzInit[xmin=-10,xmax=10,ymin=-10,ymax=10]
			\tkzDefPoint(-2.4,-7.4){A}
			\tkzDefPoint(-5.1,-0.6){B}
			\tkzDefPoint(7.3,1.7){C}

			\tkzDefPoint(4.1,-0.6){P1}
			\tkzDefPoint(0,0){P2}
			\tkzDefPoint(-3,-3.4){P3}
			\tkzDefPoint(-2,-4){P4}
			\tkzDefPoint(-1,-2){P5}
			\tkzDrawPoint(P1)
			\tkzDrawPoint(P2)
			\tkzDrawPoint(P3)
			\tkzDrawPoint(P4)
			\tkzDrawPoint(P5)
			\tkzDrawPoint(A)
			\tkzDrawPoint(B)
			\tkzDrawPoint(C)
		}
	}
}

\addtocounter{framenumber}{-1}
\env{frame}
{
	\frametitle{Kútpur hjúpur punktasafns}
	\env{itemize}
	{
		\item<1-> \emph{Kúptur hjúpur punktasafns} er minnsti kúpti marghyrningur sem
			inniheldur all punkta punktasafnsins.
		\item<1-> Maður getur ímyndað sér að maður taki teygju og strekki hana yfir punkta safnið og
			sleppi henni svo.
		\item<1-> Við munum nota aðferð sem kallast \emph{Graham's scan} til að finna kúpta hjúp punktasafns.
	}
	\center
	\scalebox{0.3}
	{
		\env{tikzpicture}
		{
			\tkzInit[xmin=-10,xmax=10,ymin=-10,ymax=10]
			\tkzDefPoint(-2.4,-7.4){A}
			\tkzDefPoint(-5.1,-0.6){B}
			\tkzDefPoint(7.3,1.7){C}

			\tkzDefPoint(4.1,-0.6){P1}
			\tkzDefPoint(0,0){P2}
			\tkzDefPoint(-3,-3.4){P3}
			\tkzDefPoint(-2,-4){P4}
			\tkzDefPoint(-1,-2){P5}
			\tkzDrawSegments[dashed](A,B B,C C,A)
			\tkzDrawPoint(P1)
			\tkzDrawPoint(P2)
			\tkzDrawPoint(P3)
			\tkzDrawPoint(P4)
			\tkzDrawPoint(P5)
			\tkzDrawPoint(A)
			\tkzDrawPoint(B)
			\tkzDrawPoint(C)
		}
	}
}

\env{frame}
{
	\frametitle{Graham's Scan}
	\env{itemize}
	{
		\item<1-> Við byrjum á að velja vendipunkt, yfirleitt látin vera punkturinn neðst til vinstri.
		\item<2-> Við látum hann fremst í safnið og röðum svo restin miðið við hornið sem þeir mynda við vendipunktinn.
		\item<3-> Við gefum okkur svo hlaða og látum aftasta, fremsta og næst fremsta punktinn á hlaðann.
		\item<4-> Við göngum síðan í gegnum raðaða punkta safnið okkur og fyrir hvert stak fjarlægjum við ofan af hlaðanum
			á meðan efstu tvö stökin á hlaðanum og stakið sem við erum á í listanum mynda hægri beygju. Þegar þau mynda
			vinstri beygju bætum við stakinu úr safninu á hlaðan.
		\item<5-> Þegar við er búin að fara í gegnum allt safnið er hlaðinn kúpti hjúpurinn.
	}
}

\env{frame}
{
	\frametitle{Kútpur hjúpur punktasafns}
	\center
	\scalebox{0.5}
	{
		\env{tikzpicture}
		{
			\tkzInit[xmin=-10,xmax=10,ymin=-10,ymax=10]
			\tkzDefPoint(0,0){P0}
			\tkzDefPoint(10,1){P1}
			\tkzDefPoint(3,2){P2}
			\tkzDefPoint(4,4){P3}
			\tkzDefPoint(7,9){P4}
			\tkzDefPoint(2,8){P5}
			\tkzDefPoint(1,13){P6}
			\tkzDefPoint(-1,13){P7}
			\tkzDefPoint(-2,9){P8}
			\tkzDefPoint(-2,4){P9}
			\tkzDefPoint(-4,5){P10}
			\tkzDefPoint(-6,2){P11}
			\tkzDrawPoint(P0)
			\tkzDrawPoint(P1)
			\tkzDrawPoint(P2)
			\tkzDrawPoint(P3)
			\tkzDrawPoint(P4)
			\tkzDrawPoint(P5)
			\tkzDrawPoint(P6)
			\tkzDrawPoint(P7)
			\tkzDrawPoint(P8)
			\tkzDrawPoint(P9)
			\tkzDrawPoint(P10)
			\tkzDrawPoint(P11)
		}
	}
}

\addtocounter{framenumber}{-1}
\env{frame}
{
	\frametitle{Kútpur hjúpur punktasafns}
	\center
	\scalebox{0.5}
	{
		\env{tikzpicture}
		{
			\tkzInit[xmin=-10,xmax=10,ymin=-10,ymax=10]
			\tkzDefPoint(0,0){P0}
			\tkzDefPoint(10,1){P1}
			\tkzDefPoint(3,2){P2}
			\tkzDefPoint(4,4){P3}
			\tkzDefPoint(7,9){P4}
			\tkzDefPoint(2,8){P5}
			\tkzDefPoint(1,13){P6}
			\tkzDefPoint(-1,13){P7}
			\tkzDefPoint(-2,9){P8}
			\tkzDefPoint(-2,4){P9}
			\tkzDefPoint(-4,5){P10}
			\tkzDefPoint(-6,2){P11}
			\tkzDrawPoint(P0)
			\tkzDrawPoint(P1)
			\tkzDrawPoint(P2)
			\tkzDrawPoint(P3)
			\tkzDrawPoint(P4)
			\tkzDrawPoint(P5)
			\tkzDrawPoint(P6)
			\tkzDrawPoint(P7)
			\tkzDrawPoint(P8)
			\tkzDrawPoint(P9)
			\tkzDrawPoint(P10)
			\tkzDrawPoint(P11)
			\tkzLabelPoint(P0){$0$}
		}
	}
}

\addtocounter{framenumber}{-1}
\env{frame}
{
	\frametitle{Kútpur hjúpur punktasafns}
	\center
	\scalebox{0.5}
	{
		\env{tikzpicture}
		{
			\tkzInit[xmin=-10,xmax=10,ymin=-10,ymax=10]
			\tkzDefPoint(0,0){P0}
			\tkzDefPoint(10,1){P1}
			\tkzDefPoint(3,2){P2}
			\tkzDefPoint(4,4){P3}
			\tkzDefPoint(7,9){P4}
			\tkzDefPoint(2,8){P5}
			\tkzDefPoint(1,13){P6}
			\tkzDefPoint(-1,13){P7}
			\tkzDefPoint(-2,9){P8}
			\tkzDefPoint(-2,4){P9}
			\tkzDefPoint(-4,5){P10}
			\tkzDefPoint(-6,2){P11}
			\tkzDrawPoint(P0)
			\tkzDrawPoint(P1)
			\tkzDrawPoint(P2)
			\tkzDrawPoint(P3)
			\tkzDrawPoint(P4)
			\tkzDrawPoint(P5)
			\tkzDrawPoint(P6)
			\tkzDrawPoint(P7)
			\tkzDrawPoint(P8)
			\tkzDrawPoint(P9)
			\tkzDrawPoint(P10)
			\tkzDrawPoint(P11)
			\tkzLabelPoint(P0){$0$}
			\tkzLabelPoint(P1){$1$}
			\tkzLabelPoint(P2){$2$}
			\tkzLabelPoint(P3){$3$}
			\tkzLabelPoint(P4){$4$}
			\tkzLabelPoint(P5){$5$}
			\tkzLabelPoint(P6){$6$}
			\tkzLabelPoint(P7){$7$}
			\tkzLabelPoint(P8){$8$}
			\tkzLabelPoint(P9){$9$}
			\tkzLabelPoint(P10){$10$}
			\tkzLabelPoint(P11){$11$}
		}
	}
}

\addtocounter{framenumber}{-1}
\env{frame}
{
	\frametitle{Kútpur hjúpur punktasafns}
	\center
	\scalebox{0.5}
	{
		\env{tikzpicture}
		{
			\tkzInit[xmin=-10,xmax=10,ymin=-10,ymax=10]
			\tkzDefPoint(0,0){P0}
			\tkzDefPoint(10,1){P1}
			\tkzDefPoint(3,2){P2}
			\tkzDefPoint(4,4){P3}
			\tkzDefPoint(7,9){P4}
			\tkzDefPoint(2,8){P5}
			\tkzDefPoint(1,13){P6}
			\tkzDefPoint(-1,13){P7}
			\tkzDefPoint(-2,9){P8}
			\tkzDefPoint(-2,4){P9}
			\tkzDefPoint(-4,5){P10}
			\tkzDefPoint(-6,2){P11}
			\tkzDrawSegments(P11,P0 P0,P1)
			\tkzDrawPoint(P0)
			\tkzDrawPoint(P1)
			\tkzDrawPoint(P2)
			\tkzDrawPoint(P3)
			\tkzDrawPoint(P4)
			\tkzDrawPoint(P5)
			\tkzDrawPoint(P6)
			\tkzDrawPoint(P7)
			\tkzDrawPoint(P8)
			\tkzDrawPoint(P9)
			\tkzDrawPoint(P10)
			\tkzDrawPoint(P11)
			\tkzLabelPoint(P0){$0$}
			\tkzLabelPoint(P1){$1$}
			\tkzLabelPoint(P2){$2$}
			\tkzLabelPoint(P3){$3$}
			\tkzLabelPoint(P4){$4$}
			\tkzLabelPoint(P5){$5$}
			\tkzLabelPoint(P6){$6$}
			\tkzLabelPoint(P7){$7$}
			\tkzLabelPoint(P8){$8$}
			\tkzLabelPoint(P9){$9$}
			\tkzLabelPoint(P10){$10$}
			\tkzLabelPoint(P11){$11$}
		}
	}
}

\addtocounter{framenumber}{-1}
\env{frame}
{
	\frametitle{Kútpur hjúpur punktasafns}
	\center
	\scalebox{0.5}
	{
		\env{tikzpicture}
		{
			\tkzInit[xmin=-10,xmax=10,ymin=-10,ymax=10]
			\tkzDefPoint(0,0){P0}
			\tkzDefPoint(10,1){P1}
			\tkzDefPoint(3,2){P2}
			\tkzDefPoint(4,4){P3}
			\tkzDefPoint(7,9){P4}
			\tkzDefPoint(2,8){P5}
			\tkzDefPoint(1,13){P6}
			\tkzDefPoint(-1,13){P7}
			\tkzDefPoint(-2,9){P8}
			\tkzDefPoint(-2,4){P9}
			\tkzDefPoint(-4,5){P10}
			\tkzDefPoint(-6,2){P11}
			\tkzDrawSegments(P11,P0 P0,P1)
			\tkzDrawSegments[dashed](P1,P2)
			\tkzDrawPoint(P0)
			\tkzDrawPoint(P1)
			\tkzDrawPoint(P2)
			\tkzDrawPoint(P3)
			\tkzDrawPoint(P4)
			\tkzDrawPoint(P5)
			\tkzDrawPoint(P6)
			\tkzDrawPoint(P7)
			\tkzDrawPoint(P8)
			\tkzDrawPoint(P9)
			\tkzDrawPoint(P10)
			\tkzDrawPoint(P11)
			\tkzLabelPoint(P0){$0$}
			\tkzLabelPoint(P1){$1$}
			\tkzLabelPoint(P2){$2$}
			\tkzLabelPoint(P3){$3$}
			\tkzLabelPoint(P4){$4$}
			\tkzLabelPoint(P5){$5$}
			\tkzLabelPoint(P6){$6$}
			\tkzLabelPoint(P7){$7$}
			\tkzLabelPoint(P8){$8$}
			\tkzLabelPoint(P9){$9$}
			\tkzLabelPoint(P10){$10$}
			\tkzLabelPoint(P11){$11$}
		}
	}
}

\addtocounter{framenumber}{-1}
\env{frame}
{
	\frametitle{Kútpur hjúpur punktasafns}
	\center
	\scalebox{0.5}
	{
		\env{tikzpicture}
		{
			\tkzInit[xmin=-10,xmax=10,ymin=-10,ymax=10]
			\tkzDefPoint(0,0){P0}
			\tkzDefPoint(10,1){P1}
			\tkzDefPoint(3,2){P2}
			\tkzDefPoint(4,4){P3}
			\tkzDefPoint(7,9){P4}
			\tkzDefPoint(2,8){P5}
			\tkzDefPoint(1,13){P6}
			\tkzDefPoint(-1,13){P7}
			\tkzDefPoint(-2,9){P8}
			\tkzDefPoint(-2,4){P9}
			\tkzDefPoint(-4,5){P10}
			\tkzDefPoint(-6,2){P11}
			\tkzDrawSegments(P11,P0 P0,P1 P1,P2)
			%\tkzDrawSegments[dashed](P1,P2)
			\tkzDrawPoint(P0)
			\tkzDrawPoint(P1)
			\tkzDrawPoint(P2)
			\tkzDrawPoint(P3)
			\tkzDrawPoint(P4)
			\tkzDrawPoint(P5)
			\tkzDrawPoint(P6)
			\tkzDrawPoint(P7)
			\tkzDrawPoint(P8)
			\tkzDrawPoint(P9)
			\tkzDrawPoint(P10)
			\tkzDrawPoint(P11)
			\tkzLabelPoint(P0){$0$}
			\tkzLabelPoint(P1){$1$}
			\tkzLabelPoint(P2){$2$}
			\tkzLabelPoint(P3){$3$}
			\tkzLabelPoint(P4){$4$}
			\tkzLabelPoint(P5){$5$}
			\tkzLabelPoint(P6){$6$}
			\tkzLabelPoint(P7){$7$}
			\tkzLabelPoint(P8){$8$}
			\tkzLabelPoint(P9){$9$}
			\tkzLabelPoint(P10){$10$}
			\tkzLabelPoint(P11){$11$}
		}
	}
}

\addtocounter{framenumber}{-1}
\env{frame}
{
	\frametitle{Kútpur hjúpur punktasafns}
	\center
	\scalebox{0.5}
	{
		\env{tikzpicture}
		{
			\tkzInit[xmin=-10,xmax=10,ymin=-10,ymax=10]
			\tkzDefPoint(0,0){P0}
			\tkzDefPoint(10,1){P1}
			\tkzDefPoint(3,2){P2}
			\tkzDefPoint(4,4){P3}
			\tkzDefPoint(7,9){P4}
			\tkzDefPoint(2,8){P5}
			\tkzDefPoint(1,13){P6}
			\tkzDefPoint(-1,13){P7}
			\tkzDefPoint(-2,9){P8}
			\tkzDefPoint(-2,4){P9}
			\tkzDefPoint(-4,5){P10}
			\tkzDefPoint(-6,2){P11}
			\tkzDrawSegments(P11,P0 P0,P1 P1,P2)
			\tkzDrawSegments[dashed](P2,P3)
			\tkzDrawPoint(P0)
			\tkzDrawPoint(P1)
			\tkzDrawPoint(P2)
			\tkzDrawPoint(P3)
			\tkzDrawPoint(P4)
			\tkzDrawPoint(P5)
			\tkzDrawPoint(P6)
			\tkzDrawPoint(P7)
			\tkzDrawPoint(P8)
			\tkzDrawPoint(P9)
			\tkzDrawPoint(P10)
			\tkzDrawPoint(P11)
			\tkzLabelPoint(P0){$0$}
			\tkzLabelPoint(P1){$1$}
			\tkzLabelPoint(P2){$2$}
			\tkzLabelPoint(P3){$3$}
			\tkzLabelPoint(P4){$4$}
			\tkzLabelPoint(P5){$5$}
			\tkzLabelPoint(P6){$6$}
			\tkzLabelPoint(P7){$7$}
			\tkzLabelPoint(P8){$8$}
			\tkzLabelPoint(P9){$9$}
			\tkzLabelPoint(P10){$10$}
			\tkzLabelPoint(P11){$11$}
		}
	}
}

\addtocounter{framenumber}{-1}
\env{frame}
{
	\frametitle{Kútpur hjúpur punktasafns}
	\center
	\scalebox{0.5}
	{
		\env{tikzpicture}
		{
			\tkzInit[xmin=-10,xmax=10,ymin=-10,ymax=10]
			\tkzDefPoint(0,0){P0}
			\tkzDefPoint(10,1){P1}
			\tkzDefPoint(3,2){P2}
			\tkzDefPoint(4,4){P3}
			\tkzDefPoint(7,9){P4}
			\tkzDefPoint(2,8){P5}
			\tkzDefPoint(1,13){P6}
			\tkzDefPoint(-1,13){P7}
			\tkzDefPoint(-2,9){P8}
			\tkzDefPoint(-2,4){P9}
			\tkzDefPoint(-4,5){P10}
			\tkzDefPoint(-6,2){P11}
			\tkzDrawSegments(P11,P0 P0,P1)
			\tkzDrawSegments[dashed](P1,P3)
			\tkzDrawPoint(P0)
			\tkzDrawPoint(P1)
			\tkzDrawPoint(P2)
			\tkzDrawPoint(P3)
			\tkzDrawPoint(P4)
			\tkzDrawPoint(P5)
			\tkzDrawPoint(P6)
			\tkzDrawPoint(P7)
			\tkzDrawPoint(P8)
			\tkzDrawPoint(P9)
			\tkzDrawPoint(P10)
			\tkzDrawPoint(P11)
			\tkzLabelPoint(P0){$0$}
			\tkzLabelPoint(P1){$1$}
			\tkzLabelPoint(P2){$2$}
			\tkzLabelPoint(P3){$3$}
			\tkzLabelPoint(P4){$4$}
			\tkzLabelPoint(P5){$5$}
			\tkzLabelPoint(P6){$6$}
			\tkzLabelPoint(P7){$7$}
			\tkzLabelPoint(P8){$8$}
			\tkzLabelPoint(P9){$9$}
			\tkzLabelPoint(P10){$10$}
			\tkzLabelPoint(P11){$11$}
		}
	}
}

\addtocounter{framenumber}{-1}
\env{frame}
{
	\frametitle{Kútpur hjúpur punktasafns}
	\center
	\scalebox{0.5}
	{
		\env{tikzpicture}
		{
			\tkzInit[xmin=-10,xmax=10,ymin=-10,ymax=10]
			\tkzDefPoint(0,0){P0}
			\tkzDefPoint(10,1){P1}
			\tkzDefPoint(3,2){P2}
			\tkzDefPoint(4,4){P3}
			\tkzDefPoint(7,9){P4}
			\tkzDefPoint(2,8){P5}
			\tkzDefPoint(1,13){P6}
			\tkzDefPoint(-1,13){P7}
			\tkzDefPoint(-2,9){P8}
			\tkzDefPoint(-2,4){P9}
			\tkzDefPoint(-4,5){P10}
			\tkzDefPoint(-6,2){P11}
			\tkzDrawSegments(P11,P0 P0,P1 P1,P3)
			%\tkzDrawSegments[dashed](P1,P3)
			\tkzDrawPoint(P0)
			\tkzDrawPoint(P1)
			\tkzDrawPoint(P2)
			\tkzDrawPoint(P3)
			\tkzDrawPoint(P4)
			\tkzDrawPoint(P5)
			\tkzDrawPoint(P6)
			\tkzDrawPoint(P7)
			\tkzDrawPoint(P8)
			\tkzDrawPoint(P9)
			\tkzDrawPoint(P10)
			\tkzDrawPoint(P11)
			\tkzLabelPoint(P0){$0$}
			\tkzLabelPoint(P1){$1$}
			\tkzLabelPoint(P2){$2$}
			\tkzLabelPoint(P3){$3$}
			\tkzLabelPoint(P4){$4$}
			\tkzLabelPoint(P5){$5$}
			\tkzLabelPoint(P6){$6$}
			\tkzLabelPoint(P7){$7$}
			\tkzLabelPoint(P8){$8$}
			\tkzLabelPoint(P9){$9$}
			\tkzLabelPoint(P10){$10$}
			\tkzLabelPoint(P11){$11$}
		}
	}
}

\addtocounter{framenumber}{-1}
\env{frame}
{
	\frametitle{Kútpur hjúpur punktasafns}
	\center
	\scalebox{0.5}
	{
		\env{tikzpicture}
		{
			\tkzInit[xmin=-10,xmax=10,ymin=-10,ymax=10]
			\tkzDefPoint(0,0){P0}
			\tkzDefPoint(10,1){P1}
			\tkzDefPoint(3,2){P2}
			\tkzDefPoint(4,4){P3}
			\tkzDefPoint(7,9){P4}
			\tkzDefPoint(2,8){P5}
			\tkzDefPoint(1,13){P6}
			\tkzDefPoint(-1,13){P7}
			\tkzDefPoint(-2,9){P8}
			\tkzDefPoint(-2,4){P9}
			\tkzDefPoint(-4,5){P10}
			\tkzDefPoint(-6,2){P11}
			\tkzDrawSegments(P11,P0 P0,P1 P1,P3)
			\tkzDrawSegments[dashed](P3,P4)
			\tkzDrawPoint(P0)
			\tkzDrawPoint(P1)
			\tkzDrawPoint(P2)
			\tkzDrawPoint(P3)
			\tkzDrawPoint(P4)
			\tkzDrawPoint(P5)
			\tkzDrawPoint(P6)
			\tkzDrawPoint(P7)
			\tkzDrawPoint(P8)
			\tkzDrawPoint(P9)
			\tkzDrawPoint(P10)
			\tkzDrawPoint(P11)
			\tkzLabelPoint(P0){$0$}
			\tkzLabelPoint(P1){$1$}
			\tkzLabelPoint(P2){$2$}
			\tkzLabelPoint(P3){$3$}
			\tkzLabelPoint(P4){$4$}
			\tkzLabelPoint(P5){$5$}
			\tkzLabelPoint(P6){$6$}
			\tkzLabelPoint(P7){$7$}
			\tkzLabelPoint(P8){$8$}
			\tkzLabelPoint(P9){$9$}
			\tkzLabelPoint(P10){$10$}
			\tkzLabelPoint(P11){$11$}
		}
	}
}

\addtocounter{framenumber}{-1}
\env{frame}
{
	\frametitle{Kútpur hjúpur punktasafns}
	\center
	\scalebox{0.5}
	{
		\env{tikzpicture}
		{
			\tkzInit[xmin=-10,xmax=10,ymin=-10,ymax=10]
			\tkzDefPoint(0,0){P0}
			\tkzDefPoint(10,1){P1}
			\tkzDefPoint(3,2){P2}
			\tkzDefPoint(4,4){P3}
			\tkzDefPoint(7,9){P4}
			\tkzDefPoint(2,8){P5}
			\tkzDefPoint(1,13){P6}
			\tkzDefPoint(-1,13){P7}
			\tkzDefPoint(-2,9){P8}
			\tkzDefPoint(-2,4){P9}
			\tkzDefPoint(-4,5){P10}
			\tkzDefPoint(-6,2){P11}
			\tkzDrawSegments(P11,P0 P0,P1)
			\tkzDrawSegments[dashed](P1,P4)
			\tkzDrawPoint(P0)
			\tkzDrawPoint(P1)
			\tkzDrawPoint(P2)
			\tkzDrawPoint(P3)
			\tkzDrawPoint(P4)
			\tkzDrawPoint(P5)
			\tkzDrawPoint(P6)
			\tkzDrawPoint(P7)
			\tkzDrawPoint(P8)
			\tkzDrawPoint(P9)
			\tkzDrawPoint(P10)
			\tkzDrawPoint(P11)
			\tkzLabelPoint(P0){$0$}
			\tkzLabelPoint(P1){$1$}
			\tkzLabelPoint(P2){$2$}
			\tkzLabelPoint(P3){$3$}
			\tkzLabelPoint(P4){$4$}
			\tkzLabelPoint(P5){$5$}
			\tkzLabelPoint(P6){$6$}
			\tkzLabelPoint(P7){$7$}
			\tkzLabelPoint(P8){$8$}
			\tkzLabelPoint(P9){$9$}
			\tkzLabelPoint(P10){$10$}
			\tkzLabelPoint(P11){$11$}
		}
	}
}

\addtocounter{framenumber}{-1}
\env{frame}
{
	\frametitle{Kútpur hjúpur punktasafns}
	\center
	\scalebox{0.5}
	{
		\env{tikzpicture}
		{
			\tkzInit[xmin=-10,xmax=10,ymin=-10,ymax=10]
			\tkzDefPoint(0,0){P0}
			\tkzDefPoint(10,1){P1}
			\tkzDefPoint(3,2){P2}
			\tkzDefPoint(4,4){P3}
			\tkzDefPoint(7,9){P4}
			\tkzDefPoint(2,8){P5}
			\tkzDefPoint(1,13){P6}
			\tkzDefPoint(-1,13){P7}
			\tkzDefPoint(-2,9){P8}
			\tkzDefPoint(-2,4){P9}
			\tkzDefPoint(-4,5){P10}
			\tkzDefPoint(-6,2){P11}
			\tkzDrawSegments(P11,P0 P0,P1 P1,P4)
			%\tkzDrawSegments[dashed](P1,P4)
			\tkzDrawPoint(P0)
			\tkzDrawPoint(P1)
			\tkzDrawPoint(P2)
			\tkzDrawPoint(P3)
			\tkzDrawPoint(P4)
			\tkzDrawPoint(P5)
			\tkzDrawPoint(P6)
			\tkzDrawPoint(P7)
			\tkzDrawPoint(P8)
			\tkzDrawPoint(P9)
			\tkzDrawPoint(P10)
			\tkzDrawPoint(P11)
			\tkzLabelPoint(P0){$0$}
			\tkzLabelPoint(P1){$1$}
			\tkzLabelPoint(P2){$2$}
			\tkzLabelPoint(P3){$3$}
			\tkzLabelPoint(P4){$4$}
			\tkzLabelPoint(P5){$5$}
			\tkzLabelPoint(P6){$6$}
			\tkzLabelPoint(P7){$7$}
			\tkzLabelPoint(P8){$8$}
			\tkzLabelPoint(P9){$9$}
			\tkzLabelPoint(P10){$10$}
			\tkzLabelPoint(P11){$11$}
		}
	}
}

\addtocounter{framenumber}{-1}
\env{frame}
{
	\frametitle{Kútpur hjúpur punktasafns}
	\center
	\scalebox{0.5}
	{
		\env{tikzpicture}
		{
			\tkzInit[xmin=-10,xmax=10,ymin=-10,ymax=10]
			\tkzDefPoint(0,0){P0}
			\tkzDefPoint(10,1){P1}
			\tkzDefPoint(3,2){P2}
			\tkzDefPoint(4,4){P3}
			\tkzDefPoint(7,9){P4}
			\tkzDefPoint(2,8){P5}
			\tkzDefPoint(1,13){P6}
			\tkzDefPoint(-1,13){P7}
			\tkzDefPoint(-2,9){P8}
			\tkzDefPoint(-2,4){P9}
			\tkzDefPoint(-4,5){P10}
			\tkzDefPoint(-6,2){P11}
			\tkzDrawSegments(P11,P0 P0,P1 P1,P4)
			\tkzDrawSegments[dashed](P4,P5)
			\tkzDrawPoint(P0)
			\tkzDrawPoint(P1)
			\tkzDrawPoint(P2)
			\tkzDrawPoint(P3)
			\tkzDrawPoint(P4)
			\tkzDrawPoint(P5)
			\tkzDrawPoint(P6)
			\tkzDrawPoint(P7)
			\tkzDrawPoint(P8)
			\tkzDrawPoint(P9)
			\tkzDrawPoint(P10)
			\tkzDrawPoint(P11)
			\tkzLabelPoint(P0){$0$}
			\tkzLabelPoint(P1){$1$}
			\tkzLabelPoint(P2){$2$}
			\tkzLabelPoint(P3){$3$}
			\tkzLabelPoint(P4){$4$}
			\tkzLabelPoint(P5){$5$}
			\tkzLabelPoint(P6){$6$}
			\tkzLabelPoint(P7){$7$}
			\tkzLabelPoint(P8){$8$}
			\tkzLabelPoint(P9){$9$}
			\tkzLabelPoint(P10){$10$}
			\tkzLabelPoint(P11){$11$}
		}
	}
}

\addtocounter{framenumber}{-1}
\env{frame}
{
	\frametitle{Kútpur hjúpur punktasafns}
	\center
	\scalebox{0.5}
	{
		\env{tikzpicture}
		{
			\tkzInit[xmin=-10,xmax=10,ymin=-10,ymax=10]
			\tkzDefPoint(0,0){P0}
			\tkzDefPoint(10,1){P1}
			\tkzDefPoint(3,2){P2}
			\tkzDefPoint(4,4){P3}
			\tkzDefPoint(7,9){P4}
			\tkzDefPoint(2,8){P5}
			\tkzDefPoint(1,13){P6}
			\tkzDefPoint(-1,13){P7}
			\tkzDefPoint(-2,9){P8}
			\tkzDefPoint(-2,4){P9}
			\tkzDefPoint(-4,5){P10}
			\tkzDefPoint(-6,2){P11}
			\tkzDrawSegments(P11,P0 P0,P1 P1,P4 P4,P5)
			%\tkzDrawSegments[dashed](P4,P5)
			\tkzDrawPoint(P0)
			\tkzDrawPoint(P1)
			\tkzDrawPoint(P2)
			\tkzDrawPoint(P3)
			\tkzDrawPoint(P4)
			\tkzDrawPoint(P5)
			\tkzDrawPoint(P6)
			\tkzDrawPoint(P7)
			\tkzDrawPoint(P8)
			\tkzDrawPoint(P9)
			\tkzDrawPoint(P10)
			\tkzDrawPoint(P11)
			\tkzLabelPoint(P0){$0$}
			\tkzLabelPoint(P1){$1$}
			\tkzLabelPoint(P2){$2$}
			\tkzLabelPoint(P3){$3$}
			\tkzLabelPoint(P4){$4$}
			\tkzLabelPoint(P5){$5$}
			\tkzLabelPoint(P6){$6$}
			\tkzLabelPoint(P7){$7$}
			\tkzLabelPoint(P8){$8$}
			\tkzLabelPoint(P9){$9$}
			\tkzLabelPoint(P10){$10$}
			\tkzLabelPoint(P11){$11$}
		}
	}
}

\addtocounter{framenumber}{-1}
\env{frame}
{
	\frametitle{Kútpur hjúpur punktasafns}
	\center
	\scalebox{0.5}
	{
		\env{tikzpicture}
		{
			\tkzInit[xmin=-10,xmax=10,ymin=-10,ymax=10]
			\tkzDefPoint(0,0){P0}
			\tkzDefPoint(10,1){P1}
			\tkzDefPoint(3,2){P2}
			\tkzDefPoint(4,4){P3}
			\tkzDefPoint(7,9){P4}
			\tkzDefPoint(2,8){P5}
			\tkzDefPoint(1,13){P6}
			\tkzDefPoint(-1,13){P7}
			\tkzDefPoint(-2,9){P8}
			\tkzDefPoint(-2,4){P9}
			\tkzDefPoint(-4,5){P10}
			\tkzDefPoint(-6,2){P11}
			\tkzDrawSegments(P11,P0 P0,P1 P1,P4 P4,P5)
			\tkzDrawSegments[dashed](P5,P6)
			\tkzDrawPoint(P0)
			\tkzDrawPoint(P1)
			\tkzDrawPoint(P2)
			\tkzDrawPoint(P3)
			\tkzDrawPoint(P4)
			\tkzDrawPoint(P5)
			\tkzDrawPoint(P6)
			\tkzDrawPoint(P7)
			\tkzDrawPoint(P8)
			\tkzDrawPoint(P9)
			\tkzDrawPoint(P10)
			\tkzDrawPoint(P11)
			\tkzLabelPoint(P0){$0$}
			\tkzLabelPoint(P1){$1$}
			\tkzLabelPoint(P2){$2$}
			\tkzLabelPoint(P3){$3$}
			\tkzLabelPoint(P4){$4$}
			\tkzLabelPoint(P5){$5$}
			\tkzLabelPoint(P6){$6$}
			\tkzLabelPoint(P7){$7$}
			\tkzLabelPoint(P8){$8$}
			\tkzLabelPoint(P9){$9$}
			\tkzLabelPoint(P10){$10$}
			\tkzLabelPoint(P11){$11$}
		}
	}
}

\addtocounter{framenumber}{-1}
\env{frame}
{
	\frametitle{Kútpur hjúpur punktasafns}
	\center
	\scalebox{0.5}
	{
		\env{tikzpicture}
		{
			\tkzInit[xmin=-10,xmax=10,ymin=-10,ymax=10]
			\tkzDefPoint(0,0){P0}
			\tkzDefPoint(10,1){P1}
			\tkzDefPoint(3,2){P2}
			\tkzDefPoint(4,4){P3}
			\tkzDefPoint(7,9){P4}
			\tkzDefPoint(2,8){P5}
			\tkzDefPoint(1,13){P6}
			\tkzDefPoint(-1,13){P7}
			\tkzDefPoint(-2,9){P8}
			\tkzDefPoint(-2,4){P9}
			\tkzDefPoint(-4,5){P10}
			\tkzDefPoint(-6,2){P11}
			\tkzDrawSegments(P11,P0 P0,P1 P1,P4)
			\tkzDrawSegments[dashed](P4,P6)
			\tkzDrawPoint(P0)
			\tkzDrawPoint(P1)
			\tkzDrawPoint(P2)
			\tkzDrawPoint(P3)
			\tkzDrawPoint(P4)
			\tkzDrawPoint(P5)
			\tkzDrawPoint(P6)
			\tkzDrawPoint(P7)
			\tkzDrawPoint(P8)
			\tkzDrawPoint(P9)
			\tkzDrawPoint(P10)
			\tkzDrawPoint(P11)
			\tkzLabelPoint(P0){$0$}
			\tkzLabelPoint(P1){$1$}
			\tkzLabelPoint(P2){$2$}
			\tkzLabelPoint(P3){$3$}
			\tkzLabelPoint(P4){$4$}
			\tkzLabelPoint(P5){$5$}
			\tkzLabelPoint(P6){$6$}
			\tkzLabelPoint(P7){$7$}
			\tkzLabelPoint(P8){$8$}
			\tkzLabelPoint(P9){$9$}
			\tkzLabelPoint(P10){$10$}
			\tkzLabelPoint(P11){$11$}
		}
	}
}

\addtocounter{framenumber}{-1}
\env{frame}
{
	\frametitle{Kútpur hjúpur punktasafns}
	\center
	\scalebox{0.5}
	{
		\env{tikzpicture}
		{
			\tkzInit[xmin=-10,xmax=10,ymin=-10,ymax=10]
			\tkzDefPoint(0,0){P0}
			\tkzDefPoint(10,1){P1}
			\tkzDefPoint(3,2){P2}
			\tkzDefPoint(4,4){P3}
			\tkzDefPoint(7,9){P4}
			\tkzDefPoint(2,8){P5}
			\tkzDefPoint(1,13){P6}
			\tkzDefPoint(-1,13){P7}
			\tkzDefPoint(-2,9){P8}
			\tkzDefPoint(-2,4){P9}
			\tkzDefPoint(-4,5){P10}
			\tkzDefPoint(-6,2){P11}
			\tkzDrawSegments(P11,P0 P0,P1 P1,P4 P4,P6)
			%\tkzDrawSegments[dashed](P4,P6)
			\tkzDrawPoint(P0)
			\tkzDrawPoint(P1)
			\tkzDrawPoint(P2)
			\tkzDrawPoint(P3)
			\tkzDrawPoint(P4)
			\tkzDrawPoint(P5)
			\tkzDrawPoint(P6)
			\tkzDrawPoint(P7)
			\tkzDrawPoint(P8)
			\tkzDrawPoint(P9)
			\tkzDrawPoint(P10)
			\tkzDrawPoint(P11)
			\tkzLabelPoint(P0){$0$}
			\tkzLabelPoint(P1){$1$}
			\tkzLabelPoint(P2){$2$}
			\tkzLabelPoint(P3){$3$}
			\tkzLabelPoint(P4){$4$}
			\tkzLabelPoint(P5){$5$}
			\tkzLabelPoint(P6){$6$}
			\tkzLabelPoint(P7){$7$}
			\tkzLabelPoint(P8){$8$}
			\tkzLabelPoint(P9){$9$}
			\tkzLabelPoint(P10){$10$}
			\tkzLabelPoint(P11){$11$}
		}
	}
}

\addtocounter{framenumber}{-1}
\env{frame}
{
	\frametitle{Kútpur hjúpur punktasafns}
	\center
	\scalebox{0.5}
	{
		\env{tikzpicture}
		{
			\tkzInit[xmin=-10,xmax=10,ymin=-10,ymax=10]
			\tkzDefPoint(0,0){P0}
			\tkzDefPoint(10,1){P1}
			\tkzDefPoint(3,2){P2}
			\tkzDefPoint(4,4){P3}
			\tkzDefPoint(7,9){P4}
			\tkzDefPoint(2,8){P5}
			\tkzDefPoint(1,13){P6}
			\tkzDefPoint(-1,13){P7}
			\tkzDefPoint(-2,9){P8}
			\tkzDefPoint(-2,4){P9}
			\tkzDefPoint(-4,5){P10}
			\tkzDefPoint(-6,2){P11}
			\tkzDrawSegments(P11,P0 P0,P1 P1,P4 P4,P6)
			\tkzDrawSegments[dashed](P6,P7)
			\tkzDrawPoint(P0)
			\tkzDrawPoint(P1)
			\tkzDrawPoint(P2)
			\tkzDrawPoint(P3)
			\tkzDrawPoint(P4)
			\tkzDrawPoint(P5)
			\tkzDrawPoint(P6)
			\tkzDrawPoint(P7)
			\tkzDrawPoint(P8)
			\tkzDrawPoint(P9)
			\tkzDrawPoint(P10)
			\tkzDrawPoint(P11)
			\tkzLabelPoint(P0){$0$}
			\tkzLabelPoint(P1){$1$}
			\tkzLabelPoint(P2){$2$}
			\tkzLabelPoint(P3){$3$}
			\tkzLabelPoint(P4){$4$}
			\tkzLabelPoint(P5){$5$}
			\tkzLabelPoint(P6){$6$}
			\tkzLabelPoint(P7){$7$}
			\tkzLabelPoint(P8){$8$}
			\tkzLabelPoint(P9){$9$}
			\tkzLabelPoint(P10){$10$}
			\tkzLabelPoint(P11){$11$}
		}
	}
}

\addtocounter{framenumber}{-1}
\env{frame}
{
	\frametitle{Kútpur hjúpur punktasafns}
	\center
	\scalebox{0.5}
	{
		\env{tikzpicture}
		{
			\tkzInit[xmin=-10,xmax=10,ymin=-10,ymax=10]
			\tkzDefPoint(0,0){P0}
			\tkzDefPoint(10,1){P1}
			\tkzDefPoint(3,2){P2}
			\tkzDefPoint(4,4){P3}
			\tkzDefPoint(7,9){P4}
			\tkzDefPoint(2,8){P5}
			\tkzDefPoint(1,13){P6}
			\tkzDefPoint(-1,13){P7}
			\tkzDefPoint(-2,9){P8}
			\tkzDefPoint(-2,4){P9}
			\tkzDefPoint(-4,5){P10}
			\tkzDefPoint(-6,2){P11}
			\tkzDrawSegments(P11,P0 P0,P1 P1,P4 P4,P6 P6,P7)
			%\tkzDrawSegments[dashed](P6,P7)
			\tkzDrawPoint(P0)
			\tkzDrawPoint(P1)
			\tkzDrawPoint(P2)
			\tkzDrawPoint(P3)
			\tkzDrawPoint(P4)
			\tkzDrawPoint(P5)
			\tkzDrawPoint(P6)
			\tkzDrawPoint(P7)
			\tkzDrawPoint(P8)
			\tkzDrawPoint(P9)
			\tkzDrawPoint(P10)
			\tkzDrawPoint(P11)
			\tkzLabelPoint(P0){$0$}
			\tkzLabelPoint(P1){$1$}
			\tkzLabelPoint(P2){$2$}
			\tkzLabelPoint(P3){$3$}
			\tkzLabelPoint(P4){$4$}
			\tkzLabelPoint(P5){$5$}
			\tkzLabelPoint(P6){$6$}
			\tkzLabelPoint(P7){$7$}
			\tkzLabelPoint(P8){$8$}
			\tkzLabelPoint(P9){$9$}
			\tkzLabelPoint(P10){$10$}
			\tkzLabelPoint(P11){$11$}
		}
	}
}

\addtocounter{framenumber}{-1}
\env{frame}
{
	\frametitle{Kútpur hjúpur punktasafns}
	\center
	\scalebox{0.5}
	{
		\env{tikzpicture}
		{
			\tkzInit[xmin=-10,xmax=10,ymin=-10,ymax=10]
			\tkzDefPoint(0,0){P0}
			\tkzDefPoint(10,1){P1}
			\tkzDefPoint(3,2){P2}
			\tkzDefPoint(4,4){P3}
			\tkzDefPoint(7,9){P4}
			\tkzDefPoint(2,8){P5}
			\tkzDefPoint(1,13){P6}
			\tkzDefPoint(-1,13){P7}
			\tkzDefPoint(-2,9){P8}
			\tkzDefPoint(-2,4){P9}
			\tkzDefPoint(-4,5){P10}
			\tkzDefPoint(-6,2){P11}
			\tkzDrawSegments(P11,P0 P0,P1 P1,P4 P4,P6 P6,P7)
			\tkzDrawSegments[dashed](P7,P8)
			\tkzDrawPoint(P0)
			\tkzDrawPoint(P1)
			\tkzDrawPoint(P2)
			\tkzDrawPoint(P3)
			\tkzDrawPoint(P4)
			\tkzDrawPoint(P5)
			\tkzDrawPoint(P6)
			\tkzDrawPoint(P7)
			\tkzDrawPoint(P8)
			\tkzDrawPoint(P9)
			\tkzDrawPoint(P10)
			\tkzDrawPoint(P11)
			\tkzLabelPoint(P0){$0$}
			\tkzLabelPoint(P1){$1$}
			\tkzLabelPoint(P2){$2$}
			\tkzLabelPoint(P3){$3$}
			\tkzLabelPoint(P4){$4$}
			\tkzLabelPoint(P5){$5$}
			\tkzLabelPoint(P6){$6$}
			\tkzLabelPoint(P7){$7$}
			\tkzLabelPoint(P8){$8$}
			\tkzLabelPoint(P9){$9$}
			\tkzLabelPoint(P10){$10$}
			\tkzLabelPoint(P11){$11$}
		}
	}
}

\addtocounter{framenumber}{-1}
\env{frame}
{
	\frametitle{Kútpur hjúpur punktasafns}
	\center
	\scalebox{0.5}
	{
		\env{tikzpicture}
		{
			\tkzInit[xmin=-10,xmax=10,ymin=-10,ymax=10]
			\tkzDefPoint(0,0){P0}
			\tkzDefPoint(10,1){P1}
			\tkzDefPoint(3,2){P2}
			\tkzDefPoint(4,4){P3}
			\tkzDefPoint(7,9){P4}
			\tkzDefPoint(2,8){P5}
			\tkzDefPoint(1,13){P6}
			\tkzDefPoint(-1,13){P7}
			\tkzDefPoint(-2,9){P8}
			\tkzDefPoint(-2,4){P9}
			\tkzDefPoint(-4,5){P10}
			\tkzDefPoint(-6,2){P11}
			\tkzDrawSegments(P11,P0 P0,P1 P1,P4 P4,P6 P6,P7 P7,P8)
			%\tkzDrawSegments[dashed](P7,P8)
			\tkzDrawPoint(P0)
			\tkzDrawPoint(P1)
			\tkzDrawPoint(P2)
			\tkzDrawPoint(P3)
			\tkzDrawPoint(P4)
			\tkzDrawPoint(P5)
			\tkzDrawPoint(P6)
			\tkzDrawPoint(P7)
			\tkzDrawPoint(P8)
			\tkzDrawPoint(P9)
			\tkzDrawPoint(P10)
			\tkzDrawPoint(P11)
			\tkzLabelPoint(P0){$0$}
			\tkzLabelPoint(P1){$1$}
			\tkzLabelPoint(P2){$2$}
			\tkzLabelPoint(P3){$3$}
			\tkzLabelPoint(P4){$4$}
			\tkzLabelPoint(P5){$5$}
			\tkzLabelPoint(P6){$6$}
			\tkzLabelPoint(P7){$7$}
			\tkzLabelPoint(P8){$8$}
			\tkzLabelPoint(P9){$9$}
			\tkzLabelPoint(P10){$10$}
			\tkzLabelPoint(P11){$11$}
		}
	}
}

\addtocounter{framenumber}{-1}
\env{frame}
{
	\frametitle{Kútpur hjúpur punktasafns}
	\center
	\scalebox{0.5}
	{
		\env{tikzpicture}
		{
			\tkzInit[xmin=-10,xmax=10,ymin=-10,ymax=10]
			\tkzDefPoint(0,0){P0}
			\tkzDefPoint(10,1){P1}
			\tkzDefPoint(3,2){P2}
			\tkzDefPoint(4,4){P3}
			\tkzDefPoint(7,9){P4}
			\tkzDefPoint(2,8){P5}
			\tkzDefPoint(1,13){P6}
			\tkzDefPoint(-1,13){P7}
			\tkzDefPoint(-2,9){P8}
			\tkzDefPoint(-2,4){P9}
			\tkzDefPoint(-4,5){P10}
			\tkzDefPoint(-6,2){P11}
			\tkzDrawSegments(P11,P0 P0,P1 P1,P4 P4,P6 P6,P7 P7,P8)
			\tkzDrawSegments[dashed](P8,P9)
			\tkzDrawPoint(P0)
			\tkzDrawPoint(P1)
			\tkzDrawPoint(P2)
			\tkzDrawPoint(P3)
			\tkzDrawPoint(P4)
			\tkzDrawPoint(P5)
			\tkzDrawPoint(P6)
			\tkzDrawPoint(P7)
			\tkzDrawPoint(P8)
			\tkzDrawPoint(P9)
			\tkzDrawPoint(P10)
			\tkzDrawPoint(P11)
			\tkzLabelPoint(P0){$0$}
			\tkzLabelPoint(P1){$1$}
			\tkzLabelPoint(P2){$2$}
			\tkzLabelPoint(P3){$3$}
			\tkzLabelPoint(P4){$4$}
			\tkzLabelPoint(P5){$5$}
			\tkzLabelPoint(P6){$6$}
			\tkzLabelPoint(P7){$7$}
			\tkzLabelPoint(P8){$8$}
			\tkzLabelPoint(P9){$9$}
			\tkzLabelPoint(P10){$10$}
			\tkzLabelPoint(P11){$11$}
		}
	}
}

\addtocounter{framenumber}{-1}
\env{frame}
{
	\frametitle{Kútpur hjúpur punktasafns}
	\center
	\scalebox{0.5}
	{
		\env{tikzpicture}
		{
			\tkzInit[xmin=-10,xmax=10,ymin=-10,ymax=10]
			\tkzDefPoint(0,0){P0}
			\tkzDefPoint(10,1){P1}
			\tkzDefPoint(3,2){P2}
			\tkzDefPoint(4,4){P3}
			\tkzDefPoint(7,9){P4}
			\tkzDefPoint(2,8){P5}
			\tkzDefPoint(1,13){P6}
			\tkzDefPoint(-1,13){P7}
			\tkzDefPoint(-2,9){P8}
			\tkzDefPoint(-2,4){P9}
			\tkzDefPoint(-4,5){P10}
			\tkzDefPoint(-6,2){P11}
			\tkzDrawSegments(P11,P0 P0,P1 P1,P4 P4,P6 P6,P7 P7,P8 P8,P9)
			%\tkzDrawSegments[dashed](P8,P9)
			\tkzDrawPoint(P0)
			\tkzDrawPoint(P1)
			\tkzDrawPoint(P2)
			\tkzDrawPoint(P3)
			\tkzDrawPoint(P4)
			\tkzDrawPoint(P5)
			\tkzDrawPoint(P6)
			\tkzDrawPoint(P7)
			\tkzDrawPoint(P8)
			\tkzDrawPoint(P9)
			\tkzDrawPoint(P10)
			\tkzDrawPoint(P11)
			\tkzLabelPoint(P0){$0$}
			\tkzLabelPoint(P1){$1$}
			\tkzLabelPoint(P2){$2$}
			\tkzLabelPoint(P3){$3$}
			\tkzLabelPoint(P4){$4$}
			\tkzLabelPoint(P5){$5$}
			\tkzLabelPoint(P6){$6$}
			\tkzLabelPoint(P7){$7$}
			\tkzLabelPoint(P8){$8$}
			\tkzLabelPoint(P9){$9$}
			\tkzLabelPoint(P10){$10$}
			\tkzLabelPoint(P11){$11$}
		}
	}
}

\addtocounter{framenumber}{-1}
\env{frame}
{
	\frametitle{Kútpur hjúpur punktasafns}
	\center
	\scalebox{0.5}
	{
		\env{tikzpicture}
		{
			\tkzInit[xmin=-10,xmax=10,ymin=-10,ymax=10]
			\tkzDefPoint(0,0){P0}
			\tkzDefPoint(10,1){P1}
			\tkzDefPoint(3,2){P2}
			\tkzDefPoint(4,4){P3}
			\tkzDefPoint(7,9){P4}
			\tkzDefPoint(2,8){P5}
			\tkzDefPoint(1,13){P6}
			\tkzDefPoint(-1,13){P7}
			\tkzDefPoint(-2,9){P8}
			\tkzDefPoint(-2,4){P9}
			\tkzDefPoint(-4,5){P10}
			\tkzDefPoint(-6,2){P11}
			\tkzDrawSegments(P11,P0 P0,P1 P1,P4 P4,P6 P6,P7 P7,P8 P8,P9)
			\tkzDrawSegments[dashed](P9,P10)
			\tkzDrawPoint(P0)
			\tkzDrawPoint(P1)
			\tkzDrawPoint(P2)
			\tkzDrawPoint(P3)
			\tkzDrawPoint(P4)
			\tkzDrawPoint(P5)
			\tkzDrawPoint(P6)
			\tkzDrawPoint(P7)
			\tkzDrawPoint(P8)
			\tkzDrawPoint(P9)
			\tkzDrawPoint(P10)
			\tkzDrawPoint(P11)
			\tkzLabelPoint(P0){$0$}
			\tkzLabelPoint(P1){$1$}
			\tkzLabelPoint(P2){$2$}
			\tkzLabelPoint(P3){$3$}
			\tkzLabelPoint(P4){$4$}
			\tkzLabelPoint(P5){$5$}
			\tkzLabelPoint(P6){$6$}
			\tkzLabelPoint(P7){$7$}
			\tkzLabelPoint(P8){$8$}
			\tkzLabelPoint(P9){$9$}
			\tkzLabelPoint(P10){$10$}
			\tkzLabelPoint(P11){$11$}
		}
	}
}

\addtocounter{framenumber}{-1}
\env{frame}
{
	\frametitle{Kútpur hjúpur punktasafns}
	\center
	\scalebox{0.5}
	{
		\env{tikzpicture}
		{
			\tkzInit[xmin=-10,xmax=10,ymin=-10,ymax=10]
			\tkzDefPoint(0,0){P0}
			\tkzDefPoint(10,1){P1}
			\tkzDefPoint(3,2){P2}
			\tkzDefPoint(4,4){P3}
			\tkzDefPoint(7,9){P4}
			\tkzDefPoint(2,8){P5}
			\tkzDefPoint(1,13){P6}
			\tkzDefPoint(-1,13){P7}
			\tkzDefPoint(-2,9){P8}
			\tkzDefPoint(-2,4){P9}
			\tkzDefPoint(-4,5){P10}
			\tkzDefPoint(-6,2){P11}
			\tkzDrawSegments(P11,P0 P0,P1 P1,P4 P4,P6 P6,P7 P7,P8)
			\tkzDrawSegments[dashed](P8,P10)
			\tkzDrawPoint(P0)
			\tkzDrawPoint(P1)
			\tkzDrawPoint(P2)
			\tkzDrawPoint(P3)
			\tkzDrawPoint(P4)
			\tkzDrawPoint(P5)
			\tkzDrawPoint(P6)
			\tkzDrawPoint(P7)
			\tkzDrawPoint(P8)
			\tkzDrawPoint(P9)
			\tkzDrawPoint(P10)
			\tkzDrawPoint(P11)
			\tkzLabelPoint(P0){$0$}
			\tkzLabelPoint(P1){$1$}
			\tkzLabelPoint(P2){$2$}
			\tkzLabelPoint(P3){$3$}
			\tkzLabelPoint(P4){$4$}
			\tkzLabelPoint(P5){$5$}
			\tkzLabelPoint(P6){$6$}
			\tkzLabelPoint(P7){$7$}
			\tkzLabelPoint(P8){$8$}
			\tkzLabelPoint(P9){$9$}
			\tkzLabelPoint(P10){$10$}
			\tkzLabelPoint(P11){$11$}
		}
	}
}

\addtocounter{framenumber}{-1}
\env{frame}
{
	\frametitle{Kútpur hjúpur punktasafns}
	\center
	\scalebox{0.5}
	{
		\env{tikzpicture}
		{
			\tkzInit[xmin=-10,xmax=10,ymin=-10,ymax=10]
			\tkzDefPoint(0,0){P0}
			\tkzDefPoint(10,1){P1}
			\tkzDefPoint(3,2){P2}
			\tkzDefPoint(4,4){P3}
			\tkzDefPoint(7,9){P4}
			\tkzDefPoint(2,8){P5}
			\tkzDefPoint(1,13){P6}
			\tkzDefPoint(-1,13){P7}
			\tkzDefPoint(-2,9){P8}
			\tkzDefPoint(-2,4){P9}
			\tkzDefPoint(-4,5){P10}
			\tkzDefPoint(-6,2){P11}
			\tkzDrawSegments(P11,P0 P0,P1 P1,P4 P4,P6 P6,P7)
			\tkzDrawSegments[dashed](P7,P10)
			\tkzDrawPoint(P0)
			\tkzDrawPoint(P1)
			\tkzDrawPoint(P2)
			\tkzDrawPoint(P3)
			\tkzDrawPoint(P4)
			\tkzDrawPoint(P5)
			\tkzDrawPoint(P6)
			\tkzDrawPoint(P7)
			\tkzDrawPoint(P8)
			\tkzDrawPoint(P9)
			\tkzDrawPoint(P10)
			\tkzDrawPoint(P11)
			\tkzLabelPoint(P0){$0$}
			\tkzLabelPoint(P1){$1$}
			\tkzLabelPoint(P2){$2$}
			\tkzLabelPoint(P3){$3$}
			\tkzLabelPoint(P4){$4$}
			\tkzLabelPoint(P5){$5$}
			\tkzLabelPoint(P6){$6$}
			\tkzLabelPoint(P7){$7$}
			\tkzLabelPoint(P8){$8$}
			\tkzLabelPoint(P9){$9$}
			\tkzLabelPoint(P10){$10$}
			\tkzLabelPoint(P11){$11$}
		}
	}
}

\addtocounter{framenumber}{-1}
\env{frame}
{
	\frametitle{Kútpur hjúpur punktasafns}
	\center
	\scalebox{0.5}
	{
		\env{tikzpicture}
		{
			\tkzInit[xmin=-10,xmax=10,ymin=-10,ymax=10]
			\tkzDefPoint(0,0){P0}
			\tkzDefPoint(10,1){P1}
			\tkzDefPoint(3,2){P2}
			\tkzDefPoint(4,4){P3}
			\tkzDefPoint(7,9){P4}
			\tkzDefPoint(2,8){P5}
			\tkzDefPoint(1,13){P6}
			\tkzDefPoint(-1,13){P7}
			\tkzDefPoint(-2,9){P8}
			\tkzDefPoint(-2,4){P9}
			\tkzDefPoint(-4,5){P10}
			\tkzDefPoint(-6,2){P11}
			\tkzDrawSegments(P11,P0 P0,P1 P1,P4 P4,P6 P6,P7 P7,P10)
			%\tkzDrawSegments[dashed](P10,P11)
			\tkzDrawPoint(P0)
			\tkzDrawPoint(P1)
			\tkzDrawPoint(P2)
			\tkzDrawPoint(P3)
			\tkzDrawPoint(P4)
			\tkzDrawPoint(P5)
			\tkzDrawPoint(P6)
			\tkzDrawPoint(P7)
			\tkzDrawPoint(P8)
			\tkzDrawPoint(P9)
			\tkzDrawPoint(P10)
			\tkzDrawPoint(P11)
			\tkzLabelPoint(P0){$0$}
			\tkzLabelPoint(P1){$1$}
			\tkzLabelPoint(P2){$2$}
			\tkzLabelPoint(P3){$3$}
			\tkzLabelPoint(P4){$4$}
			\tkzLabelPoint(P5){$5$}
			\tkzLabelPoint(P6){$6$}
			\tkzLabelPoint(P7){$7$}
			\tkzLabelPoint(P8){$8$}
			\tkzLabelPoint(P9){$9$}
			\tkzLabelPoint(P10){$10$}
			\tkzLabelPoint(P11){$11$}
		}
	}
}

\addtocounter{framenumber}{-1}
\env{frame}
{
	\frametitle{Kútpur hjúpur punktasafns}
	\center
	\scalebox{0.5}
	{
		\env{tikzpicture}
		{
			\tkzInit[xmin=-10,xmax=10,ymin=-10,ymax=10]
			\tkzDefPoint(0,0){P0}
			\tkzDefPoint(10,1){P1}
			\tkzDefPoint(3,2){P2}
			\tkzDefPoint(4,4){P3}
			\tkzDefPoint(7,9){P4}
			\tkzDefPoint(2,8){P5}
			\tkzDefPoint(1,13){P6}
			\tkzDefPoint(-1,13){P7}
			\tkzDefPoint(-2,9){P8}
			\tkzDefPoint(-2,4){P9}
			\tkzDefPoint(-4,5){P10}
			\tkzDefPoint(-6,2){P11}
			\tkzDrawSegments(P11,P0 P0,P1 P1,P4 P4,P6 P6,P7 P7,P10)
			\tkzDrawSegments[dashed](P10,P11)
			\tkzDrawPoint(P0)
			\tkzDrawPoint(P1)
			\tkzDrawPoint(P2)
			\tkzDrawPoint(P3)
			\tkzDrawPoint(P4)
			\tkzDrawPoint(P5)
			\tkzDrawPoint(P6)
			\tkzDrawPoint(P7)
			\tkzDrawPoint(P8)
			\tkzDrawPoint(P9)
			\tkzDrawPoint(P10)
			\tkzDrawPoint(P11)
			\tkzLabelPoint(P0){$0$}
			\tkzLabelPoint(P1){$1$}
			\tkzLabelPoint(P2){$2$}
			\tkzLabelPoint(P3){$3$}
			\tkzLabelPoint(P4){$4$}
			\tkzLabelPoint(P5){$5$}
			\tkzLabelPoint(P6){$6$}
			\tkzLabelPoint(P7){$7$}
			\tkzLabelPoint(P8){$8$}
			\tkzLabelPoint(P9){$9$}
			\tkzLabelPoint(P10){$10$}
			\tkzLabelPoint(P11){$11$}
		}
	}
}

\addtocounter{framenumber}{-1}
\env{frame}
{
	\frametitle{Kútpur hjúpur punktasafns}
	\center
	\scalebox{0.5}
	{
		\env{tikzpicture}
		{
			\tkzInit[xmin=-10,xmax=10,ymin=-10,ymax=10]
			\tkzDefPoint(0,0){P0}
			\tkzDefPoint(10,1){P1}
			\tkzDefPoint(3,2){P2}
			\tkzDefPoint(4,4){P3}
			\tkzDefPoint(7,9){P4}
			\tkzDefPoint(2,8){P5}
			\tkzDefPoint(1,13){P6}
			\tkzDefPoint(-1,13){P7}
			\tkzDefPoint(-2,9){P8}
			\tkzDefPoint(-2,4){P9}
			\tkzDefPoint(-4,5){P10}
			\tkzDefPoint(-6,2){P11}
			\tkzDrawSegments(P11,P0 P0,P1 P1,P4 P4,P6 P6,P7)
			\tkzDrawSegments[dashed](P7,P11)
			\tkzDrawPoint(P0)
			\tkzDrawPoint(P1)
			\tkzDrawPoint(P2)
			\tkzDrawPoint(P3)
			\tkzDrawPoint(P4)
			\tkzDrawPoint(P5)
			\tkzDrawPoint(P6)
			\tkzDrawPoint(P7)
			\tkzDrawPoint(P8)
			\tkzDrawPoint(P9)
			\tkzDrawPoint(P10)
			\tkzDrawPoint(P11)
			\tkzLabelPoint(P0){$0$}
			\tkzLabelPoint(P1){$1$}
			\tkzLabelPoint(P2){$2$}
			\tkzLabelPoint(P3){$3$}
			\tkzLabelPoint(P4){$4$}
			\tkzLabelPoint(P5){$5$}
			\tkzLabelPoint(P6){$6$}
			\tkzLabelPoint(P7){$7$}
			\tkzLabelPoint(P8){$8$}
			\tkzLabelPoint(P9){$9$}
			\tkzLabelPoint(P10){$10$}
			\tkzLabelPoint(P11){$11$}
		}
	}
}

\addtocounter{framenumber}{-1}
\env{frame}
{
	\frametitle{Kútpur hjúpur punktasafns}
	\center
	\scalebox{0.5}
	{
		\env{tikzpicture}
		{
			\tkzInit[xmin=-10,xmax=10,ymin=-10,ymax=10]
			\tkzDefPoint(0,0){P0}
			\tkzDefPoint(10,1){P1}
			\tkzDefPoint(3,2){P2}
			\tkzDefPoint(4,4){P3}
			\tkzDefPoint(7,9){P4}
			\tkzDefPoint(2,8){P5}
			\tkzDefPoint(1,13){P6}
			\tkzDefPoint(-1,13){P7}
			\tkzDefPoint(-2,9){P8}
			\tkzDefPoint(-2,4){P9}
			\tkzDefPoint(-4,5){P10}
			\tkzDefPoint(-6,2){P11}
			\tkzDrawSegments(P11,P0 P0,P1 P1,P4 P4,P6 P6,P7 P7,P11)
			%\tkzDrawSegments[dashed](P7,P11)
			\tkzDrawPoint(P0)
			\tkzDrawPoint(P1)
			\tkzDrawPoint(P2)
			\tkzDrawPoint(P3)
			\tkzDrawPoint(P4)
			\tkzDrawPoint(P5)
			\tkzDrawPoint(P6)
			\tkzDrawPoint(P7)
			\tkzDrawPoint(P8)
			\tkzDrawPoint(P9)
			\tkzDrawPoint(P10)
			\tkzDrawPoint(P11)
			\tkzLabelPoint(P0){$0$}
			\tkzLabelPoint(P1){$1$}
			\tkzLabelPoint(P2){$2$}
			\tkzLabelPoint(P3){$3$}
			\tkzLabelPoint(P4){$4$}
			\tkzLabelPoint(P5){$5$}
			\tkzLabelPoint(P6){$6$}
			\tkzLabelPoint(P7){$7$}
			\tkzLabelPoint(P8){$8$}
			\tkzLabelPoint(P9){$9$}
			\tkzLabelPoint(P10){$10$}
			\tkzLabelPoint(P11){$11$}
		}
	}
}

\env{frame}
{
	\frametitle{Graham's Scan}
	\env{itemize}
	{
		\item<1-> Það er ljóst að hlaðinn í lok reikniritsins lýsir kúptum marghyrningi.
		\item<2-> Það er þó aðeins meira mál að sýna að þetta sé í raun kúpti hjúpur punktasafnsins.
		\item<3-> Við látum það ógert í þessum fyrirlestri.
		\item<4-> Ef punktasafnið inniheldur $n$ punkta þá er reikniritið $\mathcal{O}(n \log n)$ út af því við þurfum að 
			raða punktunum. Eftir röðun er reikniritið $\mathcal{O}(n)$ því hver punktur fer inn á hlaðan einu sinni.
		\item<5-> {\bf ATH:} Útfærslan á næstu glæru gerir ráð fyrir því að það sé til óúrkynjaður kúptur hjúpur,
			m.ö.o. til eru þrír punktar í safninu sem liggja ekki á sömu línu.
		\item<6-> {\bf Æfing:} Búið til fall sem ákvarðar hvort punkta safn hafi óúrkynjaðan kúpt hjúp ({\bf Ábending:}
			minniháttar breytingar \texttt{ccw} gefa fall sem segir hvort gefnir punktar liggi á sömu línu).
	}
}

\env{frame}
{
	\frametitle{Graham's Scan}
	\code{convex_hull.h}
}

\env{frame}
{
	\frametitle{Þriðjungunarleit}
	\env{itemize}
	{
		\item<1-> Þið hafið öll heyrt um helmingunarleit.
		\item<2-> Sum ykkar hafa þó kannski ekki heyrt um þriðjungunarleit.
		\item<3-> Þriðjungunarleit finnur útgildi kúpts falls. 
		\item<4-> Ólíkt helmingurnarleit þá skiptum við leitarsvæðinu upp í þriðjunga, í stað helminga,
			og notum til þess tvo punkta.
		\item<5-> Látum $f$ vera kúpt fall á $[a, b]$.
		\item<6-> Þá gildir fyrir $s, t \in [a, b]$, $s < t$ að ef $f(s) > f(t)$ þá tekur $f$ lágildi
			á $[s, b]$, en annars tekur $f$ lágildi á $[a, t]$.
	}
}

\env{frame}
{
	\frametitle{Þriðjungunar leit}
	\code{ts.h}
}

\env{frame}
{
	\frametitle{Niðurlag}
	\env{itemize}
	{
		\item<1-> Við erum nú búnir að fara yfir það efni sem við ætluðum okkur.
		\item<2-> Ef ykkur vantar námskeið fyrir komandi annir sem tengjast lauslega keppnisforritun
			eru t.d. í boði \emph{Þýðendur}, \emph{Líkinda og tölfræði}, \emph{Fléttufræði},
			\emph{Rúmfræði}, \emph{Algebra II}, \emph{Slembiferli}, \emph{Tvinnfallagreining (Stærðfræðigreining IIIB)}
			og \emph{Töluleg greining}.
		\item<3-> Við mælum sérstaklega með \emph{Algebra I}, \emph{Greining reiknirita} og \emph{Netafræði}.
	}
}

\end{document}
